%%%%%%%%%%%%%%%%%%%%%%%%%%
%
%    Preface Page
%
%%%%%%%%%%%%%%%%%%%%%%%%%%
\section*{\MakeUppercase{Preface}}
\addcontentsline{toc}{section}{\MakeUppercase{Preface}}

The motivation behind this work, as I have become realizing not long ago, is to settle down own confusion that, in the past, normally overrun logical thinking every time a discussion was taking place on money and their \textit{flows} or \textit{circulation}. Back then, that sort of the personal confusion was striking and dominating. It was disempowering to such an extent that dropping out of the above-mentioned discussions was preferable. 

That was my experience back then. It was present, as I progressed my professional career in the field of financial and economic research from position of financial analyst at ART-Capital, one of the pioneers of securities business in my home country of Ukraine, to the position of Head of Research at ING Bank Ukraine, a subsidiary of the Dutch financial giant ING. I used to subscribe totally to the basic tenets of the standard thinking about economic processes such as gross domestic product, balance of payments, government deficits, etc. And in that standard way of thinking the concept of \textit{capital flows} was looming large permanently.

Now, I can affirm that I am ready to lay down an argument that has a bit of an empowering logic. In short, the widely accepted metaphors of motion with respect to money and finance---such as flows, mobility, circulation, etc.---are inconvenient tools. By themselves they are the core source of the confusion. This dissertation argues explicitly for a distinctly clear description.

\subsubsection*{Origins}

Due to my leanings to computer coding I landed an entry-level job in the banking industry in early 1995.\footnote{This section was written over late 2021 and early 2022. The exact date of starting it as notes was October 16th, 2021.} I became an employee of the computer department of \ac{lad_pib}, which was located in my hometown Ladyzhyn in the Vinnytsia region, which is a south-western part of Ukraine.

How to code I learned while studying in the power-stations department of \ac{vpi} in 1989-1994.  \ac{vpi} was and still is a public university. Students attending the power-stations department are graduating as electrical engineers with a specialty in working at power stations. Generally, studies at \ac{vpi} are heavy loaded with higher mathematics and physics. There, too, I was exposed to some bits of knowledge of what is now called computer science. My graduating thesis was on some optimization process in power engineering, which utilized the trials-and-errors method from mathematics. I developed a computer code that was solving that question as part of my thesis.

I graduated from \ac{vpi} in summer 1994 with an honors diploma and an electrical engineer certification. Hence, my first job was as a very entry-level electrician at \ac{lad_tpp}, then a state-run enterprise and the major employer in my hometown. After six months in this job, I grabbed immediately the first opportunity that popped up to work with computers and do coding. I was told that a local bank, which the power station used as a cash office delivering wages to its employees, was looking for a computer assistant. To me, switching from the job of an electrician in the large power station to a job in the computer department of supposedly dull place of a local bank was a decision full of promise and high expectations in my professional satisfaction. When I brought my resignation letter to my superior in the power station, he, an experienced middle-aged man, mentioned that banking associates to him with middle-aged female accountants drinking tea most of the time while on the job. I did not care much about what he said as I just wanted coding while on the job. A bit later, a recognition arrived to me however that indeed he was right in one detail that bank's employees were mostly female, mostly middle-aged. In everything else, his statement was inaccurate. 

Back in 1995, Ukraine's economy was not doing well. In fact, it was in severe depression as all new independent counties that used to be republics of Soviet Union experienced similar development. The country had just experienced hyperinflation in 1992-94. In 1995, monthly inflation was still running high (in many cases at double digit rates in month-on-month terms). 

To illustrate high inflation, let me give the following example: the monthly wage I earned working for a power station was just enough to buy a pair of shoes at the local bazaar and that was it until the next wage payment. This is how prices for ordinary things were rising disproportionately ahead in relation to the wage. 

General people, just to secure themselves from the acute inflation risk, were habitually buying what is called locally as \textit{valuta} (this is predominantly cash US dollars). They did that if they could afford not to spend those funds for supporting themselves until the next wage/pension payment. Anybody in our small town, and in any other towns of the country large and small, could exchange Ukrainian cash currency for ``valuta" at the local open-air bazaar. Usage of the word ``valuta" migrated from the Soviet period. During that period that term symbolized (a) something that the majority of people could neither see nor hold and it was heard of, and (b) something that the state-run economy of the Soviet Union used in order to pay for valuable imports.

My first encounter with cash US dollar paper notes took place in very early 1990s, while studying at \ac{vpi}. Many international students were studying there. Some of them did possess cash US dollars and were willing to pay in them to other students to help them with their homework assignments. A Lebanese student paid me for such tutoring job, according to prior agreement, by handing me a cash paper note of 10 US dollars. 

At that time, we as students at \ac{vpi} were not very much aware of banks' existence. It must had been that way because (a) all consumer related transactions were done in cash paper notes, and (b) the Ukraine's banking system at the time predominantly relied on the banking infrastructure left from Soviet Union state-run system. 

In its pure form, before the 1988 grand reform known as \textit{perestroika}, it was a system dominated by State Bank, a state banking authority and by design an entity that was carrying all basic banking transactions. The 1988 reform introduced a window of freedom to banking as a business activity. Nevertheless, the banks themselves and their operations were rather invisible to the general public. 

As a \ac{vpi} student with grade rate above certain threshold, I was paid a small state stipend as many other students. We were handed over stipend in cash currency from a cashier window in one of the \ac{vpi}'s buildings. Later as an employee of \ac{lad_tpp} I was paid a wage and the actual process of wage payment was organized this way: it was too paid in cash of local currency and through the window of cashier, which was physically situated on the site of \ac{lad_tpp} I worked for. 

Only later after becoming a bank employee, I recognized that the cashier window, in fact, was operated by the bank now I worked for. \ac{lad_pib} rented a room within the office premises of \ac{lad_tpp}. In modern parlance, the power plant outsourced cash management to the local bank.

However, as banks were largely out of sight for most of the public, a sense of money disorder was very much on display. Prices for all goods and services skyrocketed in 1992 and double- and triple-digit inflation (in year-on-year terms) had set in. It subsided to a single-digit level only in early 1998 and for a short period of time.

Nevertheless, as I started working at the bank as a computer department employee in March 1995, it was an extraordinary professional experience for me from the very beginning. My job responsibilities were effectively to oversee the daily workings of the computer system of the bank, which consisted of the key computers situated in the room occupied by a computer department and the personal computers of the account managers, who dealt with business clients. 

At this point, I have said a little bit more about the bank I worked for. Before my first day on the job as a bank's computer department employee largely I had no idea of my employer. Only after some time on the job, when I was instructed by my superior (head of computer department) what I have to do and at what time of the day, I started to realize a bit more what the business entity it was. So, I was hired by a local branch of the bank called Prominvestbank, or PIB for short. In that time, it was among top three banks in Ukraine by assets. It had a wide network of branches in key industrial areas of the country. And the branch that hired me was a part of the entire Prominvestbank system. (In fact, later I found out that top managers of the bank from the ranks of head office and the branch level ranks too called the bank they worked for as \textit{the system}.)

In 1995, Ukraine's authorities re-organized the banking system inherited from the Soviet Union by establishing the so-called ``two-tier system", where the first tier is occupied by a central bank of the country and the second tier is occupied by profit-seeking commercial banks of state and private ownership.

Prominvestbank was not a bank created from scratch by the owners. Instead, it was one of the few so-called specialized banks that were created yet in the late 1980s in the Soviet Union on the back of re-organization of the State Bank. That re-organization took place as a result of the 1988 reforms of the monetary system of the Soviet Union. When the Soviet Union disintegrated in 1991, the system of the Prominvestbank that once spread over the territory of the Soviet Union was carved along the boundaries of the newly independent states. In the newly independent Ukraine, Prominvestbank was a system that used to be just  recently a piece of a larger system. Because banking sector of the former Soviet Union was the very first sector being liberalized that sentiment that state-owned banks were about to be privatized was very strong in Ukraine already in early 1990s. The head of the state-owned Prominvestbank succeeded in getting rid of state ownership quickly at the time via turning employees of the bank into minority shareholders, where he held \textit{de-facto} a controlling stake in the bank.

As I started working for the Prominvestbank's branch in my hometown, I quickly realized the hierarchical structure of the bank's system. Each branch of the bank had its own balance sheet that was compiled every working way via the computer system. It was done every working day: right after the closing of the operational day, which took place in late evening (between 6pm and 9pm), while the opening of the operational day took place in the morning (between 8am and 9am). That is why the computer department employees were working via rotating morning/evening shifts. 

This branch had designated identifications such as \ac{mfo} 302485 and four-letter code UAAE. The former was a reference number where ``\ac{mfo}" abbreviation was a legacy of the past payment system used back in the Soviet Union. But still it was in use by the new national payment system created from the scratch by the Ukraine central bank's payment department with its software developers in early 1990s. 

The codes of each bank branch were in use by the central bank's payment system and specifically it served as an identification number for routing payment instructions via the central bank's system to which all commercial banks and their branches had access to (see details below).

One of the key responsibilities on the job assigned for me was overseeing the processing of the payments by the computer system of the branch. In essence, back then in 1995, this system consisted of a Zyxel-branded modem and two computers, one of which served as local server storing all major files and another one served as a screen into the operations of the software that handled the payments. The modem was attached to a designated phone line and periodically within every hour of daylight operations was transmitting bytes of information back and force. There were two queues of information--inbound and outbound--being transmitted via the modem. At the beginning those bytes of information were the files of dbf\footnote{This is a format, popular back then, of the database software dBASE.} format, which were not encrypted. However, in a few years the encryption technology was introduced into the national payment system as Prominvestbank's head office developers changed the hardware and software accordingly in every branch to comply with new requirements. The software the bank used was an in-house product. It was created by the staff developers at the Prominvestbank's head office. They commanded the whole system of the bank by running the network of computers of the bank and through sending out to every branch of the bank regular directive messages with instruction on the software updates (when to run specific files to update the system installed at the branch so that it was synchronized with the rest of the system of entire bank).

The national payment system as it appeared to me in 1995 was designed in the following specific way. Each branch of the Prominvestbank (and very much likely it applied to the branches of all other banks) was transmitting their inbound and outbound payment instructions via a regional office of the national central bank\footnote{By country's Constitution, there are 24 administrative units.} toward the head office of the central bank. In a few years, it changed. Because there was a drive for consolidation among the banks that had branch network. In the case of Prominvestbank, consolidation resulted in the elimination of the regional offices of the central bank as local clearing houses. Recall that a regional office of the central bank digested payment instructions originated within its area of oversight. If a payment instruction was to settle a transaction between the clients of banks of the same region it could be settled in the books of the regional branch of the central bank. Otherwise, if payment instructions are meant to make credit-debit transactions between branches of different regions then it must be processed by the head office of the central bank, which was the ultimate clearing house.

The consolidation drive among large banks aimed to bypass those regional clearing houses and becoming themselves a clearing house facility for the payments instructions that take place between own branches. For example, if a client of the one branch of Prominvestbank located in Vinnytsia region gave a payment instruction to debit own account and credit an account of the counterparty, which was served by a branch located in the, for example, Donetsk region, then such a transaction was settled on the balance sheet of the bank. It did not require Prominvestbank to touch reserve balances, an item on the assets side of its consolidated balance sheet. It required just to change balances on the liability side of the consolidated balance sheet between the payer and payee.

Hence, after the above-mentioned consolidation by Prominvestbank of all payments to be carried out via the head office of the bank. The workings of the branch's computer department in overseeing the daily payments process largely remained intact. Albeit there was one noticeable difference: all inbound and outbound payment instructions did not involve the regional office of the central bank, now they all were transmitted to the head office of the bank itself.

Daily observation of the payment system within a branch of one of the largest banks of Ukraine gave me realization that most of the operations are taking place on the digital turf, i.e. on the computer network system developed by the bank's IT professionals. At the time, in the middle of 1990s, Prominvestbank enjoyed high national standing as it was a bank with large network of branches and serving traditionally industrial enterprises all over the country. In my hometown, the branch of Prominvestbank I worked for had a reputation of ``having money", meaning it carried out payments without delay. That quality was valued among both the local state-owned enterprises, among which the above-mentioned power station was the largest, and by smaller private entrepreneurs. In fact, the branch had ``money" because \ac{lad_tpp} it served. The latter was receiving regularly large volume of payments on its checking account opened with \ac{lad_pib}, usually, by the end of the day. And as a result of that the bank branch's balance on the correspondent account with its head office branch grew proportionally.

The bank branch I am talking about provided the large enterprises it served, and which were located in close proximity with cash service. It meant that an enterprise relied on the bank branch for converting some portion of funds on the checking account of such enterprise into cash and then delivering that cash to the premises of the enterprise. This way, the enterprise got cash to pay wages to its employees. This was the technique before debit cards being cautiously introduced in the late 1990s. All in all, my takeaway from what I observed so far was cash operations of the clients of the bank branch were less sizable than their non-cash operations that were taking place digitally. The latter were primary. 

Those operations discussed in the previous paragraph were denominated in the local money unit of account.\footnote{After gaining independence in the end of 1991 after Soviet Union disintegration, Ukraine introduced own money unit of account called karbovanets on November 12th, 1992. Due to the mere fact that in the former Soviet Union, there was only one cash currency production facility based in Russia, Ukraine's authorities allowed rubles to circulate for some time and introduced own cash currency, called \textit{coupon-karbovanets}. Due to past hyperinflation, in September 1996 Ukraine's authorities replaced karbovanets with new money unit of account called \textit{hryvnia} and all prices and balance sheet items of all economic units of the country were denominated by the ratio of hundred thousand karbovanets to 1 hryvnia.} 

Meanwhile, bank branch operations with foreign cash paper notes such as US dollars were rather small and occasional. No more than few hundred US dollars a day was the size of foreign-exchange operations with cash by \ac{lad_pib}. People, who did foreign exchange at the open-air bazaar literally across the street, were outcompeting the bank branch both in bid and offer quotes. However, gradually and over time, some clients started to generate operations in non-cash and denominated in the foreign money unit of account. What amusement I experienced when the head office of the bank sent us a payment confirmation in favor of private entrepreneur worth of fifty thousand US dollars. That was the biggest price tag in US dollars I had seen, and it was non-cash and just a digital record. That occasion had strengthened my impression that money was predominantly digital records. 

Over time, operations of \ac{lad_pib} with foreign money both cash and especially non-cash became larger and more frequent. Thanks to some knowledge of English, I was offered to transfer from the computer department position to the position of account manager with responsibility of working with clients, which carried out export-import operations. It meant that these clients dealt with foreign money. Capital controls (called ``valuta controlled" in Ukrainian) were part of my new responsibilities. That was due to the fact that the country's authorities made commercial banks agents in controlling the operations of their clients with foreign money. In practice, it meant that commercials bank just followed legislation and central bank's regulation on what is allowed and what is not allowed. Since then, I was responsible for servicing the bank branch's operations for clients that were transacting in foreign-money units. They were carrying out the buy and sell operations of cash and non-cash US dollars mostly and at less extent Russian rubles. These were the most usable foreign money units of account handled by Ukrainian companies in their trade with foreign counterparts in the second half of 1990s. 

Later, I was assigned new responsibilities. Because officially I worked in one room with loan officers of the bank branch, they assigned me the emerging responsibility of dealing with what they called (and I did follow their habit) ``credit resources". Those credit officers, two of them in fact, dealt with clients who had outstanding loans with the bank. Those loans were usually overdue loans because of the weak economy. And by the way, \ac{lad_tpp} never borrowed as far as I was aware of the matter. Loans were extended to the so-called new economy entrepreneurs and their performance was not great at the time. Collateral on those loans was of two types: (1) machinery and facilities used in production, which are collectively called fixed assets, and (2) goods produced by those very borrowers, which are called inventories. Such collateral proved totally illiquid as those borrowers routinely delayed interest and principal repayment. For example, there was a company that borrowed from the bank branch. Its main business was the production of sparkling wines. It was a newly created company owned by a Georgian businessman. So, it entered into a loan agreement with a bank branch by pledging the production facility with key equipment housed inside as collateral. Additional collateral was the goods in the production process, which were a certain (quite large) number of sparkling wine bottles. That company delayed repayments on the loan agreement. Once, as a member of the loan department, I participated in the on-site inspection of the fixed assets that were put as collateral for bank loans. The production facility and equipment inside it was operational but looked somewhat worn-out (it was not brand new). In the extreme development of that loan there was a moment, when the borrower gave up completely selling its inventory and redeemed the loan. The bank branch was forced to take over the produced goods (sparkling wine bottles) physically and tried to sell them by itself. The top executive of the bank branch ordered employees of the loan department to attempt selling some number of boxes with sparkling wine bottles in the nearby bazaar, which was a place of active market trade held once a week. That sparkling wine turned out to be of lower quality than established brands already known to local consumers. In effect, experience with retail sales via bazaar of the movable part of the collateral showed it was difficult too to turn it into cash money. Selling the fixed part of collateral, which was attempted, was equally problematic. In essence, this was a type of collateral management done in practice.

Generally speaking, of all loan portfolios, a repayment of interest or of principal was a rare and outstanding event at the time. The credit officers were routinely consumed by dealing with overdue loans. Hence, they left on me dealing with the ``credit resources" matters. So, ``credit resources" were the balances on the correspondent account of the bank branch with its head office branch. The head office of Prominvestbank was the counterparty, or the correspondent bank for the bank branch I worked for.

All outbound payments the bank branch did were made via the correspondent account. The balance on that account, which is an asset item on the balance sheet of the bank branch, was reduced with every outbound payment being executed. Similarly, all inbound payments in favor of the bank branch did result in the correspondent account balance increase. These operations were automated or computerized, i.e. they did not require manual input from the bank branch employees. For instance, inbound payments in favor of the bank branch credited clients' accounts and simultaneous debiting the correspondent account. As far as the outbound payments were concerned, the procedure for handling clients' payment instructions was the following. Once a client handed over a payment instruction to the bank branch's account manager, it was input into the bank's computer system through keystrokes manually by that very account manager. If there were not sufficient balance on the client's account, then the client was anticipating incoming payments from counterparties that would build up the balance of enough size on the client's account by the end of the day. Once the client's account balance was there the payment instruction was executed by the bank branch automatically by debiting the client's account. On the other hand, to complete the transaction on its balance sheet the bank should have enough balance in its correspondent account. If there was not enough balance on that account, the outbound payment instructions were staying in the queue, and they were executed once the inbound payments built up that balance.

Overtime, Prominvestbank allowed its branches to trade balances (held on the correspondent accounts opened with the head office of the bank). At the same time, the Ukraine's central bank sent out to all commercial banks of the country the instruction on how to comply with reserve requirements. Prominvestbank imposed that instruction on its branches so that each branch had to meet regularly the reserve requirement defined as some percentage of total balance held on the termed and demand deposits. So effectively, my new responsibilities in dealing with ``credit resources" were (1) to make sure that the bank branch meets its reserve requirements according to the central bank instructions, and (2) to trade excess balances accumulated on the correspondent account. Thanks to serving large state-owned enterprises, some of which except the power station were privatized in mid-1990s, the bank branch usually accumulated by the end of the day excess balance on its correspondent account. In fact, it was other branches of Prominvestbank which were calling us and asking for lending them ``credit resources" overnight or for a week. Dealing with banking units outside the system of Prominvestbank was ruled out or banned by the head office -- it was a rule you were not allowed to break. 

Those lending operations were unsecured, but our counterparts were meeting their commitments all the time. This fact did not mean that there was no credit risk of lending between branches of Prominvestbank. Over time, it became known that some branches were short of balances on the correspondent account most of the time. Some of them were of such poor credit quality that internally at our branch it was decided by the top executive people not to deal with them no matter what. So, effectively we lent ``credit resources" mostly overnight to a quite narrow circle of branches that we know as having good standing and they paid back on time.

Effectively, my responsibilities on this new position were (1) housekeeping in terms of making sure that clients' operations with foreign money as well as internal operations of the bank were meeting all regulations of the central bank and Prominvestbank's head office, (2) to carry out buy and sell operations of foreign money in non-cash and cash forms by clients and by bank branch itself, and (3) trade excess balances on the correspondent account with other branches. 

In turned out in one year in the second half of 1990s the branch got its annual profit mostly from those two activities: non-cash foreign exchange and "credit resources" trading. It was this way because the bank branch lending to the clients did not perform well in that year as it did several years before.

Daily involvement in the business of Prominvestbank's intra-branch trading of ``credit resources"\footnote{Over time, computer department of Prominvestbank's head office even created a special software called 'Bourse of Credit Resources' for the organized activities of the branches in their daily business with balances on the correspondent accounts with the head office. Since its launch, it was required that all agreements between the branches on borrowing and lending of those balances be carried out via the bourse. Previously, those agreements were made via the phone.} gave birth to some small talk discussions we held internally within the room I shared with credit officers of the bank branch. It was due to some particular realization. It was albeit vague in terms of the theoretical backing, but as a perception it was becoming evident more and more. That realization was about a thing that all those intra-branch trading business with ``credit resources" was neutral for Prominvestbank as a whole. Because all the branches traded with each other the balances on their correspondent accounts with the head office. Those were just items on the liabilities side of the balance sheet of Prominvestbank's head office. Its own correspondent account with the central bank, an item on the assets side of the balance sheet, was unaffected, when branches traded balances in their own correspondent accounts with each other. Only, demand by the clients for cash in paper notes and outbound payments made in favor of clients served by banking units outside the Prominvestbank system did affect the balance on that correspondent account with the central bank.

It is worth mentioning the non-cash foreign-exchange operations as they were done in the bank branch. First, how did clients obtain the foreign-exchange balances in their bank accounts? Foremost, it was done thanks to the clients' exports of goods and services. For example, one of the clients of the bank branch was local meat-processing plant that was selling its produce mostly to Russia and it got paid either in Russian rubles or US dollars, depending on the contract it had Russian counterparties. The initial indication that foreign exchange inbound payment was forthcoming was a two-leg process: (1) the client as a rule informed the bank branch about expecting such a payment to take place within few days, and (2) a \ac{swift} text message was sent by the head office in favor of the bank branch (the head office had received that message thanks to communications  channels it had since it was carrying out cross-border payment via correspondent accounts with foreign banks). Then, the bank branch balance sheet automatically, through the Prominvestbank computer system, got the accounting entry, where the client's current account in foreign money unit (as a rule denominated in the US dollars) was credited,\footnote{It is \textit{credited}, because the account is on the \textit{liabilities} side of the balance sheet of the bank branch.} while the branch's correspondent account (denominated in the US dollars and opened on the balance sheet of the head office) was debited.\footnote{It is \textit{debited}, because the account is on the \textit{assets} side of the balance sheet of the bank branch.}

Once the funds denominated in the foreign money, either in Russian ruble or US dollar, got credited to the current account of the meat-processing plant, they become available for a foreign-exchange operation upon client's request. For example, it required this balance on its US dollar checking account to be converted into a balance at the checking account in domestic money of account hryvnia. Clients usually quickly requested conversion of the foreign-money denominated balances into the hryvnia balances. Recall, hryvnia is Ukraine's domestic money unit of account. However, during the period of financial crisis or increased devaluation risks over the domestic money unit, the central bank used to impose \ac{srr} of greater than zero: sometimes it was 100\%, but most of the time it was of 50\%. SRR is a percentage of the foreign-money denominated balances, which  were received by the client from abroad as exports proceeds, which are required by regulation to immediate sale on the inter-bank (non-cash) foreign-exchange market. The client that wanted to sell its "dollars" or "rubles" was usually aware of the current exchange rate as it was well known publicly to every interested person. Then, the client provided to the branch the sell order of its ``valuta" balances at the current rate. We, at the bank branch, usually knew which branch of the Prominvestbank system held a bid on either US dollars or rubles. Again, the rule of a thumb of the foreign-exchange operations by the branch of Prominvestbank was such that foreign-exchange buy and sell orders from the clients are to be matched (or met) within the system of branches of Prominvestbank. It meant that the branch selling foreign-money denominated balances, or ``valuta" it was usually called, has to find a branch of Prominvestbank which was buying those balances. The former had an offer from its client, on the account of which those balances were accounted, and the latter had a bid from its client, which was willing to acquire those balances in exchange for domestic-money denominated balances. Sometime, thanks to a diversified client base, there could be a moment when one branch simultaneously receives an offer and bids over the foreign-money balances. In this case, the seller and the buyer are matched by the branch itself, on its balance sheet and without involvement of another branch. 

To conclude, I am referring to the above-mentioned examples of operations of the bank branch of a large commercial bank in Ukraine with a purpose. That experience gave me some realizations that I am relying upon in my dissertation work. 

\subsubsection*{Russia's War}

On February 24th, 2022, Ukraine was attacked militarily by Russia's army for the second time in the previous eight years. 

The first invasion took place in February 2014, when Russia's army occupied first Crimea and then shortly afterwards some parts of the Luhansk and Donetsk regions, all of which are territories of Ukraine with sizable population. The Russia's war on Ukraine has been lasting for 10 years now. 

The civilian population suffers the most from this war. For example, my hometown is quite away from the lengthy front line, nevertheless the war affected it massively. In 2022 and 2024, Russia's army sent missiles and drones targeting \ac{lad_tpp} to create a lasting damage to this massive industrial enterprise, which supplies electricity to a number of regions and not only the Vinnytsia region alone. But the more profound way the Russia's army has been targeting civilians, killing some of them but keeping those alive in constant terror and risks of new bombardments and of collapse of the systems that provide basic services such as utilities, health care and education. See \citep{bbc2022,yahoo2024}.  

Due to that development, my key focus is on the Ukraine's lebaration from these inhuman deeds of the Russia's government. Economics of sanctions is part of my dissertation work, too.

\subsubsection*{Closing remarks}

During my writings of the dissertation, I have found a nice description of the research process, which I cannot resist citing it below: 

\begin{quote}
Just as the best way to learn a subject thoroughly is to teach it, the best way to understand a technology is to reinvent it. \citep[p.~8]{gardiner2006}
\end{quote}

It appears describing what I have been doing for some time now: not in the terms of ``reinvent[ing]" something, but in the very terms of ``learn[ing] a subject thoroughly." And I was trying as much as I could. If one finds this work useful, then it is due to the latter effort, not to the former.

\begin{flushright}
   \MyName \\
   Red Hook, New York \\
   October 18th, 2023
\end{flushright}

\addtocontents{toc}{\vspace{0.3cm}{Chapter}~\hfill{}\par}
%\addcontentsline{toc}{section}{Chapter} % <-- insert a line w/ "Chapter"
\newpage
