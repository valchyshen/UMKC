\newpage

\section{\MakeUppercase{Economics of Sanctions: The Case of Russia}}

%\subsection*{Abstract}The approach developed towards international cross-border transactions in the last chapter is now applied to sanctions. This topic is an important matter since there has been a visible rise in the fields of knowledge such as \textit{statecraft} and \textit{geoeconomics}.\index{Statecraft}\index{Geoeconomics} All of which incorporate accumulated knowledge about the ongoing cases of sanctions. Unmistakenly, there is an economic theory standing behind the sanctions as they are being defined and applied in each particular case. It is based upon neoclassical economics, where a notion of scarcity of resources is paramount. This is especially true concerning the financial dimension of sanctions, in which denying ``access to capital," or the financial resources, has been considered by the sanctioning authorities (sovereigns) as a prime tool for achieving the desired behavior of the sanctioned. This dissertation, however, argues that such a theoretical apparatus is deficient. It applies to a special case of the sanctioned, who are the users either directly or indirectly of the monetary system of the sanctioning authorities. If the sanctioned are in active disciplining of the domestic populace to be the users of the domestic money of account, then sanctions---as in the case of Russia since 2014---become less certain. Instead of the expected accumulation of forces of economic destruction in the targeted economy in some future time, the sanctionsas they are now---in an ironic turn of events---might solidify the domestic forces of economic immunization.

\subsection{Introduction}

There is tendency that certain misconceptions feature longevity. Their existance might be confusing to those whose questions about the observable socio-economic phenomena obtain indefinite answers. This dissertation aims to apeal to such questions, which have been raised,  by arguing of weak analytical capacities of such conceptions as ``Ukraine conflict," ``Ukraine crisis," and ``Fortress Russia".

As a matter of fact, Russia has been literally waging a brutal war by means of its military forces against Ukraine since February 2014. What took place in February of 2022 was a doubling down of aggression from the side of Russia's government and its military arm, the Russia's invading army, against a sovereign country of Ukraine. 

It must be said, too, that the former waged an economic war against the latter well before the millitary war of 2014. It did it through weaponization of natural gas and other strategic goods such as nuclear power.\footnote{On the thesis of weaponization of nuclear energy by Russia's government see \citep{un2023,un2024}.}\index{Russia!weaponization of strategic goods}\index{Russia!economic war against Ukraine} Hence, the meaning of weaponization of the economic resources and of the key infrastructures was recognized in Ukraine by the mid of 2000s. It was several years before \ac{gfc}, in other words, during the years of, as perceived, rising economic prosperity. The realization was that such weaponization was not an exclusive instrument of the Cold War in the hands of major military powers, see \citep{gazprom2008,farrell2019,farrell2023}. 

During 2014-15 or the first phase of the war, Russia's secret services and the miliary acted \textit{subtly} as their squads crossed the border without insignia. Russian top officials, foremost its president, denied the presence of own armed forces on the Ukraine's soil during interviews with top Western media outlets. Whatever the staged performance the Russian authorities did at the time, it was pretty clear to vast majority of Ukrainians which country's army was acting that way. As a result, Russia annexed such territories of Ukraine as Crimea peninsula at the Black Sea and Sea of Azov and parts of the Donbas region in the country's east, which reach Sea of Azov. In the second phase of the war, which has been taking place since February 2022 and through present, Russia's invading army acted \textit{openly}. Their president even broadcasted the dicision to invade few hours before the Russian military boots and bombs hit the ground in Ukraine. During that televized speach, Putin's facial expression, tone of his voice, repeated heavy sighs, and usage of words all exposed an elevated level of hatred towards the enemy: Ukraine, its society and its democratically-elected leadership, \citep[see][]{kremlin2022}. Even though, he used expressions of ``one people, one whole," while talking of Russians and Ukrainians, which might seem to people unfamiliar with the matter as a thoughful attempt to explain himself, it was a stark performance aimed to do a couple of things at once. 

First, he heated up to a near culmination point the state of domestic discourse taking place since the early 1990s, which this dissertation describes as ``paranoid style of domestic politics,"\index{Paranoid style of politics!in Russia since 1990s} the term borrowed from \citep{hofstadter2020,dharvey}. For decades, that discourse has been exploiting the feelings of the wide swathe of Russians, for whom the cries about of lost empire (of the tzarist Russia) or lost great country (of the Soviet Union) feed into national pride. Over past thirty years in Russia, auhorities approval has been spiking every time Russia's invading army got new order to act, see Figure~\ref{fig:levada} on p.~\pageref{fig:levada} and econometric analysis in Appendix \ref{reg_levada} on p.~\pageref{reg_levada}. 

Second, it was a projection towards the West, or Global North, showing the willingness and capability to execute military intervention outside of own internationally-recognized territory and expand farther the area of Ukraine under its occupations, which have been unrecognized.

During the first phase of this war about six thousand civilians lost their lives and about 1.5 million were forced to evacuate from their homes -- such was the count on the second year after the war started \citep{poroshenko2015}. 

The second phase of the Russia's war on Ukraine has had a heavier toll than previous one. And it has been mounting. The number of civilians killed by the Russian army increased markedly, if compared to previous phase. As several-month occupation of the small towns such as Bucha revealed after to their liberation by the Ukraine's army that Russia's invading army had a clear go ahead command to execute mass killing in order to spread fear and root out resistance. Along the way of those executions at small towns, the Russia's air forces have been raining missiles, glide bombs and drones onto Ukraine's civil infracsttucure, including hospotals, schools and universities. In their reports on Ukraine, intetrnational releaf organizations note that exact number of killed civilians must be well higher than own estimates due to difficulies of access to the territories, which are in war zone.  

By the end of 2022, or after ten months of this second phase of the war, the number of Ukrainian citizens that left the country amounted to 7.9 million,\footnote{To be precise: ``7,915,287 million refugees from Ukraine recorded across Europe since 24 February 2022"~\citep{unisef2022}} of which women and children were the majority \citep{unisef2022,wb2023}. This number did not count the number of internally displaced population, which peaked to 7.71 million in 2022\footnote{See {\small\url{https://dtm.iom.int/ukraine}}}. The number of Ukrainian children kidnapped and deported by the Russia's invading army in the occupied territories ranked in hundreds of thousand \citep{un2023}. The scale of ruin the Ukraine's society has been experiencing is charactrized by following senence: ``Following the escalation of war on 24 February 2022, the Ukraine refugee crisis became the largest since World War II."~\citep[p.~2]{unisef2022} That similarity with World War II, during which the entire Ukraine's territory was covered by a war zone, must be understood not only in terms of quantity of refugees -- the destruction of civil and productive infrastructure has been a targeted operation so that to produce upmost negative effect on population. For example, Ukraine's thermal and water power generating facilities were either subsantially damaged or destroyed by the combinations of drone and missile attacks by Russia.

It has been a dramatic ten-year period of 2014-24 for the Ukraine's society. In its consciousness the current war that came from the minds and hands of Putin's Russia is comparable only to the World War II, which came from the minds and hands of Nazi Germany. The latter occupied Ukraine entirely in the 1940s. The former has been (i) executing daily bombardments by drones and missiles over the entire country since 2022, and (ii) extending the area of Ukraine's territory under its occupation since 2014, which now account for a quarter. 

The response to this modern day aggression has been two-fold. From the major military and economic powers of the West---led by the \ac{usa} and followed by the \ac{eu}---it was in terms of sanctions, which were the key tool applied. From the side of Ukraine, the choice was to resist militarily and it was pre-determined. 

% --- REPEATED TEXT ------------------------------------------------------------+
%Ukraine responded to the Russia's military aggression by mobilization of its military forces and own production of arms. Both of which are quite limited if compared to Russia's, however, a the very beginning it was enough to resist. In parallel, it asked the major global powers in the West---mainly, the \ac{usa} and \ac{eu}---the for the assistance of military supplies and training to defend itself. The responce from them was (a) economic and financial sanctions as key tool targeted at Russia, and (b) measured and paced supplies of armaments to Ukraine. 

Economics of sanctions is key point of analysis of the current Chapter. It is required because there was a gap between (a) prior expacations about the impact and outcome due to sanctions, and (b) the actual developments. Those expectations were in the minds of top people who devised sanctions, they talked about them publicly and the major media outlets were airing them to the general public of different countries. 

In Ukraine, in particular, the general belief was that economically mighty West knows well how to put an economic restrain on a bad actor. One can say that it was relatively recently when Soviet Union, of which Ukraine was a part, lost the Cold War stand-off with the West by vanishing itself.\footnote{That process was complex and contained various economic and political nuances. For recent works on this subjec written by people, who experience the life in the Soviet Union, see \citep{zubok2021,feygin2024,feygin2024_}.} And general perception in Ukraine and elsewhere has been that Soviet Union collapse, first of all, was due to economics and especially moneary economics. Indeed, during the twilight of his professional career as the concluding leader of the Soveit Union ``Gorbachev \dots [was in] the need to beg for foreign credits and assistance"~\citep[p.~432]{zubok2021}. That episode has epitomized the existence of the real economic power in the hands of the West.

Every time sanctions were imposed on Russia in 2014 and 2022 the perception was the West applied massive, unprecedented economic costs on the bad actor. Under their weight the latter would stumble, halt its milicious acts. The official and semi-official statements fuelled that perception \citep{ustres2014,wsj2022}. However, as time went by it deflated itself under the weight of unmet expectations \citep{wsj2024}.

This dissertation's argument on sanctions---by referencing to the case of sanctions applied to Russia---runs contrary to the established view. The latter, in fact, consiss of two broad camps: one states that sanctions are potent and unfold over long period of time, while another disagrees with the former only in the key aspect that is sanctions backfire and hance must be restrained from overdone. Whereas, this dissertation work argues that sanctions in their present form applied to Russia are timid and mistaken in such a way that they turn useful for the Russian authorities by absorbing all the negativism the Russian public has about the ongoing economic changes that place in Russia's economy. Those changes were required before 2014, while their implementation would had been politically difficult and even impossible due to general public dissatisfaction.

This argument arises upon the evolutionary analysis of Russia's economy over past hundred years from 1920s and through 2020s. This is the period right after the 1917 Revolution, of which first seventty years belonged to Soviet Union, in which Russia was dominant, and the last thirty years belong to modern Russia. In particular, a special attention is paid to the changes of monetary system and how the payment system was re-designed (a) to fit the required purposes of the day, this is the episode lasting from late 1920s and through 1970s, and (b) to adapt with a changing international environment, the episode since mid 1980s and through present.

Out of these two large episodes in history, the latter one is noteworthy. The argument of this dissertation, which was mentioned above, refers to the \citeauthor{dharvey}'s conceptualization of a neoliberal state with its leaning to the paranoid style of politics.\index{Paranoid style of politics!as termed by David Harvey}\index{Paranoid style of politics!Russia}\index{Harvey, David!neoliberal state}. Such a state runs on the fueled and elecrified public perception of being ``besieged and threatened by enemies from within and without"~\cite[pp.~82-83]{dharvey}. Hence, a certain level of consolidation around the ruling reghime is accepted from the public in such a state. This description is not foreign to the modern-day Russia's society. The Russia's war on Ukraine originated from such a state.\footnote{Yet, in a 2007 book a scholar at Peterson Institute for International Economics concluded that Russia had experienced ``a successful movement toward capitalism," however, it failed with democracy~\cite{rae2010}. Referring to the writings of \citeauthor{slobodian2018,slobodian2023}, one can conclude that achievements in Russia were rather in line with neoliberalism, not a sizable deviation from it.}

\begin{figure}[!ht]
\centering
\includegraphics[width=1.0\textwidth]{\plotsfolder/levada_indexes}
\caption[Monthly public opinion surveys by Levada, a Russia-based polling agency]%
{Monthly public opinion surveys by Levada, a Russia-based polling agency.\par\vspace{.05in}Source: Levada, \url{www.levada.ru}.}
\label{fig:levada}
\end{figure}

\subsection{Literature Review}

This sections consists of two parts. The first one reviews the sanctions literature. Then, the second one reviews other related literature con cerning the argument of this dissertation work.

\subsubsection*{Literature on Sanctions}

The economic sanctions literature has blossomed up since February 2022. The wave of sanctions imposed on Russia due to its first military invasion of Ukraine in February 2014 turned out to be a ripple in the flat sea. It was due to the second, more larger, invasion that sanctioning Russia started to resemble a more visible wave.

There are three types of lierature in terms of expertise. One is written by the practinioners with an working experience in devising and implementing sanction regimes, such as \citep{zarate2013,lew2016,nephew2017,lew2018,fishman2025}. Second is written by think-tankers and academicians such as \citep{blackwill,mulder,demarais2022,mcdowell,farrell2019,farrell2023}. Lastly, the books written by top media correspondents from Bloomberg and other similar oulets, who had access to top decision-makers either in the U.S or Russia~\citep{mohsin2024,baker2024}. 

The common underlying thread in all of these work is the authors' reliance on the standard neoclassical apporach to economics. In particular, these auhors usually in their expressions fall back on such neoclassical economics basics such as (a) scarcity of resouces, including finance, (b) savings finance investments, (c) bank deposits finance loans, (d) it is prudent for emerging market economies (or Global South countries) that as a rule experience the shortage of domestic savings tto tap excess savings of the developed market economies (or Global North countries), and etc., and etc. These basics are from the intellectual arsenal of a ``sound" economist. Their way of thinking is disciplined by this approach to economic analysis. It seats in the back of their mind. They never depart from it even an inch. A key shortage of such an approach is absance of conceptualization of money and finance as \acfp{dcr}.

On the back of this foundation the following shared principles arise as the accepted way of thinking in this literature. 

First, the \acf{usd} is a major global reserve currency. It is a central element of the international financial system. And smooth, stable and reliable operation of latter is a global public good \citep{imf1999}. It means that access to the international capital markets, which deals largely in the \acf{usd}, has been synonymous with a lasting and expandable prosperity. By this logic, the following assumption has been that it is sensable to strive for the maintenance of the regular access to the international capital markets. Both for public and private sectors of the mentioned countries. Losing such an access has been considered as a sure pathway to economic instability, crisis and stagnation.

Second, the \acf{usd} has been enjoying a status of the major global reserve currency on the back of ``network effects," which imply interdependency between the economy from which the reserve currency originates and the economies where it ``circulates." Essentially, network effects mean ``the more that other market actors make or accept payment in a currency, the more useful and efficient it is for other market actors to do the same"~\citep[p.~21]{mcdowell}. As a result, on the back of to the development of the Eurodollar market over 1970-1980s, ``[m]ore dollars circulated outside the United States than within it"~\citep[p.~23]{farrell2023}.

The sancions literature mentioned above can be divided into two broad camps in terms of their evaluation of sanctions. 

One of them considers the currently formulated sanctions as a powerful tool affecting the economy of the sanctioned country. The time span over which the sanctioned country is expectted to be affected and to concede to the outside pressure varies dramatically. It might take place relatively quickly or it might happen longer-than-expected. Nevertheless, the assumed end point is known, which is a state of economic stagnation. This is the line observed in \citep{zarate2013,blackwill,baker2024}.

Another camp, while aknowledging the power of sanctions to affect the targeted economy eventually, argues that sanctions backfire on the economy of the country, which imposed sanctions. This counter argument is the basic line that differentiate these two camps. This viewpoint is mostly articulatted in \citep{lew2016,lew2018,mulder,demarais2022,mcdowell,farrell2023,mohsin2024}. 

Jacob Lew, former U.S. Treasury Secreary in the President Obama administration, became the key point of reference for the second viewpoint. In 2016, two years after sanctioning Russia for its first military invasion in Ukraine and de-facto annexation of the latter's two large regions: Crimea (entirely) and Donbas (partially), Lew made a public speech on sanctions in Carnegie Endowment for International Peace, a think-tank based in Washington, D.C. In that talk, he defended the viewpoint of the first camp by saying that sanctions were cafully crafted and applied in a targeted fashion so that the Russian people were not affected and only a narrow circle of bad actors inside the Russia's government and key decision-makers got affected. At the end of first part of his talk, which alluded to the viewpoint of the first camp, he claimed that ``targeted sanctions have worked as intended" in Russia and that ``the pressure they impose over time" would keep Russia vary of not repeating its aggression. Then, in the second part of this speech he warned of the dangers of overuse of sanctions, which later become the staple claim of the thesis that sanctions backfire:

\begin{quote}
It is important to make sure these tools remain available and effective. And there is a risk that overuse could ultimately reduce our capability to use sanctions effectively. While sanctions are a valuable alternative to more severe measures, including the lawful use of force, it is a mistake to think that they are low-cost. And if they make the business environment too complicated---or unpredictable, or if they excessively interfere with the \textit{flow} of funds worldwide, financial transactions may begin to \textit{move outside} of the United States entirely---which could threaten the central role of the U.S. financial system globally, not to mention the effectiveness of our sanctions in the future.~\citep[emphasis added]{lew2016}
\end{quote}

Since that time, Lew's message has been a starting point of the works written on sanctions such as \citep{mcdowell,farrell2023,mohsin2024}, explicitly, and \citep{demarais2022}, implicitly. 

With caveats these authors present sanctions on Russia---either imposed since 2014 or since 2022 invasions---severe, robust and unprecendented. ``Most shockingly" \citeauthor{farrell2023} wrote that EU supported Ukraine and joint US in sanctioning Russia by blocking ``Russia's central bank access to the money it held in Europe, paralizing billions in foreign reserves that Moscow had thought were safe from political interference"~\citep[p.~112]{farrell2023}. For \citeauthor{mcdowell} ``the most severe sanctions to date against" Russia were adopted by the U.S. Treasury in early spring 2018~\citep[p.~43]{mcdowell}. ``While those sanctions [adopted in 2022] were extraordinarily costly, Moscow's anticipatory moves limited the damage that United States could inflcit"~\citep[p.~49]{mcdowell}. Similar viewpoint is held by \citeauthor{demarais2022}, who said that ``the most robus penalties" of Russia were implented in the 2014 vintage of sanctions, whereas the ones of 2022 lacked the bite already. Both \cite{mohsin2024} and \cite{baker2024} refer to the 2022 vinage of sanctions as ``unprecendented," while acknowledging the difficulty of sanctioning such a big economy as of Russia, which deeply integrated into international commodities and financial markets.

After recognizing sanctions as severe and unprecedented, these authors, except \cite{baker2024}, turn to their key message that sanctions have side effects of negative character on the economy, which imposes sanctions, or they backfire. These are termed as de-dollarization and political risks associated with dealing via the U.S. financial system. The causation line built by this typical argument is being drawn in the following way. The U.S. is conceptualized as ``underground empire" as by \cite{farrell2023}, which at will ``weaponizes" own currency the U.S dollar against other countries by means of financial sanctions, which cut off specific individuals, firms, governments from the global financial system. This capacity on the sides of the U.S. creates an environment spread globally, where other countries and their governments and leaders are being aware of it and hence they pressured to adopt counter measures. The latter are termed as ``anti-dollar policies" as by \cite{mcdowell}. Hence, ``the link between US financial sanctions and anti-dollar policy \textit{reactions} from targeted governments is undeniable"~\citep[p.~5, emphasis added]{mcdowell}. Word ``reactions" is intentially highlighted in this quote to underline cuasation the author was drawing: by this approach political risk causes de-dollarization. This is an important note, to which this dissertation returns later.

Another shared feature of the sanctions literature of the past several years was its relaince on the narrattives spread by the Russia propagarnda well before Russia's second invasion of Ukraine in early 2022. This imposed habit of thinking is expressed in different ways.

Thus, both \cite{mcdowell} and \cite{demarais2022} cited the purposedly-catchy passages on the US dollar made by Russia's top officials in 2018. The latter, acting as one voice, were letting know the world about Russia's economy was set to reduce usage of the US dollar in foreign trade and other transactions as a responce to the Western sanctions. In late 2018 Putin himself stated: ``We aren't aiming to ditch the dollar. The dollar is dittching us."~\citep[p.~126]{demarais2022} It appears that neither Western media nor the authors of sanctions did not question the causality as presented by Russia's authoriy. Both of the above-mentioned authors conluded that political risks or possibility of being sanctioned by the U.S. authorities forced different countries, including Russia, to seek alternative to the U.S. dollar.

Then, there is another line of thinking, which reveals that writers rely upon the messaging and terminology which are most desired by the Russia authorities in their melign activities versus other countries such as Ukraine. When writers talk of Crimea without ever mentioning Ukraine, then Russia's narraive of reducing Ukraine to the notion of non-existence is being implanted. This is a common practice by all authors mentioned above. In one paragraph, on pp.~44-45 in \cite{mcdowell}, where the reason behind the 2014 vinttage of the US sancions on Russia are explained there is no Ukraine being mentioned at all with relation to Crimea as if the latter was an uninhabitant island. Similarly, \cite{demarais2022,baker2024} practices this turn of words troughout the book. \cite{demarais2022} on p.~24 has a paragraph on Crimea without Ukraine being mentioned, where Ukraine's Crimea is called ``a land that few Europeans knew existed before Russia seized it." It suggested non-Europeans might be even more ignorant. It is in the index section of \cite{demarais2022}, where the author corrects the narrative pointing out that Crimea is part of Ukraine: the entries starting with C letter contain a line ``Crimea. \textit{See} Ukraine." The index section in \cite{baker2024} contains Crimea, while mentioning Ukraine only as ``Soviet transfer to Ukraine," a term of bitter episode in history caltivated by the Russia propaganda. 

Another habit of thinking by the writers on sanctions mentioned above is their explicit concerns over humanitarian impact of sanctions upon the ordinary population of Russia. See \citep[p.~54]{demarais2022}, \citep[p.~xiii]{mohsin2024}. Whereas no or very little explicit concern was raised with respect to the ongoing humanitarian impact upon the Ukraine's population, which has been the target of the Russia's military aggression since early 2014. Only \cite{baker2024} is trying the most not to fall into this mental trap, even though she spent many years in the past of working in Moscow, the key urban area in Russia that fascinates foreign people the most.\footnote{However, \cite{baker2024} revealed that she is still under the influence of imposed way of thinking by Russia-desired narratives. She described that ``bread and \textit{salo}, or lard, [as] traditional Slavic dish."~\citep[p.~66]{baker2024} The point here is that word ``salo" is a Ukrainian one not Slavic. Moreover, the dish itself consisting of bread and salo is a Ukrainian dish and not Slavic, the term Russia's ideology uses to describe different people such as Russians, Belorussians and Ukrainians. Sometimes, under ``Slavic" the Russia's ideology also include different nations of Cenral and Eastern Europe. However, in \cite{baker2024} Slavic refers to only tree nations mentioned above.}

After imgaining sanctions as powerful, severe and unprecedented, the major signiture of the sanctions literature so far has been to think that Russia's economy was a typical case of the emerging market countries. The only difference was that it was running massive current account surpluses thanks to exports of crude oil and natural gas. There was no question raised to the validity of the following assertion: ``Designing of sanctions entails caregfully studying the strengths and weaknesses of their intended target."~\citep[p.~23]{demarais2022}

Thanks to the details revealed by \cite{baker2024}'s account of sanctions, there is a recorded evidence that sanctions of Russia were designed with the assumptions mentioned above. People who designed sanctions followed hem (see Table~\ref{tab:key_sanctions_ppl}, p.~\pageref{tab:key_sanctions_ppl}). 

Yet, in 2014 after the Russia's initial military aggression against Ukraine, the U.S. Treasury started to devise the sanctioning regime targeting the melign actor. The key body of that exercise was in the ``Treasury's international office, where the self-professed economic nerds reside"~\citep[p.~86]{baker2024}. According to \citeauthor{baker2024}, Rory MacFarquhar, Daleep Singh and Brad Setser were among those nerds. These people ``weren't sanctions specialists but they their expertise in economics and financial markets." MacFarquhar and Singh possessed financial markets experience thanks to their previous work in Goldman Sachs, while Setser's assets were being a graduate of Harvard and Oxford. Their focus was on the following:

\begin{quote}
Economists are generally focused on how to grow an economy. Now they were flipping that way of thinking on its head. There was some banter about the two [Signh and Setser] being engaged in the ``dark arts of economics." ``It's basically causing market forces to damage an economy rather than support it," Signh recalled. Or, as Setser put it, ``We know how to fight a financial crisis---can we cause one?"\par As they explored new types of financial welfare, they quickly zeroedin on the battle for capital. They stared bandying about ideas to stop the flow of funding to Russia, short of a full Iran-style block. It was a novel approach that grew from the lessons learned from the unprecedented collapse in capital flows during the 2008 financial crisis. ``The answer that emerged `Why can't we just block new financing rather than block all ransactions?`" Setser recalled. The idea was partly about constraining big state-controlled companies from borrowing from the rest of the world to pay dividends to the government. \dots\par ``Where did U.S. economic leverage intersectn with Russia's economic vulnerability?" Signh recalled thinking. ``Financial capital was and is an obvious choice." It had never been done before.~\citep[pp.86-87]{baker2024}
\end{quote}

Such was the logic of the 2014 vintage of sanctions and it was retain in the 2022 one. And as \citeauthor{baker2024} concludes: ``After 2014 Putin began devising his Fortress Russia strategy to protect the country from future sanctions."~\citep[p.89]{baker2024} Hence, \citeauthor{baker2024}'s book sided with \cite{mcdowell,demarais2022} in terms of thinking about the causation. That US sanctions came first and then Russia's responce to them followed. In other words, the US sanctions caused Russia's policy of ``ditching the dollar."

\begin{landscape}

\begin{table}[!htbp]
\captionsetup{width=1\linewidth,labelfont=bf}
  \vspace{.4in}
  \centering
  \begin{tikzpicture}
  %\draw[help lines,gray] (0,0) grid (16,7);
        %
        % Banker R ---------------------------------------------------------
        \matrix (m) [matrix anchor=north, matrix of nodes, nodes in empty cells,
             nodes = {minimum height=.3in, align=left},
             column sep=1em, row 1/.style={font=\bfseries},
             column 1/.style={align=left},
            ] at (8,7)
        {
        Name & \shortstack{Year of\\Birth} & \shortstack{Education\\(Degree)} & \shortstack{Prior\\Experience} & \shortstack{Sanctions\\Position} & \shortstack{Target\\Country} \\
        \shortstack{Juan\\Zarate}\footnote{Juan Zarate's estimation of age was made upon the following link: \url{https://hls.harvard.edu/today/juan-c-zarate-97-theres-a-lot-of-presumption-of-the-demise-of-american-power-and-im-raging-against-that/}} & 1972 & \shortstack{Harvard\\University (\acs{jd})} & \acs{doj} &  \shortstack{\acs{dns} Advisor on Com-\\bating Terrorism (2001-09)} & \shortstack{N.Korea,\\Iran} \\
        \shortstack{Jack\\Lew}\footnote{Jack Lew's CV is derived from \url{https://il.usembassy.gov/our-relationship/our-ambassador/}, accessed on October 12th, 2024.} & 1955 & \shortstack{Georgetown\\University (\acs{jd})} & \acs{omb} & \shortstack{US Treasury\\ Sec. (2013-17)} & \shortstack{Iran,\\Russia} \\ 
        \shortstack{Richard\\Nephew}\footnote{Richard Nephew's CV is from \url{https://www.state.gov/biographies/richard-nephew}, accessed on October 12th, 2024.} & N/A & \shortstack{George Washington U.\\(MA in Security Policy)} & \shortstack{\acs{nnsa},\\ \acs{dos}} & \shortstack{DoS\\(2013-15)} & Iran \\ 
        \shortstack{Edward\\Fishman}\footnote{Edward Fishman's CV is from \url{https://www.energypolicy.columbia.edu/edward-fishman/}} & 1989 & \shortstack{Cambridge, UK (M.Phil.\\in \acs{ir}), Stanford (MBA)} & \shortstack{US Treasury,\\\acs{dod}} & \shortstack{\acs{dos}\\(2013-17)} & \shortstack{Russia,\\Iran} \\
        \shortstack{Rory\\MacFarquhar}\footnote{Rory MacFarquhar's CV is from \url{https://www.linkedin.com/in/rory-macfarquhar-7aabb68/details/experience/} and \url{https://english.ckgsb.edu.cn/new/ckgsb-knowledge-series-the-state-of-us-china-economic-relations/}, all accessed on October 12th, 2024.} & 1982 & \shortstack{Yale U. (BA),\acs{lse} (MSc)\\Harvard U. (PhD)} & \shortstack{Goldman Sachs\\(2002-10)} & \shortstack{NSC\\(2011-16)} & Russia \\
        \shortstack{Brad\\Setser}\footnote{Brad Setser's CV \url{https://www.cfr.org/expert/brad-w-setser}.} & N/A & \shortstack{Harvard U. (BA)\\Oxford U. (MA,PhD in \acs{ir})} & \shortstack{CFR, fellow\\(2003-09)} & \shortstack{US Treasury\\(2011-16)} & Russia \\
        \shortstack{Daleep\\Singh}\footnote{Daleep Singh's CV is from \url{https://docs.house.gov/meetings/FA/FA14/20221116/115182/HHRG-117-FA14-Bio-SinghD-20221116.pdf}.} & 1976 & \shortstack{MIT (MBA)\\Harvard U. (MPA)} & \shortstack{Goldman Sachs,\\PGIM Chief Econ.} & \shortstack{\acs{dns} Advisor on\\Intl Econ (2021-22,2024-)} & Russia \\
        \shortstack{Wally\\Adeyemo}\footnote{Wally Adeyemo's CV is from \url{https://home.treasury.gov/about/general-information/officials/Wally-Adeyemo}.} & 1981 & \shortstack{U. of California Ber-\\keley (BA), Yale U. (\acs{jd})} & \shortstack{\acs{dns} Advisor on\\Intl Econ (2015-16)} & \shortstack{US Treasury Deputy\\Secretary (2021-)} & Russia \\
        \shortstack{Daniel\\Fried}\footnote{Daniel Fried's CV is from \url{https://www.atlanticcouncil.org/expert/daniel-fried/} and \url{https://1997-2001.state.gov/regions/eur/fried971023.html}, accessed on October 12th, 2024.} & 1952 & \shortstack{Cornell U.(BA)\\Columbia U. (MA)} & \shortstack{DoS\\(1997-2009)} & \shortstack{Policy Coordi-\\nator (2013-17)} & \shortstack{Russia,\\Iran} \\
        };
        %\node[fit=(m-1-1)(m-1-2)]%
        %{\textit{\MakeUppercase{Bank's (R) positions}}};
        \draw[thin]  (m-1-1.south -| m.west) -- (m-1-1.south -| m.east);
        \draw[thin]  (m-10-1.south -| m.west) -- (m-10-1.south -| m.east);
  \end{tikzpicture}
 \caption[Key people working on the sanctions within the US government, 2000-2024]%
  {Key people working on the sanctions within the US government, 2000-2024.\par\vspace{.05in}Note: JD = \acf{jd}, DoS = \acf{dos}, DoD = \acf{dod}, DoJ = \acf{doj}, LSE = \acf{lse}, OMB = \acf{omb}, DNS = \acf{dns}, NNSA = \acf{nnsa}.\newline Sources: \cite{zarate2013,nephew2017,mohsin2024,baker2024}.}
  \label{tab:key_sanctions_ppl}
  \vspace{.2in}
\end{table}

\end{landscape}

Indeed, the MacFarquhar-Singh-Setser thesis that "international sanctions limit [Russia's] access to capital" would tame the latter's aggression was mainsream thinking among economists and \acf{ir} experts. Among the latter group, there was a divided view of Russia between 2014 and 2022: one defined Russia as a declining power, and another one call it a myth \citep{kofman2021}. However, after Russia's second, full-scale military invasion of Ukraine on February 24, 2022, the latter camp changed its position a bit aligning with the former's view of Russia as a declining power, while adding that Russia has been increasingly dangerous. However, the overarching view of economic factors never changed even a little bit. As ever, the claim that lack of ``access to capital" via the international financial markets would spell doom on any country remained unchanged \citep{kofman2022}.

The 2022 vintatge of sanctions was under the watch of \cite{singh2022}. They were devised to impact melign Russia via five channels, of which the capital channel was the key one and topping the list. 

\begin{quote}
Channel one was the immediate delivery of a capital account shock unlike any seen before.
Together with the issuers of the dominant reserve currencies in the world, we denied foreign capital to and blocked any transactions with the Kremlin's largest state-owned banks,
representing about 80 percent of Russia's total banking assets. Almost concurrently---and
apparently to Putin's surprise---we then immobilized about \$300 billion held by its central bank,
disarming the financial fortress that Russia had built to record size as a buffer for crisis. The
ripple effects were clear to see (see next section for further details): a wave of capital outflows,
an initial nosedive in the value of the ruble, and spiking inflation -- all of which Putin could only
counter with self-defeating capital controls and emergency interest rate hikes that pushed
Russia into deep recession.~\citep[pp.~1-2]{singh2022}
\end{quote}

Another decision-maker and contributor to devising the sanctions of Russia \cite{fishman2025} echoed \citeauthor{singh2022}'s view that the 2022 vintage of sanctions intended to produce financial shock on Russia. According to \citeauthor{fishman2025}, ``[t]he 2014 sanctions \dots were like a 2 out of 10 in intensity, whereas the ones that have been imposed in the last two weeks are more like an 8 out of 10."~\citep{wsj2022} These statements reveal there was consensus. Yet, two years later the \textit{Wall Street Journal} wrote: ``Western officials and experts say the financial, economic, military and energy sanctions imposed on Russia since February 2022 have damaged Russia's economy and arms-production capacity, and will create serious problems for the Kremlin in the coming years. But they acknowledge the restrictions have hit more slowly than they hoped. \dots sanctions haven't
prevented Russia from vastly increasing military spending this year"~\citep{wsj2024}. 

Modern-day economists, which devise sanctions, appears to place strong belief in the assumptions they have. They must have asked themselve why thier prior expectations did not materialize. However, there have been no public statements about errors made and the need to do a thorough review. For example, \cite{csis2024} acknowledged analytical failures in the military intelligence community as of early 2022. Such as their incorrect assessments of both Ukraine's ability and williness to defend itself (they thought it was lower than what discovered later) and Russia's military posturing as second army after the US (they thought it was more capable). There seems nothing similar from the side of economists. There is plenty of economic writings to consider, one of which is by \citeauthor{ayres1944}, an institutionalist:

\begin{quote}
Not only is the general public increasingly aware of the importance of the machine process; there is also increasing disillusionment with the dogmas of finance. The gold standard, for example, is no longer the fetich it used to be. \textit{Only a few years ago ``sound" economists were assuring the world that Germany and Japan could not possibly sustain all-out war since neither had an adequate gold reserve}. Not only have we seen them do so nevertheless; we have also learned that possession of the greater part of the world stock of monetary gold by no means insures victory in war. The story is being widely told of the financier who on being informed of the burning of an aluminum plant replied, ``Well, it was fully covered by insurance, wasn't it?" The folly of that way of thinking is now apparent to all thoughtful people. \citep[pp.~13-14, emphasis added]{ayres1944}
\end{quote}

\subsubsection*{Related Literature}

[\uppercase{Yet to be developed}]

\subsection{Evoluion of the Russia's Payments System}

To grasp what modern-day sanctions can and cannot do it is required first to turn towards institutional details of the evolution of the monetary system of today's Russia. In parallel, we
are going to do the same for Ukraine, too. However, even a slight
investigation into the matter enforces a researcher to do a bit of extra
work. It means one needs not only to understand the transition of these
economies after the break-up of the Soviet Union from the command or
state-planning system to the market system, it is a must to study the
evolution of the monetary system during the Soviet period.

The following sections described these periods in a small detail. The
last section concludes.

\subsubsection{Period of the Soviet Union: From 1920s and Through Mid-1980s}

The analysis must start with the Soviet Union period. It is not because there was
a Cold War rivalry between the Western capitalist countries led by the \acf{us} and the
the bloc of socialist countries led by the Soviet Union. In fact, it must begin with earlier episode, prior not only the Cold War but well before World War II. That is the period spanning from the late 1920s and through middle of 1980s.

This long period in its entirety lays
foundations of the economic developments of late 1980s under Gorbachev
and his Perestroika policy, which did not prevent Soviet Union from
breaking up. Then, it had an impact on the transition period of 1990s in
now sovereign countries, including Ukraine and Russia.

The economy of the Soviet Union before the reforms of late 1980s under
Gorbachev had quite specific institutional design.\footnote{See \citep{valchyshen_ru98} for more detailed discussion of the matter.} First, all means
of production were under ownership of the government. Secondly, finance
was reduced to be a utility like service and not the tool for
speculation, i.e. non-productive capital gains.

The design of the Soviet Union economic system that survived with some
minor modifications through mid 1980s with ruble as money of
account was elaborated in the late 1920s and then in early 1930s. It was after the
lengthy period during which the Bolsheviks managed to stabilize the
economy after the 1917 revolution and subsequent several-year civil war
on the terrains of the former Russia empire. The Bolsheviks government
crushed the Ukraine's republic militarily in those years.

There was a short period in the 1920s when authorities in the Soviet
Union allow private ownership to carry out business activities. That
short episode was characterized by active commercial debts (private
\acp{iou}) creation. And authorities presiding in the State Bank understood
that process as endogenous money creation. Thus, a State Bank protocol
dated in the 1920s over the economic developments of the time
acknowledge that when bank accepted commercial \acp{iou} (called \textit{veksel}) it
meant ``exchanging commercial money by banking money"
\citep[pp.~40-44]{cbr2008}. 

This system evolved from the socioeconomic environment of the 1920s, which made authorities critical of the workings of the then credit market. At the time, the fact was that the State Bank and a number of specialized banks (subsidiaries of State Bank) were discounting commercial bills (veksels), i.e., acknowledgements of debt or \acp{iou} issued by non-financial enterprises for the sake of carrying out buy-and-sale transactions. 

The State Bank economists of the time debated monetary matters in line with the real-bills principle, quantity theory of money \citep[pp.~28-35]{woodruff1999}, and in such terms as ``exchanging commercial money by banking money"~\citep[pp.~40-44]{cbr2008}. 

The payments between branches of State Bank, which usually were in different towns, were carried out by means of centralized matching (in 1922-26) and decentralized matching (1927-29); see Chart 1 on page 8 and Chart 2 on page 8, respectively. Both of these methods proved unsatisfactory, as the required level of control and matching was not achieved~\citep[p.~103]{cbr2008}. 

\begin{figure}[ht]
    \captionsetup{width=1\linewidth,labelfont=bf}
    \centering
    \begin{tikzpicture}[scale=1]
        \draw[help lines,white] (0,0) grid (16,13);
        \draw (7.5,9) node[minimum height=2cm,minimum width=3cm,draw,thick] (GB) {};
        \node[align=center] at (7.5,9) {\textbf{State Bank}\\Head Office};
        \draw (4,12) node[minimum height=2cm,minimum width=3cm,draw,thick] (GP) {};
        \node[align=center] at (4,12) {\textbf{State}\\ \textbf{Planning}\\Authority};
        \draw (11,12) node[minimum height=2cm,minimum width=3cm,draw,thick] (GG) {};
        \node[align=center] at (11,12) {\textbf{Government}\\Ministries};
        \draw[-{Stealth[length=3mm]},ultra thick,red] (GP.east) -- (GG.west);
        \draw[-{Stealth[length=3mm]},ultra thick,red] (GG.west) -- (GP.east);
        \draw[-{Stealth[length=3mm]},ultra thick,red] (7.5,12) -- (GB.north);
        \fill[red] (7.5,12) circle (3pt);
        % Industrial Union 1
        \draw (2,2) node[minimum height=1cm,minimum width=1cm,draw,thick] {A};
        \draw (4,2) node[minimum height=1cm,minimum width=1cm,draw,thick] {B};
        \draw (6,2) node[minimum height=1cm,minimum width=1cm,draw,thick] {C};
        \draw (4,2) node[minimum height=2cm,minimum width=6cm,draw,dashed,thick] (U1) {};
        \draw (U1.north)++(0,.35) node[align=center,font=\fontsize{9}{9}\selectfont] (U1T) {\textbf{Industrial Union \#1}};
        \draw (4,2.75) node[minimum height=4cm,minimum width=6.5cm,draw,red,dashed,thick] (M1) {};
        \draw (M1.north)++(0,-.35) node[align=center,red,font=\fontsize{9}{9}\selectfont] {\textbf{Government Ministry \#1}};
        % Industrial Union 2
        \draw (9,2) node[minimum height=1cm,minimum width=1cm,draw,thick] {D};
        \draw (11,2) node[minimum height=1cm,minimum width=1cm,draw,thick] {E};
        \draw (13,2) node[minimum height=1cm,minimum width=1cm,draw,thick] {F};
        \draw (11,2) node[minimum height=2cm,minimum width=6cm,draw,dashed,thick] (U2) {};
        \draw (U2.north)++(0,.35) node[align=center,font=\fontsize{9}{9}\selectfont] (U2T) {\textbf{Industrial Union \#2}};
        \draw (11,2.75) node[minimum height=4cm,minimum width=6.5cm,draw,red,dashed,thick] (M2) {};
        \draw (M2.north)++(0,-.35) node[align=center,red,font=\fontsize{9}{9}\selectfont] {\textbf{Government Ministry \#2}};
        % State Bank branches
        \draw (2.5,7) node[minimum height=1cm,minimum width=2.2cm,draw,thick] (B1){Branch \#1};
        \draw (5.5,7) node[minimum height=1cm,minimum width=2.2cm,draw,thick] (B2) {Branch \#2};
        \draw (9.5,7) node[minimum height=1cm,minimum width=2.2cm,draw,thick] (B3) {Branch \#3};
        \draw (12.5,7) node[minimum height=1cm,minimum width=2.2cm,draw,thick] (B4) {Branch \#4};
        % ---------------------------------------------------------------------
        % Production targets
        \draw[-{Stealth[length=3mm]},ultra thick,red] (GP.south) -- (M1.north);
        \draw[-{Stealth[length=3mm]},ultra thick,red] (GG.south) -- (M2.north);
        % Credit limits arrows
        \draw [line width=1pt, double distance=3pt,
             arrows = {-Latex[length=0pt 3 0]}] (GB.west) -| (B1.north);
        \draw [line width=1pt, double distance=3pt,
             arrows = {-Latex[length=0pt 3 0]}] (GB.west) -| (B2.north);
        \draw [line width=1pt, double distance=3pt,
             arrows = {-Latex[length=0pt 3 0]}] (GB.east) -| (B4.north);
        \draw [line width=1pt, double distance=3pt,
             arrows = {-Latex[length=0pt 3 0]}] (GB.east) -| (B3.north);
        \draw [line width=1pt, double distance=3pt,
             arrows = {-Latex[length=0pt 3 0]}] (GB.east) -| (U2.north east);
        \draw [line width=1pt, double distance=3pt,
             arrows = {-Latex[length=0pt 3 0]}] (GB.west) -| (U1.north west);
        \draw [line width=1pt, double distance=3pt,
             arrows = {Latex[length=0pt 3 0]-Latex[length=0pt 3 0]}] (U1T) to (U2T);
        \node[rectangle,right delimiter=\}] (AA) at (14.25,2.75) {\tikz{\path (14.25,6) rectangle (14.25,2);}}; \node[align=left,right=20pt,font=\fontsize{9}{9}\selectfont] at (AA.west) {Industries};
        \node[rectangle,right delimiter=\}] (BB) at (14.25,8) {\tikz{\path (14.25,10) rectangle (14.25,6);}}; \node[align=left,right=20pt,font=\fontsize{9}{9}\selectfont] at (BB.west) {Finance};
        \node[rectangle,right delimiter=\}] (CC) at (14.25,12) {\tikz{\path (14.25,13) rectangle (14.25,11);}}; \node[align=left,right=20pt,font=\fontsize{9}{9}\selectfont] at (CC.west) {Control \&\\Oversight};
        % Text labels
        \node[align=center,font=\fontsize{9}{9}\selectfont,red] (T1) at (7.5,12.75) {Planning and execution\\of plans for production\\of goods \& services};
        \node[align=right,font=\fontsize{9}{9}\selectfont] (T2) at (1.5,10) {Assigning of credit\\plans \& limits}; \draw (T2.east) -- (3.5,9.1);
        \node[align=center,font=\fontsize{9}{9}\selectfont] (T3) at (7.5,5.5) {Unions were allowed\\to ``trade"/exchange credit\\limits between themselves}; 
        \draw (T3.south) -- (7.5,3.45);
    \end{tikzpicture}
    \caption[Schema of the credit limits allocation by State Bank and State Planning Authority, since 1931]%
    {Schema of the credit limits allocation by State Bank and State Planning Authority, since 1931. Note: $A,B,C,D,E,F$ are industrial enterprises. \par\vspace{.05in} Source: \citep[p.~103]{cbr2010}, illustration by author.}
    \label{fig:credit_limits}
\end{figure}

The entire credit system, where endogenous credit creation was partially based on the commercial bills, i.e., outside of banks, were back then deemed as less regulated.\footnote{Authorities aimed to deploy effective 'ruble control,' i.e., control by means of the ruble, which meant back then that State Bank was to expand control over credit creation towards the level of enterprise. It effectively meant State Bank was to take over commercial bills creation by providing direct lending to those enterprises that required credit. The previous practice of commercial bills issuance and then their rediscount in the banking system revealed that "among those commercial bills filed for rediscounting there were many so-called 'advance' and 'friendly' bills, which did not reflect real movement of goods and which were issued by enterprises for each other for the sake of credit" \citep[pp.~124-125]{gusakov1951}. This conflict between authorities' adoption of the real bills doctrine for the operations of the monetary system and the actual workings of the economy, where accommodation bills had proliferated, is discussed in \citep[pp.~21-55]{woodruff1999}.} Hence, the authorities of the country had made the decision that time had come to reform the system.

\begin{figure}[ht]
    \captionsetup{width=1\linewidth,labelfont=bf}
    \centering
    \begin{tikzpicture}[scale=1]
    \draw[help lines,white] (0,0) grid (16,13);
    %\tikz[align=center] \node[draw] {This is a\\demonstration.};
    \node[align=left,font=\fontsize{9}{9}\selectfont] at (.5,12.5) {State\\Bank};
    \draw (8,10.25) node[minimum height=5.5cm,minimum width=16cm,draw,dashed] {};
    \draw (8,11) node[minimum height=2cm,minimum width=4cm,draw,thick] (HO) {};
    \node[align=center] at (8,11) {\textbf{Head Office}\\of State Bank};
    \draw (3,9) node[minimum height=2cm,minimum width=4cm,draw] (B1) {};
    \node[align=center] at (3,9) {\textbf{Branch \#1}\\\textit{Initiating Branch}};
    \draw (13,9) node[minimum height=2cm,minimum width=4cm,draw] (B2) {};
    \node[align=center] at (13,9) {\textbf{Branch \#2}\\\textit{Completing Branch}};
    \draw[-{Stealth[length=3mm]}] (B1) -- (B2);
    \draw[-{Stealth[length=3mm]}] (B2) -- (B1);
    \draw[-{Stealth[length=3mm]}] (B1.north) to [bend left=20] (HO.west);
    \draw[-{Stealth[length=3mm]}] (B2.north) to [bend right=20] (HO.east);   
    \draw (3,6.5) node[minimum height=1cm,minimum width=1cm,draw] (A) {\textbf{A}};
    \draw (13,6.5) node[minimum height=1cm,minimum width=1cm,draw] (B) {\textbf{B}};
    \draw (A)--(B1); \draw (B)--(B2);
    \draw (A)++(1.7,0) node[align=left,font=\fontsize{9}{9}\selectfont,blue] {The\\Payer\\(Initiator)};
    \draw (B)++(1.7,0) node[align=left,font=\fontsize{9}{9}\selectfont,blue] {The\\Payee\\(Recipient)};
    % Branch #1 positions
    \matrix (m) [matrix anchor=north, matrix of nodes, nodes in empty cells,
             nodes={text width=.85in, minimum height=.2in, align=left}, 
             column sep=.5em,
             row 2/.style={nodes={font=\bfseries\fontsize{9}{9}\selectfont}},
             row 3/.style={nodes={font=\fontsize{9}{9}\selectfont}}
            ] at (3,5)
        {
        & \\
        Assets, $c_t$ & Liabilities, $d_t$ \\
        $\Delta\big\downarrow$ Head \hspace{.5in}Office account & $\Delta\big\downarrow$ Customer $A$ account \\
        };
        \node[fit=(m-1-1)(m-1-2), font=\fontsize{9}{9}\selectfont] {\textit{\MakeUppercase{Branch \#1}}};
        \draw[thin]  (m-2-2.south -| m.west) -- (m-2-2.south -| m.east);
        \draw[thin]  (m-2-2.north west) -- (m-3-2.south west);
    % Branch #2 positions
    \matrix (m) [matrix anchor=north, matrix of nodes, nodes in empty cells,
             nodes={text width=.85in, minimum height=.2in, align=left}, 
             column sep=.5em,
             row 2/.style={nodes={font=\bfseries\fontsize{9}{9}\selectfont}},
             row 3/.style={nodes={font=\fontsize{9}{9}\selectfont}}
            ] at (13,5)
        {
        & \\
        Assets, $c_t$ & Liabilities, $d_t$ \\
        $\Delta\big\uparrow$ Head \hspace{.5in}Office account & $\Delta\big\uparrow$ Customer $B$ account \\
        };
        \node[fit=(m-1-1)(m-1-2), font=\fontsize{9}{9}\selectfont] {\textit{\MakeUppercase{Branch \#2}}};
        \draw[thin]  (m-2-2.south -| m.west) -- (m-2-2.south -| m.east);
        \draw[thin]  (m-2-2.north west) -- (m-3-2.south west);
        % T-accounts
        \matrix (m) [matrix anchor=north, matrix of nodes, nodes in empty cells,
             nodes={text width=.85in, minimum height=.2in, align=left}, 
             column sep=.5em,
             row 2/.style={nodes={font=\bfseries\fontsize{9}{9}\selectfont}},
             row 3/.style={nodes={font=\fontsize{9}{9}\selectfont}}
            ] at (3,2.25)
        {
        & \\
        Debit & Credit \\
        Customer $A$ account & Head Office account \\
        };
        \node[fit=(m-1-1)(m-1-2), font=\fontsize{9}{9}\selectfont] {\textit{T-Account: Entry by Branch \#1}};
        \draw[thin]  (m-2-2.south -| m.west) -- (m-2-2.south -| m.east);
        \draw[thin]  (m-2-2.north west) -- (m-3-2.south west);
        \matrix (m) [matrix anchor=north, matrix of nodes, nodes in empty cells,
             nodes={text width=.85in, minimum height=.2in, align=left}, 
             column sep=.5em,
             row 2/.style={nodes={font=\bfseries\fontsize{9}{9}\selectfont}},
             row 3/.style={nodes={font=\fontsize{9}{9}\selectfont}}
            ] at (13,2.25)
        {
        & \\
        Debit & Credit \\
        Head Office account & Customer $B$ account \\
        };
        \node[fit=(m-1-1)(m-1-2), font=\fontsize{9}{9}\selectfont] {\textit{T-Account: Entry by Branch \#2}};
        \draw[thin]  (m-2-2.south -| m.west) -- (m-2-2.south -| m.east);
        \draw[thin]  (m-2-2.north west) -- (m-3-2.south west);
        % Labels
        \node[align=left,font=\fontsize{9}{9}\selectfont,red] (T1) at (8,4) {\textbf{(1):} Initiating\\branch makes\\a \underline{credit} record\\against its account\\with Head Office};
        \node[align=left,font=\fontsize{9}{9}\selectfont,red] at (1.975,10.8) {\textbf{(2):} Initiating\\branch sends\\a credit register\\to Head Office};
        \node[align=right,font=\fontsize{9}{9}\selectfont,red] (T3) at (8,1.5) {\textbf{(3):} Completing\\branch makes\\a \underline{debit} record\\against its account\\with Head Office};
        \node[align=left,font=\fontsize{9}{9}\selectfont,red] at (14.25,10.8) {\textbf{(4):} Completing\\branch sends\\a debit register\\to Head Office};
        \node[align=center,font=\fontsize{9}{9}\selectfont,red] at (8,12.5) {\textbf{(5):} Head Office makes a centralized match of the\\credit and debit registers received, confirms the payment};
        \node[align=center,font=\fontsize{9}{9}\selectfont,red] at (8,8.35) {The initiating branch sends\\to completing branch the details on\\the payment being initiated};
        \draw[red] (T1.west) -- (5.5,3.5); \draw[red] (T1.west) -- (5.5,2.1);
        \draw[red] (T3.east) -- (10.25,3.5); \draw[red] (T3.east) -- (10.25,2.1);
    \end{tikzpicture}
    \caption[Schema of centralized matching of an inter-branch payment in State Bank, through 1926]%
    {Schema of centralized matching of an inter-branch payment in State Bank, which was practiced trough 1926 (client A of branch \#1 makes payment on the account of client B of branch \#2).\par Source: narrative from \citep[p.~103]{cbr2010}, illustration by author}
    \label{fig:state_bank_payments1}
\end{figure}

\begin{figure}[ht]
    \captionsetup{width=1\linewidth,labelfont=bf}
    \centering
    \begin{tikzpicture}[scale=.95]
    \draw[help lines,white] (0,0) grid (16,14);
    %\tikz[align=center] \node[draw] {This is a\\demonstration.};
    \draw (8,12.25) node[minimum height=5.5cm,minimum width=16cm,draw,dashed] {};
    \node[align=left,font=\fontsize{9}{9}\selectfont] at (.5,14.5) {State\\Bank};
    \draw (8,13) node[minimum height=2cm,minimum width=4cm,draw,thick] (HO) {};
    \node[align=center] at (8,13) {\textbf{Head Office}\\of State Bank};
    \draw (3,11) node[minimum height=2cm,minimum width=4cm,draw] (B1) {};
    \node[align=center] at (3,11) {\textbf{Branch \#1}\\\textit{Initiating Branch}};
    \draw (13,11) node[minimum height=2cm,minimum width=4cm,draw] (B2) {};
    \node[align=center] at (13,11) {\textbf{Branch \#2}\\\textit{Completing Branch}};
    \draw[-{Stealth[length=3mm]}] (B1) -- (B2);
    \draw[-{Stealth[length=3mm]}] (B2) -- (B1);
    \draw[-{Stealth[length=3mm]}] (B1.north) to [bend left=20] (HO.west);
    \draw[-{Stealth[length=3mm]}] (B2.north) to [bend right=20] (HO.east);   
    \draw (3,8.5) node[minimum height=1cm,minimum width=1cm,draw] (A) {\textbf{A}};
    \draw (13,8.5) node[minimum height=1cm,minimum width=1cm,draw] (B) {\textbf{B}};
    \draw (A)--(B1); \draw (B)--(B2);
    \draw (A)++(-1.7,0) node[align=right,font=\fontsize{9}{9}\selectfont,blue] {The\\Payer\\(Initiator)};
    \draw (B)++(1.7,0) node[align=left,font=\fontsize{9}{9}\selectfont,blue] {The\\Payee\\(Recipient)};
    % Branch #1 positions
    \matrix (m) [matrix anchor=north, matrix of nodes, nodes in empty cells,
             nodes={text width=.85in, minimum height=.2in, align=left}, 
             column sep=.5em,
             row 2/.style={nodes={font=\bfseries\fontsize{9}{9}\selectfont}},
             row 3/.style={nodes={font=\fontsize{9}{9}\selectfont}}
            ] at (3,7)
        {
        & \\
        Assets, $c_t$ & Liabilities, $d_t$ \\
        $\Delta\big\downarrow$ Account of inter-branch turnover & $\Delta\big\downarrow$ Customer $A$ account \\
        };
        \node[fit=(m-1-1)(m-1-2), font=\fontsize{9}{9}\selectfont] {\textit{\MakeUppercase{Branch \#1}}};
        \draw[thin]  (m-2-2.south -| m.west) -- (m-2-2.south -| m.east);
        \draw[thin]  (m-2-2.north west) -- (m-3-2.south west);
    % Branch #2 positions
    \matrix (m) [matrix anchor=north, matrix of nodes, nodes in empty cells,
             nodes={text width=.85in, minimum height=.2in, align=left}, 
             column sep=.5em,
             row 2/.style={nodes={font=\bfseries\fontsize{9}{9}\selectfont}},
             row 3/.style={nodes={font=\fontsize{9}{9}\selectfont}}
            ] at (9.25,7)
        {
        & \\
        Assets, $c_t$ & Liabilities, $d_t$ \\
        $\Delta\big\uparrow$ Account of inter-branch turnover  & $\Delta\big\uparrow$ Customer $B$ account \\
        };
        \node[fit=(m-1-1)(m-1-2), font=\fontsize{9}{9}\selectfont] {\textit{\MakeUppercase{Branch \#2}}};
        \draw[thin]  (m-2-2.south -| m.west) -- (m-2-2.south -| m.east);
        \draw[thin]  (m-2-2.north west) -- (m-3-2.south west);
        % T-accounts
        \matrix (m) [matrix anchor=north, matrix of nodes, nodes in empty cells,
             nodes={text width=.85in, minimum height=.2in, align=left}, 
             column sep=.5em,
             row 2/.style={nodes={font=\bfseries\fontsize{9}{9}\selectfont}},
             row 3/.style={nodes={font=\fontsize{9}{9}\selectfont}}
            ] at (3,4.5)
        {
        & \\
        Debit & Credit \\
        Customer $A$ account & Inter-branch account \\
        };
        \node[fit=(m-1-1)(m-1-2), font=\fontsize{9}{9}\selectfont] {\textit{T-Account: Entry by Branch \#1}};
        \draw[thin]  (m-2-2.south -| m.west) -- (m-2-2.south -| m.east);
        \draw[thin]  (m-2-2.north west) -- (m-3-2.south west);
        \matrix (m) [matrix anchor=north, matrix of nodes, nodes in empty cells,
             nodes={text width=.85in, minimum height=.2in, align=left}, 
             column sep=.5em,
             row 2/.style={nodes={font=\bfseries\fontsize{9}{9}\selectfont}},
             row 3/.style={nodes={font=\fontsize{9}{9}\selectfont}}
            ] at (9.25,4.5)
        {
        & \\
        Debit & Credit \\
        Inter-branch account & Customer $B$ account \\
        };
        \node[fit=(m-1-1)(m-1-2), font=\fontsize{9}{9}\selectfont] {\textit{T-Account: Entry by Branch \#2}};
        \draw[thin]  (m-2-2.south -| m.west) -- (m-2-2.south -| m.east);
        \draw[thin]  (m-2-2.north west) -- (m-3-2.south west);
        % Labels
        \node[align=left,font=\fontsize{9}{9}\selectfont,red] (T1) at (6.5,8.25) {\textbf{(1):} Initiating\\branch makes\\a \underline{credit} record\\against its\\account of inter-\\branch turnover};
        %\node[align=left,font=\fontsize{9}{9}\selectfont,red] at (1.975,10.8) {\textbf{(2):} Initiating\\branch sends\\a credit register\\to Head Office};
        \node[align=left,font=\fontsize{9}{9}\selectfont,red] (T3) at (10.5,8.25) {\textbf{(3):} Completing\\branch makes\\a \underline{debit} record\\against its\\account of inter-\\branch turnover};
        \node[align=left,font=\fontsize{9}{9}\selectfont,blue] at (14,14) {\textbf{NEW:} Accounts\\with Head Office\\were substituted with\\accounts of inter-branch\\turnover};
        \node[align=center,font=\fontsize{9}{9}\selectfont,red] at (8,14.5) {\textbf{(4):} Head Office controls total\\sum of inter-branch payments};
        \node[align=center,font=\fontsize{9}{9}\selectfont,red] at (8,10.35) {\textbf{(3):} Matching of credit and debit\\registers from two branches is carried\\out by branches themselves};
        \draw[red] (T1.west) -- (4,7); %\draw[red] (T1.west) -- (5.5,2.1);
        \draw[red] (T3.west) -- (8.5,7); %\draw[red] (T3.east) -- (10.25,2.1);
        \node[rectangle,right delimiter=\}] (AA) at (12.5,4.5) {\tikz{\path (12.5,7) rectangle (12.5,2.75);}}; \node[align=left,right=22pt,font=\fontsize{9}{9}\selectfont] at (AA.west) {if customer $A$'s\\balance $>0$};
    % ...if customer A's balance =0
    % Branch #1 positions
    \matrix (m) [matrix anchor=north, matrix of nodes, nodes in empty cells,
             nodes={text width=.85in, minimum height=.2in, align=left}, 
             column sep=.5em,
             row 2/.style={nodes={font=\bfseries\fontsize{9}{9}\selectfont}},
             row 3/.style={nodes={font=\fontsize{9}{9}\selectfont}}
            ] at (3,2.5)
        {
        & \\
        Assets, $c_t$ & Liabilities, $d_t$ \\
        $\Delta\big\uparrow$ Customer $A$ account & $\Delta\big\uparrow$ Account of inter-branch turnover \\
        };
        \node[fit=(m-1-1)(m-1-2), font=\fontsize{9}{9}\selectfont] {\textit{\MakeUppercase{Branch \#1}}};
        \draw[thin]  (m-2-2.south -| m.west) -- (m-2-2.south -| m.east);
        \draw[thin]  (m-2-2.north west) -- (m-3-2.south west);
    % Branch #2 positions
    \matrix (m) [matrix anchor=north, matrix of nodes, nodes in empty cells,
             nodes={text width=.85in, minimum height=.2in, align=left}, 
             column sep=.5em,
             row 2/.style={nodes={font=\bfseries\fontsize{9}{9}\selectfont}},
             row 3/.style={nodes={font=\fontsize{9}{9}\selectfont}}
            ] at (9.25,2.5)
        {
        & \\
        Assets, $c_t$ & Liabilities, $d_t$ \\
        $\Delta\big\uparrow$ Account of inter-branch turnover  & $\Delta\big\uparrow$ Customer $B$ account \\
        };
        \node[fit=(m-1-1)(m-1-2), font=\fontsize{9}{9}\selectfont] {\textit{\MakeUppercase{Branch \#2}}};
        \draw[thin]  (m-2-2.south -| m.west) -- (m-2-2.south -| m.east);
        \draw[thin]  (m-2-2.north west) -- (m-3-2.south west);
        \node[rectangle,right delimiter=\}] (AA) at (12.5,1) {\tikz{\path (12.5,1.9) rectangle (12.5,0);}}; \node[align=left,right=22pt,font=\fontsize{9}{9}\selectfont] at (AA.west) {if customer $A$'s\\balance $=0$};
    \end{tikzpicture}
    \caption[Schema of decentralized matching of an inter-branch payment in State Bank, which was practiced in 1927-1929]%
    {Schema of decentralized matching of an inter-branch payment in State Bank, which was practiced in 1927-1929. Source: \citep[p.~103]{cbr2010}, illustration by author.}
    \label{fig:state_bank_payments2}
\end{figure}

\begin{figure}[ht]
    \captionsetup{width=1\linewidth,labelfont=bf}
    \centering
    \begin{tikzpicture}[scale=.95]
    \draw[help lines,white] (0,0) grid (16,15);
    %\tikz[align=center] \node[draw] {This is a\\demonstration.};
    \draw (8,12.25) node[minimum height=5.5cm,minimum width=16cm,draw,dashed] {};
    \node[align=left,font=\fontsize{9}{9}\selectfont] at (.5,14.5) {State\\Bank};
    \draw (8,13) node[minimum height=2cm,minimum width=5cm,draw,thick] (HO) {};
    \node[align=center] at (7,13) {\textbf{Head Office}\\of State Bank};
    \draw (9.5,13) node[minimum height=1cm,minimum width=1cm,draw,black] (MAU) {MAU};
    \draw (3,11) node[minimum height=2cm,minimum width=4cm,draw] (B1) {};
    \node[align=center] at (3,11) {\textbf{Branch \#1}\\\textit{Initiating Branch}};
    \draw (13,11) node[minimum height=2cm,minimum width=4cm,draw] (B2) {};
    \node[align=center] at (13,11) {\textbf{Branch \#2}\\\textit{Completing Branch}};
    %\draw[-{Stealth[length=3mm]}] (B1) -- (B2);
    \draw[-{Stealth[length=3mm]}] (B2) -- (B1);
    \draw[-{Stealth[length=3mm]}] (B1.north) to [bend left=20] (HO.west);
    \draw[-{Stealth[length=3mm]}] (HO.east) to [bend left=20] (B2.north);   
    \draw (3,8.5) node[minimum height=1cm,minimum width=1cm,draw] (A) {\textbf{A}};
    \draw (13,8.5) node[minimum height=1cm,minimum width=1cm,draw] (B) {\textbf{B}};
    \draw (A)--(B1); \draw (B)--(B2);
    \draw (A)++(-1.7,0) node[align=right,font=\fontsize{9}{9}\selectfont,blue] {The\\Payer\\(Initiator)};
    \draw (B)++(1.7,0) node[align=left,font=\fontsize{9}{9}\selectfont,blue] {The\\Payee\\(Recipient)};
    % Branch #1 positions
    \matrix (m) [matrix anchor=north, matrix of nodes, nodes in empty cells,
             nodes={text width=.85in, minimum height=.2in, align=left}, 
             column sep=.5em,
             row 2/.style={nodes={font=\bfseries\fontsize{9}{9}\selectfont}},
             row 3/.style={nodes={font=\fontsize{9}{9}\selectfont}}
            ] at (3,7)
        {
        & \\
        Assets, $c_t$ & Liabilities, $d_t$ \\
        $\Delta\big\downarrow$ MFO account & $\Delta\big\downarrow$ Customer $A$ account \\
        };
        \node[fit=(m-1-1)(m-1-2), font=\fontsize{9}{9}\selectfont] {\textit{\MakeUppercase{Branch \#1}}};
        \draw[thin]  (m-2-2.south -| m.west) -- (m-2-2.south -| m.east);
        \draw[thin]  (m-2-2.north west) -- (m-3-2.south west);
    % Branch #2 positions
    \matrix (m) [matrix anchor=north, matrix of nodes, nodes in empty cells,
             nodes={text width=.85in, minimum height=.2in, align=left}, 
             column sep=.5em,
             row 2/.style={nodes={font=\bfseries\fontsize{9}{9}\selectfont}},
             row 3/.style={nodes={font=\fontsize{9}{9}\selectfont}}
            ] at (9.25,7)
        {
        & \\
        Assets, $c_t$ & Liabilities, $d_t$ \\
        $\Delta\big\uparrow$ MFO account  & $\Delta\big\uparrow$ Customer $B$ account \\
        };
        \node[fit=(m-1-1)(m-1-2), font=\fontsize{9}{9}\selectfont] {\textit{\MakeUppercase{Branch \#2}}};
        \draw[thin]  (m-2-2.south -| m.west) -- (m-2-2.south -| m.east);
        \draw[thin]  (m-2-2.north west) -- (m-3-2.south west);
        % T-accounts
        \matrix (m) [matrix anchor=north, matrix of nodes, nodes in empty cells,
             nodes={text width=.85in, minimum height=.2in, align=left}, 
             column sep=.5em,
             row 2/.style={nodes={font=\bfseries\fontsize{9}{9}\selectfont}},
             row 3/.style={nodes={font=\fontsize{9}{9}\selectfont}}
            ] at (3,4.5)
        {
        & \\
        Debit & Credit \\
        Customer $A$ account & MFO account \\
        };
        \node[fit=(m-1-1)(m-1-2), font=\fontsize{9}{9}\selectfont] {\textit{T-Account: Entry by Branch \#1}};
        \draw[thin]  (m-2-2.south -| m.west) -- (m-2-2.south -| m.east);
        \draw[thin]  (m-2-2.north west) -- (m-3-2.south west);
        \matrix (m) [matrix anchor=north, matrix of nodes, nodes in empty cells,
             nodes={text width=.85in, minimum height=.2in, align=left}, 
             column sep=.5em,
             row 2/.style={nodes={font=\bfseries\fontsize{9}{9}\selectfont}},
             row 3/.style={nodes={font=\fontsize{9}{9}\selectfont}}
            ] at (9.25,4.5)
        {
        & \\
        Debit & Credit \\
        MFO account & Customer $B$ account \\
        };
        \node[fit=(m-1-1)(m-1-2), font=\fontsize{9}{9}\selectfont] {\textit{T-Account: Entry by Branch \#2}};
        \draw[thin]  (m-2-2.south -| m.west) -- (m-2-2.south -| m.east);
        \draw[thin]  (m-2-2.north west) -- (m-3-2.south west);
        % Labels
        \node[align=left,font=\fontsize{9}{9}\selectfont,red] (T1) at (6.5,8.25) {\textbf{(1):} Initiating\\branch makes\\a \underline{credit} record\\against its\\MFO account};
        \node[align=left,font=\fontsize{9}{9}\selectfont,red] at (1.975,12.8) {\textbf{(2):} Initiating\\branch sends\\a credit register\\to Head Office};
        \node[align=left,font=\fontsize{9}{9}\selectfont,red] (T3) at (10.5,8.25) {\textbf{(5):} Completing\\branch makes\\a \underline{debit} record\\against its\\MFO account};
        \node[align=left,font=\fontsize{9}{9}\selectfont,blue] at (13.5,14.2) {\textbf{NEW:} (1) Accounts of inter-\\branch turnover are MFO\\accounts, (2) MAU = mecha-\\nical accounting unit};
        \node[align=center,font=\fontsize{9}{9}\selectfont,red] at (8,14.5) {\textbf{(3):} At Head Office, MAU updates\\the register of the branch MFO accounts};
        \node[align=left,font=\fontsize{9}{9}\selectfont,red] at (14.25,12.8) {\textbf{(4):} Completing\\branch receives\\a debit register\\from Head Office};
        \node[align=center,font=\fontsize{9}{9}\selectfont,red] at (8,10.35) {\textbf{(6):} Initiating branch carries out\\decentralized matching/verification of\\payment after receiving a debit register\\from completing branch};
        \draw[red] (T1.west) -- (4,7); %\draw[red] (T1.west) -- (5.5,2.1);
        \draw[red] (T3.west) -- (8.5,7); %\draw[red] (T3.east) -- (10.25,2.1);
        \node[rectangle,right delimiter=\}] (AA) at (12.5,4.5) {\tikz{\path (12.5,7) rectangle (12.5,2.75);}}; \node[align=left,right=22pt,font=\fontsize{9}{9}\selectfont] at (AA.west) {if customer $A$'s\\balance $>0$};
    % ...if customer A's balance =0
    % Branch #1 positions
    \matrix (m) [matrix anchor=north, matrix of nodes, nodes in empty cells,
             nodes={text width=.85in, minimum height=.2in, align=left}, 
             column sep=.5em,
             row 2/.style={nodes={font=\bfseries\fontsize{9}{9}\selectfont}},
             row 3/.style={nodes={font=\fontsize{9}{9}\selectfont}}
            ] at (3,2.25)
        {
        & \\
        Assets, $c_t$ & Liabilities, $d_t$ \\
        $\Delta\big\uparrow$ Customer $A$ account & $\Delta\big\uparrow$ MFO account \\
        };
        \node[fit=(m-1-1)(m-1-2), font=\fontsize{9}{9}\selectfont] {\textit{\MakeUppercase{Branch \#1}}};
        \draw[thin]  (m-2-2.south -| m.west) -- (m-2-2.south -| m.east);
        \draw[thin]  (m-2-2.north west) -- (m-3-2.south west);
    % Branch #2 positions
    \matrix (m) [matrix anchor=north, matrix of nodes, nodes in empty cells,
             nodes={text width=.85in, minimum height=.2in, align=left}, 
             column sep=.5em,
             row 2/.style={nodes={font=\bfseries\fontsize{9}{9}\selectfont}},
             row 3/.style={nodes={font=\fontsize{9}{9}\selectfont}}
            ] at (9.25,2.25)
        {
        & \\
        Assets, $c_t$ & Liabilities, $d_t$ \\
        $\Delta\big\uparrow$ MFO account & $\Delta\big\uparrow$ Customer $B$ account \\
        };
        \node[fit=(m-1-1)(m-1-2), font=\fontsize{9}{9}\selectfont] {\textit{\MakeUppercase{Branch \#2}}};
        \draw[thin]  (m-2-2.south -| m.west) -- (m-2-2.south -| m.east);
        \draw[thin]  (m-2-2.north west) -- (m-3-2.south west);
        \node[rectangle,right delimiter=\}] (AA) at (12.5,1) {\tikz{\path (12.5,1.9) rectangle (12.5,0);}}; \node[align=left,right=22pt,font=\fontsize{9}{9}\selectfont] at (AA.west) {if customer $A$'s\\balance $=0$};
    \end{tikzpicture}
    \caption[Schema of decentralized matching of an inter-branch payment in State Bank, which was practiced since 1931]%
    {Schema of decentralized matching of an inter-branch payment in State Bank, which was practiced since 1931. Source: \citep[p.~103]{cbr2010}, illustration by author.}
    \label{fig:state_bank_payments3}
\end{figure}

However, the central authorities were dissatisfied
with lowered level of control over the economic processes and by
elevated price inflation. Hence, there was a crucial reform done in the
1930s that banned issuance of \acp{iou} outside of the State Bank system. In
effect, by strictly enforcing debtor-creditor relationship via the State
Bank only, the authorities imposed tight ``ruble control" over the real
resource procurement and re-allocation.

Explaining historical materialism \citeauthor{marx1848} argued that during transformations of societies from one mode of production to another there is gradual however profound change in the ruling ideas that govern whole society. Transformation from capitalism to communism, they argued, would required rupture from the established and traditional ideas and view points. As far as, the money system of the society is concerned, such a rupture means that traditional banking system with its central banking and private banking parts evolves into a system, where:

\begin{quote}
[c]entralization of credit [is] in the hands of the State, by means of a national bank with State capital and an exclusive monopoly. \citep[p.~26]{marx1848}
\end{quote}

The monetary reforms of the 1930s in the Soviet Union following exactly the above proposition. As a result, the financial system was structured into two segments: domestic and
international. Its domestic segment consisted of two banks: Gosbank (or
interchangeably, State Bank) and Vneshekonombank. The former handled on
its balance sheet all the transactions for government agencies as well
as for the government-controlled industries that were directed via their
respective ministry. The latter handled operations in the domestic units
with outside world. The domestic part of the system was centrally
directed and not strictly a profit-seeking institution. It was concerned
rather with development. The international segment consisted of the
commercial banks owned by Soviet Union government as purely
profit-seeking institutions, which used to, and still do, hire
foreigners for their operations as a practice in those markets \citep{krotov2007,krotov2011}.

Domestically, it was the government via respective ministries overseeing
certain sectors of the economy and the state-planning department (called
Gosplan) that determined assortment, quantities, and prices of the
output. These were yearly assignments (plans). Similarly, on this
temporal basis Gosplan also allocated credit limits per production
union, which was a sector wide agglomeration of enterprises located over
certain area that were connected technologically into a production
chain. State Bank via its branches was accommodating enterprises with
lending within those limits. There was a practice of exchanging credit
limits between production unions, if there was a union in shortage of
the credit limit and there was a union with excess of credit limit.

The State Bank was organize as a single institution with a network of
branches, each of which maintained own balance sheet. A branch
maintained a balance sheet relationship with its customers. The payments
on behalf of the customer of State Bank were organized via the MFO
system, where the MFO abbreviation stands in Russian for *mezhfilialnyi
oborot* or inter-branch turnover in English. It facilitated smooth
functioning of payments between counter-parties in different parts of
the country spanning several time zones. Each branch of State Bank had
an MFO account on its balance sheet. If an enterprise, a client of the
State Bank branch, had not enough balance on its account, the branch
accommodated it with short-term loan to make the payment, charging a low
single-digit interest rate from the customer. If the customer delays the
repayment of the loan than interest rate charge increased. The payment
entry was accounted for on the balance sheet of the branch via the MFO
account (debit the current account of the bank's customer and credit the
MFO account). The MFO account was not an asset only account, it was
rather an asset-and-liability account. Its balance could appear on
either side of the balance sheet. Once a year, State Bank carried out a
procedure of netting out the accumulated balance on the MFO accounts of
its branches.

In addition to the MFO payment system, which was a gross payment system,
there was additional practice of net clearing. It was understood that:
(a) if there was a group of enterprises that operate regularly as
suppliers and customers to each other; and (b) if their payments
"collide with each other [in time]" or, in other words, their payments
to each other were taking place nearly simultaneously, then the net
clearing of these mutual payments would be beneficial to the parties
involved. The benefits of net clearing were: (1) improving the
timeliness of payment settlements; (2) economizing on the costs of
borrowed funds by the enterprises; and (3) lowering the labor input in
the paperwork both by the enterprises and State Bank officers.
Eventually, a number of such entities were established and their only
function was net clearing or matching of the mutual obligations between
enterprises and then bringing up only the net volume for the final
settlement/payment via the MFO system. These entities were called
Bureaus of Mutual Settlement. They enhanced operations of the industrial
enterprises as payments speed up, according to the State Bank documents
of the time. At that time, in the 1930s, the Soviet Union's economy
experienced rapid industrialization, while ``[f]ollowing the Second World
War the Soviet Union made rapid strides in economic development and
attained levels of industrial production second only to the United
States" \citep[p.~612]{dillard1967}.

Alongside with industrial development, the Soviet Union authorities
carried out policy of agriculture produce expropriation from the
Ukrainian rural population, which has been traditionally focused on
grain harvesting. Grain consolidation helped to sell it into the foreign
buyers and obtaining foreign bank credits, which were needed to buy
capital goods required for industrialization. In Ukraine, that period of
mass and artificial starvation in the 1930s is called Holodomor, i.e.
enforced deadly hunger.

The Soviet ruble was an inconvertible money of account. As a deputy
head of State Bank put it: there was ``Chinese wall" between ruble and
foreign currencies back then~\citep[p.~19]{krotov2008b}. So, in technical
terms, the ruble was inconvertible, neither into a foreign-currency unit
of account nor into precious metals such as gold (this was despite the
fact that cash notes of ruble of different denomination contained small
print saying ``bank tickets are backed by gold, precious metals and other
assets of State Bank." ). A domestic foreign-exchange market was absent,
and holding of foreign currencies by private individuals was illegal.
Foreign-exchange transactions were under strict control and limited in
terms of: (a) transactions and operations related to commercial
cross-border flow of goods and services; and (b) the transactions being
handled by a designated set of banking units, the above-mentioned
Vneshekonombank and \ac{ibec} (a Moscow-based International Bank for Economic
Cooperation). Entrepreneurship as well as financial speculation by
private individuals was also illegal ~\citep[p.~48]{bashkirova2012}. In general, as \citeauthor{dillard1967} put it, Soviet Union was ``a non-business economy"~\citep[p.~633]{dillard1967}.

From the early 1930s through the late 1980s, the payment system within the domestic part of the financial system of the \ac{ussr} functioned under the design of net clearing and active control and management\footnote{Officially, it was referred to as ``control by the means of ruble" \citep[p.~8]{usoskin1946}, \citep[pp.~99-110]{cbr2010}, and \citep[pp.~12-26]{krotov2008b}.} 


\begin{figure}[ht]
\vspace{.2in}
\captionsetup{width=1.0\linewidth,labelfont=bf}
    \centering
    \begin{tikzpicture}
        \draw[help lines,white] (0,0) grid (16,6);
        % First
        \node (A) at (0,2) {A};
        \node (B) at (4,2) {B};
        \node at (2,-1) {(1)};        
        \draw (A) -- (B) node[midway,above] {\textsuperscript{$debt: d_{0,B}$ \hspace{.25in} $credit: c_{0,A}$}};
        \draw[red] (0.2,1.9) -- (3.8,1.9) node[midway,below] {\textsuperscript{$credit: c_{0,B}$ \hspace{.25in} $debt: d_{0,A}$}};
        % Second
        \node         (A)         at (6,2) {A};
        \node         (C)         at (10,2) {C};
        \node         (B)         at (8,4) {B};
        \node         (D)         at (8,0) {D};
        \node at (8,-1) {(2)}; 
        \draw (A) -- (B) node[midway,sloped,above] {\textsuperscript{$d_{0,B}$ \hspace{.4in} $c_{0,A}$}}; 
        \draw (B) -- (C) node[midway,sloped,above] {\textsuperscript{$d_{0,C}$ \hspace{.4in} $c_{0,B}$}}; 
        \draw (C) -- (D) node[midway,sloped,below] {\textsuperscript{$c_{0,C}$ \hspace{.4in} $d_{0,D}$}}; 
        \draw (D) -- (A) node[midway,sloped,below] {\textsuperscript{$c_{0,D}$ \hspace{.4in} $d_{0,A}$}}; 
        % Third
        \node         (A)         at (12,2) {A};
        \node         (C)         at (16,2) {C};
        \node         (B)         at (14,4) {B};
        \node         (D)         at (14,0) {D};
        \node at (14,-1) {(3)};
        \draw (A) -- (B); \draw (B) -- (C); \draw (C) -- (D); \draw (D) -- (A);
        \draw (A) -- (C); \draw (B) -- (D); 
        \draw[red] (12.2,1.9) -- (15.8,1.9); % A--C
        \draw[red] (14.1,.2) -- (14.1,3.8);  % B--D
        \draw[red] (12.1,2.25) -- (13.7,3.85); % A--B
        \draw[red] (12.1,1.75) -- (13.7,.15);  % A--D
        \draw[red] (14.3,.15) -- (15.8,1.65);  % D--C
        \draw[red] (14.3,3.85) -- (15.9,2.25); % B--C
        % Labels
        \draw[black] (3,5.5) -- (4,5.5); \draw[red] (8,5.5) -- (9,5.5);
        \node[black,align=left,font=\small] at (5.5,5.5) {DCR due today};
        \node[black,align=left,font=\small] at (11.45,5.5) {Opposite DCR due today};
    \end{tikzpicture}
    \vspace{.1in}
    \caption[Types of payments clearing by the Soviet Union's State Bank]{Types of payments clearing by the Soviet Union's State Bank: (1) Bilateral clearing (set-off), (2) Chain clearing, (3) Multilateral clearing (series of set-offs). For details see, respectively, Figures~\ref{fig:bilateral_clearing},~\ref{fig:chain_clearing} and~\ref{fig:multilateral_clearing} below.\par \vspace{.05in}Source: narrative from \citep{schwartz1946,usoskin1946}, illustration by author.}
    \label{fig:setoff_state_bank}
\end{figure}

\begin{figure}[ht]
\vspace{.1in}
\captionsetup{width=1.0\linewidth,labelfont=bf}
    \centering
    \begin{tikzpicture}[scale=.9]
        % Clearing Scheme
        \draw[help lines,white] (0,0) grid (16,14);
        \node (A) at (0,12) {A};
        \node (B) at (6,12) {B};
        \node (R) at (3,14) {R}; \node (RR) at (7,14) {\small State Bank};
        %\node at (2,5) {(1)};        
        \draw (A) -- (B) node[midway,above] {\textsuperscript{$debt: d^{100}_{0,B}$ \hspace{.5in} $credit: c^{100}_{0,A}$}};
        \draw[red] (.3,11.9) -- (5.7,11.9) node[midway,below] {\textsuperscript{$credit: c^{90}_{0,B}$ \hspace{.5in} $debt: d^{90}_{0,A}$}};
        \draw[->,dashed] (RR) -- (R);
        % Register Matrix
        \matrix (m) [matrix anchor=north, matrix of nodes, nodes in empty cells,
             nodes = {text width=.35in, minimum height=.16in, align=right, anchor=center, font=\fontsize{9}{9}\selectfont}, column sep=.5em
            ] at (5,10)
        {
             &        &       &        &      &        &     \\
             &        &       &        &      &        &     \\
        \textbf{Unit} & Credit & Debt  & Credit & Debt & Credit & Debt \\
        A    & X       & X &   100  & 90   & 10 & X\\
        B    & 90     &  100  & X       & X    & X & 10 \\
        \textbf{Gross}    & 90     &  100  & 100       & 90     \\
         };
        \node[fit=(m-2-2)(m-2-3), font=\fontsize{9}{9}\selectfont] {Unit \textbf{A}};
        \node[fit=(m-2-4)(m-2-5), font=\fontsize{9}{9}\selectfont] {Unit \textbf{B}};
        \node[fit=(m-2-6)(m-2-7), font=\fontsize{9}{9}\selectfont] {\textbf{Net}};
        \node[fit=(m-1-1)(m-1-7), font=\fontsize{9}{9}\selectfont]%
            {\textit{\MakeUppercase{Register Matrix}}\vspace{.2in}};    
        \draw[thin]  (m-3-3.south -| m.west) -- (m-3-3.south -| m.east);
        \draw[thin]  (m-6-3.south -| m.west) -- (m-6-3.south -| m.east);
        \draw[thin]  (m-2-1.north east) -- (m-6-1.south east);
        \draw[thin]  (m-2-3.north east) -- (m-6-3.south east);
        \draw[thin]  (m-2-5.north east) -- (m-6-5.south east);
        % Labels
        \fill[red] (2.75,11.9) circle (2pt);
        \fill[] (3,12) circle (2pt);
        \draw[red,dashed] [arrows={-Stealth[]}] (2.75,11.9) -- (3.5,10.25);
        \draw[dashed] [arrows={-Stealth[]}] (3,12) -- (3.75,10.25);
        \node[fit={(m-4-5.north west) (m-4-5.south east)},
            ultra thick, inner sep=0pt, rounded corners=0mm, draw=red]{};
        \node[fit={(m-5-2.north west) (m-5-2.south east)},
            ultra thick, inner sep=0pt, rounded corners=0mm, draw=red]{};
            %draw=red,label={[cyan,align=center]270:Popular\\Choice!}]{};
        \node[fit={(m-4-4.north west) (m-4-4.south east)},
            ultra thick, inner sep=0pt, rounded corners=0mm, draw=black]{};
        \node[fit={(m-5-3.north west) (m-5-3.south east)},
            ultra thick, inner sep=0pt, rounded corners=0mm, draw=black]{};
         % Gross transaction table
        \matrix (m) [matrix anchor=north, matrix of nodes, nodes in empty cells,
            nodes={text width=.5in, align=right, minimum height=.16in, anchor=center, font=\fontsize{9}{9}\selectfont},
            column 1/.style={align=left},
            row 6/.style={nodes={font=\bfseries\fontsize{9}{9}\selectfont}}
            ] at (8,6.25)
        {
                     &      &    &  &        &    &     &      &      & \\
                     &      &    &  &        &    &     &      &      & \\
        \textbf{Unit}&Credit&Debt&NW& Credit$^\ast$ &Debt$^\dag$&NW$^\ddag$   &Credit&Debt  &NW    \\
        A            & 0    & 0  & 0& 100.008& 90 &0.008&10.008& 0    &0.008 \\
        B            & 0    & 0  & 0&  90.007& 100&0.007&0     & 9.993&0.007 \\
        \textbf{Total}&0    & 0  & 0& 190.015&190 &0.015&10.008& 9.993&0.015 \\
         };
         \node[fit=(m-1-1)(m-1-10), font=\fontsize{9}{9}\selectfont]%
            {\textit{\MakeUppercase{Gross payments}}\vspace{.2in}};
        \node[fit=(m-2-2)(m-2-4), font=\fontsize{9}{9}\selectfont] {\textbf{State Bank's $R$\\Opening Positions}};
        \node[fit=(m-2-5)(m-2-7), font=\fontsize{9}{9}\selectfont] {\textbf{Change of Positions}};
        \node[fit=(m-2-8)(m-2-10), font=\fontsize{9}{9}\selectfont] {\textbf{State Bank's $R$\\Closing Positions}};
        \draw[thin] (m-2-1.north east) -- (m-6-1.south east);
        \draw[thin] (m-2-4.north east) -- (m-6-4.south east);
        \draw[thin] (m-2-7.north east) -- (m-6-7.south east);
        \draw[thin] (m-6-3.south -| m.west) -- (m-6-3.south -| m.east);
        % Net transaction table -------------------------------------------------
        \matrix (m) [matrix anchor=north, matrix of nodes, nodes in empty cells,
            nodes={text width=.5in, align=right, minimum height=.16in, anchor=center, font=\fontsize{9}{9}\selectfont},
            column 1/.style={align=left},
            row 6/.style={nodes={font=\bfseries\fontsize{9}{9}\selectfont}}
            ] at (8,2.5)
        {
                     &      &    &  &        &    &     &      &      & \\
                     &      &    &  &        &    &     &      &      & \\
        \textbf{Unit}&Credit&Debt&NW& Credit$^\ast$ &Debt$^\dag$&NW$^\ddag$   &Credit&Debt  &NW    \\
        A            & 0    & 0  & 0&       0& 10 &    0& 0    & 10   &0 \\
        B            & 0    & 0  & 0&  10.001&   0&0.001&10.001& 0    &0.001 \\
        \textbf{Total}&0    & 0  & 0&  10.001& 10 &0.001&10.001& 10   &0.001 \\
         };
         \node[fit=(m-1-1)(m-1-10), font=\fontsize{9}{9}\selectfont]%
            {\textit{\MakeUppercase{Net Payments}}\vspace{.2in}};
        \node[fit=(m-2-2)(m-2-4), font=\fontsize{9}{9}\selectfont] {\textbf{State Bank's $R$\\Opening Positions}};
        \node[fit=(m-2-5)(m-2-7), font=\fontsize{9}{9}\selectfont] {\textbf{Change of Positions}};
        \node[fit=(m-2-8)(m-2-10), font=\fontsize{9}{9}\selectfont] {\textbf{State Bank's $R$\\Closing Positions}};
        \draw[thin] (m-2-1.north east) -- (m-6-1.south east);
        \draw[thin] (m-2-4.north east) -- (m-6-4.south east);
        \draw[thin] (m-2-7.north east) -- (m-6-7.south east);
        \draw[thin] (m-6-3.south -| m.west) -- (m-6-3.south -| m.east);
        \node[gray,align=left,font=\fontsize{9}{9}\selectfont] at (14,10.5)% 
            {Note:\\ \\$^{\ast}$ Change in the positions\\ of State Bank's credits\\on units $A$ and $B$ consists\\of their overnight borrowings at\\interest rate of 3\%, which are\\equal to debts due by the units,\\see Register Matrix, plus accrued\\interest rate charge;\\ \\$^{\dag}$ State Bank's debt positions\\to units $A$ and $B$ are their\\credits, see Register Matrix;\\ \\$^{\ddag}$ NW is Net Worth of State\\Bank, which is equal to the\\interest rate income charged\\ from units $A$ and $B$.};
    \end{tikzpicture}
     \caption[Bilateral clearing example via State Bank's \acf{bms}]%
    {Bilateral clearing example via State Bank's \acf{bms}. Source: narrative from \citep{schwartz1946,usoskin1946}, illustration by author.}
    \label{fig:bilateral_clearing}
    \vspace{.1in}
\end{figure}

\begin{figure}[ht]
\vspace{.2in}
\captionsetup{width=1.0\linewidth,labelfont=bf}
\centering
\begin{tikzpicture}[scale=.9]
        % Clearing Scheme
        \draw[help lines,white] (0,0) grid (17,13);
        \node (A) at (1,10) {A};
        \node (B) at (4,13) {B};
        \node (D) at (4,7)  {D};
        \node (C) at (7,10) {C};
        \draw (A) -- (B) node[midway,sloped,above] {\textsuperscript{$d^{100}_{0,B}$ \hspace{.6in} $c^{100}_{0,A}$}}; 
        \draw (B) -- (C) node[midway,sloped,above] {\textsuperscript{$d^{120}_{0,C}$ \hspace{.6in} $c^{120}_{0,B}$}}; 
        \draw (C) -- (D) node[midway,sloped,below] {\textsuperscript{$c^{140}_{0,C}$ \hspace{.6in} $d^{140}_{0,D}$}}; 
        \draw (D) -- (A) node[midway,sloped,below] {\textsuperscript{$c^{160}_{0,D}$ \hspace{.6in} $d^{160}_{0,A}$}}; 
        \node (R) at (9,13) {R}; \node (RR) at (12,13) {\small State Bank}; \draw[->,dashed] (RR) to (R);
       % Register Matrix
        \matrix (m) [matrix anchor=north, matrix of nodes, nodes in empty cells,
             nodes = {text width=.33in, minimum height=.16in, align=right, anchor=center, font=\fontsize{9}{9}\selectfont}, column sep=.5em
            ] at (8.5,5)
        {
             &  &  &  &  &  &  &  &  &  &  \\
             &  &  &  &  &  &  &  &  &  &  \\
        \textbf{Unit} & Credit & Debt  & Credit & Debt & Credit & Debt & Credit & Debt& Credit & Debt \\
        A    &  X &  X & 100&    &    &    &    & 160&    & 60 \\
        B    &    & 100&  X &  X & 120&    &    &    &  20&    \\
        C    &    &    &    & 120&  X &  X & 140&    &  20&    \\
        D    & 160&    &    &    &    & 140&  X &  X &  20&    \\
        \textbf{Gross}    
             & 160 &100 & 100& 120&120& 140& 140& 160&    &    \\
         };
        \node[fit=(m-2-2)(m-2-3), font=\fontsize{9}{9}\selectfont] {Unit \textbf{A}};
        \node[fit=(m-2-4)(m-2-5), font=\fontsize{9}{9}\selectfont] {Unit \textbf{B}};
        \node[fit=(m-2-6)(m-2-7), font=\fontsize{9}{9}\selectfont] {Unit \textbf{C}};
        \node[fit=(m-2-8)(m-2-9), font=\fontsize{9}{9}\selectfont] {Unit \textbf{D}};
        \node[fit=(m-2-10)(m-2-11), font=\fontsize{9}{9}\selectfont] {\textbf{Net}};
        \node[fit=(m-1-1)(m-1-11)]%
            {\textit{\MakeUppercase{Register Matrix}}\vspace{.2in}};    
        \draw[thin]  (m-3-3.south -| m.west) -- (m-3-3.south -| m.east);
        \draw[thin]  (m-8-3.south -| m.west) -- (m-8-3.south -| m.east);
        \draw[thin]  (m-2-1.north east) -- (m-8-1.south east);
        \draw[thin]  (m-2-3.north east) -- (m-8-3.south east);
        \draw[thin]  (m-2-5.north east) -- (m-8-5.south east);
        \draw[thin]  (m-2-7.north east) -- (m-8-7.south east);
        \draw[thin]  (m-2-9.north east) -- (m-8-9.south east);
        % Labels
        \fill[] (5.5,8.5) circle (2pt);
        \draw[dashed] [arrows={-Stealth[]}] (5.5,8.5) -- (8.45,5.5);
        \node[fit={(m-6-8.north west) (m-6-8.south east)},
            ultra thick, inner sep=0pt, rounded corners=0mm, draw=black]{};
        \node[fit={(m-7-7.north west) (m-7-7.south east)},
            ultra thick, inner sep=0pt, rounded corners=0mm, draw=black]{};
    \end{tikzpicture}
    \caption[Chain clearing example via State Bank's \acf{bms}: Part 1]%
    {Chain clearing example via State Bank's acf{bms}: Part 1 (\textit{See next page for continued exposition}). Note: One of the four debt-credit pairs, between $C$ and $D$, is highlighted via black rectangles. \par\vspace{.05in}Source: narrative from \citep{schwartz1946,usoskin1946}, illustration by author.}
    \label{fig:chain_clearing}
 \end{figure}
 \begin{figure}[ht]
 \ContinuedFloat
 \captionsetup{width=1.0\linewidth,labelfont=bf}
 \centering
 \begin{tikzpicture}[scale=.9]
        % Clearing Scheme
        \draw[help lines,white] (0,0) grid (17,10);
        % Gross transaction table
        \matrix (m) [matrix anchor=north, matrix of nodes, nodes in empty cells,
            nodes={text width=.48in, align=right, minimum height=.16in, anchor=center, font=\fontsize{9}{9}\selectfont},
            column 1/.style={align=left},
            row 8/.style={nodes={font=\bfseries\fontsize{9}{9}\selectfont}}
            ] at (8.5,10)
        {
                     &      &    &  &        &    &     &      &      & \\
                     &      &    &  &        &    &     &      &      & \\
        \textbf{Unit}&Credit&Debt&NW& Credit$^\ast$ &Debt$^\dag$&NW$^\ddag$   &Credit&Debt  &NW    \\
        A            & 0    & 0  & 0& 100.008& 160&0.008&    0 & 59.992    &0.008 \\
        B            & 0    & 0  & 0& 120.010& 100&0.010& 20.010     & 9.993&0.010 \\
        B            & 0    & 0  & 0& 140.012& 120&0.012& 20.012     & 9.993&0.012 \\
        B            & 0    & 0  & 0& 160.013& 140&0.013& 20.013    & 9.993&0.013 \\
        \textbf{Total}&0    & 0  & 0& 520.043& 520 &0.043&60.035& 59.992&0.043 \\
         };
         \node[fit=(m-1-1)(m-1-10)]%
            {\textit{\MakeUppercase{Gross payments}}\vspace{.2in}};
        \node[fit=(m-2-2)(m-2-4), font=\fontsize{9}{9}\selectfont] {\textbf{State Bank's $R$\\Opening Positions}};
        \node[fit=(m-2-5)(m-2-7), font=\fontsize{9}{9}\selectfont] {\textbf{Change of Positions}};
        \node[fit=(m-2-8)(m-2-10), font=\fontsize{9}{9}\selectfont] {\textbf{State Bank's $R$\\Closing Positions}};
        \draw[thin] (m-2-1.north east) -- (m-8-1.south east);
        \draw[thin] (m-2-4.north east) -- (m-8-4.south east);
        \draw[thin] (m-2-7.north east) -- (m-8-7.south east);
        \draw[thin] (m-8-3.south -| m.west) -- (m-8-3.south -| m.east);
        %\node[fit=(m-8-8)(m-8-10),fill=gray];
        % Net transaction table
        \matrix (m) [matrix anchor=north, matrix of nodes, nodes in empty cells,
            nodes={text width=.48in, align=right, minimum height=.16in, anchor=center, font=\fontsize{9}{9}\selectfont},
            column 1/.style={align=left},
            row 8/.style={nodes={font=\bfseries\fontsize{9}{9}\selectfont}}
            ] at (8.5,4.5)
        {
                     &      &    &  &        &    &     &      &      & \\
                     &      &    &  &        &    &     &      &      & \\
        \textbf{Unit}&Credit&Debt&NW& Credit$^\ast$ &Debt$^\dag$&NW$^\ddag$   &Credit&Debt  &NW    \\
        A            & 0    & 0  & 0&      0& 60&   0&    0 & 60  & 0 \\
        B            & 0    & 0  & 0& 20.002& 0&0.002& 20.002 & 0&0.002 \\
        B            & 0    & 0  & 0& 20.002& 0&0.002& 20.002 & 0&0.002 \\
        B            & 0    & 0  & 0& 20.002& 0&0.002& 20.002 & 0&0.002 \\
        \textbf{Total}&0    & 0  & 0& 60.006& 60 &0.006&60.006& 60&0.006 \\
        };
        \node[fit=(m-1-1)(m-1-10)]%
            {\textit{\MakeUppercase{Net payments}}\vspace{.2in}};
        \node[fit=(m-2-2)(m-2-4), font=\fontsize{9}{9}\selectfont] {\textbf{State Bank's $R$\\Opening Positions}};
        \node[fit=(m-2-5)(m-2-7), font=\fontsize{9}{9}\selectfont] {\textbf{Change of Positions}};
        \node[fit=(m-2-8)(m-2-10), font=\fontsize{9}{9}\selectfont] {\textbf{State Bank's $R$\\Closing Positions}};
        \draw[thin] (m-2-1.north east) -- (m-8-1.south east);
        \draw[thin] (m-2-4.north east) -- (m-8-4.south east);
        \draw[thin] (m-2-7.north east) -- (m-8-7.south east);
        \draw[thin] (m-8-3.south -| m.west) -- (m-8-3.south -| m.east);
    \end{tikzpicture}
    \caption[Chain clearing example via State Bank's \acf{bms}: Part 2]%
    {Chain clearing example via State Bank's \acf{bms}: Part 2 (\textit{Continued from previous page}).\par\vspace{.05in}Note: $^{\ast}$ Change in the positions of State Bank's credits on units $A,B,C,$ and $D$ consists of their overnight borrowings at interest rate of 3\%, which are equal to debts due by the units,see Register Matrix, plus accrued interest rate charge; $^{\dag}$ State Bank's debt positions to units $A,B,C,$ and $D$ are their credits, see Register Matrix; $^{\ddag}$ NW is Net Worth of State Bank, which is equal to the interest rate income charged from units $A,B,C,$ and $D$.\par\vspace{.05in}Source: narrative from \citep{schwartz1946,usoskin1946}, illustration by author.}
    \label{fig:chain_clearing}
\end{figure}

\begin{figure}[ht]
\vspace{.2in}
\captionsetup{width=1.0\linewidth,labelfont=bf}
\centering
\begin{tikzpicture}[scale=.9]
        % Clearing Scheme
        \draw[help lines,white] (0,0) grid (17,13);
        \node (R) at (9,13) {R}; \node (RR) at (9,11.75) {\small State Bank}; 
        \draw[->,dashed] (RR) to (R);
        \node (A) at (1,10) {A};
        \node (B) at (4,13) {B};
        \node (D) at (4,7)  {D};
        \node (C) at (7,10) {C};
        \draw (A) -- (B) node[midway,sloped,above] {\textsuperscript{$d^{100}_{0,B}$ \hspace{.6in} $c^{100}_{0,A}$}}; 
        \draw (B) -- (C) node[midway,sloped,above] {\textsuperscript{$d^{120}_{0,C}$ \hspace{.6in} $c^{120}_{0,B}$}}; 
        \draw (C) -- (D) node[midway,sloped,below] {\textsuperscript{$c^{140}_{0,C}$ \hspace{.6in} $d^{140}_{0,D}$}}; 
        \draw (D) -- (A) node[midway,sloped,below] {\textsuperscript{$c^{160}_{0,D}$ \hspace{.6in} $d^{160}_{0,A}$}}; 
        \draw (B) -- (D) node[midway,sloped,above] {\textsuperscript{$d^{50}_{0,D}$ \hspace{.8in} $c^{50}_{0,B}$}};
        \draw (A) -- (C) node[midway,sloped,above] {\textsuperscript{$d^{100}_{0,C}$ \hspace{.8in} $c^{100}_{0,A}$}};
        % Opposite DCRs
        \draw[red] (1.3,10.1) -- (3.8,12.6); %(A) -- (B);
        \draw[red] (1.3,9.9) -- (3.8,7.4); %(A) -- (D);
        \draw[red] (3.85,12.45) -- (3.85,7.45); %(B) -- (D);
        \draw[red] (1.4,9.9) -- (6.7,9.9); %(A) -- (C);
        \draw[red] (4.2,12.6) -- (6.7,10.1); %(B) -- (C);
        \draw[red] (4.2,7.4) -- (6.6,9.8); %(C) -- (D);
        % Opposite DCRs as standalone scheme
        \node (A) at (11,10) {A};
        \node (B) at (14,13) {B};
        \node (D) at (14,7)  {D};
        \node (C) at (17,10) {C};
        \draw[red] (A) -- (B) node[midway,sloped,above] {\textsuperscript{$c^{40}_{0,B}$ \hspace{.6in} $d^{40}_{0,A}$}}; 
        \draw[red] (B) -- (C) node[midway,sloped,above] {\textsuperscript{$c^{90}_{0,C}$ \hspace{.6in} $d^{90}_{0,B}$}}; 
        \draw[red] (C) -- (D) node[midway,sloped,below] {\textsuperscript{$d^{100}_{0,C}$ \hspace{.6in} $c^{100}_{0,D}$}}; 
        \draw[red] (D) -- (A) node[midway,sloped,below] {\textsuperscript{$d^{80}_{0,D}$ \hspace{.6in} $c^{80}_{0,A}$}}; 
        \draw[red] (B) -- (D) node[midway,sloped,above] {\textsuperscript{$c^{150}_{0,D}$ \hspace{.9in} $d^{150}_{0,B}$}};
        \draw[red] (A) -- (C) node[midway,sloped,above] {\textsuperscript{$c^{70}_{0,C}$ \hspace{.9in} $d^{70}_{0,A}$}};
       % Register Matrix
        \matrix (m) [matrix anchor=north, matrix of nodes, nodes in empty cells,
             nodes = {text width=.33in, minimum height=.16in, align=right, anchor=center, font=\fontsize{9}{9}\selectfont}, column sep=.5em
            ] at (8.5,5)
        {
             &  &  &  &  &  &  &  &  &  &  \\
             &  &  &  &  &  &  &  &  &  &  \\
        \textbf{Unit} & Credit & Debt  & Credit & Debt & Credit & Debt & Credit & Debt& Credit & Debt \\
        A    &  X &  X & 100&  70&  80& 160&  80& 160&    & 10 \\
        B    &  40& 100&  X &  X & 120&  90&  50& 150&  130&    \\
        C    &  70& 100& 90 & 120&  X &  X & 140& 100&  120&    \\
        D    & 160&  80& 150&  50& 100& 140&  X &  X &  & 140 \\
        \textbf{Gross}    
             & 270 &280 & 340& 210&320& 300& 270& 410&    &    \\
         };
        \node[fit=(m-2-2)(m-2-3), font=\fontsize{9}{9}\selectfont] {Unit \textbf{A}};
        \node[fit=(m-2-4)(m-2-5), font=\fontsize{9}{9}\selectfont] {Unit \textbf{B}};
        \node[fit=(m-2-6)(m-2-7), font=\fontsize{9}{9}\selectfont] {Unit \textbf{C}};
        \node[fit=(m-2-8)(m-2-9), font=\fontsize{9}{9}\selectfont] {Unit \textbf{D}};
        \node[fit=(m-2-10)(m-2-11), font=\fontsize{9}{9}\selectfont] {\textbf{Net}};
        \node[fit=(m-1-1)(m-1-11)]%
            {\textit{\MakeUppercase{Register Matrix}}\vspace{.2in}};    
        \draw[thin]  (m-3-3.south -| m.west) -- (m-3-3.south -| m.east);
        \draw[thin]  (m-8-3.south -| m.west) -- (m-8-3.south -| m.east);
        \draw[thin]  (m-2-1.north east) -- (m-8-1.south east);
        \draw[thin]  (m-2-3.north east) -- (m-8-3.south east);
        \draw[thin]  (m-2-5.north east) -- (m-8-5.south east);
        \draw[thin]  (m-2-7.north east) -- (m-8-7.south east);
        \draw[thin]  (m-2-9.north east) -- (m-8-9.south east);
        % Labels
        \fill[red] (12.5,8.5) circle (2pt);
        \fill[] (5.5,8.5) circle (2pt);
        \draw[red,dashed] [arrows={-Stealth[]}] (12.5,8.5) -- (8.55,5.5);
        \draw[dashed] [arrows={-Stealth[]}] (5.5,8.5) -- (8.45,5.5);
        \node[fit={(m-7-3.north west) (m-7-3.south east)},
            ultra thick, inner sep=0pt, rounded corners=0mm, draw=red]{};
        \node[fit={(m-4-8.north west) (m-4-8.south east)},
            ultra thick, inner sep=0pt, rounded corners=0mm, draw=red]{};
            %draw=red,label={[cyan,align=center]270:Popular\\Choice!}]{};
        \node[fit={(m-6-8.north west) (m-6-8.south east)},
            ultra thick, inner sep=0pt, rounded corners=0mm, draw=black]{};
        \node[fit={(m-7-7.north west) (m-7-7.south east)},
            ultra thick, inner sep=0pt, rounded corners=0mm, draw=black]{};
    \end{tikzpicture}
    \caption[Multilateral clearing example via State Bank's \acf{bms}: Part 1]%
    {Multilateral clearing example via State Bank's \acf{bms}: Part 1. (\textit{See next page for continued exposition}) Note: Two of the twelve debt-credit pairs, between $C$ and $D$ and $D$ and $A$, are highlighted via black and red rectangles.\par\vspace{.05in}Source: narrative from \citep{schwartz1946,usoskin1946}, illustration by author.}
    \label{fig:multilateral_clearing}
 \end{figure}
 \begin{figure}[ht]
 \ContinuedFloat
 \captionsetup{width=1.0\linewidth,labelfont=bf}
 \centering
 \begin{tikzpicture}[scale=.9]
        % Clearing Scheme
        \draw[help lines,white] (0,0) grid (17,10);
        % Gross transaction table
        \matrix (m) [matrix anchor=north, matrix of nodes, nodes in empty cells,
            nodes={text width=.48in, align=right, minimum height=.16in, anchor=center, font=\fontsize{9}{9}\selectfont},
            column 1/.style={align=left},
            row 8/.style={nodes={font=\bfseries\fontsize{9}{9}\selectfont}}
            ] at (8.5,10)
        {
                     &      &    &  &        &    &     &      &      & \\
                     &      &    &  &        &    &     &      &      & \\
        \textbf{Unit}&Credit&Debt&NW& Credit$^\ast$ &Debt$^\dag$&NW$^\ddag$   &Credit&Debt  &NW    \\
        A            & 0    & 0  & 0& 280.023& 270&0.023& 10.023&      0&0.023 \\
        B            & 0    & 0  & 0& 210.017& 340&0.017&      0&129.983&0.017 \\
        B            & 0    & 0  & 0& 300.025& 320&0.025&      0& 19.975&0.025 \\
        B            & 0    & 0  & 0& 410.034& 270&0.034&140.034&      0&0.034 \\
        \textbf{Total}&0    & 0  & 0&1200.099&1200&0.099&150.057&149.958&0.099 \\
         };
       \node[fit=(m-1-1)(m-1-10)]%
            {\textit{\MakeUppercase{Gross payments}}\vspace{.2in}};
        \node[fit=(m-2-2)(m-2-4), font=\fontsize{9}{9}\selectfont] {\textbf{State Bank's $R$\\Opening Positions}};
        \node[fit=(m-2-5)(m-2-7), font=\fontsize{9}{9}\selectfont] {\textbf{Change of Positions}};
        \node[fit=(m-2-8)(m-2-10), font=\fontsize{9}{9}\selectfont] {\textbf{State Bank's $R$\\Closing Positions}};
        \draw[thin] (m-2-1.north east) -- (m-8-1.south east);
        \draw[thin] (m-2-4.north east) -- (m-8-4.south east);
        \draw[thin] (m-2-7.north east) -- (m-8-7.south east);
        \draw[thin] (m-8-3.south -| m.west) -- (m-8-3.south -| m.east);
        %\node[fit=(m-8-8)(m-8-10),fill=gray];
        % Net transaction table
        \matrix (m) [matrix anchor=north, matrix of nodes, nodes in empty cells,
            nodes={text width=.48in, align=right, minimum height=.16in, anchor=center, font=\fontsize{9}{9}\selectfont},
            column 1/.style={align=left},
            row 8/.style={nodes={font=\bfseries\fontsize{9}{9}\selectfont}}
            ] at (8.5,4.5)
        {
                     &      &    &  &        &    &     &      &      & \\
                     &      &    &  &        &    &     &      &      & \\
        \textbf{Unit}&Credit&Debt&NW& Credit$^\ast$ &Debt$^\dag$&NW$^\ddag$   &Credit&Debt  &NW    \\
        A            & 0    & 0  & 0& 10.001&  0&0.001&    10.001 & 0  & 0.001 \\
        B            & 0    & 0  & 0&      0&130&    0& 0 & 130& 0 \\
        B            & 0    & 0  & 0&      0& 20&    0& 0 & 20& 0 \\
        B            & 0    & 0  & 0&140.012&  0&0.012&140.012 & 0& 0.012 \\
        \textbf{Total}&0    & 0  & 0&150.013&150&0.013&150.013& 150 & 0.013 \\
        };
        \node[fit=(m-1-1)(m-1-10)]%
            {\textit{\MakeUppercase{Net payments}}\vspace{.2in}};
        \node[fit=(m-2-2)(m-2-4), font=\fontsize{9}{9}\selectfont] {\textbf{State Bank's $R$\\Opening Positions}};
        \node[fit=(m-2-5)(m-2-7), font=\fontsize{9}{9}\selectfont] {\textbf{Change of Positions}};
        \node[fit=(m-2-8)(m-2-10), font=\fontsize{9}{9}\selectfont] {\textbf{State Bank's $R$\\Closing Positions}};
        \draw[thin] (m-2-1.north east) -- (m-8-1.south east);
        \draw[thin] (m-2-4.north east) -- (m-8-4.south east);
        \draw[thin] (m-2-7.north east) -- (m-8-7.south east);
        \draw[thin] (m-8-3.south -| m.west) -- (m-8-3.south -| m.east);
    \end{tikzpicture}
    \caption[Multilateral clearing example via State Bank's \acf{bms}: Part 2]%
    {Multilateral clearing example via State Bank's \acf{bms}: Part 2 (\textit{Continued from previous page}).\par\vspace{.05in}Note: $^{\ast}$ Change in the positions of State Bank's credits on units $A,B,C,$ and $D$ consists of their overnight borrowings at interest rate of 3\%, which are equal to debts due by the units,see Register Matrix, plus accrued interest rate charge; $^{\dag}$ State Bank's debt positions to units $A,B,C,$ and $D$ are their credits, see Register Matrix; $^{\ddag}$ NW is Net Worth of State Bank, which is equal to the interest rate income charged from units $A,B,C,$ and $D$.\par\vspace{.05in}Source: narrative from \citep{schwartz1946,usoskin1946}, illustration by author.}
    \label{fig:multilateral_clearing}
\end{figure}

One basic feature of this payment system was the following: State Bank was a single entity operationally, the balance sheet of which was used to record all debits and credits among all domestic economic units.\footnote{This system was different from its counterparts in the Western countries by concentration of credit and clearing functions under one roof. It worth to note that, for example in the US, some authors were critical of the existed credit and clearing system as being less efficient: ``Much of this difficulty [arising from out-ot-town check collections] could be eliminated if the United States had a branch banking system instead of the present system, which is composed of some 27,000 small, independent banks. If there were a hundred banks with fifteen thousand branches, the matter would be greatly simplified. Many checks would be redeposited in another branch of the drawee bank and no collection problem would be involved. Checks on different banks could be put through centrally located clearing houses, as they are in Canada."~\citep[p.~19]{patterson1918}} Describing this system, \citeauthor{woodruff1999} retains the original abbreviation \ac{mfo} and the Russian name it stands for, mezhfilialnyi oborot???inter-branch transfers or turnover in English \citep[pp.~66-67]{woodruff1999}. This paper follows this practice by referencing it as the \ac{mfo} system.

Mechanics of government finance were the following. A former Deputy Head
of State Bank explained: ``[T]he \ac{mfo} system allowed budget spending on
the local level to be carried out independently of budget revenues.
Balancing of revenues and expenditures was taking place [later on] in
the Center, in State Bank, which credited Ministry of Finance if the
balance of the latter was in shortage"~\citep[p.~16]{krotov2008b}.

After the World War II, the Soviet Union established a bloc of socialist
countries of the Eastern Europe and later included Vietnam, Cuba.
Together these were organized under the umbrella of the \acf{cmea}, which used to be referred to by the abbreviated name of Comecon. At the beginning the trade within Comecon,
i.e. between different money systems each of which has own money unit of
account, was carried out via Western banks and relying on the funds
denominated in foreign money units of account (not of the countries
members of Comecon). Later on, in 1963 Soviet Union established the
above-mentioned \ac{ibec} that took over over the trade flows between Comecon
member states by establishing new money of account *transferable
ruble*. The latter was not convertible into the Soviet ruble, the money
unit of account used in the USSR, nor into the units of accounts of the
Western world. Again, the member states of Comecon retained their
national money units of account for domestic transactions, while dealing
in Western money units for transactions outside the \ac{cmea}. See \citep{vincze,kalinski}.

During the 1970s both the Soviet Union and the members of the Comecon
had accumulated foreign-currency debt on the back of: (a) past poor
harvests; (b) support of consumption by domestic consumers via importing
goods; and (c) domestic fixed-investments that required foreign
technology. For more on these developments, for instance, see \citep{coombs1976,gaidar2002,harold}.

Dillard explained that agriculture was one of ``the weakest links of the
Soviet economy"~\citep[p.~628]{dillard1967}. As a result, it forced the country
"[i]n 1963 [to enter] the world market to buy grain to feed its growing
population" (ibid). It was quite an unusual episode in the global
economy of the time. Thus, Coombs discusses Russia's significant
offerings of gold during the 1960s to the London gold market,
"reflecting wheat harvest failures in 1963 and 1965."~\citep[p.~154]{coombs1976}.
Later, however, the authorities of the Soviet Union turned to
imports-buying through Western debt instead of selling its stock of
gold. Already in the 1970s, the Soviet Union was already in its fourth
or fifth year of trying to gradually reduce political tensions with the
Western powers. Meanwhile, the Polish government, facing unpopularity,
embarked on ``a build-up of Western debt, a substantial proportion of
which was used for consumption," and eventually, ``Poland become the
world's third-largest wheat importer"~\citep[pp.~89-90]{harold}. The Soviet
Union helped Comecon members with refinancing of debt in foreign money
of account. For example, Poland was assisted in this way for several
billion US dollars \citep{stasi1981}. But already in the early 1980s, top
leadership of the Soviet Union faced situation, which later would be
called as sudden stop: ``Banks have suddenly stopped giving us loans
(USA, FRG [West Germany])." \citep{stasi1981} quoted in \citep[pp.~15-16]{zubok2021}.

The command economy of the Soviet Union of its late period was described
as \textit{hunting} for the ``hard" currency receipts via exports of commodities
such as crude oil and natural gas, see \citep[p.~222]{dibb1988} and \citep{zubok2021}. Back
in the 1980s, the hard/soft currency dichotomy implied: (a) ``hard
currency" meant the Western banks' liabilities and the central banks'
liabilities as physical (paper) currency denominated in the money units
of account of the major economies such the US dollar and Deutsche mark,
(b) ``soft currency" meant the state banks' liabilities denominated in
the money units of account of the socialist bloc of countries, as in~\citep{dibb1988}.


\subsubsection{Reforms of 1980s \& 1990s, the August 1998 Default}

Since late 1970s and especially early 1980s, the Soviet Union leadership
nurtured the idea of economic reform \citep{zubok2021}, while specifics of
the reform were vague and debatable. However, key strategy was to
incentivize profit-seeking behavior among the state-owned industrial
enterprises, which were deemed wasting resources while enjoying
accommodation in resources allocation provided by the state. Money as
credit was considered as part of the resources alongside with physical
resources. While foreign banks \acp{iou} (credits) were indeed scarce, as it
was described above, the Soviet Union authorities tended to think of the
domestic bank \acp{iou} (credits) in the same way. In early 1980s, few years
prior to Perestroika launch by Gorbachev, the mid-1980s leader of the
Soviet Union, then another leader of Soviet Union Andropov instructed
State Bank to provide much less accommodation to domestic enterprises
than it used to be. Same kind of incentives were
sought to apply widely throughout Soviet Union:

\begin{quote}
 ``Other ministers should come to you," [Andropov] said to the Minister
 of Finance, ``crawling on their bellies, \textit{begging for money}."~\citep[p.~18, emphasis added]{zubok2021}
\end{quote}

Later on, in early 1986 or just few years after Andropov death, newly
appointed leader of Soviet Union intriduced policy of Perestroika. It
aimed to realize absolute economic calculation among the publicly-owned
enterprises, i.e. push them into profit-seeking mode of operations
instead of obtaining state subsidies. It never intended to introduce
private property on those means of production, see \citep{gorb}. This is what
Andropov, a hard-line leader, was seeking too.

The second half of 1980s experienced an extension of the past shortages
of consumer goods, not elimination of them. As past system---with
Gosplan and State Bank---was still in place, the policy of establishing
business-focused economy had shifted to embrace the previously
unthinkable ideas. First of them was market-determined prices instead of
state-dictated ones on broad range of goods and services. Second of them
was making legal private property on the means of production.

Meanwhile, as these ideas were still digested by policy-makers and
general public, the key institutional change took place. The monetary
system was reformed first out of all other sectors. The rationale was
that business production must be supported by business credit
allocation. While previously there was state monopoly on credit
allocation to enterprises, in the second half of 1990s a law was adopted
that allowed creation of cooperatives, a form of private ownership
organization with for-profit operations. In addition, the changes into
the law allowed cooperatives to create banks. The old banking system was
deem slow and wasteful, hence, there must be modern banks to spring up.
There was no banking oversight from the State Bank as this practice was
unknown to the officials of the old monetary system. It resulted in the
development when new banking industry attracted either highly-smart and
business-capable people with PhD in physics and people with criminal
past and respective incentives \citep{valchyshen_ru98}. These new banks used
to open accounts with State Bank and its new departments -- they
effectively used State Bank balance to net clear the payments their new
customers, to whom they promised more attractive interest rates on
balances than State Bank did. In addition to that the authorities were
gradually relaxing the previously tight restrictions on foreign trade,
which inevitably spawn innovations of allowing access to the \acp{iou} in
foreign money units of account.

However, let us return to the main policy of the 1980s of incentivizing
the state-owned enterprises to become more efficient and operate in a
business, competitive environment. It became popular if not dominant
among the domestic economists of the Soviet Union and the foreign ones
that worked on the economy of the Soviet Union and advised its
policy-makers. That thinking adopted the approach developed by Hungarian
economist Janos Kornai, who argued that socialist economies operated
under *soft budget constraint* (SBC). It is when state-owned enterprises
dominating these economies obtain credit from state bank and subsidies
from the state government on a regular basis. Moreover, the price system
controlled by the state does not incentivize surpluses of goods.
Instead, shortages result.

So, the key of economists within Soviet Union were aligning with a
policy to remedy the Soviet Union economy. It was short with two
prescriptions: (1) moving to the market-based economic system as quickly
as possible, and (2) impose hard budget constraint on state-owned
enterprises, while their privatization was a distant prospect back than.

Economists of Post Keynesian tradition from the quite early criticized
the concept of hard budget constraint as one having no relevance at all
if one bases analysis on the endogenous money view. As the following
quote borrowed from the 1991 article states, both types of the economies
(capitalist or socialist) could be characterized with such monetary
conditions that are, applying SBC terminology, on the softer side of the
budget constraint continuum.

\begin{quote}
 If credit money is truly endogenous in a capitalist system, then a
 hard budget constraint does not exist in capitalist or socialist
 systems; i.e., given the cost of obtaining credit from the banking
 system, firms have a soft budget constraint in \textit{both} economies.
 \citep[p.~330, emphasis added]{szego}
\end{quote}

Indeed, high-profile proposals on the reforming of Soviet Union such as
in \citep{peck} considered the translation towards market economy through the
prism of hardening of budget constraint. For example, \citeauthor{nordhaus}
contributing to \cite{peck} advised for the fiscal policy ``the target should
be a balanced budget"~\citep[p.~104]{nordhaus}, while for the monetary policy
his advise was:

\begin{quote}
 With respect to credit policies, a substantial tightening of credit
 will be possible and desirable when firms operate with hard-budget
 constraints. Once individual enterprises are subject to hard-budget
 constraints, Gosbank should make credit available only to firms that
 can repay credits; this implies curbing credits to unprofitable
 enterprises.~\citep[p.~108]{nordhaus}
\end{quote}

From the \ac{mmt}'s sectoral balances approach, where domestic non-government
sector is represented by two sectors of households and enterprises in
addition to the government and foreign sectors, then prospect of two
sectors changing their balances from deficit to zero (balanced) or even
surplus suggest that either surplus of the household sector should
contract or foreign sector should increase its deficit against domestic
economy. Given the fact that over 1980s Soviet Union economy was
struggling in obtaining foreign \acp{iou} (credits) from abroad, then we can
assume that foreign sector impact on the Soviet Union economy not that
decisive. Then with major three sectors of the domestic economy, where
two---government and enterprises---are set to reduce their gross
spending to achieve net spending at zero, which balanced budget or
surplus budget outcome as incomes are equal or greater than
expenditures. It means that the remaining sector of households faces
decline in gross incomes. Dynamically, such a prospect spells unstable
rather than stable economy.

Add to this the last minute financial reform of early 1991, when the
Finance Minister of Soviet Union pushed through the policy of calling in
all the 50 and 100 paper currency notes of the State Bank in circulation
and replacing them with currency of newer print. A person was allowed to
exchange these notes in the total amount no more than 10,000 rubles. The
reason of this reform was concern over growing ``overhand of currency"
amid rising prices and persistent shortage of goods. (Already three-year
private banking system operating on the margins of State Bank was left
without due attention.) This policy caused the ``most uproar"~\citep[p.~189]{zubok2021} among the general public.\footnote{Yet, \citeauthor{zubok2021} considers Valentin Pavlov, Minister of Finance and author of this reform, as ``one of the few who knew how the Soviet monetary system worked and what the real causes of its crisis were"~\citep[p.~175]{zubok2021}.}

Politically, Soviet Union was brewing with national revival of
independence and sovereignty that was previously curtailed with military
force. In Ukraine it was in 1920s, while in Baltic countries it was in
1940s. Economic hardship with long-lasting shortages of consumer goods,
which became acute and open during 1980s, fueled wide backlash against
the central authorities in Moscow. In 1991, political disintegration of
the Soviet Union was largely peaceful. Political divorce between Ukraine
and Russia back in 1991 was completely peaceful.

There is wrong money take with respect to Soviet Union collapse such as:

\begin{quote}
 The power of money was central and crucial to the behavior of Soviet
 elites during the last years of the Soviet Union. Had the Kremlin
 ruler made different choices, to tap into this power, turning the
 existing elites into stakeholders of the transition, instead of
 alienating them, even the KGB officers would supported state
 capitalism and privatization, just as they later did under Yeltsin and
 Putin.~\citep[p.~437]{zubok2021}
\end{quote}

It assumes that economic theory was roughly correct in drawing the path
for transition towards the market economy. It does not take into account
the possibility of this economic theory being a bit incorrect.

\cite{dibb2006}: ``Practically *overnight*, the Soviet Union lost its world
power status and, fragmented as Russian society was, plunged into a
nightmare of a collapsing economy, rampant inflation and the
disappearance of savings, jobs and pensions" (emphasis added). It is a
grand mistake to put all these events into one short period of time,
which is 1991, and then allow this one sentence to govern over our
understanding of FSU and Russia developments. Secondly, approximating
``Russian society" to ``the Soviet Union" is another omission usually made
by two groups: (1) Russians themselves, and (2) people who where not not
born and raised inside the Soviet Union.

In 1990-91, today's president of Russia Putin himself had experience of
structuring a cross-border transactions while he served in the city
council of Leningrad, the second largest city in Russia and the FSU.
Now, it is known as city of St Petersburg. Back then the economy of the
collapsing Soviet Union was swiftly disintegrating economically and
politically, see respectively \citep{woodruff2000} and \citep{zubok2021}. The
major urban centers of the former Soviet Union such as city of Moscow
and Leningrad have experienced shortages of consumer goods supplies. It
was due to mass hoarding of vital goods produced domestically as
state-run infrastructures of banking and payments and provisioning of
food and other products were evolving from purely state owned and
centralized to privately owned and decentralized albeit in an extreme
sense of the word. In May 1991, or few months before its ultimate
collapse, the Soviet Union government made a desperate move to sustain
key urban areas with food supplies. It assigned a number of for-profit
businesses, which were just emerging at the time in the Soviet Union, to
be middlemen between foreign food suppliers and itself. The foreign food
suppliers were accepting foreign funds in payment. Both the Soviet Union
government and those for-profit businesses were short of foreign funds.
In order to obtain them, the Soviet Union government guaranteed exports
of valuable raw materials.\footnote{These included ``oil, timber, metals, cotton, and other natural resources"~\citep[p.~121]{gessen2014}.} This way the foreign funds were raised
and available for payment of food imports. In the city of Leningrad in
1991, such a for-profit business was Kontinent, one of the many
companies created by the city council's foreign relations department
that was led by Putin. It entered into an agreement with a German food
supplier. Some biographic accounts of Putin contain mentioning this
transaction. the Russia side committed to deliver natural resources,
while the German side committed to deliver food products. The
transaction involved money of account and mutual money payments. It was
essentially a monetary transaction, even though some authors called them
as barter.\footnote{By definition ``[b]arter exchange of commodities, at a ratio
    agreed by the parties involved, *does not* involve a common money of
    account" \citep[p.~107, emphasis added]{ingham2020}. Hence, barter
    description of the above mentioned transaction is a mistaken
    approach.} It did contain money of account that both sides agreed
to use for denomination of the value of goods purchased and sold. The
Russian side desired to get access to the balances in the German bank
denominated in the money of account of Germany. For the German side it
was rather normal to agree for the use of German money of account. Once
the Russian side delivered the natural resources under its side of the
agreement, the German side credited 90 million Deutsche marks to
Vneshekonobank bank account opened with a German bank. Then the Russian
side paid for the food supplies to the German side. Records showed that
above-mentioned company Kontinent and its supervisors were flexible and
creative in terms of choosing destination for the food products imported
from Germany. There was a mini scandal in the city of Leningrad back in
1991. Amid food rationing in the home city, the Leningrad-based
Kontinent diverted supplied to the city of Moscow instead of Leningrad,
the original destination. The key point of this episode is that people,
who practically handled transactions like these from the Russian side,
understood money as a balance sheet entry as the above mentioned 90
millions Deutsche marks were considered as \textit{credits} or a balance of
funds credited to the account of Russian side at the German bank. See
\citep[pp.~104-105,118-124]{gessen2014} and \citep{gessen2017}.

Meanwhile, the whole region of former Soviet Union was experiencing
hyperinflation during the first half of the 1990s. Dollarization first
of all of personal savings was sweeping Ukraine, Russia and all other
newly independent countries. After the disintegration of the Soviet
Union, over some time the newly independent countries adopted their own
money units of account, while Russia retained ruble. Ukraine changed
money of account from ruble to karbovanets. Incomes were collapsing and
in terms of real (price adjusted) contraction of annual volume of gross
domestic product (GDP) the entire region of the former Soviet Union was
in depression by mid 1990s. It was not even a recession, it was truly
depression as cumulative loss in annual GDP was in double-digit
territory. Human conditions worsened for majority of population in every
newly independent states. Both households and private businesses
effectively adopted US dollar as their operational money unit of
account.

As Ukraine and Russia, as well as other former members of Soviet Union,
now being separate states while still largely interconnected by trade
relationships, the economic developments of contraction persisted and
even accelerated. In Russia, president Yeltsin appointed reform-minded
economist Yegor Gaidar as active head of the government. Gaidar was
widely credited by implementing the price liberalization and launching
the process of privatization of the state-owned enterprises. Russia was
a cheer leader in economic reforms for the rest of former Soviet Union
members. Other nations watched what Russia's government was implementing
and tried to emulate albeit with some lengthy delay. This development is
vividly described by the former chief economic adviser to the then
president of Ukraine, Leonid Kuchma:

\begin{quote}
 We, especially in the first years of independence [of early 1990s],
 copied in good faith all that what was done by the [Russian] 'elder
 brother.' Moreover, our teachers were the same.~\citep[p.~232]{galchynskiy}
\end{quote}

Indeed, Russia was pioneer in price liberalization, mass privatization
program, positive real interest rate policy by the central bank,
\footnote{Implemented in order to incentivize savings in the national money
    unit of account, to discourage dollarization and fight
    hyperinflation.} stock market development and then government bond market. As authorities
turned to the policy of tight credit and high interest rates (trying to
make them positive in real, inflation-adjusted terms), the private
sector was free to pursue a variety of strategies for the sake of
obtaining profits. As commercial bank \acp{iou} (credits) were costly, the
non-bank businesses proliferated in practice of issuing commercial \acp{iou}
and settling mutual obligations in a quite similar way that previously
done by Bureau of Mutual Settlements that operated under the roof of
State Bank and Gosplan from 1930s to 1980s. These new businesses settled
commercial obligation within the borders of newly independent states and
across of their borders too. The most lucrative state-owned enterprises
with their produce in high demand abroad were snapped into private hands
via privatization process. There was pressure to privatize fast. It was
coming from the theory that former state-owned enterprises must be
brought into the market and competitive environment as quickly as
possible to get the economy out of the crisis. As it was discussed
above, the hypothesis of soft budget constrain was governing that
theory. Collapsing demand and hyperinflation were adding urgency to do
the thinking that privatization is quick fix. Lastly, theory reasoned
that once hard budget constrain was imposed then it would quicken the
recovery. Former state-owned enterprises were sold via auctions that
usually attracted few bidders and selling prices of these productive
assets were low but albeit impressive for the ordinary people amid
economic depression. That was general development with the private
sector in the newly independent countries of former Soviet Union.
However, it was Russia public and private sectors---policymakers and
bigwig business people that quickly earned the name of oligarchs---that
were in the lead and impressed their counterparts in the other former
members of the Soviet Union. The center of intellectual activity in
terms of policy-making and in terms of making big money quickly was in
Moscow. For example, Ukraine's policy-makers and business people
normally observed Russia's successes in the strides, respectively, to
the market economy and concentration of wealth with a bit of jealousy.

In the second half of 1990s the \textit{major} economic development was taking
place with respect to government finance. That process was called
shifting from the inflationary financing towards market-based financing.
The former referred to the practice of obtaining direct credit from the
central bank, the operational legacy of the State Bank, which operated
one balance sheet effectively for entire economy. The latter referred to
the practice adopted in the West, where government was selling bonds
denominated in the national money of account to the private
investors, resident and non-resident ones. As additional tool against
high-inflation environment, the laws were adopted throughout former
Soviet Union that made illegal direct lending of the central bank to
government as well as direct purchase of the government bonds via the
primary market.

Russia was first to launch the government bond market. At the begging
the government was able to sell short-term securities or zero-coupon
bonds of maturity of up to one year.\footnote{These were known as GKOs, which in Russian stands for government
    short-term bonds or obligations. The entire episode of the Russia's
    monetary history since transition to the market economy in early
    1990s earned a name as the GKO market. It lasted several years from
    1994 when the Russian government started issuing GKOs denomianted in
    rubles and through August 1998, when the government chose to declare
    default on them.} As interest rates were high at
double digits the government securities were selling at the market
rates. This brought foreign interest in the market that was offering
lucrative yields. Foreign bids into the GKOs also bid up the value of
the exchange rate of ruble versus the US dollar. The central bank as its
own policy mix was supporting a policy of exchange rate corridor, which
aided to the GKO investors' calculation of the expected returns as
future value of ruble to US dollar was expected to survive around the
current level (of level of entry into the GKO positions). At the
beginning the GKO market created an environment of economic progress:
(1) the government was said financing its deficits via non-inflationary
means, (2) exchange rate stabilized at last after a multi-year
devaluation period, (3) private investors, domestic and foreign, were
harvesting high-yield returns and this encouraged more position-making
into GKOs from domestic and foreign bidders (Russia newly created
private banks did that and foreign major investment banking names did
too). Russia's commercial banks made additional business by making
positions on promising delivering US dollar in the future to their
counter-parties.

Ukraine followed the government bond market launch with some delay. But
the major details were similar in terms of short-term maturity of the
bond denominated in the national money of account and high interest
rates at which these bond were sold. It also helped to stabilize the
exchange rate for some period of time as the central bank adhered to the
same policy of high positive real interest rates and pegged exchange
rate against the US dollar.

The exterior of the new means of government financing did hide the
underlying operations of the market. As explained in \citep{valchyshen_ru98},
using Russia's case as an example, the market size at some (and not very
much distant) point reached the level when the government was not able
to redeem the bonds due at a particular date from accumulated balances
at the central bank account, even if proceeds from the primary auction
of GKOs are taken into account. To support smooth functioning of the GKO
market, the central government and the central bank coordinated their
balance sheets in the following way: under an established and
long-lasting agreement the central bank provided the government with
short-term (initally intra-day) loan enough to redeem the bonds. Then,
the government repaid the central bank loans once enough balance was
accumulated. That practices worked for a while until the fall of 1997,
when Asian crisis hit shaking international confidence in riskiness of
emerging markets. At the time, Russia's politics was messy with Yeltsin
barely managing to survive re-elections against ever popular Communist
Party. The latter was collecting votes from the dissatisfed voters,
number of which was fairly big as market economy exposed extreme social
inequality between super rich few oligarchs, who benefited from the
hurried transition from the command economy to the market one, and the
vast majority of rest of the population. Yeltsin kept pressing with
liberal market reforms as economy was struggling and key exports were
not covering imports and Russian government was openly short of foreign
\acp{iou} (credits) it needed to pay past debts and support the exchange rate
convertibility. In late 1997, foreign investors turned quite cautious to
the GKO market. Some were withdrawing from the ruble positions and not
interested in new GKO positions. The exchange rate corridor was under
risk of collapsing a policy commitment. Yeltsin was begging foreign \acp{iou}
(credits) nearly in the same fashion as Gorbachev in his tenure. In
first half of 1998, the central bank reacted to the market uncertainty
with surprise interest rate hike of massive proportion by raising it
from 28\% to 150\% via several spreadout steps. But the hike to the 150\%
level from 50\% in late May of 1998 was considered as a sign of brewing
problem. That rate survived no more than 10 days and was lowered to 60\%.
But the course of the events was indicating that exchange rate policy of
corridor was about to break down. Exchange rate policy was a priority as
it was observed by the general public on a daily basis. Behind the
curtains, the central bank and the central government were at near war
because of the GKO market. It did not function as before anymore in
terms of attracting funds on the account of the government. Instead,
investors were expecting redemption and some were eyeing converting
ruble balances with resident banks into US dollar balance of
non-resident banks, i.e. withdrawing from the Russia's domestic bond
market. The internal loan from the central bank to the government was
quite big and not payable back by the government. What was previously an
inter-day loan turned into a longer term loan and the central bank top
officials worried that a balance sheet disclosure would reveal that the
law, which banned such a credit relationship, was broken. It was the
central bank to officials that pushed the government to declare default
on the ruble-denominated government bonds. On August 17, 1998, the
Russia government declared non-payment decision and invited investors
for restructuring of the debt. As a result, exchange rate policy changed
by re-pricing the ruble at much weaker rate to the US dollar. Inflation
spiked again and economic recession followed. Politically, that Russian
financial crisis of August 1998 paved the way to change in the
government, where liberal reformers were replaced by people of more
experience in security services such as prime-minister Primakov. Lastly,
president Yeltsin handed over the presidency powers to Putin, then
largely unknown security service person of younger generation. The key
economic and monetary take away of this political change is that both
Primakov and Putin preserved economic policy-making machinery brought by
Yeltsin.

In Ukraine, the bond market too experienced a swift departure of
investors and exchange rate peg was broken. Inflation spike and
recession followed. The government restructured its market debt.

\citep{shlapentokh2000}: ``Although anti-Western and anti-American feelings are on the increase
in the country, it has little to do with a concern for Russia's lost glory. The
Russia eventually will confront the problem is much more mundane. The reality is that economic reforms have not brought a better life for the majority, and the collapse of the ruble [in 1998 and earlier in the first half of the 1990s] made the situation worse. Yet this has not led to a nationalistic or radical heat as one could assume if his only source of information was articles in Russian newspapers. My observations in Moscow led me to believe that the average Russians remain quite passive, seeing nationalistic rhetoric as merely a ploy of this or that political force to use patriotic feelings as a vehicle to bring to power those who utter it. And power is coveted by all of these politicians not to improve the life of the masses or to make Russia great again but for another reason. The majority of simple folks with whom I conversed believed that the elite (both pro-Western and anti-Western) want power merely for their own personal enrichment."

\citep{shlapentokh2000}: ``I discussed the situation in the Balkans with her and recalled my
Communist friend's report about the patriotic gathering in front of the U.S. embassy when the American flag was burned. Her reply was contemptuous.
"Big deal to burn the flag. I would like to see
one person who would burn one American
dollar." Her point was clear. Patriotic rhetoric
is cheap, but no one is willing to sacrifice his
interests, even the smallest iota, for the
county." \citeauthor{shlapentokh2000} description of Russia's society featured such terms as ``a craving for Western dollars" and realization that ``a Russian Nazi could come to power."

\subsubsection{From Early 2000s and Through the \acs{gfc}}

Late 1990s and early 2000s turned out to be a period of turnaround as
commodities markets for crude oil, iron ore, grains, etc. appreciated in
price. The entire region of the former Soviet Union experienced
export-led growth. Russia's economy benefited from foreign trade so much
that its government paid down its foreign debt and enjoyed trade
surpluses of massive proportions. Its leadership must have consciously
adhered to this policy as it did not want to appear again in situation,
in which both Andropov and Gorbachev founds themselves, of begging
constantly for foreign \acp{iou} (credits) to be able just to redeem own debt
was falling due or buy essential, everyday food.

In this period, Russia's policy-making remained broadly aligned with
intellectual underpinnings of liberal market economy. While few former
reformers active in government during 1990s survived into the corridors
of power in Moscow during 2000s. The key person, who spearheaded liberal
reforms along the lines initiated by Gaidar, was Alexey Kudrin. He has
been credited with pushing through reform such as land market (turning
land into tradable commodity), monetizing social benefits (it turned
highly unpopular and harmed popularity of the Kremlin leadership, mainly
Putin), budget rule (deemed the most successful policy of all), pension
age increase (again highly unpopular with general public and cost Putin
popularity).

Meanwhile, its private sector was enjoying lucrative market environment
too. Private owners of the major industrial now privatized enterprises
were welcomed in the major financial centers as respected business
people to do new initial public offerings (IPOs) of shares and US dollar
borrowings via the Eurobond market or via syndicated loan market. As it
was noted above, private sector was adopting the US dollar as its
operational money of account. Even middle- and low-income
households followed this lead as domestic purchases of personal cars and
personal housing was priced and dealt with in the US dollars, including
cash currency.

This market practice was followed nearly universally over the other
newly independent countries, where private sector operations were
thought of in the US dollar terms primarily. Russia's private and
state-run businesses were expanding into the countries of former Soviet
Union, including Ukraine. They were expanding their balance sheets
relying on the US dollar as money of account. For example,
state-run financial institution of Russia called VEB as well as
privately run commercial bank of Russia NRB acquired commercial banks in
Ukraine from the private sellers. Over time, operations of these two
commercial banks revealed that they were expanding their balance sheets
predominantly in the foreign money units of account. Primarily it was US
dollar, not Russian ruble. While they pretty neglected using Ukraine's
national money of account.

In the run-up to September 2008, when \acf{gfc} culminated with Lehman Brothers bankruptcy, both Russian and Ukrainian private sectors were expanding their US dollar balance sheets quite
liberally. The GFC produced deep devaluations and deep recessions.

Thanks to oil and natural gas exports, Russia had sizable official
foreign-exchange reserves. It spent them in 2008-09 to support its
private sector, where leading industrial companies appeared
highly-indebted to private financial institutions abroad and used their
assets as collateral for borrowings. That experience paved the way
towards critical re-consideration of the private monetary relations.
Disciplining of the domestic business to use ruble as their operational
money of account had been initiated in the aftermath of \ac{gfc}.

Meanwhile, Ukraine had never had enough official foreign-exchange
reserves. Since 2005, its economy moved from having foreign trade
surpluses to having chronically foreign trade deficits.

\subsubsection{The First Phase of the Russia's War in 2014}

Just three years after \ac{gfc} Russia's economy faced two slow-burning
crises:

-   social-political crisis of low approval rating of the Kremlin
    leadership and personally Putin; and

-   economic crisis of slow growth, which was subject of internal
    discussion of Kremlin top economic advisers.\footnote{Namely, by Andrei Belousov, who now holds position of first deputy prime minister of Russian government: \url{http://government.ru/en/gov/persons/123/events/}}

The first crisis was a public reaction over the results of general
elections into the Russian parliament in December 2011. These were
deemed as fraudulent. The ruling party United Russia, the party
representing two personalities of power Putin and Medvedev, won those
elections. But the environment in the public was negative to the ruling
elite. That was the year when Putin served his last year as prime
minister, while president Medvedev was an outgoing official who was not
going to challenge Putin in the presidential elections in March 2012.
So, just three months before presidential elections Russia social life
saw massive protests against authorities. Public dissatisfaction was
fueled too but recent announcement made by Putin in the second half of
2011 that he would take part in March 2012 presidential elections,
seeking re-election. The December 2011 protests calmed down as police
arrested the protest leaders swiftly. Authorities showed showed they
were not going to tolerate public protests. In March 2012, Putin won
presidency. It is noteworthy that the December 2011 parliamentary
elections and the March 2012 presidential elections showed that
Communist Party of Russia has remained second largest political group
after the ruling party United Russia. Protests against the ruling party
took place over 2012 and 2013 but were losing in terms of visibility.
However, surveys such as by Levada, one of the most trustworthy opinion
pollsters in Russia, were recording a wider dissatisfaction over the
ruling party and Putin personally. In 2013, Levada's Putin approval
index hit the lowest point since early 2000s, which was an extraordinary
fact. Levada was forced to revise its survey and drop the index
altogether in 2013, while its components were still produced on monthly
basis and published.\footnote{The Puting approval index was calculated and reported by Levada
    on a monthly basis from two variables: (1) share of the surveyed
    people who approve Putin's policy and (2) share of surveyed people
    who disapprove Putin's policy. The difference between (1) and (2)
    resulted in net approval index.}

Since his re-election as president in March 2012 and following
inauguration in May of same year Putin declared a multi-year policy of
grand ambition that addresses Russian general public nostalgia about the
dissolved Soviet Union. His program aimed to create Customs Union with
the countries of former Soviet Union. Medvedev, a usual substitute for
Putin and who speak what Putin does not not want to say himself as of
yet, was talking about this new policy as a better version of the
European Union. It meant that Russia's government was thinking of
monetary union, too, where ruble was considered as the shared money unit
of account.

At that same year, Ukraine and a number of other countries were in talks
with European Union officials about signing an association agreement
with the EU. In Ukraine, that move was considered as yet one step to
become more closer with European institutions and away from the Russia
dominated ones.

Meanwhile, economic growth in 2012 and early 2013 was slow in Russia
alone. Moreover, during first half of 2013 real GDP was negative in real
seasonally adjusted quarter-on-quarter teams every quarter: in first and
second quarters of 2013. Technically this meant recession, while in
Russia both financial media (domestic and foreign) and professional
economists kept silence on the phenomenon. A bit later, the government's
statistical office revised the data and published new series of
quarterly national accounts, according to which there was no more two
consecutive quarters of real GDP decline \citep[p.~30]{valchyshen_2014}. Also,
by mid 2013 there were two economic factors were observed: (1) Russia
ruble was sizably dislocated up by the metric of real effective exchange
rate, suggesting lost competitiveness according to standard economic
theory \citep[pp.~9-10]{valchyshen_2013}, (2) top economic policy-makers of
Russia, first of all Belousov and then economic minister Ulyukaev,\footnote{He was lately sentenced to several-year term in jail for high-profile corruption scandal.}
were discussing internal, not external, issue that drags Russian economy
and in particular they flagged lost control over costs as the core
issue, which is an indirect reference to \citep[pp.~9-10]{valchyshen_2015}.

In this environment, Russia authorities pressured Ukraine authorities to
give up the EU association agreement and join the Russia-led project of
Customs Union. Ukraine's president Yanukovych refusal to sign EU
association agreement, what was concession to Putin, caused massive and
extremely determined public protests\footnote{Disclosure: author of this paper took part himself in Euromaidan as a supporter of protests during the winter of 2013-14 and article about this experience appeared in \textit{The Westchester Guardian} on February 20, 2014 (link: bit.ly/3q4oE5D).} that ended up with numerous
losses of life on the side of protesters, while president Yanikovych
gave up his powers and fled the country to Russia.

Russia itself not only gave refugee to fugitive Yanukovych, it invaded
into Ukraine with army troops without insignia in February 2014. They
crossed the Ukrainian borders into territories of Crimea and Eastern
Donbas, effectively occupying them since then till now. Ukraine's army
withdrew from Crimea while aimed to fight back in Donbas. The Western
powers, the US and the EU, imposed sanctions that were targeting
financial accounts of Russian officials and restricting access of the
Russian borrowers to international financial system.

\begin{quote}
With its currency near an all-time low, its stock market down twenty percent this year and a marked rise in interest rates, Russia has already started to bear the economic costs of its unlawful effort to undermine Ukraine's security, stability, and sovereignty.~\citep[quoting David S. Cohen, Under Secretary for Terrorism and Financial Intelligence]{ustres2014}
\end{quote}

The above-mentioned quote borrowed from the statement from the US
Treasury dated 2014 reveals common assumption of the Russia economic
conditions on the eve of 2014. This assumption is that Russia's economy
right before the 2014 sanctions was in a kind of normal point, i.e. its
economic parameters were ``right". Indeed, Russian ruble lost in value
versus the US dollar during the later part of 2014 and in 2015, too. Its
stock market dropped and interest rates both on domestic instruments in
rubles and in foreign one in US dollars jumped in response. What if, as
it was explained above, ruble was \textit{not} in ``right" point? What if the
stock market is not a concern for the majority of general public that
votes? What if interest rates on the borrowings in the foreign money of
account is not a point of concern for the Russia authorities? What if by
free-floating the exchange rate of ruble, Russia authorities move away
from the nearly chronic dependence on one commodity exports? It is not
known exactaly whether Russia government were making fun of sanctions or
not, but already in early 2015 Russia finance minister Siluanov was
widely talking the following way:

\begin{quote}
 ``Dutch disease is over," Siluanov said in an interview, referring to
 the fallout from a commodity boom that pushes up exchange rates and
 stalls competitiveness.~\citep{bbg2015,tass2015}
\end{quote}

At this moment, we must recall that 2014-15 was characterized by two
swift developments in the international financial system: (1) steady and
sizable appreciation of the US dollar versus major money units of
account as measured by the DXY index (see Appendix), and (2) similarly
steady and sizable drop in the crude oil price. These developments were
rather problematic for an economy that relies on crude oil exports as
revenues and dollar-based borrowings. This is exactly where Russia used
to be and GFC revealed its vulnerabilities. The following statement by
Russia's deputy finance minister Alexey Moiseev,\footnote{Short bio of Alexey Moiseev is here: \url{https://www.fsb.org/profile/alexey-moiseev/}} made eight years
after the GFC, has been symptomatic and telling about the long-lasting
trends and changes taking place in the Russia's economy:

\begin{quote}
 I do not see a stagnation of the Russia's financial market. I observe
 instead that financial market has been actively developing. If we
 consider the financial market as a whole, we see there are several
 processes that are taking place. These processes realize our
 long-standing dream and they lead to what where we want to arrive
 eventually. In particular, this is sort of import substitution of the
 foreign [financial] markets. Here, I disagree with the statement that
 Eurobond borrowing is a good instrument for raising funds [by a
 Russia's business entity]. You know, the financial crisis of 20014-15
 showed that even to the companies with large share of export revenues
 Eurobonds were very risky instruments for raising funds. There were
 periods when hedging of foreign exchange risks was cheap.
 Nevertheless, we see that even, when they were cheap, they worked out
 at the end not the way the companies expected. I can share information
 that largest companies that hedged FX risks have suffered losses.
 \dots I conclude that our shared task must be putting to rest attempts
 of returning to the international financial markets with new
 borrowings. We have lastly a chance to substitute foreign [financial]
 markets, to which we accustomed ourselves and where it was
 comfortable, with domestic [financial] markets. Today, our financial
 markets is very comfortable. My meeting with foreign investors reveal
 that foreigners do receive the same level of high-quality service in
 terms of clearing, settlement, payments, etc as in major financial
 centers such as London and New York.~\citep{moiseev}
\end{quote}

The key outcomes from the 2014 invasion of Ukraine for Russia are the
following. First, its economy moved to the more flexible exchange rate
policy than it used to be. There is no yet a proper definition of
free-floating in economics. Ruble before 2014 was considered largely
free-floating money unit. However, next few years showed that was
another stage of free floating, which is pretty different from the past
one in terms of exchange rate upper and lower boundaries. Second,
popularity of Kremlin and Putin personally skyrocketed in 2014 (see
Figure 5 in Appendix). Effectively, public discontent with authorities
largely evaporated.

It, however, went south by 2018 again. Authorities pushed through the
pension reform, where retirement age was raised. It was adopted during
the meeting of the government headed by Medvedev, who served as prime
minister after serving as president for a while. The meeting was held
right on the eve of the opening of the World Cup held in Russia.
Nevertheless, the government decision caused uproar among the general
public. Kremlin and Putin lost popularity to a significant degree, see Figures~\ref{fig:levada} and~\ref{fig:reg_levada1} respectively on p.~\pageref{fig:levada} and p.~\pageref{fig:reg_levada1}.

\subsubsection{The Second Phase of the Russia's War, 2022}

Russia invaded Ukraine on February 24th, 2022 with a bigger military
force than in February 2014. The Western powers---the US and the
EU---responded with wave of sanctions with a freeze of the Russia's
official foreign-exchange reserves on the correspondent accounts with
banks in their jurisdictions. In the wake of the invasion, there was a
usual spike recorded in public approval of Kremlin and Putin, in
particular (Figure 5). However, this paper recognizes that correlation
does not mean causation. Why than such striking dependency exist?

\cite{dibb2022} put it this way: ``[T]here is obsession of Ukraine [in the
Putin's Russia]. ... Putin like some other Russians probably many has a
peculiar attitude toward Ukraine."\footnote{On the video, watch since 19:36 timing.}

Russia propaganda in the Russian language and later in foreign language
media has been traditionally portraying in derogatory terms the
sovereignty of Ukraine. There has been a whole tradition in the Soviet
Union to antagonize, denegrade, mock Ukrainian leaders that fought for
country's sovereignty in 1920s. It gained new talking points, when
Ukraine factually gained sovereignty in 1991.

Under Putin's political regime it has been progressively doubling down
on these terms. Internationally, there are two big political events that
made modern Ukraine known: Orange Revolution of 2004 and Euromaidan
Revolution of 2014. In Russia these events are called colored
revolutions and talked about in derogatory terms by the Putin's regime.
In fact, the Ukrainian people of different nationality and speaking
normally Ukrainian and Russian, who gathered for those two events, knew
from the very beginning that their protest was against the Putin's
regime spread into Ukraine. Since 2012, when Putin got re-elected as
president of Russia the tide of anti-Ukrainian stance and the vocabulary
of derogatory terms were on the rise and reaching new highs. It has been
in a grand display on the eve of the war invasions by Russia into
Ukraine since 2014 and 2022. In Russia, domestically the atmosphere seen
in the public discourse is properly characterized by the following
terms:

\begin{quote}
[T]he paranoid style of \dots politics, in which the nation is
depicted as besieged and threatened by enemies from within and
without."~\citep[pp.~82-83]{dharvey}
\end{quote}

With all domestic policies that has been implemented in Russia in the
1990s and since 2000s, which have been unpopular but still implemented
for the sake of macro-financial stability, modern day Russia's
government do qualify for David Harvey's ``a growing column of neoliberal
state aparathuses worldwide"~\citep[p.~29]{dharvey}. Ukraine's governments
under different presidents, too, qualify. Even though, Russia explains
that it wages war against the West at large, not Ukraine in particular,
its intellectual underpinning for the organization of economic
relationships is derived from the very modern-day West. Let us recall
that second most popular political force in modern-day Russia is
Communist Party of Russia. Putin's regime speaks of some possibility of
restoration of the Soviet Union, which is popular among the voters who
follow the Communist Part, but he never speak of abolition of the
private property institution on the means of production of large-scale
type. Instead, domestically Russia's government retains the language
that carefully preserves the business friendly environment. Business
institutions are developing along the best practices and guidelines
borrowed from the West. Financialization and organization of the economy
upon market-based finance has been there for a lengthy while already and
never curtails conceptually. Putin's regime has been explicitly
underlining that it is open to business as usual.

Monetarily, Russia's domestic economy experience the following
development on the back of the war on Ukraine lasting since 2014. Yet 15
years ago, about the time of GFC, general public as a whole held 90-120
billion in US dollar equivalent as commercial bank \acp{iou} denominated in
the foreign money units of account. Normally, \acf{usd}, euro and pound
sterling were the top ones. As of November 1st, 2022 the size of the
household deposits with local banks in foreign money units dropped by
about 40\% to 57 billion in US dollar equivalent, while yet at the
beginning of 2022 it amounted to 94 billion \citep{zadornov}. This speaks to
the above-mentioned development of disciplining domestic businesses and
private individuals to adopt national money of account as
operational instead of the foreign one.

\citep[emphasis added]{tulin2022}: ``The role of these currencies [\ac{usd}, \ac{eur} and \ac{gbp}] as payment instrument for the Russia's economy will be declining increasingly and, it must be underlined, naturally, i.e. even without regulatory nudging from the side of the Russia's authorities. The barriers for the usage of these currencies in international payments are due to foreign decisions: as a result of both the official sanctions and decisions by private counter parties of the Russian banks. \dots [Russia's] banks well before their customers had experienced how traditional world reserve currencies had become really dengerous, toxic assets for the Russia's economy. They are trying to use every possibility to deny accepting foreign currencies, which their customers such as private individuals and businesses are brining to them. This is because they cannot perform in due respect upon their liabilities such as foreign-currency current [checking] accounts and time deposits. There are two reasons for that. First, it is shortage of cash foreign currency: we were cut off from channels of inter-bank, wholesale deliveries of cash banknotes. There is some inflow of foreign currency cash into banks, but it is not a large one -- just a shallow creek instead of wide rivers of the near past. Second, transfers abroad and usage of bank cards for purchases abroad are a problem. So, there are [Russia's bank] foreign-currency liabilities to the citizens [of Russia], but banks cannot perform upon them in the form, which was determined priorly by demand and time-deposit conracts, for the reasons that not depend on these banks. Due to this, there are now not popular but logical steps on squeezing out foreign-currency customers -- such as foreign-currency [time] deposits are being removed from the banks' product lines offered to customers and charging fees for maintaining foreign-currency [current/checking] accounts. Majority of customers have been not happy about these practices by banks. In fact, by squeezing the customers from their foreign-currency savings via such economic means, the banks are doing \textit{good deed}, what is well required, [this term in Russian has religious connotation] not so much for themselves but for their very customers. Since the banks appeared on the forefront of sanctions offencive, they realized well before all the rest how dengerous is the usage of toxic foreign currencies and they warn their customer of this danger."

\subsection{Conclusion}

Russia has been carrying its war on Ukraine as matter of fact for
several years now. It has been sanctioned with numerous rounds for its
open destruction of the Ukraine's society and underlying economy. For
the sanctions of 2022 as well as for the sanctions of 2014 type the
following consideration apply:

\begin{quote}
Sender governments [that impose sanctions] face numerous barriers to
imposing economic sanctions that are sufficiently costly to force a
target government to make concessions. Senders' first major challenge
is \textit{knowing} how costly the sanctions against a target state need to
be in order to make the target concede.~\citep[p.~169, emphasis added]{sanctions2021}
\end{quote}

Hence, the sanctions' most simple formula $\text{Costs}>\text{Benefits}$ worked for
Russia so far in the following way $\text{Costs}<\text{Benefits}$.

There is underlying theory behind sanctions, where conceptually scarcity
of resources, including money and finance, is the first stepping stone
of entire theoretical corpus. Economic logic behind sanctions forges
expectations that Russia's government will experience government budget
deficit deterioration, leading to ``money printing" and hence high
inflation. There is kind of thinking that things that seemingly ruined
the Soviet Union economy must be in play with Russia's this time.

\ac{mmt} rebukes this stance by arguing ``finance is not scarce resource"
\citep[p.~139]{wray2020}. Its conceptualization of money hierarchy as well as
the sectoral balances and endogenous money concepts (not discussed in
due detail in this paper) gives a rational for proper understanding of
the Soviet Union economy demise in the 1980s and tendency to accelerated
inflation of the early 1990s.

Proper knowing of the motives of Russia's war on Ukraine includes
realization of Russia own economic and social crises,\footnote{From the speech by Andrey Belousov on April 22, 2013: ``Today's situation of economic slowdown, unfortunately, at a great extent is not because of the global economy, it is due to \textit{internal} factors"~\citep[emphasis added]{kremlin2013}.} from which Russian population attention diverted by the Putin's regime onto manufactured crisis/war in Ukraine.
