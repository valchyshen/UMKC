\newpage

\section{\MakeUppercase{Economics of Sanctions: The Case of Russia's Invasion of Ukraine}}\label{sec:sanctions}

\subsection*{Abstract}
\subsection{Introduction}
\subsection{Review of sanctions literature}
\subsection{Logic \& underlying economic theory of sanctions}

\subsection{Lasting Russia crisis instead of "Ukraine crisis"}

For example, \citep{prates2023} relates to the subject matter as "Ukraine/Ukrainian war/crisis".

\begin{quote}
A macroeconomic phenomenon that has gone down in economic history as the Great Depression\index{Great Depression!of 1873-1896} of 1873-1896 contributed to dissatisfaction with free trade and the gold standard. The name may be misleading, because that depression was not an economic collapse but a gradual and continual decline in world prices. From 1873 until 1896 prices dropped by 22 percent in the United Kingdom, 32 percent in the United States, more elsewhere. This depression of prices that gave the episode its name caused serious problems. Prices and earnings declined, but debt burdens remained constant. Expectations of further price declines caused uncertainty and pessimism. \citep[p.~8]{freiden2006}
\end{quote}

Writing in his book published in 2006, \citeauthor{gardiner2006} shared his view of the Russia's economy as ``a success stroy" \citeyearpar[p.~233]{gardiner2006}.\index{Russia}\index{Russia!Seen as normal case} In particular, the rationale was in the following passage:

\begin{quote}
A country which appeared to be indire financial condition in the 1990s is now becoming a sucess story. That is Russia. Its national currencyb looked worthless, the tax take was tiny, and the main currency of the people was \acl{usd} bills, more it was said than circulated in the United States itself. The monetarist ghouls were hovering around the body, eager to thrust upon it dollar loans to tide it over. A few economists in the West urged Russia to have nothing to do with these remedies. The prescription of their lead speakers was, firstly, a moratorium on foreign debt repayment, secondly, no loans from the \acl{imf} or any foreign bank or government, and thirdly to monetise the inter-company debts of Russia's large utilities and other big businesses, thus making them the basis of its monetary system. The latter means doing what we have shown was happenning in the 1830s in Britain. The first would gain time for the huge export industries of Russia to get going on a sounder basis. It was obvious that Russia would soon have a large trade surplus from its primary producers, and that has proved true; it is now one of the ten largest surpluses in the world. It needs a debt moratorium no longer. \par
The adjustment from inept communist ideology to effective creditary structures has slowly been taken place, but one can guess that it was not easy in a country where the idea of charging depreciation on capital equipment did not exist, and where even double-entry bookkeeping had been regarded as a false capitalistic doctrine. Because industry was not charged for its capital assets, the Russians tended to throw capital at all production problems, whereas the Japanese threw management expertise at them. To make matters even worse for the Russians, after 1989 they were assaulted by an invasion of western free-market theorists who saw the opportunity to create, in one great leap forward, a free market economy such as has never existed in the whole history of the world. \citeyearpar[pp.233-234]{gardiner2006}
\end{quote}

\subsubsection*{Evolution in 1980-90's: Transition from command to market \newline \& Default of August '98}
    
\subsubsection*{Evolution in 2000-10's: From boom-bust modus operandi to \newline the \textit{sanctioned} economic model}

\subsubsection*{Evolution in 2020's: Logic of the \textit{sanctioned} economic model}  
    
\subsection{Conclusion: The need for change in the underlying economic theory}

\begin{figure}[!ht]\centering
\includegraphics[width=1.0\textwidth]{\plotsfolder/levada_indexes}
\caption[Monthly public opinion surveys by Levada, a Russia-based polling agency]%
{Monthly public opinion surveys by Levada, a Russia-based polling agency \par \small Longer description}
\end{figure}

\citeauthor{budra2022} explains the post-\ac{gfc} state of the policital economy juncture in Russia. Its state propaganda presents the Russia's model of governance as statism that is an  opposite to the Western neoliberal order. While the latter is based upon individualism and egoistic materialism, the former has collectivism as a distinguishing feature.  However, this rethoric on the side of Russia's propaganda aired domestically and abroad "is originally combined with neoliberal practices in the Kremlin's socio-economic policies" \citep[p.~46]{budra2022}. Those practices include, among many others, "the active introduction of the principles of economic austerity and the model of public-private partnerships" (ibid).

\citeauthor{hofstadter2020} on the paranoid style of politics\index{paranoid style of politics} as an international phenomenon observed throughout history.\index{paranoid style of politics!Hofstadter, Charles}

\citep[p.~505-506,512,]{hofstadter2020}: "\dots an arena for uncommonly angry minds. \dots the reality of the style and to illustrate its frequent historical recurrence. \dots  the central preconception of the paranoid style -- the existence of a vast, insidious, preternaturally 
effective international conspiratorial network designed to perpetrate acts of the most fiendish character. \dots  I would like to emphasize agam that the paranoid style is an international phenomenon. Nor is it confined to modern times."

\begin{quote}
Let us now abstract the basic elements in the paranoid style. 
The central image is that of a vast and sinister conspiracy, a gigantic and yet subtle machinery of.influence set in motion to undermine and destroy a way of life. One may object that there \textit{are} conspiratorial acts in history, and there is nothing paranoid about taking note of them. This is true. All political behavior reqmres strategy, many strategic acts depend for their effect upon a period of secrecy, and anything that is secret may be described, often with but little exaggeration, as conspiratorial. The disanguishing thing about the paranoid style is not that its exponents see conspiracies or plots here and there in history, but that they regard a "vast" or "gigantic" conspiracy as \textit{the motive fiorce} in historical evens. History \textit{is} conspiracy, set in motion by demonic forces of almost transcendent power, and what is felt to be needed to defeat it is not the usual methods of political give-and-take, but an all-out crusade. The paranoid spokesman sees the fate of this conspiracy in apocalyptic terms -- he traffics in birth and death of whole worlds, whole political orders, whole systems of human values. He is always manning the barricades of civilization. \citep[p.~525, emphasys original]{hofstadter2020}
\end{quote}

Among mainstream and heterodox economists the view is held that because of sanctions Russia has been diversifying its holdings of official foreign reserve reserves and switching to the non-US dollar international payments. See respectively \citep{mcdowell} and \citep{bracarense2022,bracarense2024}.

\citep[p.~3, emphasis added]{neumann1957}: ""Since political power is control of other men, political power (as contrasted with power over external nature) is \textit{always} a two-sided relationship." \citeauthor{neumann1957} supports this statement with a reference to \citep{lasswell1948}, who describes this relationship as "cue-giving and cue-taking"

On the inequality in Russia see \citep{remington2018}.

The mid-2000s debate "Russia as a normal country" -- see \citep{shleifer2005}, \citep{zhurav2007}. In the second half of 2010s, the conclusion was "[Russia was ]ending the twenty-five-year project of normalization", see \citep{medvedev2019}

On the mathematization of economics and Russian mathematicians' contribution see \citep{boldyrev2024}.

On the deep dollarization in the emerging market economies see \citep{camb2023}.

Mainstream economics take on the modern-day topic of sanctions, see \citep{NBERw31852}.

"Also Russia, has been trying to 
de-dollarize in the last couple of years. However, as we will discuss below they did not 
manage to lower the deposit dollarization ratio on a sustained basis." \citep{bis2007} While in other mainstream literature on dollarization during 1990s and early 2000s Russia frequent in the researchers' dataset describing this phenomena, see \citep{NBERw10015}.

\begin{quote}
\dots despite its immense destructive power, the war's 
impact on European productive capacities was limited. According to the 
United Nations, by 1947, a mere two years after the end of hos til i ties, 
industrial production in Europe had returned to its 1938 level, and by 1948 
it had surpassed the pre war level by about 13 per cent (exclud ing Germany). 
(If Germany is included, by 1947, Europe had reached 83 per cent and, by 
1948, 96 per cent of the 1938 level; see United Nations 1949: 137). \cite[p.~41]{alacevich2023}
\end{quote}

Standard view of sanctions regularly invokes the pipes and plumbers analogy as well as metaphors of motion while describing the global monetary system, see \citep{hess2023,hess2024}. Hence, \citeauthor{hess2024} describes:

\begin{quote}
The plumbing of these financial networks is therefore central to the ability to implement, oversee, and, enforce sanctions. But it's a complicated system, with pipes consisting of both public and private actors. Oversight and co-ordination are therefore necessary to maximise sanctions effectiveness, as well as ensuring that they don't impact financial markets more widely. \par
Functioning as intended, these networks would have been rerouted in a way that prevented the Russian state, its "partner-in-crime" Belarus, and Kremlin-connected actors from receiving new \textit{inflows} of funds, while keeping the pipes clear for the \textit{outflows} of foreign currency reserves. Coupled with an effective policy aimed at reducing oil and gas revenues, this might, eventually, deplete the funding of the Russian war machine. \citep[emphasis original]{hess2024}
\end{quote}

\citeauthor{walker2020} describes Atlas Network and its orgamizationl powers in terms of impacting the climate of opinion, see \citep[pp.~173-179]{walker2020}. With prime interest focused at ecological economics, the author uncovers the Atlas Network activities in the region of Oceania, where major economies are of Australia and New Zealand among other smaller nations. It has been maintaining such a climate of public opinion that, in effect, is sheilding fossil-fuel corporations from the social activism and public policies aimed at figting against global heating. Note the use of the stronger word "heating" instead of usual "warming". Then, the author observes in passing that "the influence of the Atlas Network is limited in nation-states with already existing authoritarian governments and extensive fossil-fuel resources (such as China and Russia)", see \citep[p.~178]{walker2020}. This reference makes that case that China as a nation-state governed by an authoritarian government and Russia as a nation-state, where fossil-fuels are in abaundence. What \citeauthor{walker2020}'s work is missing is historical evolution of the Atlas Network activities in Russia. The latter has vast network of domestic security services, which under Putin's leadership of Russia since very late of 1990s must have been paying close attention of the foreign influence on domestic climate of opinion. In his book, on pp.176-177, \citeauthor{walker2020} mentioned the artcile \citep{fang2017}. Its title of "Sphere of Influence" has very familiar connotattion to any person, who knows Russian language and have lived in the Russian language information space for some period of time. This is because the domestic opinion in Russia has been shaped quite aggressively bu the domestic secuities aparathus and the term 'sphere(s) of influence' has been one of the most commonly-used in the open and explicit way. In the book co-authored by Plehwe there is a map of continents and countries with dots showing how many thank tanks associated with the Atlas Network are operating there. The map's area that is territory of Russia's nation-state appears completely empty. It creates an impression similar to the above-mentioned statement made by \citeauthor{walker2020} that influence of the Atlas Network is next to none in Russia. It is an erronous impression.

There has been a visible line of thinking among the Western academia of equating all people of different nationalities who lived within the former SOviet Union via a collective term Russians. For example, \citeauthor{buck-morss1995}, per the quote below, speaks explicitly of the 300 million Russians as of early 1990s, while at the time the Soviet Union as an state was gone and 15 sovereign nation-states were recognized by international community of nation-states. Russia as a new sovereign state had about 150 million of population and even there not all people are ethnically Russian. 

\begin{quote}
In Moscow in 1993 the plan for economic transformation to capitalist markets was described by officials and the press in a representationally impoverished form as,
 simply, "the big bang" (English original). This mystical, invisible, sonar
 boom, imported by economists from Harvard, was supposed to provide
 for three hundred million Russian people some kind of cosmic rebirth
 out of the ashes of seventy years of Soviet rule. Heralded as the beginning
 of the new era, it seemed to the average citizen, on the contrary, to lead
 society ever deeper into a black hole. With no new vision of social life,
 with no way of refiguring their identity, Russians have responded by re
 treating into an equally mystical but culturally familiar collective identity
 of ethnic unity, one that finds a frightening voice in the political rhetoric
 of Vladimir Zhirinovsky. \citep[p.~466]{buck-morss1995}
\end{quote}

\citep{wolff1999} provides an account of the late 19th and early 20th century politics of the Russian Empire in the Far East, in particular within the Manchuria privince of China. The author refers to that policy as "liberal", which was unusual practice to the established way of doing things in the vast Russian Empire. Russia's Finance Minister Witter defended what would be now called a soft power approach towards Manchuria, which the Russian Empire was eyeing to absorb from then weak China. Prior to his lead of the Finance Ministry, Witte was at the top of the Russian railroad development and management. Yet the Witte's Finance Ministry in 1894 issued a budget report, where the Russian railroad network was refered to as "a very powerful weapon \dots for the direction of the economic development of the country" \citep[p.~5]{wolff1999}. This was the times, when the governemnt of the Russian Empire included War Minister along other Ministers like Witte. The scolars of the 19th century imperialism in East Asia disagree whether Witte's liberal policy was as imperialistic by its innet as by those within the Russian Empire goverment, "who favored less subtle means" of Manchuria. This historical case invokes similarities of the sovereign nation-state Russia policy towards Ukraine since 1990s. The means of execution of imperialist policy has changed to some extent. Russian liberal thinkers and government officials, for example, Boris Nemtzov expressed the view in the 1990s, while arguing to the domestic hardliners, that via privatization Russia was able to return Crimea albeit without a military intervention.

\citep{newsweek2024}: "A number of large banks in China have stopped accepting \textit{payments} from sanctioned Russian financial institutions, and banks in Armenia and Kyrgyzstan are no longer accepting cards that use the Russian Mir payment system, Moscow's alternative to Visa and Mastercard, after they suspended operations in the country over the war in Ukraine. India, once a top purchaser of Russian oil, is reported to have stopped paying for Russian premium crude oil. Meanwhile, Russian oil firms are facing delays of up to several months to be paid for crude and fuel as banks in China, Turkey and the United Arab Emirates (UAE) fear retaliation from the United States, Reuters reported on March 27.". This commentary shows how basic is the concept of payment for practical undertakings of different kinds.

On the topic of weaponization of the US dollar see \citep{ali2022,ali2023}.

On the US conservatives see \citep[see Chapter 3][pp.~83-124]{fang2013}: "unlike the New Deal era, when a popular progressive president [F. D. Roosevelt] outflanked the small clique of elite businessmen trying to obstruct his reform, the Koch brothers have triumphed over Obama in many significant ways." Then the following detailing of the Koch brothers' origins as a private business conglomerate, which was established by Fred Koch (1900-1967), father of the above-mentioned Koch brothers. Two out of his four sons---David (born 1935) and Charles (1940-2019)---followed the father's footsteps and expanded further both the business and activisim in conservative circles. The following observation by \citeauthor{fang2013} is worth mentioning:

\begin{quote}
After graduating from MIT, Fred invented a more efficient process for converting crude oil into gasoline. His innovation quickly earned him powerful enemies in eastern oil companies, which harassed him with a barrage of patent lawsuits. Seeking greater business opportunities elsewhere, in the 1920s and 1930s, Fred embarked for the Soviet Union, where he contracted to help erect fifteen oil stills in Grozny, Tuape, Batoum, Baku, and Yaroslav. There, he said, he witnessed Stalin's purges and learned firsthand about the communist plot to take over America. \dots However, it should be noted that Fred traveld throughout the Soviet empire doing business with a large cast of characters, and for two decades never saw any problem with accepting their money. His sudden hatred for his business partners, which did not become public until the 1950s, was never fully explained.
\citep[pp.~94-95]{fang2013}
\end{quote}

Another account on the activism of the US conservatives, see \cite{mayer2020}, provides a little b it of additioonal details on Fred Koch's business career. It confirms the case of a legal battle with America's major oil companies of that time, which are named as eastern oil companies by \citeauthor{fang2013}. At the same time, it elaborates a bit further on the period of 1920s and 1930s: 

\begin{quote}
In 1930, his company, then called Winkler-Koch, began training Russian engineers and helping Stalin's regime set up fifteen modern refineries under the first of Stalin's five-year plans. The program was a success, forming the backbone of the future Russian petroleum industry. The oil trade brought crucial hard currency into the Soviet Union, enabling it to modernize other industries. Koch was reportedly paid \$500,000, a princely sum during America's Great Depression. But by 1932, Soviet officials decided it would be more advantegous to copy the technology and build future refinaries themselves. Fred Koch continued to provide technical assistance to the Soviets as they constructed one hundred plants, according to one report, but advisory work was less profitable. \dots After leaving the USSR [in 1932] \dots [d]uring the 1930s, Fred Koch traveled frequently to Germany on oil business. Archival records document that in 1934 Winkler-Koch \dots provided the engineering plans and began overseeing the construction of a massive oil refinery owned by a company on the Elbe River in Humburg.
\cite[pp.~34-35]{mayer2020}
\end{quote}

\cite{drezner2014}, being part of the thematic volume \citep{brands2024}, provides an analysis based upon the economics perspective of standard theoretical framework. 

\citep[p.~115]{cappella2016} recounts "each primary method of war finance [that is] taxation, domestic debt, external funding, and printing \dots".

Additonal sources: \cite{steil2006}, \cite{armijo2014}, \cite{miller2007}, \cite{dobson2015}, \cite{aler1993}, \cite{polk1941}.

There is slight contradiction in the sanctions literature with respect to the 2nd World War episode of assets freeze by the US administration.  In 1940-41, Roosevelt administration froze the US bank accounts of the Nazi Germany, its two allies (Italy and Japan) and the 20 countires they had conquered by that time. Total volume of frozen assets amounted to 4,936 million US dollars taken at current prices. The share of frozen assets belonging to Germany and its two allies was 310 million US dollars or just above 6 percent. The share of the frozen assets belongning to the conqued 20 countries was 4,626 million US dollars or 94 percent \citep[see][]{polk1941}. At the same time, \citeauthor{aler1993} omitted it altogether in its introductory breif on the history of US governments' usage of assets freeze against sanctioned countries \citep[pp.~2-3]{aler1993}. The assets freeze of 1941 is mentioned once in passing on page 10 of this work. Similarly, \citeauthor{dobson2015} makes just a quick reference to the 1940 freeze of the European assets in the Ukraine by mentioning Denmark and Norway after their occupation by Germany \cite[see][p.~41]{dobson2015}.

\citeauthor{galbraith2022} analyzes the emergence of the financial system after the February 2022 second  war invasion by Russia into Ukraine. His work omits Ukraine-related analysis of the already decade-long war, assuming in passing that  "eventually the fighting dies away in Ukraine" \citep[p.~327]{galbraith2022}.

\citep[p.~171]{carim1999} from \citep{klotz1999} explores political economy of sanctions imposed by the Western powers against South Africa's apartheid government, a gold producing country. It concludes with a generalization:

\begin{quote}
The usefulness of ``bankers sanctions" beyond South Africa, however, will depend on the particular nature of the economic and political violnarability of the target. South Africa, for example, remained less vulnarable to financial strains than most devloping countries during the 1970s and 1980s because of its gold production and the sophistication of its domesctic capital markets. Many other countries would not have the mitigating strength of a resource base that increases in value during times of global economic turbulence or the institutional capacity to oversee a two-tiered exchange rate. Thus, we should expect most other countries to be \textit{more susceptible} to financial pressures, if their economic and institutional weaknesses are accurately identified. (emphasis original)
\end{quote}

See \cite{viner1929} on ``the relations between \textit{haute finance}
 and \textit{haute politique}" \citep[p.~408, emphasis original]{viner1929}. Early financial sanctions on the eve of First World War were discussed in the following way: "the objective of the financial boycott was sometimes
 not so much to inflict punishment for participation in a hostile
 alliance or to win them away from it, as to \textit{withhold} from
 potential foes the means whereby to build up economic or
 military strength" \citep[p.~448, emphasis added]{viner1929}.
 
\cite[pp.~21-22]{labaqui2014} from \cite{armijo2014} observed: ``Most scholars and analysts had expected that, after the dramatic neoliberal reforms in the late 1990s and early 2000s, greater financial integration and economic liberalization had made it too costly, both economically and politically, for Latin American governments to reverse such reforms. In contrast, [Argenita and venezuela] managed to implement unorthodox economic policies and to reduce the power of global capital and \acp{ifi} not only due to the good fortune that they have found in high commodity prices, but also thanks to domestic political support from those feeling marginalized under neoliberal economic policies."

\citeauthor{katada2014} shows neglect with respect to the ``South-South" aggressive tactics by large counrties of weaponizing finance with respect to the countries that they intentially weaken in order to impose their own rule. What is a 21st century imperialism. They observe: ``North-South financial ties are already in some measure being replaced by South-South ones." \citep[p.~11]{katada2014}. They continued:

\begin{quote}
Larger emerging powers, especially but not only China, also may anticipate wielding the sword of currency power in their regions. Thus the many contemprory proposals for local currency invoicing of trade championed by Russia, China, India, South Africa, and other members of the financial G20 that are \textit{not} advanced industrial countries surely come with a savings in direct transaction costs -- after all, why should the Ukraine, for example, need to purchase or earn U.S. dollars in order to buy natural gas from Russia? Yet, apparently cheaper ruble-denominated debt obligations in time could bring an unexpectedly high level of political vulnerability later. For example, Russia's recent history of economic relations with its regional periphery has been quite aggresive. \citep[p.~12, emphasis original]{katada2014}
\end{quote}

The above quote from the analysis by \citeauthor{katada2014} reveals that authors, first, left behind a sizable number of details and, secondly, applied consistently the language used by the Russia state-financed think-tanks. As far as the first point is concerned the facts of Russia pushing other smaller countries as Ukraine into foreign-currency debt and weakining them by all means possible did not appear in the analysis. The famous in Ukraine's political and economic life is the natural gas agreement between governments of Ukraine and Russia riched in 2009 after a Russia's natural gas war against Ukraine in the prior months.\index{Russia!natural gas war against Ukraine of 2009} The price and payments were in the U.S. dollars as before. The new were two features: (1) the price formula was tied to the international energy market prices and (2) the buying side (Ukraine's government) committed to the floor system of volumes of natural gas to be purchased. From the beginning, Ukraine was burdened with regular payments in the U.S. dollars. The committed volume to be purchased and price to be paid made the total bill to pay of such a sizable proportions that Ukraine's government was all the time in the situation of borrowing U.S. dollars in order to pay. Borrowing from \citeauthor{kregel2019}, in effect Russia was pushing Ukraine into the financial position that is of Ponzi position or, at least, extreme speculative posititon. This case was not along and following by lengthy trade war initiated by Russia and then in 2014 its military aggression followed that doubled down in 2022 and full-scaled war aimed at complete elimination of Ukraine as a nation and total destruction of its economy. As far as the second point is concerned, the terms such as ``the Ukraine," ``regional periphery" and ``cheaper ruble-denominated debt" have been originating from them and both politically allignment with ruling authoritarianism of this country. The above-mentioned think tanks mirror their narrative-building practices with those think tank around the world that maintain what \citeathor{mirowski2009} termed as neoliberal thought collective. All in all, as of today the title of the work by \citeauthor{katada2014} ``New Kids on the Block" appears out of this world -- Russia is hardly a new kid that is making financial statecraft, it has been consiously executing a war against Ukraine under different guises.  

\cite{hancock2009} on the Russia's regional ``integration" project, which is after its war against Ukraine started in 2014 is nothing but an imperialist project of re-conquering a nation that has been continiously seeking its independence from the first Russia empire and then from the one styled under Soviet Union.

She observes: ``around 1993-1994, when Yeltsin increasingly came under pressure from nationalists, who demanded that Russia reassert itself on the world stage in general and particularly in its own neighborhood" \citep[p.~125]{hancock2009}. 

\citep{dunlop1997}: ``".

\citep[p.~61]{lachowski2007}: ``In the spring of 2005 Ukrainian Foreign Minister Borys Tarasyuk protested against Russian demonstrations of strength (the Russian Special Forces had landed 
illegally in the Crimea during an exercise) and notified Russia that Ukraine would 
not extend the 1997 agreement when it expires in 2017."

\citep{shlapentokh2000}: ``Although anti-Western and anti-American feelings are on the increase
in the country, it has little to do with a concern for Russia's lost glory. The
Russia eventually will confront the problem is much more mundane. The reality is that economic reforms have not brought a better life for the majority, and the collapse of the ruble [in 1998 and earlier in the first half of the 1990s] made the situation worse. Yet this has not led to a nationalistic or radical heat as one could assume
if his only source of information was articles in Russian newspapers. My observations in Moscow led me to believe that the average Russians remain quite passive, seeing nationalistic rhetoric as merely a ploy of this or that political force to use patriotic feelings as a vehicle to bring to power those who utter it. And power is coveted by all of these politicians not to improve the life of the masses or to make Russia great again but for another reason. The majority of simple folks with whom I conversed believed that the elite
(both pro-Western and anti-Western) want power merely for their own personal enrichment."

\citep{shlapentokh2000}: ``I discussed the situation in the Balkans with her and recalled my
Communist friend's report about the patriotic gathering in front of the U.S. embassy when the American flag was burned. Her reply was contemptuous.
"Big deal to burn the flag. I would like to see
one person who would burn one American
dollar." Her point was clear. Patriotic rhetoric
is cheap, but no one is willing to sacrifice his
interests, even the smallest iota, for the
county.", ``a craving for Western dollars,", ``a Russian Nazi could come to power."