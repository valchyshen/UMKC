\newpage

Explaining historical materialism, Max and Engels argued that during transformations of societies from one mode of production to another there is gradual however profound change in the ruling ideas that govern whole society. Transformation from capitalism to communism, they argued, would required rupture from the established and traditional ideas and view points. As far as, the money system of the society is concerned, such a rupture means that traditional banking system with its central banking and private banking parts evolves into a system, where:

\begin{quote}
[c]entralization of credit [is] in the hands of the State, by means of a national bank with State capital and an exclusive monopoly. \citep[p.~26]{marx_communist_2008}
\end{quote}

Right after the World War II, the \citeauthor{AA1946} provided an officially authorized article on the banking system of the \ac{ussr}. It was published in the quarterly journal American Affairs \citep{AA1946}. The main lines are:

\begin{quote}
[It] is a
conventional and necessarily solvent banking
system under a regime of absolute state capitalism. Marxian ideology has nothing to do
with it. The principles, the methods and technique, even the terms are all from the book of
capitalism-e.g., central bank, gold reserve,
note issue, cash balances, working capital, longand short-term credit for industry and agriculture, savings, surplus, profit and loss. Almost you might think you were reading an
elementary text on banking in a systern of free
private enterprise; and in fact the only technical difference is that here the state is
everything. It owns the bank, all the cash balances, all the reserves, all the working capital.
Industry, it is the state. Profit is state profit.
When there is borrowing and lending at interest it is the state borrowing from itself and
lending to itself, and when you read that the
funds with which the State Bank finances the
economy ``are chiefly derived from the free
resources of the economy itself" you need a
"moment's thought to make the translation.
The ``free resources of the economy itself consist of all that is left over from what the people are permitted by the state to consume.\par In the first ten years or so of its existence, the
State Bank was only one of the banks engaged in
economic financing. There were other smaller banks
which had been especially formed to finance specific
branches of industry. Furthermore, in addition to
bank loans, there was a system of financing by bills
of exchange drawn by one business concern on an-
other. The existence of this system naturally interfered with the utilization of the banks as instruments of rigid control over the financial activities
of business organizations.
Accordingly, in 1930, steps were taken to reform
the whole credit system. The reform was based
upon two underlying principles: (1) All short-term
financing was an exclusive prerogative of the State
Bank, and (2) bills of exchange were abolished and
no business firm had the right to grant credit to
another. All short-term financing thus became bank
financing, concentrated in the State Bank, of which
all the business organizations in the country became
the direct clients. The other banks were converted
into exclusively long-term financing institutions.~\citep{AA1946}
\end{quote}

\subsection{The 1998 Russia's Local-Currency Default: Through the \ac{mmt} Angle}

\subsection*{Absract}

Russia's local-currency sovereign debt default in August 1998 begs explanation from the \acf{mmt} angle. Until very recently, the prevailing explanation has been inclined to the view that the Russian crisis took place primarily because of fiscal policy weakness, i.e., the inability of Russian authorities to reign in government deficit and impose due tax collection on the economy. In the \ac{mmt} literature, Russia's local-currency default is touched lightly upon, such as in (Mosler, 2007), the author of which, as a fixed-income investor himself back then, took losses on Russian local-currency fixed-income investments at the time of the default (Lowrey, 2013).This paper aims to provide an alternative view to the established one. It concludes that evolution and institutional detail of the design and operations of the payment system help to distil the underlying forces that paved the way to the event of default. While the mix of those forces is complex, it ranged from theoretical basis of reforms to the size of foreign currency indebtedness of that government and its commitment to local-currency convertibility into major foreign currencies, the \acf{usd}. Nevertheless, the most decisive force was political decision to default. 

\subsection{Inroduction}

In Russia, the collective social conscience epitomizes the financial crisis of August 1998 with a deep devaluation of the foreign-exchange rate of the ruble, the national money of account, and acute economic mismanagement. 

Since August 17, 1998, everyday Russian lexicon has firmly embraced the concept of 'default,' which was largely unknown to the general public, as a euphemism to mean an event with extreme negative consequences to the general well-being~\citep[p.~19]{bashkirova2012}. 
Internationally, especially in the financial markets, it has been referred to as 'the 1998 Russian default.' It initiated a chain of events that culminated in the year of 2000 with a debt restructuring of Russia sovereign debt. Bondholders agreed to a 50 per cent ``haircut" in the net present value terms, while prevailing haircuts in other countries' debt restructuring ranged 20-40 per cent (Sturzenegger \& Zettelmeyer, 2006, p. 25).

The entire region of the former Soviet Union, which by that time consisted of the current independent states, each with its own national money of account, felt the impact of the Russian crisis. Devaluations and recessions were unescapable features throughout the region. This was due to the size of Russia's economy, which was the largest among the countries of the former Soviet Union region, with still sizable trade and financial linkages between Russia and the rest of the Soviet Union member countries. 

Moreover, the lack of a language barrier within this region (the Russian language is still used widely) and the long-established practice, by habit of thought, of emulating Russia, perpetuated such socioeconomic developments and then the effects in this region that were first featured in Russia prior to its 1998 crisis. It is vividly described by the former chief economic advisor to the then president of Ukraine, Leonid Kuchma: ``We, especially in the first years of independence [of early 1990s], copied in good faith all that what was done by the [Russian] 'elder brother.' Moreover, our teachers were the same." (Galchynskiy, 2013, p. 232)

Section one of this paper describes the transitional period of the Russian economy during early 1990s. It is an important episode for inquiry into the origins of the 1998 Russian default, and aims to answer the question of what the underlying developments were paving the way by late 1990s to the event of default.

Section two discusses the operational details of the cooperation between the Ministry of Finance (MoF) and Central Bank of Russia (\ac{cbr}), while dealing with the interest rate and foreign-exchange markets.

The last section concludes and summarizes key points of this paper.

\subsection{Early Transition Period}

It is worthwhile briefly sketching out the noteworthy evolution of the Russian economy along the timeframe starting from a few decades prior to 1998. Yet, on the eve of 1990s, political elites residing in Moscow were still directly running the social and economic affairs of the \ac{ussr}, and indirectly managing the socioeconomic affairs of its satellite states. Since the early 1990s, i.e., after the breakup of the \ac{ussr} and the entire Soviet Bloc, the political elites had a narrower scope in running a chiefly Russia economy. 

Before the 1990s Moscow was overseeing the \ac{ussr}'s whole economy, comprising 15 nation units called republics, which functioned as a monetary union with the Russian ruble as the single money unit of account. 

In addition, it was also presiding over the economic structures, comprising additional, smaller states of the Soviet bloc, which circulated their own money units of account domestically (the Polish zloty, Hungarian forint, Czechoslovakian krone, Romanian leu, etc.). The cross-border commercial transactions within this bloc were carried out in monetary terms through debits and credits denominated in a shared and notional money unit of account (more details below), while international trade with countries outside the bloc were transacted in the money units of account of the major Western economies and through the established global money centers such as London, New York, etc.

\subsubsection{Institutional Detail of the Financial System Before the 1980s Reforms}

By the 1980s, the leadership of the former Soviet Union moved to enact economic reforms in a bid to escape a lasting economic stagnation. Similar currents of varying degree were taking place in the smaller countries of Central and Eastern Europe (such as East Germany, Poland, Hungary, Romania, etc.) that were part of the so-called Soviet Bloc, a political and economic entity led by the ruling political machinery of Moscow, capital of Soviet Union. 

Formally, the Soviet Bloc countries were knit together under the umbrella of the \acf{cmea}, which used to be referred to by the abbreviated name of Comecon. Its inception was ``in Moscow in January 1949," and became a structure for economic cooperation among its members on supplies of raw materials and goods as well as services shared with each other. 

Since its inception and throughout the 1950s and early 1960s, transactions between Comecon members were executed within a bilateral clearing agreement system and settled in the money of account of the clearing ruble, which was already an existing monetary unit, the Soviet ruble. Domestically, each member state of \ac{cmea} used its own national money unit of account. 

For transactions with counterparties outside of the \ac{cmea}, the member states utilized foreign money units of accounts from the West. By 1963, this system turned out to be problematic, and was reworked into a multilateral clearing system of interstate trade to be backed by short- and medium-term credits of an international bank denominated in a common money unit of account. 

Under this new-founded system, interstate transactions were carried out in monetary terms via a designated Moscow-based entity, the \ac{ibec}, which was established in 1963. It handled accounts of banks of the member-states (one bank for each member state) under a designated, common money unit of account, the transferable ruble. 

The latter was not convertible into Soviet ruble, the money unit of account used in the \ac{ussr}, nor into the units of accounts of the Western world. Again, the member states retained their national money units of account for domestic transactions, while dealing in Western money units for transactions outside the \ac{cmea}~\citep[see][]{vincze1984,kalinski2014}. 

Under Comecon, the \ac{ussr}'s financial system was structured into two segments: domestic and international. Its domestic segment consisted of two banks: Gosbank (or interchangeably, State Bank) and Vneshekonombank. The former handled on its balance sheet all the transactions for government agencies as well as for the government-controlled industries that were directed via their respective ministry (\ac{cbr}). 

The latter handled operations in the domestic units with outside world as such: the international segment was represented by: (i) Moscow-based international banks (the above-mentioned \ac{ibec} and \ac{iib}); and (ii) a myriad of banks based in the key financial centers of the world such as London, Paris, Frankfurt, Singapore, and Beirut, among others, and they were owned by either Gosbank or Vneshekonombank. The largest of such banks was Moscow Narodny Bank, which was operated mainly in the City of London,\footnote{In fact, Moscow Narodny Bank (MNB) originated in Moscow (Russia) in 1911, or well before the \ac{ussr} was created. Originally, MNB was established as a bank aiming to supply funds to the small credit institutions as well as to cooperatives of different types. It opened its first international branch in the City of London in 1915. Then in 1916, it opened a representative branch in New York City (Krotov, 2011, pp. 17,34).} and Paris-based Eurobank. 

While domestic banks in this system were state-owned, centrally directed, and not strictly profit-seeking institutions, they were concerned rather with development. The international segment consisted of the commercial banks as purely profit-seeking institutions, which used to, and still do, hire foreigners for their operations as a practice in those markets (Krotov, 2007), (Krotov, 2011). 

The Soviet ruble was an inconvertible money unit of account. As a deputy head of State Bank put it: there was ``Chinese wall" between ruble and foreign currencies back then (Krotov, 2008b, p. 19). So, in technical terms, the ruble was inconvertible, neither into a foreign-currency unit of account nor into precious metals such as gold (this was despite the fact that cash notes of ruble of different denomination contained small print saying ``bank tickets are backed by gold, precious metals and other assets of State Bank.").\footnote{Author's translation from 50 rubles cash note printed in 1991, source: \url{http://numismaclub.com/imgs/a/d/h/s/u/ussr_1991_russia_50_rubles_roubles_russian_lenin_paper_money_soviet_union_note_1_lgw.jpg}}

A domestic foreign-exchange market was absent, and holding of foreign currencies by private individuals was illegal. Foreign-exchange transactions were under strict control and limited in terms of: (a) transactions and operations related to commercial cross-border flow of goods and services; and (b) the transactions being handled by a designated set of banking units, the above-mentioned Vneshekonombank and \ac{ibec}, \ac{iib}.

Entrepreneurship as well as financial speculation by private individuals was also illegal \citep[p.~48]{bashkirova2012}. In general, as Dillard put it, Soviet Union was ``a non-business economy" (1967, p. 633).

Within the Soviet Union, State Bank in effect had its balance sheet handling the credits and debits for the entire economy. Its service was considered as a utility similar, for example, to the state-run railway company. Both functioned not for the sake of short-term profit. Their operations covered the whole country, and, quite interestingly, they designated military-type uniforms for top personnel for official ceremonies (inaugurations, receptions, etc.). 

The industrial sector had its own peculiarities. It was designed to function within the realm of cooperation as well as through a division of labor which differed from the realm of competition as well as division of labor known in the market-based Western economies. 

The breakup of the \ac{ussr} in 1991 into 15 independent states revealed that the industries of each new state were exporting a large share of its output (Viksnins, 1997, p. 25). The same was true for many other members of Comecon. 

From the early 1930s through the late 1980s, the payment system within the domestic part of the financial system of the \ac{ussr} functioned under the design of net clearing and active control and management\footnote{Officially, it was referred to as ``control by the means of ruble" (Usoskin, 1946, p. 8).} (Usoskin, 1946), (\ac{cbr}, 2010, pp. 99-110), (Krotov, 2008b, pp. 12-26). 

One basic feature of this payment system was the following: State Bank was a single entity operationally, the balance sheet of which was used to record all debits and credits among all domestic economic units.\footnote{This system was different from its counterparts in the Western countries by concentration of credit and clearing functions under one roof. It worth to note that, for example in the US, some authors were critical of the existed credit and clearing system as being less efficient: ``Much of this difficulty [arising from out-ot-town check collections] could be eliminated if the United States had a branch banking system instead of the present system, which is composed of some 27,000 small, independent banks. If there were a hundred banks with fifteen thousand branches, the matter would be greatly simplified. Many checks would be redeposited in another branch of the drawee bank and no collection problem would be involved. Checks on different banks could be put through centrally located clearing houses, as they are in Canada." (Patterson, 1918, p. 19)} Describing this system, Woodruff retains the original abbreviation MFO and the Russian name it stands for, mezhfilialnyi oborot???inter-branch transfers or turnover in English (Woodruff, 1999, pp. 66-67). This paper follows this practice by referencing it as the MFO system.

This system evolved from the socioeconomic environment of the 1920s, which made authorities critical of the workings of the then credit market. At the time, the fact was that the State Bank and a number of specialized banks (subsidiaries of State Bank) were discounting commercial bills (veksels), i.e., \acp{iou} issued by non-financial enterprises for the sake of carrying out buy-and-sale transactions. 

The State Bank economists of the time debated monetary matters in line with the real-bills principle, quantity theory of money (Woodruff, Money Unmade: Barter and the Fate of Russian Capitalism , 1999, pp. 28-35), and in such terms as ``exchanging commercial money by banking money" (\ac{cbr}, 2008, pp. 40-44). 

The payments between branches of State Bank, which usually were in different towns, were carried out by means of centralized matching (in 1922-26) and decentralized matching (1927-29); see Chart 1 on page 8 and Chart 2 on page 8, respectively. Both of these methods proved unsatisfactory, as the required level of control and matching was not achieved (\ac{cbr}, 2010, p. 103). 

The entire credit system, where endogenous credit creation was partially based on the commercial bills, i.e., outside of banks, were back then deemed as less regulated.\footnote{Authorities aimed to deploy effective 'ruble control,' i.e., control by means of the ruble, which meant back then that State Bank was to expand control over credit creation towards the level of enterprise. It effectively meant State Bank was to take over commercial bills creation by providing direct lending to those enterprises that required credit. The previous practice of commercial bills issuance and then their rediscount in the banking system revealed that "among those commercial bills filed for rediscounting there were many so-called 'advance' and 'friendly' bills, which did not reflect real movement of goods and which were issued by enterprises for each other for the sake of credit" (Gusakov \& Dymshits, 1951, pp. 124-125). This conflict between authorities' adoption of the real bills doctrine for the operations of the monetary system and the actual workings of the economy, where accommodation bills had proliferated, is discussed in (Woodruff, Money Unmade: Barter and the Fate of Russian Capitalism , 1999, pp. 21-55).} Hence, the authorities of the country had made the decision that time had come to reform the system.

The credit reform of the early 1930s resulted in State Bank consolidating the operations of specialized banks and introducing direct lending to enterprises, which carried out operations under the five-year, official plan and according to the mandate of the central government. The prev\acp{iou} practice of discounting of commercial \acp{iou} (veksels) was discontinued. In fact, ``commercial credit was banned" (\ac{cbr}, 2010, p. 105). 

The reform brought forth a new system of interbank payments, which was referred to as the MFO system (see Chart 3 on page 9). State Bank's ability to control operations of the economic units in the country had advanced. Hence, another step in the advancement of the payment system was the introduction of net clearing for the commercial payments. 

It was understood that: (a) if there was a group of enterprises that operate regularly as suppliers and customers to each other; and (b) if their payments ``collide with each other [in time]" or, in other words, their payments to each other were taking place nearly simultaneously, then the net clearing of these mutual payments would be beneficial to the parties involved. The benefits of net clearing were: (1) improving the timeliness of payment settlements; (2) economizing on the costs of borrowed funds by the enterprises; and (3) lowering the labor input in the paperwork both by the enterprises and State Bank officers. 

Back then, the introduction of net clearing yielded a better execution of payment settlements. At that time, in the 1930s, the Soviet Union's economy experienced rapid industrialization, while ``[f]ollowing the Second World War the Soviet Union made rapid strides in economic development and attained levels of industrial production second only to the United States" (Dillard, 1967, p. 612)

The banking system based upon State Bank and its MFO payment system allowed the government spending to function in the following way. Given the fact that Soviet Union covered a large territory geographically, spanning several time zones, government spending on the local level were carried out by the regional governments. A former Deputy Head of State Bank Vyacheslav Zakharov explained, ``the MFO system allowed budget spending on the local level to be carried out independently of budget revenues. Balancing of revenues and expenditures was taking place [later on] in the Center, in State Bank, which credited Ministry of Finance if the balance of the latter was in shortage" (Krotov, 2008b, p. 16).

The credit relations between State Bank and its clients-the enterprises of (in the market economy terminology) the non-financial sector-were carried under umbrella of another state authority called Central Planning Authority (or ``Gosplan" by Russian abbreviation). See Chart 4, p. 12. As the first step, the Gosplan and the central government's ministries, which overlooked the operations of the respective sector of the economy, regularly developed annual plans of production and supplies of the material (real) resources in accordance with strategic five-year plans approved by the Congress of the Communist Party of \ac{ussr}, the country's highest authority at the time. Gosbank coordinated with Gosplan and ministries the development of these plans. Then, given the material resource plans, the next step kicked in. It consisted of the above-mentioned state authorities as they were translating the material resource plans into the plans of credit allocations to the enterprises of the respective ministries. It is vital to mention that each sectoral ministry had overseen the enterprises that were organized into unions (I use word union as an equivalent to the Russian ``obiedineniye" or ``trest") or, a probably better English word is, consortiums. So, eventually, it was both a union/consortium and an enterprise of this consortium that received a limit of credit allocation from the State Bank. So that, union's credit limit amounted to the credit limits assign to each of the enterprise summed up together. Quite importantly, it was the level of a union/consortium that had its credit limit being fixed and not subject to revision by Gosplan and Gosbank within current period. It was supposed to wait till next year's Gosplan review of the nation-wide plans of material resources and credit allocation limits. This was sort of discipline measure built into the system. However, alongside there was flexibility of re-negotiation of the credit limit on the level of an enterprise. In case of need to expand own credit limit for the operational needs, an enterprise had to act through its union/consortium to get that extension, which was not guaranteed. So, once an enterprise reached out to the union/consortium, the latter could obtain a required credit limit increase only if it had found another union/consortium that was willing to reduce own credit limit by that amount. (This kind of negotiations between unions/consortiums on the credit limit had not have neither an immediate monetary pay-back nor future pay-back. How precisely these relationships were accounted for between the sides of the agreement requires additional investigation.) So, against the background of State Bank's lacking an institution of reserve balances\footnote{What is known as Federal Reserve funds in the US money system.} by design, it was an institution of the credit limit that, as it appears, resulted in a development of the notion of ``financial resources" among the policy-makers and people on top positions in the non-financial sector and inside State Bank itself. That development was gradual, as it seems. Because, in the very early period of State Bank creation, the monetary economists were acknowledging the credit/money creation processes being taken place endogenously and even outside the perimeter of state-run bank system. By 1980s, there was seemingly a very wide (near complete) adoption of the logic of loanable funds theory among the country's leadership, policymakers, State Bank officers and managers of the enterprises.

At the same time, the credit limit system was relaxed by the authorities towards the enterprises (or rather whole union/consortium of enterprises) that were considered as priority. Because that area of industry promised a break-through future development. Such was, for instance, an oil and natural extraction industry in the Siberian region of Tyumen. The following extract is from the recollections of the official involved in the deployment of oil and gas extraction efforts:

\begin{quote}
And no one counted the money. You may not know it, but the oil sector had an open ruble account for many years of the Soviet period. There was \textit{no} limitation of expenditures at all.~\citep[p.~226, emphasis added]{aven2015}
\end{quote}

In the above quote the phrase ``open ruble account" suggests that above-mentioned procedure of allocating a credit limit did not apply to the oil sector that was developing via expansion at the time in the Siberia. There was no self-imposed constrain in that case. Nevertheless, such an approach-of relaxed discipline and heightened flexibility of credit provisioning-was a rather rare circumstance and did not apply across the country. 

\begin{figure}[ht]
\vspace{.2in}
\captionsetup{width=1.0\linewidth,labelfont=bf}
    \centering
    \begin{tikzpicture}
        \draw[help lines,white] (0,0) grid (16,6);
        % First
        \node (A) at (0,2) {A};
        \node (B) at (4,2) {B};
        \node at (2,-1) {(1)};        
        \draw (A) -- (B) node[midway,above] {\textsuperscript{$debt: d_{0,B}$ \hspace{.25in} $credit: c_{0,A}$}};
        \draw[red] (0.2,1.9) -- (3.8,1.9) node[midway,below] {\textsuperscript{$credit: c_{0,B}$ \hspace{.25in} $debt: d_{0,A}$}};
        % Second
        \node         (A)         at (6,2) {A};
        \node         (C)         at (10,2) {C};
        \node         (B)         at (8,4) {B};
        \node         (D)         at (8,0) {D};
        \node at (8,-1) {(2)}; 
        \draw (A) -- (B) node[midway,sloped,above] {\textsuperscript{$d_{0,B}$ \hspace{.4in} $c_{0,A}$}}; 
        \draw (B) -- (C) node[midway,sloped,above] {\textsuperscript{$d_{0,C}$ \hspace{.4in} $c_{0,B}$}}; 
        \draw (C) -- (D) node[midway,sloped,below] {\textsuperscript{$c_{0,C}$ \hspace{.4in} $d_{0,D}$}}; 
        \draw (D) -- (A) node[midway,sloped,below] {\textsuperscript{$c_{0,D}$ \hspace{.4in} $d_{0,A}$}}; 
        % Third
        \node         (A)         at (12,2) {A};
        \node         (C)         at (16,2) {C};
        \node         (B)         at (14,4) {B};
        \node         (D)         at (14,0) {D};
        \node at (14,-1) {(3)};
        \draw (A) -- (B); \draw (B) -- (C); \draw (C) -- (D); \draw (D) -- (A);
        \draw (A) -- (C); \draw (B) -- (D); 
        \draw[red] (12.2,1.9) -- (15.8,1.9); % A--C
        \draw[red] (14.1,.2) -- (14.1,3.8);  % B--D
        \draw[red] (12.1,2.25) -- (13.7,3.85); % A--B
        \draw[red] (12.1,1.75) -- (13.7,.15);  % A--D
        \draw[red] (14.3,.15) -- (15.8,1.65);  % D--C
        \draw[red] (14.3,3.85) -- (15.9,2.25); % B--C
        % Labels
        \draw[black] (3,5.5) -- (4,5.5); \draw[red] (8,5.5) -- (9,5.5);
        \node[black,align=left,font=\small] at (5.5,5.5) {DCR due today};
        \node[black,align=left,font=\small] at (11,5.5) {Opposite DCR due today};
    \end{tikzpicture}
    \vspace{.1in}
    \caption[Types of payments clearing by the Soviet Union's State Bank]{Types of payments clearing by the Soviet Union's State Bank: (1) Bilateral clearing (set-off), (2) Chain clearing, (3) Multilateral clearing (series of set-offs).\par \vspace{.05in}Source: narrative from (CBR, 2010, pp. 105-106), illustration by author.}
    \label{fig:setoff_state_bank}
\end{figure}

Later on, the market reformers who ran the Russian government in the early 1990s argued that this system was not compatible with the market reforms. The latter were required to turn the formerly planned economy of Russia into a market-based economy~\citep[p.~56]{gaidar2011}. Other critics pointed out that it encouraged profligate industrial enterprises which ``were used to being credited at the time of shipment, to getting paid for their bills of lading, as it were, with credit being advanced by the state banks automatically"~\citep[p.~30]{viksnins1997}. By the time (or to be more precise, during the 1970-80s) both the Soviet Union and the members of the Comecon had accumulated foreign-currency debt on the back of: (a) past poor harvests; (b) support of consumption by domestic consumers via importing goods; and (c) domestic fixed-investments that required foreign technology. For more on these developments, for instance, see \citep{coombs1976,poznanski1996,kotkin2001,gaidar2002,harold2003}. 

\begin{figure}[ht]
\vspace{.1in}
\captionsetup{width=1.0\linewidth,labelfont=bf}
    \centering
    \begin{tikzpicture}[scale=.9]
        % Clearing Scheme
        \draw[help lines,white] (0,0) grid (16,14);
        \node (A) at (0,12) {A};
        \node (B) at (5,12) {B};
        \node (R) at (2.5,14) {R}; \node (RR) at (7,14) {State Bank};
        %\node at (2,5) {(1)};        
        \draw (A) -- (B) node[midway,above] {\textsuperscript{$debt: d^{100}_{0,B}$ \hspace{.5in} $credit: c^{100}_{0,A}$}};
        \draw[red] (.2,11.9) -- (4.8,11.9) node[midway,below] {\textsuperscript{$credit: c^{90}_{0,B}$ \hspace{.5in} $debt: d^{90}_{0,A}$}};
        \draw[->] (RR) -- (R);
        % Register Matrix
        \matrix (m) [matrix anchor=north, matrix of nodes, nodes in empty cells,
             nodes = {text width=.35in, minimum height=.16in, align=right, anchor=center, font=\fontsize{9}{9}\selectfont}, column sep=.5em
            ] at (5,10)
        {
             &        &       &        &      &        &     \\
             &        &       &        &      &        &     \\
        \textbf{Unit} & Credit & Debt  & Credit & Debt & Credit & Debt \\
        A    & X       & X &   100  & 90   & 10 & X\\
        B    & 90     &  100  & X       & X    & X & 10 \\
        \textbf{Gross}    & 90     &  100  & 100       & 90     \\
         };
        \node[fit=(m-2-2)(m-2-3), font=\fontsize{9}{9}\selectfont] {Unit \textbf{A}};
        \node[fit=(m-2-4)(m-2-5), font=\fontsize{9}{9}\selectfont] {Unit \textbf{B}};
        \node[fit=(m-2-6)(m-2-7), font=\fontsize{9}{9}\selectfont] {\textbf{Net}};
        \node[fit=(m-1-1)(m-1-7), font=\fontsize{9}{9}\selectfont]%
            {\textit{\MakeUppercase{Register Matrix}}\vspace{.2in}};    
        \draw[thin]  (m-3-3.south -| m.west) -- (m-3-3.south -| m.east);
        \draw[thin]  (m-6-3.south -| m.west) -- (m-6-3.south -| m.east);
        \draw[thin]  (m-2-1.north east) -- (m-6-1.south east);
        \draw[thin]  (m-2-3.north east) -- (m-6-3.south east);
        \draw[thin]  (m-2-5.north east) -- (m-6-5.south east);
         % Gross transaction table
        \matrix (m) [matrix anchor=north, matrix of nodes, nodes in empty cells,
            nodes={text width=.5in, align=right, minimum height=.16in, anchor=center, font=\fontsize{9}{9}\selectfont},
            column 1/.style={align=left},
            row 6/.style={nodes={font=\bfseries\fontsize{9}{9}\selectfont}}
            ] at (8,6.25)
        {
                     &      &    &  &        &    &     &      &      & \\
                     &      &    &  &        &    &     &      &      & \\
        \textbf{Unit}&Credit&Debt&NW& Credit$^\ast$ &Debt$^\dag$&NW$^\ddag$   &Credit&Debt  &NW    \\
        A            & 0    & 0  & 0& 100.008& 90 &0.008&10.008& 0    &0.008 \\
        B            & 0    & 0  & 0&  90.007& 100&0.007&0     & 9.993&0.007 \\
        \textbf{Total}&0    & 0  & 0& 190.015&190 &0.015&10.008& 9.993&0.015 \\
         };
         \node[fit=(m-1-1)(m-1-10), font=\fontsize{9}{9}\selectfont]%
            {\textit{\MakeUppercase{Gross payments}}\vspace{.2in}};
        \node[fit=(m-2-2)(m-2-4), font=\fontsize{9}{9}\selectfont] {\textbf{State Bank's $R$\\Opening Positions}};
        \node[fit=(m-2-5)(m-2-7), font=\fontsize{9}{9}\selectfont] {\textbf{Change of Positions}};
        \node[fit=(m-2-8)(m-2-10), font=\fontsize{9}{9}\selectfont] {\textbf{State Bank's $R$\\Closing Positions}};
        \draw[thin] (m-2-1.north east) -- (m-6-1.south east);
        \draw[thin] (m-2-4.north east) -- (m-6-4.south east);
        \draw[thin] (m-2-7.north east) -- (m-6-7.south east);
        \draw[thin] (m-6-3.south -| m.west) -- (m-6-3.south -| m.east);
        % Net transaction table -------------------------------------------------
        \matrix (m) [matrix anchor=north, matrix of nodes, nodes in empty cells,
            nodes={text width=.5in, align=right, minimum height=.16in, anchor=center, font=\fontsize{9}{9}\selectfont},
            column 1/.style={align=left},
            row 6/.style={nodes={font=\bfseries\fontsize{9}{9}\selectfont}}
            ] at (8,2.5)
        {
                     &      &    &  &        &    &     &      &      & \\
                     &      &    &  &        &    &     &      &      & \\
        \textbf{Unit}&Credit&Debt&NW& Credit$^\ast$ &Debt$^\dag$&NW$^\ddag$   &Credit&Debt  &NW    \\
        A            & 0    & 0  & 0&       0& 10 &    0& 0    & 10   &0 \\
        B            & 0    & 0  & 0&  10.001&   0&0.001&10.001& 0    &0.001 \\
        \textbf{Total}&0    & 0  & 0&  10.001& 10 &0.001&10.001& 10   &0.001 \\
         };
         \node[fit=(m-1-1)(m-1-10), font=\fontsize{9}{9}\selectfont]%
            {\textit{\MakeUppercase{Net Payments}}\vspace{.2in}};
        \node[fit=(m-2-2)(m-2-4), font=\fontsize{9}{9}\selectfont] {\textbf{State Bank's $R$\\Opening Positions}};
        \node[fit=(m-2-5)(m-2-7), font=\fontsize{9}{9}\selectfont] {\textbf{Change of Positions}};
        \node[fit=(m-2-8)(m-2-10), font=\fontsize{9}{9}\selectfont] {\textbf{State Bank's $R$\\Closing Positions}};
        \draw[thin] (m-2-1.north east) -- (m-6-1.south east);
        \draw[thin] (m-2-4.north east) -- (m-6-4.south east);
        \draw[thin] (m-2-7.north east) -- (m-6-7.south east);
        \draw[thin] (m-6-3.south -| m.west) -- (m-6-3.south -| m.east);
        \node[gray,align=left,font=\fontsize{9}{9}\selectfont] at (14,10.5)% 
            {Note:\\ \\$^{\ast}$ Change in the positions\\ of State Bank's credits\\on units $A$ and $B$ consists\\of their overnight borrowings at\\interest rate of 3\%, which are\\equal to debts due by the units,\\see Register Matrix, plus accrued\\interest rate charge;\\ \\$^{\dag}$ State Bank's debt positions\\to units $A$ and $B$ are their\\credits, see Register Matrix;\\ \\$^{\ddag}$ NW is Net Worth of State\\Bank, which is equal to the\\interest rate income charged\\ from units $A$ and $B$.};
    \end{tikzpicture}
    \caption[Bilateral clearing example via State Bank's \acf{bms}]%
    {Bilateral clearing example via State Bank's Bureaus of Mutual Settlements.\par\vspace{.05in}Source: narrative from \citep{schwartz1946,usoskin1946}, illustration by author.}
    \label{fig:bilateral_clearing}
    \vspace{.1in}
\end{figure}

\begin{figure}[ht]
\vspace{.2in}
\captionsetup{width=1.0\linewidth,labelfont=bf}
\centering
\begin{tikzpicture}[scale=.9]
        % Clearing Scheme
        \draw[help lines,white] (0,0) grid (17,13);
        \node (A) at (1,10) {A};
        \node (B) at (4,13) {B};
        \node (D) at (4,7)  {D};
        \node (C) at (7,10) {C};
        \draw (A) -- (B) node[midway,sloped,above] {\textsuperscript{$d^{100}_{0,B}$ \hspace{.6in} $c^{100}_{0,A}$}}; 
        \draw (B) -- (C) node[midway,sloped,above] {\textsuperscript{$d^{120}_{0,C}$ \hspace{.6in} $c^{120}_{0,B}$}}; 
        \draw (C) -- (D) node[midway,sloped,below] {\textsuperscript{$c^{140}_{0,C}$ \hspace{.6in} $d^{140}_{0,D}$}}; 
        \draw (D) -- (A) node[midway,sloped,below] {\textsuperscript{$c^{160}_{0,D}$ \hspace{.6in} $d^{160}_{0,A}$}}; 
        \node (R) at (9,13) {R}; \node (RR) at (12,13) {State Bank}; \draw[->] (RR) to (R);
       % Register Matrix
        \matrix (m) [matrix anchor=north, matrix of nodes, nodes in empty cells,
             nodes = {text width=.33in, minimum height=.16in, align=right, anchor=center, font=\fontsize{9}{9}\selectfont}, column sep=.5em
            ] at (8.5,5)
        {
             &  &  &  &  &  &  &  &  &  &  \\
             &  &  &  &  &  &  &  &  &  &  \\
        \textbf{Unit} & Credit & Debt  & Credit & Debt & Credit & Debt & Credit & Debt& Credit & Debt \\
        A    &  X &  X & 100&    &    &    &    & 160&    & 60 \\
        B    &    & 100&  X &  X & 120&    &    &    &  20&    \\
        C    &    &    &    & 120&  X &  X & 140&    &  20&    \\
        D    & 160&    &    &    &    & 140&  X &  X &  20&    \\
        \textbf{Gross}    
             & 160 &100 & 100& 120&120& 140& 140& 160&    &    \\
         };
        \node[fit=(m-2-2)(m-2-3), font=\fontsize{9}{9}\selectfont] {Unit \textbf{A}};
        \node[fit=(m-2-4)(m-2-5), font=\fontsize{9}{9}\selectfont] {Unit \textbf{B}};
        \node[fit=(m-2-6)(m-2-7), font=\fontsize{9}{9}\selectfont] {Unit \textbf{C}};
        \node[fit=(m-2-8)(m-2-9), font=\fontsize{9}{9}\selectfont] {Unit \textbf{D}};
        \node[fit=(m-2-10)(m-2-11), font=\fontsize{9}{9}\selectfont] {\textbf{Net}};
        \node[fit=(m-1-1)(m-1-11)]%
            {\textit{\MakeUppercase{Register Matrix}}\vspace{.2in}};    
        \draw[thin]  (m-3-3.south -| m.west) -- (m-3-3.south -| m.east);
        \draw[thin]  (m-8-3.south -| m.west) -- (m-8-3.south -| m.east);
        \draw[thin]  (m-2-1.north east) -- (m-8-1.south east);
        \draw[thin]  (m-2-3.north east) -- (m-8-3.south east);
        \draw[thin]  (m-2-5.north east) -- (m-8-5.south east);
        \draw[thin]  (m-2-7.north east) -- (m-8-7.south east);
        \draw[thin]  (m-2-9.north east) -- (m-8-9.south east);
    \end{tikzpicture}
    \caption[Chain clearing example via State Bank's Bureaus of Mutual Settlements]{Chain clearing example via State Bank's Bureaus of Mutual Settlements (\textit{See next page for continued exposition}). \par\vspace{.05in}Source: narrative from (CBR, 2010, pp. 105-106), illustration by author.}
    \label{fig:chain_clearing}
 \end{figure}
 \begin{figure}[ht]
 \ContinuedFloat
 \captionsetup{width=1.0\linewidth,labelfont=bf}
 \centering
 \begin{tikzpicture}[scale=.9]
        % Clearing Scheme
        \draw[help lines,white] (0,0) grid (17,10);
        % Gross transaction table
        \matrix (m) [matrix anchor=north, matrix of nodes, nodes in empty cells,
            nodes={text width=.48in, align=right, minimum height=.16in, anchor=center, font=\fontsize{9}{9}\selectfont},
            column 1/.style={align=left},
            row 8/.style={nodes={font=\bfseries\fontsize{9}{9}\selectfont}}
            ] at (8.5,10)
        {
                     &      &    &  &        &    &     &      &      & \\
                     &      &    &  &        &    &     &      &      & \\
        \textbf{Unit}&Credit&Debt&NW& Credit$^\ast$ &Debt$^\dag$&NW$^\ddag$   &Credit&Debt  &NW    \\
        A            & 0    & 0  & 0& 100.008& 160&0.008&    0 & 59.992    &0.008 \\
        B            & 0    & 0  & 0& 120.010& 100&0.010& 20.010     & 9.993&0.010 \\
        B            & 0    & 0  & 0& 140.012& 120&0.012& 20.012     & 9.993&0.012 \\
        B            & 0    & 0  & 0& 160.013& 140&0.013& 20.013    & 9.993&0.013 \\
        \textbf{Total}&0    & 0  & 0& 520.043& 520 &0.043&60.035& 59.992&0.043 \\
         };
         \node[fit=(m-1-1)(m-1-10)]%
            {\textit{\MakeUppercase{Gross payments}}\vspace{.2in}};
        \node[fit=(m-2-2)(m-2-4), font=\fontsize{9}{9}\selectfont] {\textbf{State Bank's $R$\\Opening Positions}};
        \node[fit=(m-2-5)(m-2-7), font=\fontsize{9}{9}\selectfont] {\textbf{Change of Positions}};
        \node[fit=(m-2-8)(m-2-10), font=\fontsize{9}{9}\selectfont] {\textbf{State Bank's $R$\\Closing Positions}};
        \draw[thin] (m-2-1.north east) -- (m-8-1.south east);
        \draw[thin] (m-2-4.north east) -- (m-8-4.south east);
        \draw[thin] (m-2-7.north east) -- (m-8-7.south east);
        \draw[thin] (m-8-3.south -| m.west) -- (m-8-3.south -| m.east);
        %\node[fit=(m-8-8)(m-8-10),fill=gray];
        % Net transaction table
        \matrix (m) [matrix anchor=north, matrix of nodes, nodes in empty cells,
            nodes={text width=.48in, align=right, minimum height=.16in, anchor=center, font=\fontsize{9}{9}\selectfont},
            column 1/.style={align=left},
            row 8/.style={nodes={font=\bfseries\fontsize{9}{9}\selectfont}}
            ] at (8.5,4.5)
        {
                     &      &    &  &        &    &     &      &      & \\
                     &      &    &  &        &    &     &      &      & \\
        \textbf{Unit}&Credit&Debt&NW& Credit$^\ast$ &Debt$^\dag$&NW$^\ddag$   &Credit&Debt  &NW    \\
        A            & 0    & 0  & 0&      0& 60&   0&    0 & 60  & 0 \\
        B            & 0    & 0  & 0& 20.002& 0&0.002& 20.002 & 0&0.002 \\
        B            & 0    & 0  & 0& 20.002& 0&0.002& 20.002 & 0&0.002 \\
        B            & 0    & 0  & 0& 20.002& 0&0.002& 20.002 & 0&0.002 \\
        \textbf{Total}&0    & 0  & 0& 60.006& 60 &0.006&60.006& 60&0.006 \\
        };
        \node[fit=(m-1-1)(m-1-10)]%
            {\textit{\MakeUppercase{Net payments}}\vspace{.2in}};
        \node[fit=(m-2-2)(m-2-4), font=\fontsize{9}{9}\selectfont] {\textbf{State Bank's $R$\\Opening Positions}};
        \node[fit=(m-2-5)(m-2-7), font=\fontsize{9}{9}\selectfont] {\textbf{Change of Positions}};
        \node[fit=(m-2-8)(m-2-10), font=\fontsize{9}{9}\selectfont] {\textbf{State Bank's $R$\\Closing Positions}};
        \draw[thin] (m-2-1.north east) -- (m-8-1.south east);
        \draw[thin] (m-2-4.north east) -- (m-8-4.south east);
        \draw[thin] (m-2-7.north east) -- (m-8-7.south east);
        \draw[thin] (m-8-3.south -| m.west) -- (m-8-3.south -| m.east);
    \end{tikzpicture}
    \caption[Chain clearing example via State Bank's Bureaus of Mutual Settlements]{Chain clearing example via State Bank's Bureaus of Mutual Settlements (\textit{Continued from previous page}).\par\vspace{.05in}Note: $^{\ast}$ Change in the positions of State Bank's credits on units $A,B,C,$ and $D$ consists of their overnight borrowings at interest rate of 3\%, which are equal to debts due by the units,see Register Matrix, plus accrued interest rate charge; $^{\dag}$ State Bank's debt positions to units $A,B,C,$ and $D$ are their credits, see Register Matrix; $^{\ddag}$ NW is Net Worth of State Bank, which is equal to the interest rate income charged from units $A,B,C,$ and $D$.\par\vspace{.05in}Source: narrative from (CBR, 2010, pp. 105-106), illustration by author.}
    \label{fig:chain_clearing}
\end{figure}

\begin{figure}[ht]
\vspace{.2in}
\captionsetup{width=1.0\linewidth,labelfont=bf}
\centering
\begin{tikzpicture}[scale=.9]
        % Clearing Scheme
        \draw[help lines,white] (0,0) grid (17,13);
        \node (R) at (9,13) {R}; \node (RR) at (9,11.75) {State Bank}; \draw[->] (RR) to (R);
        \node (A) at (1,10) {A};
        \node (B) at (4,13) {B};
        \node (D) at (4,7)  {D};
        \node (C) at (7,10) {C};
        \draw (A) -- (B) node[midway,sloped,above] {\textsuperscript{$d^{100}_{0,B}$ \hspace{.6in} $c^{100}_{0,A}$}}; 
        \draw (B) -- (C) node[midway,sloped,above] {\textsuperscript{$d^{120}_{0,C}$ \hspace{.6in} $c^{120}_{0,B}$}}; 
        \draw (C) -- (D) node[midway,sloped,below] {\textsuperscript{$c^{140}_{0,C}$ \hspace{.6in} $d^{140}_{0,D}$}}; 
        \draw (D) -- (A) node[midway,sloped,below] {\textsuperscript{$c^{160}_{0,D}$ \hspace{.6in} $d^{160}_{0,A}$}}; 
        \draw (B) -- (D) node[midway,sloped,above] {\textsuperscript{$d^{50}_{0,D}$ \hspace{.8in} $c^{50}_{0,B}$}};
        \draw (A) -- (C) node[midway,sloped,above] {\textsuperscript{$d^{100}_{0,C}$ \hspace{.8in} $c^{100}_{0,A}$}};
        % Opposite DCRs
        \draw[red] (1.3,10.1) -- (3.8,12.6); %(A) -- (B);
        \draw[red] (1.3,9.9) -- (3.8,7.4); %(A) -- (D);
        \draw[red] (3.85,12.45) -- (3.85,7.45); %(B) -- (D);
        \draw[red] (1.4,9.9) -- (6.7,9.9); %(A) -- (C);
        \draw[red] (4.2,12.6) -- (6.7,10.1); %(B) -- (C);
        \draw[red] (4.2,7.4) -- (6.6,9.8); %(C) -- (D);
        % Opposite DCRs as standalone scheme
        \node (A) at (11,10) {A};
        \node (B) at (14,13) {B};
        \node (D) at (14,7)  {D};
        \node (C) at (17,10) {C};
        \draw[red] (A) -- (B) node[midway,sloped,above] {\textsuperscript{$c^{40}_{0,B}$ \hspace{.6in} $d^{40}_{0,A}$}}; 
        \draw[red] (B) -- (C) node[midway,sloped,above] {\textsuperscript{$c^{90}_{0,C}$ \hspace{.6in} $d^{90}_{0,B}$}}; 
        \draw[red] (C) -- (D) node[midway,sloped,below] {\textsuperscript{$d^{100}_{0,C}$ \hspace{.6in} $c^{100}_{0,D}$}}; 
        \draw[red] (D) -- (A) node[midway,sloped,below] {\textsuperscript{$d^{80}_{0,D}$ \hspace{.6in} $c^{80}_{0,A}$}}; 
        \draw[red] (B) -- (D) node[midway,sloped,above] {\textsuperscript{$c^{150}_{0,D}$ \hspace{.9in} $d^{150}_{0,B}$}};
        \draw[red] (A) -- (C) node[midway,sloped,above] {\textsuperscript{$c^{70}_{0,C}$ \hspace{.9in} $d^{70}_{0,A}$}};
       % Register Matrix
        \matrix (m) [matrix anchor=north, matrix of nodes, nodes in empty cells,
             nodes = {text width=.33in, minimum height=.16in, align=right, anchor=center, font=\fontsize{9}{9}\selectfont}, column sep=.5em
            ] at (8.5,5)
        {
             &  &  &  &  &  &  &  &  &  &  \\
             &  &  &  &  &  &  &  &  &  &  \\
        \textbf{Unit} & Credit & Debt  & Credit & Debt & Credit & Debt & Credit & Debt& Credit & Debt \\
        A    &  X &  X & 100&  70&  80& 160&  80& 160&    & 10 \\
        B    &  40& 100&  X &  X & 120&  90&  50& 150&  130&    \\
        C    &  70& 100& 90 & 120&  X &  X & 140& 100&  120&    \\
        D    & 160&  80& 150&  50& 100& 140&  X &  X &  & 140 \\
        \textbf{Gross}    
             & 270 &280 & 340& 210&320& 300& 270& 410&    &    \\
         };
        \node[fit=(m-2-2)(m-2-3), font=\fontsize{9}{9}\selectfont] {Unit \textbf{A}};
        \node[fit=(m-2-4)(m-2-5), font=\fontsize{9}{9}\selectfont] {Unit \textbf{B}};
        \node[fit=(m-2-6)(m-2-7), font=\fontsize{9}{9}\selectfont] {Unit \textbf{C}};
        \node[fit=(m-2-8)(m-2-9), font=\fontsize{9}{9}\selectfont] {Unit \textbf{D}};
        \node[fit=(m-2-10)(m-2-11), font=\fontsize{9}{9}\selectfont] {\textbf{Net}};
        \node[fit=(m-1-1)(m-1-11)]%
            {\textit{\MakeUppercase{Register Matrix}}\vspace{.2in}};    
        \draw[thin]  (m-3-3.south -| m.west) -- (m-3-3.south -| m.east);
        \draw[thin]  (m-8-3.south -| m.west) -- (m-8-3.south -| m.east);
        \draw[thin]  (m-2-1.north east) -- (m-8-1.south east);
        \draw[thin]  (m-2-3.north east) -- (m-8-3.south east);
        \draw[thin]  (m-2-5.north east) -- (m-8-5.south east);
        \draw[thin]  (m-2-7.north east) -- (m-8-7.south east);
        \draw[thin]  (m-2-9.north east) -- (m-8-9.south east);
    \end{tikzpicture}
    \caption[Multilateral clearing example via State Bank's Bureaus of Mutual Settlements]{Multilateral clearing example via State Bank's Bureaus of Mutual Settlements. (\textit{See next page for continued exposition})\par\vspace{.05in}Source: narrative from (CBR, 2010, pp. 105-106), illustration by author.}
    \label{fig:multilateral_clearing}
 \end{figure}
 \begin{figure}[ht]
 \ContinuedFloat
 \captionsetup{width=1.0\linewidth,labelfont=bf}
 \centering
 \begin{tikzpicture}[scale=.9]
        % Clearing Scheme
        \draw[help lines,white] (0,0) grid (17,10);
        % Gross transaction table
        \matrix (m) [matrix anchor=north, matrix of nodes, nodes in empty cells,
            nodes={text width=.48in, align=right, minimum height=.16in, anchor=center, font=\fontsize{9}{9}\selectfont},
            column 1/.style={align=left},
            row 8/.style={nodes={font=\bfseries\fontsize{9}{9}\selectfont}}
            ] at (8.5,10)
        {
                     &      &    &  &        &    &     &      &      & \\
                     &      &    &  &        &    &     &      &      & \\
        \textbf{Unit}&Credit&Debt&NW& Credit$^\ast$ &Debt$^\dag$&NW$^\ddag$   &Credit&Debt  &NW    \\
        A            & 0    & 0  & 0& 280.023& 270&0.023& 10.023&      0&0.023 \\
        B            & 0    & 0  & 0& 210.017& 340&0.017&      0&129.983&0.017 \\
        B            & 0    & 0  & 0& 300.025& 320&0.025&      0& 19.975&0.025 \\
        B            & 0    & 0  & 0& 410.034& 270&0.034&140.034&      0&0.034 \\
        \textbf{Total}&0    & 0  & 0&1200.099&1200&0.099&150.057&149.958&0.099 \\
         };
       \node[fit=(m-1-1)(m-1-10)]%
            {\textit{\MakeUppercase{Gross payments}}\vspace{.2in}};
        \node[fit=(m-2-2)(m-2-4), font=\fontsize{9}{9}\selectfont] {\textbf{State Bank's $R$\\Opening Positions}};
        \node[fit=(m-2-5)(m-2-7), font=\fontsize{9}{9}\selectfont] {\textbf{Change of Positions}};
        \node[fit=(m-2-8)(m-2-10), font=\fontsize{9}{9}\selectfont] {\textbf{State Bank's $R$\\Closing Positions}};
        \draw[thin] (m-2-1.north east) -- (m-8-1.south east);
        \draw[thin] (m-2-4.north east) -- (m-8-4.south east);
        \draw[thin] (m-2-7.north east) -- (m-8-7.south east);
        \draw[thin] (m-8-3.south -| m.west) -- (m-8-3.south -| m.east);
        %\node[fit=(m-8-8)(m-8-10),fill=gray];
        % Net transaction table
        \matrix (m) [matrix anchor=north, matrix of nodes, nodes in empty cells,
            nodes={text width=.48in, align=right, minimum height=.16in, anchor=center, font=\fontsize{9}{9}\selectfont},
            column 1/.style={align=left},
            row 8/.style={nodes={font=\bfseries\fontsize{9}{9}\selectfont}}
            ] at (8.5,4.5)
        {
                     &      &    &  &        &    &     &      &      & \\
                     &      &    &  &        &    &     &      &      & \\
        \textbf{Unit}&Credit&Debt&NW& Credit$^\ast$ &Debt$^\dag$&NW$^\ddag$   &Credit&Debt  &NW    \\
        A            & 0    & 0  & 0& 10.001&  0&0.001&    10.001 & 0  & 0.001 \\
        B            & 0    & 0  & 0&      0&130&    0& 0 & 130& 0 \\
        B            & 0    & 0  & 0&      0& 20&    0& 0 & 20& 0 \\
        B            & 0    & 0  & 0&140.012&  0&0.012&140.012 & 0& 0.012 \\
        \textbf{Total}&0    & 0  & 0&150.013&150&0.013&150.013& 150 & 0.013 \\
        };
        \node[fit=(m-1-1)(m-1-10)]%
            {\textit{\MakeUppercase{Net payments}}\vspace{.2in}};
        \node[fit=(m-2-2)(m-2-4), font=\fontsize{9}{9}\selectfont] {\textbf{State Bank's $R$\\Opening Positions}};
        \node[fit=(m-2-5)(m-2-7), font=\fontsize{9}{9}\selectfont] {\textbf{Change of Positions}};
        \node[fit=(m-2-8)(m-2-10), font=\fontsize{9}{9}\selectfont] {\textbf{State Bank's $R$\\Closing Positions}};
        \draw[thin] (m-2-1.north east) -- (m-8-1.south east);
        \draw[thin] (m-2-4.north east) -- (m-8-4.south east);
        \draw[thin] (m-2-7.north east) -- (m-8-7.south east);
        \draw[thin] (m-8-3.south -| m.west) -- (m-8-3.south -| m.east);
    \end{tikzpicture}
    \caption[Multilateral clearing example via State Bank's Bureaus of Mutual Settlements]{Multilateral clearing example via State Bank's Bureaus of Mutual Settlements (\textit{Continued from previous page}).\par\vspace{.05in}Note: $^{\ast}$ Change in the positions of State Bank's credits on units $A,B,C,$ and $D$ consists of their overnight borrowings at interest rate of 3\%, which are equal to debts due by the units,see Register Matrix, plus accrued interest rate charge; $^{\dag}$ State Bank's debt positions to units $A,B,C,$ and $D$ are their credits, see Register Matrix; $^{\ddag}$ NW is Net Worth of State Bank, which is equal to the interest rate income charged from units $A,B,C,$ and $D$.\par\vspace{.05in}Source: narrative from (CBR, 2010, pp. 105-106), illustration by author.}
    \label{fig:multilateral_clearing}
\end{figure}

\citeauthor{dillard1967} explained that agriculture was one of ``the weakest links of the Soviet economy"~\citep[p.~628]{dillard1967}. As a result, it forced the country ``[i]n 1963 [to enter] the world market to buy grain to feed its growing population" (ibid). It was quite an unusual episode in the global economy of the time. Thus, Coombs discusses Russia's significant offerings of gold during the 1960s to the London gold market, ``reflecting wheat harvest failures in 1963 and 1965."~\citep[p.~154]{coombs1976}. Later, however, the authorities of the Soviet Union turned to imports-buying through Western debt instead of selling its stock of gold. Poznanski points out that ``in 1970, the Soviet Union was already in its fourth or fifth year of trying to gradually reduce political tensions with the Western powers."

Harold writes that since the 1970s, the Polish government, facing unpopularity, embarked on ``a build-up of Western debt, a substantial proportion of which was used for consumption," and eventually, ``Poland become the world's third-largest wheat importer"~\citep[pp.~89-90]{harold2003}.

It was also at the insistence of the East European states and 'with an eagerness bordering on na\"{i}vet\'{e}' that agreement was reached to trade in convertible currency from January 1991; Comecon was dissolved at its final session in June. 
(Millard, 1996, p. 208).

\subsubsection{Reform of the Banking System of the Late 1980s}

The Soviet Union's top leadership was at a crossroads by the mid-1980s as lasting stagnant economic conditions overwhelmed the domestic populace. Economic stagnation at that time in the country was not associated with Western-styled social problems of the time, such as job and income insecurity and price inflation; rather, it was associated with shortages of consumer goods (both durable and non-durable ones). The country's leadership had initiated reforms known as perestroika, in which the idea of reforming the monetary system topped the agenda. 

The above-mentioned system, in which the State Bank served as its foundational element, was deemed unsatisfactory. In general, reformers sought to re-introduce an environment where private individuals would follow profit-seeking motives. Amid widespread state ownership over means of production in all sectors of the official economy, the testing ground was to allow the cooperative form of business organizations to exist as an entry point for the private entrepreneurship to flourish on the official and regulated terrain of the economy. 

Hence, the country's economy, while it was prev\acp{iou}ly predominantly non-business, was opening up to a new institution of profit-seeking motives. At its core, this idea meant that cooperatives were non-financial businesses; however, there was a crucial legislative turn in the realization of this reform: the authors of the legislation on cooperatives included a clause that allowed cooperatives to engage in the banking business as well. 

\begin{quote}
[In 1988] Law on cooperation was published, containing a now-famous phrase that cooperatives could create banks.~\citep[p.~56]{krotov2008b}
\end{quote}

That legislative turn is referred to in the literature as ``famous" because it was at the time unexpected, being a radical solution to the contemporary banking specialists at the time. However, this move by the authorities was a part of the greater reform agenda for the banking system. The following account briefly summarizes that broader agenda:

\begin{quote}
Reforms and structural change of 1987-88 aimed to create two-tier banking system, where the commercial banking function of the State Bank would be handed over to the newly-created specialized [state-owned] banks. Thus, three new specialized banks were created: Agroprombank, Prombudbank\footnote{The Ukrainian author uses Ukrainian transliteration of Promstroybank.} and Bank of Social Capital Investment (Zhytlosotsbank\footnote{The Ukrainian author uses Ukrainian transliteration of Zhylsotsbank.}), which were to start lending to [state-owned] enterprises of major sectors of the economy. \dots At the same time, Law on cooperatives was adopted, which allowed creation of cooperative banks for servicing the credit needs of the cooperative enterprises, which could not access to the services of the state-owned specialized banks. In addition, state-owned enterprises gain a right to create their own banks. These cooperative and specialized banks differed in extent from specialized banks. First of all, their activities were less constrained: they could provide short- and long-term loans, accept cash currency and provide deposit services as well as carry out foreign exchange operations. Secondly, clients of these banks were not limited to one particular segment of the economy. Above said banking institutions could service private individuals and enterprises of any sector of the economy.~\citep[p.~121]{petryk2010}
\end{quote}

All in all, in the late 1980s, the Soviet Union's authorities committed wholeheartedly to the outright past-paced liberalization of the banking sphere. Instead of the State Bank system, the multi-layer credit system, where there were few specialized state banks (spetsbanks) and numerous new banks (created by cooperatives and state-owned enterprises called cooperative and commercial banks, respectively). 

The former group consisted of the so-called systemic banks with a network of branches and relatively large balance sheets. The latter group of banks had much smaller balance sheets. Also, the latter group of banks experienced less oversight from the authorities as a comprehensive legal framework of the Soviet Union's commercial business operations was simply lacking. 

Following  the concept of the hierarchy of money by \citep{bell}, the money hierarchy of the Soviet Union at the time had evolved rapidly from its past structure, in which State Bank's \acp{iou} (non-cash and cash) were the dominant means of discharging debts by non-financial sector units as well as households, towards one in which at least three distinct layers of \acp{iou} were in use (see Chart 9 below). 

Thus, state-owned, specialized banks used State Bank's \acp{iou} to discharge debts between each other. Then, smaller cooperative and commercial banks initially opened accounts with specialized banks in order to use spetsbank \acp{iou} to make payments. Then, households and enterprises of var\acp{iou} sectors used: (a) \acp{iou} of the State Bank in the form of cash notes to carry out different sorts of cash-related transactions (wage payments by enterprises and shopping by households); and (b) \acp{iou} of spetsbanks, cooperative, and commercial banks in the form of accounting records.

The liberalized environment of the operations of new banks spanned from local-currency cash operations to foreign-currency operations, including cash currency and nostro accounts on the books of non-resident banks. While state-owned, specialized banks practiced their usual restraint from risky operations, it was the new, smaller banks that operated more aggressively by carrying out operations that promised quick and sizable returns. 

Liberalization allowed some non-financial units of the economy, which were dissatisfied with the slow and risk-averse services of the specialized banks as well as with their interest rate policy, to take a shortcut by creating their own bank. As a result, some of the newly created banks, out of the group of cooperative and commercial banks, earned the reputation of being ``pocket" banks:

\begin{quote}
After bank reforms in 1988, many enterprises had been able to create ``pocket" banks that would extend [them] loans on easy terms.~\citep[p.~105]{woodruff1999}.
\end{quote}

It is crucial, too, to note the next development in the late 1980s reforms:

\begin{quote}
Price liberalization had been taking place in the banking sphere [first]. It started in 1988, when prices in many sectors of the economy were [still] strictly over-regulated.~\citep{krotov2008b}
\end{quote}

Together, these details help explain the bias of the reforms taking place in the country on the eve of its own demise. In short, it was the banking sphere that was liberalized first and quite rapidly. Price liberalization in retail goods and services took place only few years after the Soviet Union's disintegration, i.e., in early 1990s. 

By 1990, there were already 200 commercial banks operating in the entire country, while there was no regulatory framework for commercial banking in place. In some cases (if not in the majority of them), the newly created commercial banks obtained their banking licenses after a period of time following their inception date and after their operations were already in full swing~\citep[p.~498]{krotov2008b}. 

Additionally, lack of general regulatory oversight over the newly created industry of for-profit non-government banking business highlighted the fact that there were no regulatory requirements, neither for the management of the banks opening their doors for business nor for their operations. 

As a result, the commercial banking industry attracted a broad range of professionals to further their careers, including the likes of plasma physicists holding the local equivalent of a PhD degree from the country's top-notch university rising to the ranks of the commercial bank's chairman~\citep[pp.~497-504]{krotov2008b}. However, several banks featured incompetent management as well as management infiltrated by fraudsters and criminals~\citep[p.~23]{krotov2008b}. 

Bankers who used to run commercial banks in the late 1980s recall that there was no oversight to their operations whatsoever. And hence, true commercial banking attracted those who managed to carry out operations with complete disregard for the consequences -- they acted by the rule of thumb that what is not restricted is allowed~\citep[pp.~158-159]{krotov2008b}.

By that time, inside the Soviet Union, there were 15 republics in which local governments and opposition politicians competed with the central government in the Kremlin (known as 'the center') more or less aggressively. Russia was the largest republic in terms of output, followed by the second-largest economy of Ukraine, and then the third-largest of Kazakhstan. At the time, Russia's opposition leader, Boris Yeltsin, was a loud critic of 'the center.' In other republics of the Soviet Union political leaders raised their own objections to 'the center.' 

In the monetary sphere, Russia's opposition to the center-the leadership of the Soviet Union-had developed into a recognized form by the end of 1980s. Russia's leadership under Yelstin pushed even harder to challenge the 'center' by creating parallel bodies of authority. 

Eventually, in mid-1990, the Soviet Union's leadership moved onto another phase of banking sector reforms initiated already in 1987-88. That move was probably a dire bid to preserve the union of 15 republics, albeit under a new agreement, given the accelerating push by the republics led by Russian Federation for decentralization, if not outright sovereignty. 

In late 1990, the Soviet 's leadership carried out a broader package of swift reforms called The Five Hundred Days. It was a bid to revive fortunes the of the Soviet Union within the next 500 days. However, it did not succeed, and some had had doubts from the very beginning of the effort \citep{gaidar1996}. 

As part ofhis package, on December 11th, 1990, the Soviet Union signed into law two new acts: (1) ``On State Bank of \ac{ussr},"\footnote{See full text in Russian: \url{http://www.libussr.ru/doc_ussr/usr_17753.htm}.}  and (2) ``On Banks and Banking."\footnote{See full text in Russian: \url{http://www.libussr.ru/doc_ussr/usr_17754.htm}.}  In effect, the idea behind those laws was to provide a liberalized legal framework for the already growing commercial banking sector. 

The State Bank was supposed to be transformed into a central bank for the entire Soviet Union, overseeing a system of the 15 republic central banks, which the law prescribed to create.

Russia's leadership created the \acf{cbr}\footnote{\acf{cbr} was established on July 13th, 1990. The parliament of the Russian Federation (called Supreme Soviet) adopted the law on the Central Bank of the Russian Federation on December 2nd, 1990.} in a challenge to the State Bank of the Soviet Union. In 1990, it appeared that Russia's commercial banks were dealing with two central banking institutions: State Bank was still acting as the central bank of the entire Soviet Union, and newly created \ac{cbr} was gaining ground as the central bank for banks within the Russian Republic. 

Russian banking professionals describe that episode in the following way: they had excessive reserve balances with both State Bank and \ac{cbr} \citep[p.~491]{krotov2008b}. This outcome was thanks to the rivalry between State Bank and \ac{cbr} in attracting banks to settle accounts with other banks on the own balance sheets. 

Competition between political leaders of 'the center,' led by Gorbachev, and of the Russian republic, led by Yeltsin, spilled over into the monetary sphere in the following manner: the latter required the banks, including the spetsbanks, within jurisdiction of the Russian republic, to register themselves with the \ac{cbr}. To encourage this process, the Russian authorities lowered thetax rate versus the tax rate applied by Soviet Union law and exempted the re-registered banks to pay the latter. 

Local authorities in other republics followed suit: for example, Ukraine, as the second-largest republic in the Soviet Union by output, had adopted its own banking legislation according to which the \acf{nbu}, its central bank, was established on March 20th of 1991.

\begin{quote}
In the summer and fall 1990, \dots Russian authorities launched a massively successful campaign to reregister subdivisions of the Soviet banking system as new commercial banks subordinate to Russian structures, in very short order destroying most of the remaining instruments of monetary control.~\citep[p.~74]{woodruff1999}
\end{quote}

The \ac{mfo} payment system, too, evolved along the changes the banking sector had been experiencing. At that time, the State Bank was shifting to function as a central bank presiding over the state-owned specialized banks. However, that shift brought temporary bottlenecks to the payment system. It was an extraordinary event, as recounted by the state banking practitioners of the time in \citep{krotov2008b}, which revealed that specialized banks were not up to smooth payment operations.

There were two issues related to the above delays. One was inexperience. Out of three spetsbanks, only two-namely, Agroprombank and Zhylsotsbank-were formerly parts of State Bank that participated in the MFO payment system. Hence, their staff had experience in payments processing. One bank, Promstroybank, formerly a part of State Bank, did do that. Hence, its staff lacked that experience. The second was an increased number of the bank branches that participated in the MFO system. As a result, in 1988, there were delays in payments lasting months. 

The switch from the State Bank only banking system to one in which the former system was divided into the State Bank as central bank and spetsbanks as banks serving clients was somewhat stressful for the system itself and its clients \citep[pp.~20-21]{krotov2008b}. That development revealed that future modification of the payment system was required.

The above-mentioned change in the MFO system took place hand in hand with a change in the state budget procedures. Thus, the previous practice of government expenditures being made independently of revenues (as discussed above) was consciously abandoned by the reformers and was substituted for the process by which budget expenditures were now carried out according to the size of incoming budget revenues. In other words, now state budget expenditure had to be pre-financed or to be created from previously accumulated funds.

The aim of that switch in procedure, as explained by Deputy Head of State Bank Vyacheslav Zakharov, was ``to increase responsibility of the state financial bodies during the budget execution. However, that responsibility was about the mere under-financing by the [state] budget of the planned expenditures when [its] revenues underperformed."~\citep[p.~ 19]{krotov2008b}.

Hence, a balanced budget approach was an instrument of the late 1980s reformers in the Soviet Union. The above observation by the former State Bank executive supports the claim that from the very beginning the 1987-88 reforms to the banking sphere of the Soviet Union aimed at making central banking independent from the government \citep[p.~60]{fedorov1994}, in what is generally regarded now as an austerity-biased approach.

\cite[p. 40]{wallich1992} shows that already in 1988-90 the Soviet Union budget was shifting in terms of expenditure structure, while expanding in nominal terms,\footnote{In 1990, total expenditures were at 485.6 billion of rubles (50.6\% of GDP), up from 445.9bn (51\%) and 465.1bn (49.5\%) in 1988 and 1989, respectively.} in such a way that accelerated recessionary tendencies instead of attenuating them. Thus, the items classified as ``National economy" and ``Investments" were heading south in nominal terms in the 1989-90 periods if compared with their 1988 levels. Together these two items of the state expenditure amounted in 1988 to 279.4 billion of rubles, which translated to a 31.9\% share of GDP and more than half of total expenditures. Next year, which was 1989, their total declined to 268.1 billion of rubles or to 28.5\% of GDP, which was still more than a half of the total expenditures. In 1990, declined continued as their total now stood at 230.4 billion rubles. Overall, cuts in the ``National economy" expenditure amounted in two years of 1989-90 to 3.6 percentage points of GDP, while cuts in the ``Investments" expenditure were of 4.3 percentage points of GDP.

When in the final year of the Soviet Union's existence its banking system witnessed the appearance of \ac{cbr}, which rivalled the State Bank, commercial banks started to open accounts with \ac{cbr}. As was mentioned above, \ac{cbr} also lured commercial banks by tax incentives. In addition, Russia's legislative body (the Supreme Soviet of the Russian Federation) was forcing them to re-register under the Russian republic law and ignore the Soviet Union's law. 

The banks in Russia enjoyed the \ac{cbr} accommodating them with reserve balances. And, these banks started to settle payments between each other on the \ac{cbr} balance sheet, according to the above-mentioned account by Zakharov. He also mentioned that in addition to central bank settlements, ``the commercial banks carried out payments via correspondent accounts opened at \ac{cbr} and then between banks as well as via settlement organizations they created themselves" (Krotov, 2008b, p. 19). 

This description, as it appears, was about three ways of settling payments the Russian banks had access to. First, via correspondent accounts, they opened at \ac{cbr}. Second, via correspondent accounts they opened with each other. Third, by clearing houses established by a group of banks. 
As far as the second option is concerned, another former State Bank official, Dimtry Tulin, who holds a senior position in \ac{cbr} today,\footnote{He is first deputy of the governor of Bank of Russia, see: \url{https://www.cbr.ru/about_br/tulindv/}.} was discussing that option albeit as a theoretical possibility \citep[pp.~42-43]{tulin2012}. 

Conceptually, this second option of settling payments between two legally independent commercial banks (in the post 1987-88 reforms world) was analogous to the MFO payment system (in the world preceding the 1987-88 reforms) in the sense that two balance sheets of banking units were recording mutual obligations simultaneously. In both cases, there was no immediate need for the central bank's IOUs (reserve balances). 

The difference was in the nature of the backstopping of those balance sheets of the banking units. In the latter ("old," unreformed) system, the balance sheets of the banking units belonged to the branches of the State Bank, which by an established procedure periodically netted out the outstanding balances of inter-branch indebtedness. 

In the former (``new," reformed) system, those two banking units were bonded together by mutual indebtedness (one bank was holding an IOUs of another bank), which had to be cleared either by an opposite transaction or by delivery of IOUs of the central bank (reserve balance).

Together, these two accounts---by Zakharov and Tulin---point out that innovation by commercial banks, or rather utilization of all known possibilities by them, given that due oversight by the authorities was lacking, was taking place, allowing them to realize their profit-seeking motives.

During the heated debates between State Bank and \ac{cbr} over supremacy, which were mentioned above, the term ``ruble has boundaries" was used, most likely implying the boundaries of Russia and not of Soviet Union (Krotov, 2008b, p. 160). 

However, from practical grounds, that assertion was mistaken as at least for the ruble cash currency, which was produced solely by a cash production facility physically located in Russia. Still, in 1990, all 15 republics used cash rubles, but that practice did not survive into the next year.

All in all, reforms of late 1980s did not safeguard the economy of the Soviet Union from disintegration.

\subsection{The Early 1990s Period}

During the period of 1990 and 1991, both the Soviet Union itself and its ongoing reformation of the monetary system disintegrated along the borders of the former 15 republics. Then, these republics became newly independent states. They formally maintained a political dialogue via a new organization called the Commonwealth of Independent State (CIS). 

There, the monetary sphere underwent the evolution of a new impetus. The Soviet Union's money unit of account, the ruble, had its role. During the period of rule of the Soviet Union, the ruble was a share unit of account among all the republics. 

Back then, the unit of account literally stated that its name (ruble) had varying titles in other native languages of the nations other than Russian. For example, it was manat for the Azerbaijan republic in Turkish language, it was karbovanets for the Ukrainian republic in Ukrainian, and so on (the Baltic republics had it just as the ruble in their local language transliteration). 

Hence, when the Russian republic declared political sovereignty, retaining the ruble as the national money unit of account, in other parts of the former Soviet Union  there was an automatic own money unit of account as implied by the definition of the Soviet Union's ruble.\footnote{This institutional detail was pointed out during author's interview of former deputy governor and chief accountant of National Bank of Ukraine (Kravets, 2020). Broadly defined, similar detail was articulated to by Mas (1995, p. 487), see footnote 7.}

In technical terms, shifting to their own money unit of account for all republics except Russia proved problematic from the very beginning. As far as Russia was concerned it retained the ruble as its unit of account, and it owned and operated the only cash production facility in the entire Soviet Union.

At least the Ukrainian authorities accepted this as a starting point of the Ukraine republic emerging as a newly independent state, or rather regaining its political and monetary sovereignty lost in the early 1920s. So, in Ukraine, once the Soviet Union ceased to exist, the money unit of account became the  karbovanets, not ruble. However, after some delay, the full introduction of Ukraine's own money unit of account didn't take place until November 1992, while its central bank was launched in March 1991.

Yet in 1991, during the final (and quite chaotic) year of the Soviet Union's formal existence, its legislative body adopted the ``Law of valuta regulation"\footnote{See full text in Russian: \url{http://www.libussr.ru/doc_ussr/usr_18302.htm}. In the countries of the former Soviet Union, there has been long-standing tradition of referring to foreign-currency liabilities such as cash and non-cash ones as ``valuta."} (on March 1st, 1991). It formally acknowledged what was going on in the increasingly commercialized banking sector all over Soviet Union???that for private businesses as well as households, it legally allowed operations with liabilities denominated in foreign currency. In addition to that, Soviet Union regulations of 1991 were relaxed so that enterprises that exported their produce were allowed to manage foreign-currency proceeds on their own. 

It should be noted here that such liberalization of the monetary sphere was taking place amid sizable foreign debt that the Soviet Union had accumulated in the past and that were due to be serviced. So, the Soviet Union's authorities, while allowing export-oriented enterprises to take care of their own proceeds, required such enterprises to surrender a share of the foreign-currency proceeds in order to make those payments on external debt. 

This showed that authorities were trying to encourage entrepreneurship among industries that participated in world trade that would yield larger export proceeds. Effectively, the former ``Chinese wall" between ruble and foreign currency operations (the regime of inconvertibility of ruble IOUs into foreign-currency IOUs either domestic or foreign) was eliminated.

By early 1992, State Bank had dissolved, and the central banks of the newly independent states (excluding three Baltic countries: Estonia, Latvia, and Lithuania) were cooperating on its very basic legacy, i.e., in settling outstanding mutual obligations among the economic units of these economies.

Indeed, the wave of central bank creation from the former local branches of the giant State Bank took place in all 15 republics very shortly after they declared their own political sovereignty from the crumbling Soviet Union. At least the two largest republics, Russia and Ukraine, did just that. In Russia, the time span between the two decisions amounted to one month. In Ukraine, it took less than nine months.

While the introduction of the national money units of account in the rest of the republics except Russia took some time; for a relatively short period of time, the former Soviet Union territory turned into a ruble-zone, where the ruble was a shared money unit. Ukraine switched to own its money unit in November 1992, and it took more than one year and a half to make that happen. Payments were taking place according to all available options described in the previous section (see p.24).

The main development there was most of the economy of the former Soviet Union was populated by the economic units such as enterprises and non-productive organizations that were already established. New commercial businesses were just springing up. As the economy of the former Soviet Union was delineated among 15 new independent states, it turned out that these nations had a sizable foreign trade component???by some measures, about ~70\% of the domestic output had delivery destinations and correspondent payments abroad. 

Those producer-consumer relationships, which previously were within one jurisdiction and their payment transactions being handled by a unified monetary system of the State Bank, now had to be carried out and handled by a multi-layered monetary and payments system, spanning commercial and central banks of different jurisdictions. 

If previously the State Bank's charge (interest rate on credit) on the payments was regulated by the authorities, meaning being kept stable (nearly fixed) and in the low-single-digit area, now the commercialized banking system had become less predictable, with interest rates on bank credit being higher. 

Despite this change, the established trade and payment system, by virtue of technological and social practices (e.g., operationally, industrial production could not be left idle even for even a few hours, as consequences would be catastrophic) the working population that served such an industrial production facility could not sit idle, either. 

So, all kinds of inputs into production required payments to be made, even if they were now cross-border payments. So, suppliers and consumers among the enterprises continued production and accounting of debts and credits on their balance sheets, and by the previous custom established, now commercialized (previously state-owned) banks to become new, smaller commercial banks for making transactions.

In 1991, sovereign Russia's official leaders-then newly elected President Boris Yeltsin and soon-to-be appointed new (albeit acting) Prime Minister Yegor Gaidar-have continued the general line of the reforms of 1987-88 that aimed at the furthering of market reforms, separating the central bank from the government, and establishing the financial markets. Indeed, they were especially focused on furthering the market reforms, particularly if one considers Gaidar as one of the young economists of Gorbachev's team of reformers, who were pushing the reforms since the mid-1980s by the only means they had access to~\citep[pp.~59-70]{gaidar1996}. 

Gaidar himself acknowledges in his 1996 memoir that he switched from Gorbachev's team to Yeltin's, as the latter increasingly became more politically relevant, gaining more ground, while the former's prospects were increasingly fading in the tumultuous years of 1990-91.

The trajectories of economic ideas being developed both by Yeltsin and Gaidar were such that they aimed to challenge the established orthodoxy of current thought in the Soviet Union. The former was a pure politician sensing that new course had to be realized. The latter was pure theoretician, with leanings towards neoclassical economic thought.\footnote{Back then economists of ex-Soviet Union referred to this school of thought as ``modern economics." This term survived into present, see (Aven, 2014).} Both happened to grasp economic matters in the following ways:

\begin{quote}
Even before the Soviet Union collapse [future President of Russia] Yeltsin sought out US officials, arranging to meet quietly with US Federal Reserve Board chairman Alan Greenspan, former chairman Paul Volcker, and New York Federal Reserve president Gerald Corrigan at the White House in June 1991 in order to learn more about the Federal Reserve. Yeltsin and Corrigan formed a quick bond, facilitating early cooperation between the two sides.~\citep[p.~185, emphasis added.]{johnson2019}
\end{quote}

The new Moscow leaders were eager to embrace the capitalist world. The first Prime Minister, Yegor Gaidar, had secretly been learning Western-style economics along with a group of young bright Moscow colleagues.\footnote{In the 1980s, there were several groups of Soviet economists of younger generation that worked on the agenda of reforming the economy of USSR. They gathered informally and formally as well as inside the country and outside. As far as the outside venues are concerned, "[the Vienna-based \ac{iiasa}] was the only place where Soviet scientists could freely and openly deal with their foreign counterparts. At that time there was zero opportunity to go good universities outside of Soviet Union. There was field known in Russia as modern economics. \ac{iiasa} was the place where professionals could meet. There were tens or hundreds of Russian scientists of different ages came [to \ac{iiasa}]. It was extremely important this exchange and this influence. It was not only scientific knowledge sharing, but it was also culture shock, when I came here first time in 1975, because it was so different. \dots We arranged research groups on different subjects [of reforming the Soviet Union economy] such as international trade, macroeconomic stabilization, privatization, etc. We had several conferences every four-five month with serious discussions and [it became a blueprint on reforming USSR]. In the end the book was published by Yale University Press and it was called \textit{What Is To Be Done?} [see \citep{peck1991}] And in \textit{de-facto} it was a really plan." \citep[emphasis added]{aven2014}. 

Indeed, what became Chapter 2 of \citep{peck1991}) it originally was a policy-making memorandum prepared by the \ac{iiasa} study group on the Soviet Union economy. It was translated into Russian and "was given to senior officials, including President Gorbachev, in December 1990" (ibid, p.19). 

Being part of \ac{iiasa} was a sign of prestige and professional achievement for the economists in the Soviet Union. Just having a possibility to be based in Vienna (Austria) and participate in the \ac{iiasa}'s research projects and discussions with economists from the Western countries was considered as an intellectual luxury. Denial of such a possibility from the Soviet Union authorities was considered as a considerable failure in the professional career \citep[p.~225]{aven2015}.} \citep[p.~113, emphasis added]{wyplosz2003}.

In his own words, Gaidar described his economic leanings by refereeing to the works of Austrian-British economist and philosopher Fredrich von Hayek and of the Hungarian economist Janos Kornai, who is credited with a soft budget constraint thesis as the fundamental shortcoming of a socialist economy~\citep{gaidar2006}. 

Fascination with Hayek by relatively young economists in that period was rather common in that part of the world. The following observation by Crockett on the think-tanks' impact on economic intellectuals is rather pertinent:

\begin{quote}
When the Berlin Wall came down in 1989, there was an army of committed, international economic liberals reared in the Hayekian tradition, armed with clipboards and portable phones, waiting to move into Eastern Europe and the disintegrating Soviet Union to convert their ailing economies to the virtues of capitalism. \dots The 'Liberty and Society' seminars run by IHS and 'The International Workshops' run by ATLAS\footnote{\cite{cockett1995} explains that IHS, Institute of Human Studies, and ATLAS were an outgrowth of \ac{mps}, a circle of liberal intellectuals spearheaded by Hayek.} became international in character, with a strong East European and Russian\footnote{Russian here must mean related to the Soviet Union, a usual shortcut reference met in the English-language literature of the time.} representation, schooling the future economic liberal thinkers and politicians of the twenty-first century. \citep[pp.~307-308]{cockett1995}
\end{quote}

Gaidar's reforms concerned the entire economy of now independent Russia, not only its banking sector. Moreover, Gaidar's government served as a benchmark to follow by other former republics of the Soviet Union. A long-standing notion had developed back then that Russia was well ahead of the other republics in terms of instituting market reforms.

At the authorities of Russia and of the other newly independent states of the former Soviet Union imposed a tight money policy on the domestic industrial sector \citep{viksnins1997,woodruff1999}. Russian reformers relied on the proposition that supposed status quo in the banking system at the time is one of major causes of economic failure. Hence, it had to be fixed. That status quo was described in \citep{peck1991} as this:

\begin{quote}
The banking system accommodates the demands of enterprises in a way that allows ballooning credit and no constraints on enterprise spending. \dots Banks must refuse to issue credit to enterprises that have poor economic prospects. \citep[pp.~20,26]{peck1991}
\end{quote}

For example, in Latvia, the central bank, formerly a local subsidiary of the State Bank of \ac{ussr}, effectively withdrew its basic credit accommodation of its domestic industrial enterprises sector in 1992, which were primarily of a military-defense profile and used to supply their military goods and services output to Russia. In part, such posturing was nationalistic, as Latvia, a newly independent state, (along with others in the region, too) viewed its membership in the \ac{ussr} as essentially military occupation by Russia:

\begin{quote}
The Bank of Latvia, even as a successor to the Gosbank in Latvia, had to explain to [large state enterprises in Riga] that it could not guarantee payment by the clients of these enterprises. Furthermore, it would certainly not provide credit to Soviet military-industrial enterprises.~\citep[p.~30]{viksnins1997}.
\end{quote}

In Russia at the time, the government, under the direction of liberal economist and acting Prime Minister Yegor Gaidar, acted in a similar vein, albeit from a purely ideological standpoint of mainstream macroeconomics:

\begin{quote}
[In 1994] Gaidar welcomed expressions of suffering [on the back of government's policy of monetary restrictions] on the part of defense industrialists as signs that the economy was indeed moving away from unnecessary production.~\citep[p.~94]{woodruff1999}.
\end{quote}

Such zealous application of the market reforms meant the following. The long-standing and well-established procedures as practiced by State Bank since 1930s and through best part of 1980s had to be put to rest. The above-mentioned soft budget constraint that was associated with this credit system had been particularly under targeted front attack by the reformers. Those practices included credit accommodation at boringly predictable rate and payment settlements with netting on the enterprise level (via bureaus of mutual settlement). Instead, gross payment settlements (later in real time) with each payment had to be pre-financed by competing banks at varying cost of credit. Effectively, the prescription for reformers was: ``real interest rates (equal to money interest rates less the rate of inflation) must be positive" \citep[p.~108]{peck1991}.

So, when price liberalization in the banking sector (meant interest rate liberalization for commercial banks) started in 1988, the price liberalization in the consumer goods and services started in 1992. There was about a four-year time lag between each other. And it is rather strange to realize that high-profile proposals\footnote{It was indeed a high-profile policy blueprint as it was developed under active cooperation between Western and Soviet economists. The idea behind that cooperation was to table or turn over these proposals to the top leadership of USSR, including Gorbachev the leader of the country at the time. That task was fulfilled. See footnote 17 above.} on reforming of the Soviet Union economy, which were written in 1990 under umbrella of Vienna-based \ac{iiasa}\footnote{\ac{iiasa} stands for International Institute of Applied System Analysis, see \url{https://iiasa.ac.at/}. Please, refer to the footnote 17 above that highlights \ac{iiasa} role in the \ac{ussr} reform efforts.} and featured contribution of William D. Nordhaus, was assuming that any sort of reform left the banking system still untouched. Those proposals, see \cite{peck1991}, consisted of statements like ``it is neither possible nor necessary not necessary in the short run to privatize the banking system in order to have tough credit policies" (ibid, p.~31). ``[T]he existing banking sector" was considered as still state-run (ibid, pp.~108-109).

This blueprint on how to transition the formerly state-run economy to the market-based economy had two foundational and mutually related ideas (or concepts) in mind: (a) ``monetary/inflation overhang" and (b) ``soft-budget constraint". This is what later was termed as ``excess demand policy narrative"~\citep[p.~228]{gabor2012}.\footnote{\cite{gabor2012} explains that such a narrative "came to guide policy decisions [the of formerly planned economies] because it articulated well with the priorities of international financial institutions influential in the transformation" of those economies. Those priorities are formulated by the author as "financialization of the formerly planned economies" (ibid, p. 227-228). The process of "financialization," in general terms, has been termed as "rise of money manager capitalism" among the \ac{mmt} economists of the ``Kansas City" strand. This tradition stands on the shoulders of Hyman Minsky, who developed the stages of capitalism approach \citep[see][]{wray2009,wray2020}.  Emerging by the 1980s, money manager capitalism by itself is the second phase of the globalized finance capitalism. Its first phase lasted from late 19th century through early 1920s and ended up in the grand crisis of Great Depression and profound reform of finance by New Deal policies of 1930s. The second phase of globalized finance capitalism, lasting for a about 50 years now, featured a crisis of grand proportion, which is the crisis of 2007-09 named Global Financial Crisis. Nevertheless, that second phase did not have an equivalent of the New Deal. It has extended into nowadays.} Back then, this kind of policy narrative  considered the planned economies of \ac{ussr} and its satellites as being mired in the inefficiencies-such as shortages of the wide range of consumer goods-due to lack of the (a) established market clearing mechanisms and (b) hard budget constraints on the side of enterprises. 

The key assumption was that only the household sector of the economy experienced hard budget constraint as incomes of the private individuals were primarily wage incomes, while the industrial sector in general did not experience such a constraint, i.e. of a hard variety.

By 1980s, the shortages from non-durable to durable goods become quite severe and seemed endless. Many households of both rural and urban areas with appropriate climate\footnote{Ukraine is most fit to this description.} practiced self-sufficiency in food\footnote{The habit they endowed from their parents and previous generations.} by laboring not only on their jobs but also in the gardens (small land plots they were allocated by the authorities for that purpose). Before acute shortage period of late 1980s and very early 1990s, households year after year practiced growing such staple food as potatoes, cucumbers, tomatoes. It evolved into growing own cattle to self-sustain meat consumption on the back of the acute shortages of late 1980s and of the sharp and broad income decline in the early 1990s. At the same time, the industrial sector, according to the narrative, was operating under ``soft budget constraint."

The concept was coined in the 1970s and derived from Hungary, a planned economy in that time. It gained wide acceptance including international financial institutions (\ac{imf} and \acf{wb}) that were set to deal with reforms of the ailing economists of socialist bloc in 1980s. In its original formulation it stated that a firm, a state-run enterprise operating in the planned economy, ``it must cover its expenditures out of its initial endowment and revenue" and ``[i]f it fails to do so and a deficit arises, it cannot survive without intervention"~\citep[p.~1097]{kornai2003}. Under ``intervention" it was understood that a supporting organization does not allow a financial failure. These could be one or several state agencies as well as one or several credit institutions which extend a lifeline of any form: ``support," ``rescue" and ``bailout" (ibid, p. 1097), \citep[p.~6]{kornai1986}. Firms faced an environment of \acf{hbc}, when they do not receive support. \acf{sbc} provided firms with an environment of continued support by all forms. Both \ac{hbc} and \ac{sbc}, according to the concept, were two extreme sides of the continuum of the budget constraint in terms of stringency. In real situation, budget constraint could be of a certain degree and hardening or softening over time (ibid, p. 7). By late 1980s reformers of the \ac{ussr} widely adopted the \ac{sbc} approach and the assumption that enterprises of the Soviet Union operated under financial conditions that were extremely close to the extreme end of the above-mentioned continuum, i.e. on its soft end. 

Such reasoning fit well to explaining observed shortages of consumer goods, which progressed into an acute form, and their prices being creeping up in retail outlets (such as bazaars) that did not belong to the state-run retail network of stores. Because households faced \ac{hbc} while enterprises faced \ac{sbc}, such an environment induced inflationary outcome. Households did lack access to goods hence their demand was in excess of the supply creating pent-up demand. They also lacked a wide range of options of financial instruments of saving. Hence, there was a monetary overhang: ``purchasing power in excess of the total supply of goods and services" \citep[p.~234]{gabor2012}. Industrial enterprises ``had no price constraints" (ibid). They faced plan targets instead. This created ``an over-fulfilment incentive system generat[ing] an environment where state-owned enterprises continuously bid against each other for [material] resources" (ibid). 

Hence, the reformers of the \ac{ussr} too accepted the policy solution - that is of budget constraint hardening. Moreover, some high-profile reform proposals considered hardening in more acute form than original theoretical concept. Thus, the above-mentioned \ac{iiasa} policy blueprint discovered that entire balance sheet of the banking system of \ac{ussr} as of December 1990 consisted of credit to enterprises that amounted to 41\% of total assets and deposits of enterprises were at 26\% of total liabilities side of the balance sheet, see Table 1 below. Hence, the proposal was to cancel debts of enterprises owed to the banking system, at least to those who would face outright bankruptcy given their past accumulation of debt due to loss making operations. So that it would allow them to start operating in the market system ``with a clean slate"~\citep[p.~155]{peck1991}. At the same time, the proposal considered that some enterprises accumulated ``substantial [deposit] balances that would in effect allow them to operate with a soft budget constraint by drawing down their [bank] balances" (ibid). Hence, these were to be cancelled too (alongside with debts). In addition, there was an idea to harden even further the budget constraint for the enterprises. It was raised to eliminate any room for soft budget constraint. Hence, the proposal put an eye on the ``accumulated excess inventories which can be sold off [by the enterprises]" (ibid). And this way that enterprises would go around the imposed hard budget constraint. This idea did not receive further elaboration, however.

\begin{table}[ht]
\captionsetup{width=.85\linewidth,labelfont=bf}
  \vspace{.2in}
  \centering
  \begin{tikzpicture}
  \draw[help lines,white] (0,0) grid (16,4);
        \matrix (m) [matrix anchor=north, matrix of nodes, nodes in empty cells,
             nodes = {minimum height=.2in},
             column sep=1em,
             row 2/.style={align=left, font=\bfseries},
             column 1/.style={anchor=base west,{nodes={text width=1.7in}}},
             column 2/.style={anchor=base east},
             column 4/.style={anchor=base west,{nodes={text width=1.7in}}},
             column 5/.style={anchor=base east},
            ] at (8,4)
        {
         & & & & \\
        Assets & & & Liabilities & \\
        Credit to government	& 524.9	& &Deposits of households  & 380.2 \\
        Credit to enterprises	& 367.4 & &Deposits of enterprises & 235.8 \\
        Credit to households	& 10.6  & &Currency in circulation & 133.0 \\
			                  &       & &Other items\textsuperscript{\dag}             & 153.9 \\
        Total                  & 902.9 & &Total                    &	902.9 \\
         };
        \node[fit=(m-1-1)(m-1-5)]%
            {\textit{\MakeUppercase{State Bank \& Other Banks}}};
        \draw[thin]  (m-2-2.south -| m.west) -- (m-2-2.south -| m.east);
        \draw[thin]  (m-2-3.north east) -- (m-7-3.south east);       
  \end{tikzpicture}
 \caption[Consolidated Balance Sheet of the Soviet Union Banking System, December 1990 (in billions of rubles)]%
  {Consolidated Balance Sheet of the Soviet Union Banking System, December 1990 (in billions of rubles)\par \vspace{.05in}Note: Names of the balance sheet items are original. \textsuperscript{\dag} inter-enterprise and inter-bank settlement accounts, budgetary transit account, and misc. These accounts reflected balances of the \ac{mfo} payment system.\par \vspace{.05in} Source:~\citep[p.~155]{peck1991}}
  \label{tab:bs_state_bank}  
  \vspace{.2in}
\end{table}

The basis of the reformers assumption as it appears did not correspond with practices established by Gosplan and Gosbank (State Bank) on assigning credit limits on enterprises. That practice had elements of discipline and flexibility. Only favored sectors had ``open ruble accounts" with State Bank, which could be considered as some form of extreme soft budget constraint. It did not correspond with established lexicon among the State Bank employees of high, middle and low ranks who predominantly had by habit a thinking of baking through loanable funds theory\footnote{In 1995-2000, the author of this dissertation worked as a middle-ranking officer in the branch office of Prominvestbank, at the time one of the top three banks by assets. The branch operated in a small industrial town with about 25,000 inhabitants. Prominvestbank was one of three formerly state-owned specialized banks that were re-organized in late 1980s and the privatized in early 1990s right after the collapse of the Soviet Union. I started there at the entry position in the information technology department and rose the career ladder to become an account manager overseeing (a) export-import financial operations of the enterprises, who held accounts with the bank, (b) household operations with foreign currencies in cash and non-cash forms, (c) managing local-currency reserve balances with a corresponding bank (which by design was head office of Prominvestbank). All key personnel (especially of older generation) of the branch had previously worked in the system of State Bank and they obtained respective specialized education before joining work force. They all talked and thought about-what is now called reserve balances with a correspondent bank-as "credit resources". However, operationally, those balances were just used for either for (i) daily payment settlements (both within Prominvestbank system of branch offices and outside it, i.e. with other commercial banks), (ii) for lending to other branches of Prominvestbank as the branch office I worked for used to end nearly every day with excess balances, (iii) for meeting biweekly reserve requirement ratio.} (Kovtun, 2020). That lexicon and habit of thought suggest that constrained credit creation was institutionalized too by college and university teaching of future banking officers.

From the perspective of modern-day development in the economics, the concept of soft budget constraint is mentioned explicitly only in the relation to China, the only significant country that still qualify as an economy still led by the communist party and still retains features of the planned economy type. At the same time, \ac{gfc} turned some major business units of the US, the leading market-based economy of the world, into too-big-to-fail units. A wave of bailouts spread from the US into the \ac{eu} in the wake of \ac{gfc}. Strangely enough, the soft/hard budget constraint concept was not invoked. As slowdown of the productivity growth became an issue in the developed part of the world partly because the post-\ac{gfc} recovery was slow the economics literature produced a series of publications under the unifying theme of "zombie firms". By widely accepted definition, these are the firms that are ``at least 10 years old and whose profits (EBIT) are insufficient to cover interest payments"~\citep[p.~6]{borio2018}. It turned out that ``zombie firms on the rise and surviving for longer" (ibid). Careful examination of the ``zombie firms" concept suggest that, broadly speaking, it is just another name of the ``soft budget constraint" concept that was useful since 1970s and till very early 2000s. The difference is that policy proposals stemming from the ``zombie firms" literature do not end up with significant hardening of the budget constraint conditions (as it was the case back in late 1980s and early 1990s).

Moreover, economists of Post Keynesian tradition from the quite early criticized the concept of hard budget constraint as one having no relevance at all if one bases analysis on the endogenous money view. As the following quote borrowed from the 1991 article states, both types of the economies (capitalist or socialist) could be characterized with such monetary conditions that are, applying \ac{sbc} terminology, on the softer side of the budget constraint continuum. 

\begin{quote}
If credit money is truly endogenous in a capitalist system, then a hard budget constraint does not exist in capitalist or socialist systems; i.e., given the cost of obtaining credit from the banking system, firms have a soft budget constraint in both economies.~\citep[p.~330, emphasis added]{szego1991}
\end{quote}

The modern money theory literature by utilizing the sectoral balances approach argues that it is thanks to the firms' decision to deficit finance their activities the other part of the economy (let's call it non-firms sector) accumulate financial claims on the firms sector. The usual sectoral balances presentation differentiates between three sectors of the economy: government, domestic non-government and external sectors. One may make one step further and analyze the economy through four sectors by sub-dividing the domestic non-government sector into two sectors: households and domestic businesses. According to policy chosen in the late 1980s and in 1990 to reform the already struggling economy of the USSR, two sectors of the economy (government and enterprises) had to change so that their net financial flows simultaneously turn towards surpluses. Effectively, the reforms proposals based on the soft budget constraint concept both lacked the stock-flow consistency.

\subsection{On the Eve of Default}

As liberalization of the consumer prices was introduced in 1992, another policy reform took place - the authorities (the central bank with an agreement of the government) turned to the positive real interest rate maintenance regime in a bid to suppress inflation.

The policy of bringing interest rates to a rational, positive level became a fact only on May 22, 1993, after the famous joint economic policy declaration of the government and central bank. (Fedorov, 1994, p. 61)

The period of 1991-95 was catastrophic for Russia alone and for the whole region that previously was Soviet Union. Real GDP was declining persistently year after year. Hyperinflation engulfed the whole region as measured by consumer and producer prices indices. The charts (Chart 12 and Chart 13) below that depict the monthly consumer inflation in the month-on-month terms and GDP in the first half of the 1990s.

A shift towards austerity on the level of the central government was already taking place in late 1980s and technically speaking USSR entered into recession by end of the 1980s.

As it was discussed above, the 1987-88 reforms had introduced liberalization into the monetary sphere and austerity bias into the operations of the state budget expenditures that culminated with substantial cuts in the state budget's major expenditures (as discussed on p. 23). The Gaidar-led reforms double-down on them:

In the Russian republic a government of radical young economic reformers was installed in January 1992, who jettisoned the old command structure, freed most domestic prices, removed obstacles to foreign trade, cut the military budget to a fraction of its earlier level, abolished state trading, legalised all forms of private trading, and began a process of privatisation which eventually sold off most state enterprises at knockdown prices. 
(Maddison, 2001, p. 157), emphasis added

The above-mentioned bank balances held by enterprises as of December 1990 (see Table 1, p. 35), assuming that in large extent they were denominated in ruble, were effectively deflated as their purchasing power dropped on the back of hyperinflation. Bank balances held by households just followed suit. Those few who, well ahead of others, was able to convert ruble cash or ruble-denominated resident bank balances\footnote{General term to depict balances held by households or enterprises at domestic bank (state- or privately-owned} into \acf{usd} cash or \ac{usd}-denominated non-resident bank balances, were more successful to withstand the crisis than the rest. The latter were forced indeed to start from the clean slate. By different estimates by the middle of 1990 the level of dollarization (volume of the \acf{usd} cash held by households of \ac{ussr}) ranged between four tenth of a billion to four billions of US dollars \citep[p.~92]{peck1991}. That was just a beginning of the gigantic wave of dollarization of economic relations in the former Soviet Union area. Thus, in three years in Russia alone it was reported a sizable rise of bank operations denominated in foreign currency (mainly the \acf{usd}):

In the aggregate balance sheet figures of Russian commercial banks at the beginning of June 1993, [balances associated with] foreign transactions accounted for 48.1 per cent of assets, and 40.4 per cent of liabilities (Boffito, 1995, p. 127)

The above-mentioned account did not consider US dollar cash holdings by households. And those were on the quick rise too as hyperinflation raged in 1992.

Commercial banking in Russia, and in other former parts of USSR, become involved in the foreign-exchange and money market operations that promised quick and high return. What is called a "normal" business line by banks of lending to businesses was secondary. It was high risk to both sides of this market-based relationship. Yet, some raised the question "[h]ow effective and reliable is the banking system?" \citep[p.~61]{varese2001} and the answer was quite illustrative:

Entrepreneurs interviewed in 1992 in St Petersburg reported that banking delays in payment between different cities of the Russian Federation were as long as four or five months. In several cases, payments initiated in the first quarter of 1992 were not cleared by November 1992. Problems with inter-republic transfers were even more severe. Some entrepreneurs reported that they routinely travelled to collect payments directly from their customers in other republics [of ex-Soviet Union]. (ibid) 

\begin{quote}
\dots That is why banks operate in an unofficial way. They assign the task of recovering debt to an especially reliable and competent \textit{brigadir},\footnote{``Brigadir" is a Russian word of prisoners' slang origin, and it stands for a leader of a mob or of a group bullies hired to solve a certain task.} who adjusts his tactics depending on the situation. For example, in one case, he simply scares the person; in another, he physically 'strains' him. The debtor is taken out of town (to the cemetery, to the forest, or by the river), beaten, threatened, and beaten again. In another case, the debtor is forced to apply for credit to another bank, in order to return money to the bank who hired the \textit{brigadir}. This debtor is doomed to receive soon the visit of the brigade of the other bank.~\citep[p.~69, emphasis original]{varese2001}
\end{quote}

With policy interest rate being brought up so that after inflation deduction the real interest rate stays positive, see Chart 14 below, and with stance to harden budget constraint on both the government and the enterprise sectors the resulting impact on the wider economy was profound. The government budget, as it turned out, was still at deficit despites the reformers' efforts to eliminate it. This was an outcome observed in Russia as well as in many major parts of former USSR in the first half of 1990s. Liquidity preference was extremely high among the households of different ranks of social strata (with local currency funds being converted into foreign-currency ones, predominantly cash US dollars). 

At the same time, enterprises aimed to maintain daily operations increasingly turned towards settling mutual credits and debits through a network of the emerged private firms that performed exactly the same functions as did the bureaus of mutual settlements under the roof of State Bank. The difference was that now there was a profit motif shared between (a) the private entrepreneurs who owned and run the firms settling payments between enterprises, and (b) the top management of the still state-owned enterprises, which realized that state was restraining itself from the past practice of total command and it is long-term and non-reversal process. It turned out that profits in that business of mutual settlements were fat and quick enough to the owners of these firms later own buying out the equity of the enterprises for which they used to assist settling payments. That process was burn thanks to the abrupt change in the payment and settlement system, which took place on the back of State Bank being displaced by national baking systems utilizing own money units of account in the 1990-1992 period. It gained additional steam in mid-1990 as coincidence to the major change in the government finance sphere, where a law was adopted that banned central bank credit to the government, including purchase of government bonds, alongside of establishing the domestic government bond market. The latter was institutionalized as primary market where banks and other institutions bid for government bonds. It resulted in the establishing of government securities than yield more than the inter-bank interest rates on reserve balances. Hence, by 1997 broadly speaking there were two epicenters of quick and high returns. First one was provided by the new market of government bonds did become a kind of news being covered by the emerged national business media - that was the case not only in Russia, but it was also observed in Ukraine. (Recall that these two were two largest economies of the former USSR.) The second one was provided by business of credits-debits settlements and its was much less visible to the general public at the time. All in all, the following observations described the economy well: 

\begin{quote}
In Russia on the eve of its August 1998 devaluation, from 50 to 70 percent of all transactions in industry employed \textit{alternate} means of payment.\citep[p.~6, emphasis added]{woodruff2005}\par
In Russia [during 1990s] money circulation was of peculiar nature. Money served only 30\% share of payments, while 70\% share was served by alternative forms of payments. The former consisted by one third of cash and two thirds of non-cash payments. The latter consisted by 57\% of mutual settlements of goods and money obligations and by 43\% of quasi-money such as private IOUs (veksels) and government IOUs (KO, GKO, OFZ, etc)\footnote{All these abbreviations stand for the names of government tradable securities of different maturity at issue date.}~\citep[p.~144]{izvekov2009}
\end{quote}

Despite the fact that legal ban on the direct credit of central bank to the government account on the balance sheet of the central bank, the details of daily and intra-day operations of the Russia's local-currency government bond market reveal the following practice (it is reconstructed from the explanations by Aleksashenko (2009, 2018a, 2018b), who was deputy chairman of Central Bank of Russia in 1995-98):

1.	Day of repayment of government bonds and day of new primary market auction were close enough to each other so that balance sheet items of mutual relationship between central bank and government nets out. Usually, it was done intra-day.

2.	The central bank and the government exchanged IOUs so that their balance sheets at once increased by the amount equal to size of government bond repayments (interest plus principal) due to bondholders. The central bank marked up the assets item "Credit to government" and the liabilities item "Deposit to government". The government's balance sheet marked up too with assets item "Deposit at the central bank" and liabilities item "Loan from central bank".

3.	The government instructs the central bank (a) as a counterparty, to mark down its liabilities item "Deposit to government" and mark up its liabilities item "Reserve balances of banks", and consequently (b) as the register of bondholders, to instruct those commercial banks that either own directory or counterparties to ultimate bondholders to mark up their balance sheets - the assets item "Reserve balances at central bank" and liabilities item "Deposit of clients" (of bondholders).

4.	Next step, the government instructs the central bank to carry out the primary auction on selling government bonds at competitive bids that resulting in establishing some interest rate attached to the bonds of some maturities (ranging from less than one year, short term zero-coupon bonds sold at discount, to more than one year, medium-term coupon-bearing bonds). 

5.	Once, auction determined the size and the effective bids from the participating banks, the central bank (a) instructs banks to mark down their balance sheets - the assets item "Reserve balances at central bank" and liabilities item "Deposit of clients" (of bondholders), and (b) marks down its liabilities item "Reserve balances of banks" and marks up the liabilities item "Deposit of government" - effectively it is a swap between two items of liabilities for the central bank.

6.	Lastly, the central bank and government mark down their balance sheets by the size of outstanding mutual IOUs, which were created by step 1 (above): the central bank marks down the assets item "Credit to government" and liabilities item "Deposit to government", while the government marks down the assets item "Deposit at the central bank" and liabilities item "Loan from central bank".

The leftover of mutual IOUs between central bank and government resulting after steps from one through six (above), as it turned out at the end, could be on both sides of the balance sheets. At the beginning of the launch of the market financing of the budget deficit, the sequence of the above-mentioned 1-6 steps usually ended up with either zero leftover or some marginal leftover, when central bank's liabilities item "Deposit of government" and government's assets item "Deposit at the central bank." For some time, there was no occurrence of opposite situation, when central bank's balance sheet ends up with assets item "Credit to government" and government does with liabilities item "Loan from central bank". 

As the steps 1-6 were carried out intra-day hence the end of the day balance sheets of the central bank and of the government were not formally in violation with recently adopted legal norm. The whole set-up of the market assumed that government was a price taker. The set-up was supposed to ensure that incremental increase of nominal funds to market participants. Nominal rates, thanks to the dominant principle of sustaining positive real interest rates, attached to bonds were lucrative and immediately attracted animal spirits of international bond investors (an international activity which is now called hunt for yield). The latter were acting through certain schemes as initially their direct participation was not legal. Shortly, their direct participation was allowed. 

With the past due debt denominated in foreign currencies due to foreign lenders (the so-called legacy debt Russia accepted from Soviet Union) and rather low commodity prices for Russia key export items such oil. The local-currency government bond market became a vehicle of maintaining the exchange rate (or ruble convertibility) commitment. The central bank assumed that exchange rate stabilization within a pre-declared corridor (in nominal terms) as one of key means of calming down the high inflation environment in the economy. With high nominal interest rate attached into the primary placements of the Russian local-currency government bonds total expenditure on servicing the tradable debt in the local currency rose quickly (in terms of the share of revenues and expenditures). 

So that in the above-mentioned 1-6 steps the size of mutual intra-day credits expanded - however, the positive market sentiment, being still alive, allowed for some time net out those credits. Very few voices questioned that sentiment. The general attitude worsened on the back of series of domestic and external events turned around that perception among the yield hunters. 
Among those events were (1) Asian financial crisis of 1997 that produce some wave of international investors' withdrawal from the local market, and (2) domestic political developments in Russia, where President Boris Yeltsin was showing that his position was rather shaky. 

So that, when another day of government debt repayment came the same day held primary auction failed to produce that net out effect as described above in step. Instead, insufficient size of local-currency bonds placed among the market participants left the central bank and the government at the end of the day with balances, when central bank's assets item "Credit to government" contained some balance, which was mirrored in the government's balance sheet liabilities item "Loan from central bank". 

While the Asian crisis had its toll, Russia's leader Yeltsin was trying to fix domestic issues by re-shuffling of the government, which took place in the sprint of 1998. (The central bank management was left untouched.) The new government, as it was widely hoped, would bring order to government finances, where tax collection was underperforming. 

After a series of days of bond repayments and simultaneously-held bond auctions, which just proved of poor sentiment and short market demand, those balances kept growing forcing the central bank officials to be nervous that these balance sheet positions would be eventually become known to the public, showing that central bank and government violated the law. It was the central bank top officials that considered this development as unsustainable, while the government officials considered it as not qualifying a legal violation. The former urged the government to restrain, while the latter suggested they must stick to the already functioning procedure. There was an attempt to affect the bond market sentiment from the side of the central bank. As previous hike of 20ppts\footnote{Ppts = percentage points. 1 percentage point equals to 100 basis points (bps). Hence, $20 \text{ppts} = 2,000 \text{bps}$.} failed, it had an idea that a shock hike of larger magnitude would make the trick. Hence, it raised its key policy rate (called refinancing rate back then) from 50\% to 150\% in late May 1998. The market response was rather muted and shock hike did not succeed very much. Hence, the rate was cut shortly 60\% in early June. As interest rate manipulation proved not-workable, the very top officials of the central bank started to elaborate the way out from the, as they perceived, unsustainable situation. A decision to default on the local-currency debt as only viable option was agree between the top central bank officials in their meeting in London offices of the Russian daughter bank. Later then it was transferred to the president's office and the head government, which occupied his post for just few months.

\subsection{Conclusion}

The established narrative in the literature on why and how Russia's local-currency default did take place sounds more or less like this: 

There are two major reasons why the transition was more painful in the former USSR than in Eastern Europe. One was the weakness of monetary and fiscal policy which led to hyperinflation. The other was what the EBRD calls the "capture" of the state by a new business oligarchy. Both of these were serious impediments to efficient resource allocation and helped to channel income to a privileged elite. 
(Maddison, 2001, p. 158)

However, the above discussion applied intentionally (a) an evolutionary approach to detail design and operations of key institutions (especially the payment systems) and (b) the framework of modern money theory into analyzing the subject. It concludes that event of default conforms rather to this alternative view than to the established view. 

The payments and settlement system utilized by the Soviet Union during 1930-80s was unique in an extent. It was a hybrid mix of gross and net clearing that operated without reserve (or settlement) balances as it was executed within a balance sheet of the single banking institution State Bank. Reading the modern-day account on the efficient Canadian payments and settlement system by Lavoie (2019) evokes realization that there are some very broad similarities between these two systems. Such as, first, usage of gross and net clearing and, second, irrelevance of reserve balances. However, the institutional details that deliver these similarities are rather large. 

\subsection{Appendix}

\subsubsection{IMF on the GKOs dollarization}

IMF head Michel Camdessus commented as follows on the Russian policymakers' achievements back in 1997 (IMF, 1997):

As regards stabilization, the achievements are impressive: inflation has been cut from an average of 30 percent per month in 1992 to 1-1 1/2 percent in recent months, and the exchange rate has stabilized. Underlying these achievements is the fact that the enlarged government deficit, including the regional governments and extra budgetary funds, has been substantially reduced: from 19 percent of GDP in 1992 to about one-third that level last year. Moreover, as the deficit has been brought under better control, the central bank has been able to move away from financing the budget deficit and to focus instead on its proper objective: that is, achieving greater price stability. I would like here to pay tribute to the remarkable work of the CBR in winning the battle against hyperinflation and in gaining solid credibility in the family of the world's central banks.

On July 13, 1998, or nearly one month before Russia's government announced its debt moratorium, IMF Deputy Director Stanly Fisher, during his briefing with reporters, discussed a plan by the Russian government for a voluntary basis conversion of the Gosudarsvenniye Kratkosrochniye Obligatsiy (GKOs) (maturing though June 1999) into longer-term, US dollar-denominated liabilities at 7 years and 20 years. Also, he made vague reference to the issue raised by a reporter on GKOs dollarization (IMF, 1998):

QUESTION: "I'd like to get back to the GKO a little bit. You will definitely correct me if I am wrong, but as I understood it, the GKO debt was sort of effectively dollarized by hedging activity, that when a U.S. bank, for instance--they are very active in the GKO market--when a U.S. bank went in and purchased GKOs, they offset their exposure in rubles through dollar hedge positions, and often, those dollar hedges were with Russian banks--as a matter of fact, primarily with Russian banks, big Russian banks. Now, if the Russian Government were to devalue the ruble, that would really stick, and with these hedge positions in place, Russian banks effectively could collapse. Is that a concern, or is that another reason why you had to move quickly to: (a) avoid a ruble devaluation and then, secondly, sanction what is now officially a dollarization of the GKO by the conversion?" 

MR. FISCHER: "There are more reasons." 

QUESTION: "It is pretty complex, yes." 

\subsubsection{Sergey Aleksashenko, First Deputy to Chairman of the CBR (1995-98)}

Sergey Aleksashenko, who served in 1995-98 as First Deputy Head of the Central Bank of Russia (CBR), discusses his first-hand experience with the 1998 crisis that led to the country's domestic default in his book The Battle for the Ruble and a series of video recordings/stories on his YouTube channel. 

In (Aleksashenko, 2018a), he explains:

On May 19th, 1998 CBR raised his key policy rate from 30\% to 50\%. With a hindsight, I can say that this was the moment when crisis had become unavoidable, i.e., nobody, neither CBR, Russian government, IMF could not re-direct the developments into different path. \dots 1997 was a good year for Russia's economy. At last we gained macroeconomic stabilization, inflation went down and by the year end it was 11\%, during 2nd half of the year inflation was below 10\%. FX reserves of the central bank in 1st half of 1997 increased from \ac{usd} 15bn to \ac{usd} 25bn or by 2/3 and it was colossal size by the time. The economy started to grow after previous five years of contraction. We have passed the heaviest phase of the crisis [of early 1990s], the economic restructuring and started moving ahead. And the first foreign crisis that tested Russian economy's strength of Asian crisis of September-November 1997. In Asia the crisis was ravaging - companies, banks and governments could not pay on their debts. IMF sent its largest missions and started to realize its large programs there. It seemed to us that Asia was far away and our economy in good and stable shape and it should withstand. However, over two weeks since Asian crisis started foreign investors, who held 1/3 of the portfolio of GKO [bonds in circulation], started massively selling out their holdings. Over three weeks, CBR lost 1/4 of its FX reserves - out of those \ac{usd} 25bn we lost \ac{usd} 7bn and it was large stress. At that time, within CBR an internal breakup took place. Some part of the board of directors of CBR held a view that central bank should hold [defend] interest rates and sacrifice FX reserves, which were considered excessive while interest rates were valuable to the economy. Other part of the board of directors held the opposite view: the FX reserves are more important and valuable to the economy and for its long-term sustainability. Therefore, first three weeks of the [Asian] crisis were spent on sorting out which view CBR should adhere. During this time CBR lost reserves. Lastly, we decided that interest rates should be freed and the interest rates at the GKO market sharply went up. If prior Asian crisis they were in the 15-18\% range, the after the CBR's move [of freeing interest rates] they increased by 10-12 per cent up. After this investors' exit from the GKO market stopped and central bank's FX reserves stabilized. Despite the fact that some pressure on the ruble exchange rate and at the GKO market was continuing still, in general it was regarded that Russia, among all EM countries, had [withstand] the Asian regional crisis and it was that island of stability where one can hide. At the beginning of 1998 the situation was a bit unstable, but since February [of 1998] the central bank again was able to increase its FX reserves. And it seemed that everything was calming down at the GKO market. However, another [bad] news arrived - in February [of that year] the crude oil prices started to tilt down at slow pace. We, at CBR, faced this news not lightheartedly but rather calmly. IMF officials who came to Moscow were saying that CBR has to pay attention to this trend [in oil markets] and if oil price would not hold at \ac{usd} 22/bbl (as was the assumption for our balance of payments forecast) and drop to \ac{usd} 16/bbl then it would have a hard blow to the Russian economy. Later on [by summer of 1998] the crude oil price dropped to \ac{usd} 12/bbl and by early 1999 it dropped even further to \ac{usd} 8/bbl. \dots Back in March 1998, with a growth of FX reserves it seemed we were able to withstand. Now, I understand that seasonality factor - when in March-April there is prevalence of supply over demand in the local FX market - helped us to feel ourselves more comfortably. \dots On March 23rd, President Yeltsin dismissed Prime-Minister Chornomyrdin and his whole cabinet of ministers. It had become a serious blow to what we call political stability. Investors were comfortable with status quo, where President Yeltsin and Prime-Minister Chornomyrdin cohabitated. \dots That government led by PM Chornomyrdin was not highly reformed minded, some reforms were put on breaks as parliament was controlled by left-wing majority. Nevertheless, this government's successes in terms of macroeconomic stabilization and state budget control in general allowed investors to look into the future with optimism. All in all, Yeltsin dismissed PM Chornomyrdin [who was a 60-year old official back then] and proposed to appoint unknown [a 36-year old] Sergey Kiriyenko as new Prime-Minister. And it had caused real shock. This is because it was obvious that Kiriyenko had not had due experience, no experience in government at all. He had not realized the scale of problems facing him. The parliament considered this proposal by President Yeltsin as insult. First two attempts to approve Kiriyenko as next head of the government via parliament voting had failed. Only from the third attempt, on April 24th, he was approved in the parliament as its majority did not want to see the parliament being dissolved because of early elections and left-wing majority feared they would not repeat their success at the previous elections. This political turbulence immediately had its impact on the domestic business activity. Tax revenues of the federal government started sharply declining. If in the first three months of 1998 the tax authorities collected tax revenues more than in the same period of 1997, then in April-May the volumes of collected tax revenues dropped sharply. At the time, 1/3 of total state budget expenditures were expenditure on state debt servicing [he did not elaborate whether it's 'debt servicing' means interest and principal payments or just interest payments]. Then any decrease of cash receipts - those taxes that are paid into state budget - it had a blow on expenditures, which had to be reduced and the federal budget started to appear in situations when its wage as well as pension payments were delayed. Naturally, it did not lead to the improvement in the wider economy. \dots In mid-May of 1998, there was again a massive drop [of prices] in the GKO market. Investors started massively selling government bonds. I do not know what triggered that. It was rumored that there some reports of the analysts, some articles in the media. Prices on the government bonds went down. CBR raised its policy rate to 50\% from 30\% and on May 27th it raised again up to 150\%. This had become a shock for the whole financial market. We [at CBR] did not hide that our key goal was to calm down the panic, because we did not the basis for the panic. We succeeded to cool down the GKO and FX markets and then CBR cut its key policy rate, which effectively was held at such high level for just several days. However, the situation was shaky. It was understood that chances to exit from the crisis unscathed were very slim.

In (Aleksashenko, 2018b),  he talks about the operational detail established between the Central Bank and Ministry of Finance (MoF),which aimed for smoother functioning of the local-currency government bond market:

Real [first] default happened on June 17th [1998], two months prior the all-known date [August 17th, 1998]. At that moment there was an agreement, which exists today, between the Ministry of Finance and the central bank [of Russia] on servicing of the operations on the GKO-OFZ market. By this agreement, the central bank is an agent of the Ministry of Finance and technically it carries out the redemption of all issues of [GKO-OFZ securities]. All auctions back then and now are carried out in accordance with this principle that on the maturity date the Ministry of Finance is making the primary placement of the new securities - respectively, at the beginning the old securities are redeemed and then investors as a rule re-invest into new securities. Thus, on June 17th of 1998 the Ministry of Finance for the first time failed to attract enough funds to pay out to the investors. Technically, the central bank paid to the investors. And this difference just a bit below RUB5bn. That is RUB5bn was the gap between the funds the central bank paid out to investors who held old securities up to maturity and the funds the Ministry of Finance attracted via selling new securities. Eventually, the latter got itself into a position of owing these funds to the central bank. Naturally, this outcome was a surprise to the central bank. And then it was nothing surprising that next days these all funds came to the foreign exchange market. At that time the exchange rate of the [Russian ruble to the US] dollar stood at a bit over 6 rubles [per dollar] and practically speaking, nearly \ac{usd} 850m of central bank's FX reserves were gone on that day. In a week, there was a repeat of this exercise [of failed GKO-OFZ auction] and Ministry of Finance's liabilities to the central bank increased again by RUB3.5bn. An attempt by the central bank to enter into constructive negotiations with Ministry of Finance in a bid to know the perspectives [of these liabilities] and urge it to redeem the debt failed to yield any result. \dots By law, the central bank was not allowed to credit Ministry of Finance except the cases when this was directly prescribed by federal laws. At that moment it was well understood that any mistake amid approaching crisis it would result in a situation when the law enforcement agencies pay greater attention to the unlawful acts of the central bank and its management would put under investigation to the fullest extent of the law. \dots The board of directors of the central bank adopted a tough decision that it would put collection order onto accounts of Ministry of Finance and start debiting all incoming tax revenues in order to redeem that debt. It made Ministry of Finance furious. It was obvious that it end up without the funds and this would be a catastrophic situation. Then, somehow Minister of Finance Mr. Zadornov brought to the central bank a letter from the Accounts Chamber of the Russian Federation that stated that Central Bank of Russia credits to the Ministry of Finance during the GKO-OFZ maturity was not a loan from the former to the latter. Mr. Zadornov said to the central bank people: 'Here is the paper for you in order to defend yourself [before law enforcement agencies]'. We [at the central bank] did not follow such offer by Ministry of Finance. After 1998 [default on August 17th], there were eight criminal cases opened against the central bank officials and there was any case against the Ministry of Finance officials. And the fact that we [at central bank] did not break the laws was good defense for us in the future. I do not know what would had happened if we had broken the law, but it is clear we would face a hard lot then. On June 29th of 1998 Mr. Zadornov during his interview uttered a word of devaluation. \dots It was a shock to the financial community. \dots Only few people of top management inside the central bank and Ministry of Finance knew about that indebtedness [between these two bodies]. We considered that it was inappropriate to discuss this issue on public domain and there is no need to rock the boat. \dots It was clear that crisis was approaching \dots and we attempted to understand how that crisis would unfold in order to soften its impact. At that time it was clear that to carry out ruble devaluation in early July 1998 was impossible, it would had frozen the financial markets altogether, foreign investors would had started in mass selling out of government securities, Ministry of Finance would inevitably default on its external debt (volumes of redemptions of domestic and external debts were equal and the external debt schedule was quite tough - payments on the external debt were small but they were regular so that Ministry of Finance was constantly servicing it). In the very beginning of August 1998 second default took place: Vneshekonombank that was a payment agent of the Ministry of Finance did not pay on one of the tranches of the external debt. So that, we [Aleksashenko and Alexander Potiomkin, who was responsible at the central bank for foreign exchange policy] at the beginning of July while being in London [at the meeting of CBR's London-based daughter bank] came to the agreement that without default on domestic debt it was impossible to came out of this situation. Default and restructuring of the domestic debt as well as of the external debt were inevitable conditions to exit from the crisis, because it was clear that there was no scenario when Ministry of Finance could carry out redemption of the domestic debt on a weekly basis in the volume of approximately RUB6bn (equivalent of 1bn \ac{usd}). Even with a devaluation these were huge sums. It was clear that to pass the law that allowed the central bank to credit Ministry of Finance in such volumes would be practically equivalent to suicide. And we understood that if we would had entered the path of debt monetization and if central bank would start lending to Ministry of Finance and these funds would be used to redeem all debt we would create hyperinflation in another time [during 1990s] and all our successes of 1996, 1997 and early 1998 in terms of suppressing inflation and certain macroeconomic stabilization would be cancelled out. We, on the level of top central bank management, decided that at any case the country had to have resources and we had drawn a [floor] line for official FX reserves amounting to \ac{usd} 10bn of liquid foreign currency assets. Word 'liquid' is key as CBR's FX reserves at the time consisted among other things of 2bn \ac{usd} that were illiquid, they were problematic, and they could not be converted into liquid assets soon. So that the line was drawn of how much long the central bank was able to resist [market pressure] and it amounted to 13bn \ac{usd} [that could be spent on defense]. So, the receipt [against the crisis] was drawn - devaluation and restructuring of domestic debt.

\subsubsection{Yevgeniy Primakov, Prime-Minister (1998-99)}

Former prime-minister Yevgeniy Primakov, who served as head of the Russian government in 1998-99, or right after the government default in August 1998, provided these details about the Russian economy and its business practices that dominated in the country on the eve of the 1998 default:

- The following paragraph gives readers an idea of the probability that the default decision was adopted by the officials who drew on neoliberal economics only:

"By the middle of 1998 it became clear that economic course of the 'liberals' had brought the country to a dead end. In these very conditions on August 17th the GKO-OFZ payment moratorium was announced [by then serving prime-minister Sergey Kirienko]. This step deepened many negative processes in the economic, social and political spheres in Russia at catastrophic extent. I think that the authors of the August 17th decision did not foresee its real consequences. They decided to go ahead without consulting with representatives of other economic schools of thought, they considered this unnecessary and humiliation. Such a behavior, regrettably, is characteristic feature of all [neo-liberal] 'reformers', who rose to the top of economic policy- and decision-making in Russia in the 1990s and who previously had not serious work experience. Their qualities are, first, neglect of domestic opinions that foreign to them, and, second, self-love in economics that opened to them the way not only to establish themselves, which is half the trouble, but to the thoughtless and reckless experimentation within the boundaries of a giant country." (Primakov, 2002, p. 32).

- Some of operational details of how local-government bonds were bought by the commercial banks:

"The very Minfin [Ministry of Finance] placed the budget funds on the zero-interest deposits in commercial banks. The banks, in their turn, used-and I want especially to emphasize-these state funds to buy out GKOs [government short-term zero-coupon obligations / bonds]. Fabulous returns gained up to 150-200 per cent per annum." 
(Primakov, 2002, p. 27), original italics.

- On the geographical extent of financial speculation:

"During the restructuring of the Russian banking system we aimed to get rid of 'imbalance', which was due to excessive concentration of capital of commercial banks in the center [usually meant to be the city of Moscow]. By the way, as a result of August 17 event [default] the banks in the periphery turned out to be the least affected. This is thanks to the fact that they were allowed to take part in the GKO-OFZ speculations at much lesser extent." (Primakov, 2002, p. 53).

