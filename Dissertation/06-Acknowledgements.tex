%%%%%%%%%%%%%%%%%%%%%%%%%%
%
%    Acknowledgement Page
%
%%%%%%%%%%%%%%%%%%%%%%%%%%
\section*{\MakeUppercase{Acknowledgements}}
\addcontentsline{toc}{section}{\MakeUppercase{Acknowledgements}}

I would like to thank my committee members Dr. Mathew Forstater, Dr. Scott Fullwiler, Dr. Zhongjin Li, Dr. Marc Garcelon, and Dr. Linwood F. Tauheed for mentoring me in my research efforts and, in particular, for their valuable comments and suggestions during my work on dissertation. 

I am especially indebted to Dr. Scott Fullwiler, my academic advisor and committee chair (through 2024), whose own work guided me throughout my PhD studies at \ac{umkc}. It required me to discipline my way of thinking via the analytical framework of balance sheets. This allowed me to further this analysis and support it with analysis of debt-credit relationships, a core of my dissertation work. His suggestions to widen the perimeter of literature used in research on financial markets and regular advice of developing stylized facts about the subject matter of analytical inquiry will stay with me in the future.  

I would like to thank Dr. Mathew Forstater for guiding through different theoretical approaches and historical aspects of their origins and subsequent development. I appreciate every instance of suggested readings I never heard of from Dr. Forstater, which usually concluded our in-person conversations. 

I am grateful to Dr. Zhongjin Li and every opportinity she provided to me talking over my work and for her support of my research process. A very practical rule, which she handed over to me, reminds me regularly of its continued validity and it says: ``at some point, you must start actively writing instead of actively reading.'' Also, I appreciated Dr. Li's challenging questions about the practical usage of the research work, which I tried to address in my dissertation. 

I would like to address my warm gratitudes to Dr. Marc Garcelon for his inspiring lectures in the classes on classical and contemprory social theory, which allowed me to navigate my research on the topic of money as a social relationship. 

And, I appreciate very much the advise provided by Dr. Linwood F. Tauheed on the institutionalist thought, in particular of John R. Commons with respect to business transations. 

I am grateful to Dr. James Sturgeon for an insipring course on advanced institutional economics and the leadership of the students' institutionalist group that gathered in the early 2019 fall semester. At the very opening meeting of the group, Dr. Sturgeon asked participating students to pick one of his proposed list of topics to write an paper to be presented in the upcoming AFIT conference. I chose the topic of \ac{mps}, which happened to appear on my research radar a couple of years before. The idea to look at the subject matter from the institutionalist view point inspired me to read a series of related literature. Since Covid-19 pandemic derailed my writings on the matter, this paper is still in the pipeline. It happened that first person I met while vising \ac{umkc} was Dr. Sturgeon and I am deeply thankful for that memorable meeting.

I would like express my gratitude to Dr. Erik Olsen, chair of economics department, for inspiring courses on Marx theory, mathematical economics and workers cooperatives as part of advanced microeconomics. Demanding examination assisted me in studies and my dissertation work.

Let me express my deep gratitude to entire body of Economics department for the studies that I had, inspiring lectures, and demanding home assignments with examinations. To a sizable extent, I am thankful, too, to my colleauges among PhD students in the economics department for mutual assistance and friendship. Lastly, my special thanks are to Jill Folsom, economics department's officer, for immediate assistance with my inquiries. 

\newpage