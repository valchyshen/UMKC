%%%%%%%%%%%%%%%%%%%%%%%%%%
%
%        Abstract Page
%
%%%%%%%%%%%%%%%%%%%%%%%%%%
\begin{center}
\vspace*{0.01in}
\large{\MakeUppercase{\MyThesisTitle}}\\
\vspace{24pt}

\normalsize{
  \MyName, Candidate for the \MyDegree\ Degree\\
  University of Missouri--Kansas City, \MyDegreeAwardYear\\
  \vspace{24pt}
  \MakeUppercase{Abstract}
  \vspace{12pt}
}

\end{center}
\addcontentsline{toc}{section}{ABSTRACT}  %the name 'ABSTRACT' will appear in the toc, as a chapter

\doublespacing

In economics, the concept of capital mobility (flows or movements) has been central not only for international economic matters but for domestic ones, too.  When invoked with the prefix "hyper-," it unmistakenly suggested that capital, which is being discussed, is money since capital hyper-mobility is about modern-day finance that, as said, "moves" money around the world at lightning speed.  Such conceptualization is deeply misleading since it misrepresents how international finance works in the real world.   This topic of \textit{proper} description of international money is still a lacuna of economic literature. The latter traditionally relies on the centuries old metaphors of motion to the subject of money and finance, whether it talks of domestic or international transactions.  This dissertation aims to fill this gap.  It starts with the intellectual legacy of \citeauthor{innes1913}, who provides the very basics that allow us to conceptualize money in a precise way as a social relationship of the debt-credit type.  It is free of metaphors of motion.  Such conceptualization relates to the core of the writings of \citeauthor{commons1951}, \citeauthor{keynes1936}, \citeauthor{minsky1986} and others. Specifically, \citeauthor{innes1913}'s writings allow us to distill a principle of payment, which arises from debt-credit relationships.  Technological progress has been changing the tools---from clay bricks to tally sticks and from records on paper to records in the internet ``cloud''---of payments but not the principle itself.  Modern-day payments, even when enabled through smartphone apps, are implemented according to that principle. It has three avenues: creation, re-use, and set-off of \acfp{dcr}. 

Once cross-border payments are analyzed as \acp{dcr}, the money of account appears to have substantial meaning. It would reveal the \acp{dcr} that are explicitly hierarchical, placing some countries in subordination to other(s) and potential for economic dynamics of gapping assymetry between these two groups of countries. Irrespective of the country, if its national money of account is in use for cross-border transactions of all types (related to trade or finance), then such a country's banking industry serves as a collective clearing house. It means, that for this country, there is no such thing as capital ``inflow''  or ``outflow''---sometimes called capital imports and exports respectively---since all transactions are due to decisions by \textit{domestic} public or business units whether to expand or contract their balance sheets. It has flexibility to manage its economy. While others do not, even if they have accumulated a sizable pile of foreign exchange reserves. The analytical framework of this dissertation is applied to economics of santions with Russia's economy being a case study.

As a conclusion, an approach laid down in this dissertation is an attempt to describe money transactions---either domestic or cross-border---in a manner that is free of metaphors of motion. By itself, this dissertation is a start of research program for the future. Its potential application might be located for such topics as (i)~economic sanctions of melign state actors, (ii)~net resource transfer from developing countries (Global South) to developed ones (Global North), and others.   

\newpage
