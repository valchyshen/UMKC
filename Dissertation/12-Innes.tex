\newpage \index{Thailand|see{Siam}}

\section{\MakeUppercase{Legacy of A. Mitchell Innes}}\label{sec:innes}

\subsection*{Abstract}

Alfred Mitchell Innes (1864-1950)\index{Mitchell Innes, Alfred} was a \textit{chartalist/creditist}---to borrow the term in its combined appearence from \cite{goodhart2005} and \cite{earley1994}---of the resolute and unwavering type\index{Mitchell Innes, Alfred!creditist}. His rare writings on the matters of monetary theory and practice are called as dibblings in some corners of economic literature and the best articles on money ever written in others. Interest to his work revived among some heterodox economists over past several decades. After \ac{gfc} even mainstream economists started referencing it. This chapter reviews his both published and \textit{un}published works. The latter were found in the archives and never referenced in the literature. It pounts out there are useful parts of the \citeauthor{innes1913}'s work that still remain unexplored and left behind. In particular, his explanation of \textit{payment}\index{Payment} as the \textit{set-off}\index{Set-off} procedure, which naturally fits to the description of money as credit and arises from the fundamental recognition that money is a social, debt-credit relationship. This explanation is more clear than usual description of money as ``means of payment." In addition, \citeauthor{innes1913} explicitly rejected the notion of money as ``medium of exchange." These details of the analysis allow us to look critically and beyond the vail of the customary metaphors of motion used with respect to money such as capital ``flows/mobility".

\subsection{Introduction}

In the midst 1910s Alfred Mitchell Innes (1864-1950), a career serviceman of the \ac{bfo}, produced an intervention into the US economic thinking by publishing two articles in the \ac{nyc}-based financial journal. Back then, \citeauthor{keynes1914} did pick up one of them and left a short commentary on his contribution \citep{keynes1914}. Since then, \citeauthor{innes1913}'s  writings were left unnoticed by the economic profession for a long period of time. Today, economists of \ac{mmt} build on the Mitchel Innes theory. \ac{mmt} states its approach to macroeconomic analysis stands on the shoulders of such thinkers as John Maynard Keynes, Karl Marx, A. Mitchell Innes, Georg F. Knapp, Abba Lerner, Hyman Minsky, Wynne Godley \citep[p.~1]{wray2015}. Among these giants, \citeauthor{innes1913} stands out by the following recognition made by Wray: 

\begin{quote}
I believe the 1913 and 1914 articles by Innes stand as the \textit{best} pair of articles on the nature of money written in the twentieth century. \citep[p.~223, emphasis added]{wray2004}
\end{quote}

The inquiry of this chapter focuses on both the published and unpublished work by \citeauthor{innes1913} on the monetary theory, policymaking and commercial practices.

His best known work is the two articles he wrote, while being on his diplomatic assignment in the Washongton D.C., and published within a span of less than one-year period, which was between May 1913 and January 1914. He titled them respectively ``What is Money?" and ``Credit Theory of Money". Such was his highly concentrated outburst of his firmly established ideas on the monetary matters. 

\citeauthor{keynes1914} himself recognized \citeauthor{innes1913} work as having ``much foundations" after reviewing his 1913 just-published article \citep{keynes1914}. The \ac{mmt} literature embraced his work since the second half 1990s \citep{wray1998}. In addition, \citeauthor{ingham2004_} named it as ``two iconoclastic articles" \citep[p.~175]{ingham2004_}, while \citeauthor{goodhart2005} considered this work as ``extraordinary," because: 

\begin{quote}
\dots in them [Mitchell] Innes attacks the metallist/Mengerian theories head-on, just when the gold standard was riding high. It has been much easier to do so subsequently, after the adoption of fiat money has made the chartalist/credit role of money so much more obvious. \citep[p.~759]{goodhart2005}
\end{quote}
 
Aside of these infrequent episodes of recognition, economic writers rarely looked at his work because of enduring dominance of the monetary orthodoxy in that time period and nowadays. 

However, after the \ac{gfc} interest to the work by \citeauthor{innes1913} revived not only within the heterodox economics writings but in the orthodox ones, too, albeit on the margins. Such interest was spotted, remarkably, in such a sound-money bastion as Cato Institute at its 36th Annual Monetary Conference held on 15 November 2018 in Washington D.C. A keynote speaker from the \ac{bis} delivered a lecture to the audience upon the prepared paper, where the second sentence in the opening paragraph reads ``What is money?" and the explicit citation of Innes' work takes place on the eighth page and albeit in a footnote \citep{borio2018}.

To date the most elaborative discussion of the \citeauthor{innes1913} work by economic profession is a collection of essays written by academic economists associated with economics department of \ac{umkc} under the title \textit{The State and Credit Theory of Money} \citep{wray2004}. It provides detailed overview of the issues Mitchell Innes touched upon in his two articles. There is small introduction into his family origins and his personal career path. Too, this volume has some critique of his work.

This chapter aims to fill one important space still left unoccupied by the \citeauthor{innes1913} scholars, as it seems. There is an important question on the discussion of the background, or context, against which \citeauthor{innes1913} acted in 1913-14, when he submitted those articles to publication one after one. Another part of this question is what urged him to intervene into the public discussion on the money matters. To some the answer to this question appear short and obvious -- say, he tried to make a statement and make himself visible. Still, an answer like that does leaves a lot of interesting developments behind. The following discussion aims to answer that question with a bit greater dive into the literature that matters. 

Another aim of this chapter is to show that \citeauthor{innes1913} scholars missed to pick up the description of the payment as a process, which has been detailed in his writings. It is essential to understand (i) mechancs of an individual transaction, and (ii) everyday intensity and dynamism associated with money markets. It also is useful to respond to the critique of  \citeauthor{innes1913} work.

To summarize, this chapter builds out upon the analysis of the above-mentioned two published papers (published 1913 and 1914) and unpublished documents found in the U.K. National Archive\footnote{See \url{https://www.nationalarchives.gov.uk/}} and the archive of People's National Museum (Manchester, United Kingdom). The collection of the unpublished and (still unknown) papers written by \citeauthor{innes1913} include those that were (a) intended for the internal usage within the \ac{bfo}, and (b) written upon his retirement from the \ac{bfo}:

\begin{enumerate}%[label=(\roman*)]
\item \ac{bfo} Internal Report \textit{Notes on the East} \citep{innes1909}
\item \ac{bfo} Internal Report \textit{Banking and Currency Systems of the United States} \citep{innes1910}
\item \ac{bfo} Internal Report \textit{On Uruguay} \citep{innes1914a}
\item \ac{bfo} Correspondencs: during 1909-12 \citep{innes1909_,innes1911}
\item Personal Correspondencs of 1940 with general secretary of the Labour Party \citep{innes1940}
\end{enumerate}

\subsection{Historical Background: The Rise of International Financial Advisers}\label{sec:money_docs}

Available short biographies of \citeauthor{innes1913} state he was born in 1864 in England in the family of Scottish origin.\footnote{Despite the Scottish origin of the parents \citeauthor{innes1913} associated himself with Englishmen in his own writings.} According to \cite[p.~4]{wray2004}, his grandfather served as cashier of the Royal Bank of Scotland during 1808-1827 and then became a director of the bank. In his own correspondence, \citeauthor{innes1913} mentioned just once and a little bit about the professional backgorund of his parents. He wrote that ``[m]y father (and I think, but I am not sure, my grandfather before him) was one of the Scottish Archers, otherwise The King's Scottish Bodyguard" \citep{innes1911}.  

His parents were wealthy enough to provide their son with a private education. At the age of twenty six, \citeauthor{innes1913} started his professional career in the British Empire administration by entering the \ac{bfo} in 1890. His first foreign assignment arrived next year. It was to Egypt\index{Egypt}. There, he was to serve as secretary to the British Consul-General Evelyn Baring, who shortly earned a title of Lord Cromer in 1892.\index{Lord Cromer}\index{Baring!Evelyin|see{Lord Cromer}} The latter's name embodied the very British Empire among the contemprories \citep{courtney1918,tignor1963,roger2004}. After a five-year period of serving in Egypt\index{Egypt}, \citeauthor{innes1913} was promoted within the ranks of the \ac{bfo} to serve as a financial advisor to King of Siam\index{Siam}.\footnote{Siam was renamed to Thailand in 1939 \citep[p.~267]{queralt2022}.} This promotion came about in 1896 upon the recommendation of Cromer, who happenned to discuss the matter with a Belgian advisor to King of Siam\index{Siam} \citep[pp.~67-68]{primel2016}. In Siam\index{Siam}, \citeauthor{innes1913} spent three years through 1899. Same year, he returned to Egypt\index{Egypt} to serve as Under-Secretary of State for Finance in the Egypt\index{Egypt}ion government \citep[pp.~3-7]{wray2004} with Comer still Consul-General \citep{roger2004}. In 1908, \citeauthor{innes1913} obtained an assignment to the Americas continents, where he served as Councillor of the British Embassy in Washington D.C. through 1913 and then in Uruguay as Minister Plenipotentiary to the President of Uruguay through 1919. After that he retired from the \ac{bfo} ending his international diplomtic career that lasted 28 years. It allowed him to observe and learn from the traditions, custom and governence of diverse nations from those vast parts of the world called back then as Orient and Americas. 

Writings by \citeauthor{innes1913} reveal he possessed innate abilities and elevated desire to learn continiously from the enviroments he happened to socialize. His most preferred method was personal engagement with the subject matter of his interest and thorough observations of any possible details. There are two major topics in his writings: (i) monetary theory and practice \citep{innes1910,innes1913,innes1914}, and (ii) justice systems and social coherence \citep{innes1907,innes1909,innes1913_}. In both, one can find those traits. These personal characteristics must had been not the last ones because of which he was selected within the \ac{bfo} for the first assignment of what later became a lengthy career in international diplomatic service. 

Connection between \citeauthor{innes1913} and Cromer requires a bit of detail. Judging from the biographies of Cromer, who was of the age of 50 when Mitchell Innes became a member of his administration of Egypt\index{Egypt}, the former firmly relied on the same very characteristics throughout his own career. Cromer's biography states from early in career he showed that he had that ``drive for self-education" \citep[pp.~19-20]{roger2004}. Indeed, that drive was strong and the reason was hidden in the Cromer's childhood.

It is worth to note that Lord Cromer was born in England in 1841 as Evelyn Baring, ``a member of the famous City banking family of Baring Brothers" \& Co  \citep[p.~viii]{roger2004}.\index{Baring!Evelyin|see{Lord Cromer}} It was Evelyn's grandfather Francis and his brother John \citep[p.~19]{zetland1932}, who ``founded the banking house of John and Francis Baring \& Co. in London in 1762" \citep[p.~3]{roger2004}.\index{Baring!Francis}\footnote{It was Francis Baring, who was the key person behind the rise of the financial firm Barings as a financial power. His father John Baring arrived to England from Germany to settle as merchant and cloth manufecture. Francis was born in 1740 and at young age ``was sent to London to study commerce under a leading firm of city merchants" \citep[p.~11]{traill1897}. He developed himself into an exceedingly successful business person and had become ``an authority on questions of Currency and finance, and a contributor of weighty additions to the literature of the subject." (ibid, p. 12)} 

Evelyn's father Henry, son of Francis, was an absent parent and due to old age died when Evelyn was seven years old. Similarly, Evelyn's mother ``seems to have maintained a distance \dots by regarding him as something of a dunce and, in his own words, the fool of the family" \citep[p.~7]{roger2004}. Due to absent parents and personal trait of being a wild child, Evelyn grew up having little education and later as a teenager he spent his time in the millitary academy eyeing career in the British army, not in the finance business. 

Evelyn Baring had seven siblings, of whom two brothers Edward and Thomas developed themselves for the top roles in the business of Baring Brothers \& Co.\index{Baring!Brothers \& Co.}\index{Baring!Edward}\index{Baring!Thomas} Thomas was just two years older than Evelyn, while Edward was older by fourteen years. With absent parents Evelyn and Thomas were growing up together and bonded so much that ``though their subsequent careers diverged widely \dots they maintained throughout life a weekly correspondence" \citep[p.~22]{zetland1932}. 

Very early in his career in the British military, while serving in the Greek island of Corfu, Evelyn Baring ``began to be aware of his lack of learning and became determined to do something about it" and over time, for example, he learned both modern and classic Greek languages. In addition, ``[a]s a Baring, Evelyn was also in a good position to learn something about the great world of politics and banking" \citep[pp.~19-20,35]{roger2004}. He spent four years in India, while serving in imperial administration as private secretary to Viceroy of India, who was his uncle and member of Baring family. 

Over time, Evelyn Baring earned repuation as a qualified person to deal with issues of international debts, in particular, debts of Egypt\index{Egypt} \citep[p.~143-144]{tignor1963}. In 1877 his named was mentioned as a proper candidate to deal with Egypt\index{Egypt}'s public debts, where the domestic government as debtor was increasingly behind the committed payments. And in 1878 he become a member of the multilateral comission dealing with that question \citep[p.~28]{primel2016}. At the time, this appointment faced severe criticism as some in England questioned Evelyn's aptitudes to this post on the grounds that he was trained to serve in the British Army artilerry \citep[p.~245]{courtney1918}. Nevetherless, that appointment did materialize and Evelyn Baring arrived to Egypt\index{Egypt} to serve in the multilateral debt commission, where he ``was a representative of the British bondholders but not of his own government" \citep[pp.~97]{roger2004}. This fact speaks about Evelyn's experience and reputation as well as knowledge in the matters of international finance. However, it might be claimed that him being a Baring did matter, too. Cromer's\index{Lord Cromer} biographers acknowledged that by the time, being already in the age of maturity, Evelyn Baring proved himself to be rather a financier by ``instinct and inheritance." His ``first object everywhere and always \dots was to set things on a firm financial basis" as his ``soul abhorred" the loosely ``kept accounts, vague [financial] statements as to the extent of a debit balance" \citep[p.~252]{courtney1918}. Eventually this type work in Egypt\index{Egypt}, where imperial govenence and international finance were extensively intertwined, catapulted Evelyn to the official position of Consul-General in 1883, one year after the British military envaded Egypt and launched occupation. According to a contemprory biographer, this position was ``perhaps the most important \textit{financial} position in the British Empire outside the United Kingdom" \citep[p.~337, emphasis added]{traill1897}. Effectively, Evelyn Baring (later Lord Cromer\index{Lord Cromer}) had become the ruler of Egypt\index{Egypt} and his tenure lasted through 1907. A short description by the same biographer tells it all about this turn of events:

\begin{quote}
In all probability [Evelyn Baring] was then under the impression that his work would be consultative rather than constructive, and that it would consist rather in keeping financial watch and ward over the government of Egypt\index{Egypt} in the hands of native administrators than, as it turned out to be, in superintending the instruction of Egyptians by European officials in the very rudiments of the administrative art. \citep[pp.~213-214]{traill1897}
\end{quote}

Cromer's professional background and family ties reveal that he belonged in term of the social status to what \cite{queralt2022} termed as gentlmently class. The latter did ``specialized in commercial activities (finance, shipping, and insurence) and civil service (government and military)" \citep[p.~99]{queralt2022}.

The details above on the Cromer's\index{Lord Cromer} background might seem exsessive to the reader. Neverthelles, they are made with a purpose to underline two main features of the environment of Cromer's administration of Egypt\index{Egypt}, in which \citeauthor{innes1913} managed to serve for thirteen years (this excludes his service in Siam\index{Siam}). 

Hence, the first feature of Cromer's administration comes from its major occupation with Egypt\index{Egypt}'s finances in the domestic and international domains. It dealt with Egypt\index{Egypt}'s debts accumulated in the past for foreign lenders, while the main task was domestic that is placing the country on the footing of commercialized production.\footnote{The Egypt\index{Egypt}'s cotton growers were the fist to become an industry worth of participating in overseas trade. See \cite{primel2016} for the details about that process.} \citeauthor{innes1913} himself wrote in one of his essays that his entire experience of serving in Egypt\index{Egypt} was associated with the Ministry of Finance of that country.\footnote{See \citep[p.~3]{innes1907}.} 

Second feature is less explicit and being deducted analytically from Cromer's boigraphies and \citeauthor{innes1913} own writings. This is continious learning, which was a personal virture and professional style to both. 

For Cromer it meant realizing his lack of proper education. As discussed above it was a shortage from his childhood. Quite early in his career in the British army he discovered how extremely ignorant he was comparing to his colleagues, the young officers around him. In his own diary he noted that these officers ``were much better educated than himself" and then ``a number of things which they knew and I ought to have known" \citep[p.~25]{roger2004}. He managed to eliminate that shortage to an extent by learning while on the job.\footnote{Cromer humself wrote that ``I was for some while in Egypt\index{Egypt} before I fully realised how little I understood my subject; and I found, to the last day of my residence in the country, that I was constantly learning something new." \citep[p.~7]{cromer1908} It appears that his fears of knowing less than needed and required followed him from early to the late stages of his adulthood.}

For \citeauthor{innes1907} its meaning can be deducted from his own writings. For example, he wrote in on of this essays: ``[n]o man is fit to teach unless he is also willing to learn, and when he ceases learning his life's work is done" \citep[p.~45]{innes1907}. His willingness to learn manifested itself in his written elaborations on the topic of money and finance and on the issue of justic systems and social coherence. The quote mentioned above is from his critical essay \textit{To see with theirs' Eyes} on the British rule over the Egyptians, where the verbs 'to teach' and 'to learn' were addressed to those imperial relationships between the governors and the governed present in Egypt\index{Egypt} at that time. In other words, that quote made a statement that the Englishmen as the governors could not govern the Egyptians as the governed without learning from them.

At this point, we can make an intermidiate conclusion. From 1891 and through 1908, \citeauthor{innes1913} was involved in finance, while serving in Egypt\index{Egypt} and Siam\index{Siam} at top advisory positions in the governing bodies of those countries. To be precise, this was a type of finance, which \cite{polanyi1944} called \textit{haute finance}\index{Haute finance}\footnote{See \cite[pp.~9-19]{polanyi1944} on the discussion of \textit{haute finance} as an institution of international economic system of the late nineteenth and early twentieth centuries.} or what modern day economic history scholars such as \cite{queralt2022} named as the Bond Era, which lasted about a hundred years through 1914.\index{Bond Era 1815-1914}\footnote{In particular, the Bond Era lasted from the end of the Napleonic Wars in 1815 and through erruption of the World War I in 1914 \citep[p.~3]{queralt2022}. During this period the world experienced an eiphoria in sovereign lending via the means of bond issuance by the countries that lacked their own private financial industry. Such bonds were bought by the foreign private financiers. There is a contrasting difference in the conceptualizations of the same historical period as between \citeauthor{queralt2022}'s ``Bond Era" and \citeauthor{polanyi1944}'s ``haute finance." The former speaks of the high frequency of interstate conflict concentrated between 1815 and 1914, while the latter says the conditions laid down by the private financiers produced the hundred year's peace. It might be argued that \citeauthor{polanyi1944} refered to major powers of that world which were the largest empires. Whereas \citeauthor{queralt2022}'s analysis was focused on the smaller states that were on the periphery of the major empires and which were forced to modernize themselves under the pressure from them.} Haute finance was a private business, where private financiers from leading financial centers such London, Paris, Berlin, Vienna and later New York oragnized bonds issuance for the urgent needs of the sovereign governments of the economically ``backward", as it was said back then, or the catching-up nations. Such bonds usually attracted buyers because they promised a higher-than-average return (or yield). Non-payments on these bonds were not infrequent. The prevailing practice on the side of private financers in terms of dealing with non-payments was to rely on the soft and hard powers of their domestic government. Respectively, it meant applying diplomacy and the military. In extreme cases such as Egypt\index{Egypt} in 1882, military intevention and occupation was the means by which complete overhaul of domestic governence was undertaken to make past, current and future external debts servicable. In other words, such an overhaul supposedly would guarantee that committed future payments would be hornored by the debtor. Under the institution of \textit{haute finance}, the ``military intervention was considered an accepted practice of debt collection by the international community until the early part of the twentieth century" \citep[p.~107]{queralt2022}. And fears of military coercion applied by the major imperial powers was shared in many less mighty nations across continents. This made those nations cautious and extremely aware of military threats and concequences. Hence, the process of state building with modernized and capable accounting, taxation and commercial systems was quick, if not urgent, and required external advice from qualified and experienced people. 

While in Egypt\index{Egypt}, \citeauthor{innes1913} was one of numerous personnel of Englishmen in the British administration led by Cromer.\index{Lord Cromer} His trait of continous learning was backed by critical eyes. Having a daily job of dealing with \textit{haute finance} in such hot spot as Egypt must had made his curiousity about finance and money growing bigger and bigger. There are no documented in the archives that reveal that Cromer's family connections in the world of \textit{haute finance} aided to \citeauthor{innes1913}'s learning of the subject matter. However, it is reasonable to suggest that some operational details of finance in terms of how is actually done by major financial houses such as Barings Brothers \& Co was explained and understood by people involved in the Cromer's administartion dealing with finance. Collecting those bits and pieces along with other ones for own elaboration on the topic of his prime interest---money and justice---is evident and recognized in the writing by \citeauthor{innes1913}. 

His service in Egypt was outstanding and noteworthy enough so that Cromer picked Mitchell Innes to be sent to Siam\index{Siam} as a financial advisor to the King, who was actively looking for an European advisor on the financial matters.\footnote{\cite{brown1992} describes this appoitment in some detail in Section 2.7 pp. 38-40. It is interesting that Cromer was approached on that question by Gustave Rolin-Jacquemyns, a Belgian who at the time was General Advisor to King Siam\index{Siam}, serving from 1892 through 1902, and it was who he insisted on the appointment of an experienced European person as a dedicated advisor on the financial matters. Prior to Siam\index{Siam}, Rolin-Jacquemyns served in the Egypt\index{Egypt} himself and apperantly knew Cromer personally. See \cite{hubert1965} for the detailed discussion of the work done by Rolin-Jacquemyns in Siam\index{Siam}.} Back then, Kingdom of Siam had a totally different pathway towards modernization and state building than that of Egypt. It managed to stay away from borrowing from the European financiers as local elite treated them with suspicion considering external indebtedness as a form of subjugation to Western powers \citep[p.~262]{queralt2022}. Also, it was not taken over by a major imperial power. However, it felt the pressure that was still mounting mainly from the British and the French empires. The incumbent leaders of the nation decidedly launched the Western-styled state building process in the second half of the 1880s. In early 1890s, due to French army invasion and annexation of parts of Siam, the process of state-building just accelerated. Against this background, \citeauthor{innes1913} arrived to Siam in June\footnote{See \citep[p.~39]{brown1992}.} of 1896 to become a Financial Advisor to King Chulalongkorn, the King of Siam. Such a position never existed in that nation before.\footnote{\cite{vella1955} does not mention \citeauthor{innes1913} by name, while recognizing his input to Siam's modernization drive of the 1890s: ``" \citep[p.~]{vella1955}.} The government of Siam at the time was establishing two major ministries of Finance and Interior with key task of running an effective tex system. \cite{brown1992} describes that \citeauthor{innes1913} was directly reporting to the King. \cite{siam1978} provides an evidence that within a month upon his arrival to Siam \citeauthor{innes1896} prepared and submitted in September\footnote{To be precise, it took place on September 6th, 1896 \citep[p.~375]{innes1896}.} of 1896 \textit{Report on the financial system of Siam}, which effectively discusses mainly the existing tax system such as exisiting list of taxes and who is obliged to pay them \citep{innes1896}. This report did not touch anything else with respect to finance but taxes. Siam as the nation was predominantly rural relying largly on subsistance farming and fishing. Major commercial activities were monopolies in opium farming, gambling and lottery. In his short report \citeauthor{innes1896} was quite critical of the existing tax system as well as of the capacity of the Ministry of Finance. The tax system overtaxed cirtain items and activities by imposing taxes on an item at different stages of economic activities. He did not approve the established operations of the Ministry by saying that ``there does not exist a Ministry of Finance in the proper sense of the word" \citep[p.~374]{innes1896}. To him, ``it was just an office for the reciept of the portions\footnote{Tax revenue collection was not centralized yet as ``the revenue is collected partly by one Ministry, partly by another, while the local treasuries are practically independent of Ministry of Finance" \citep[p.~374]{innes1896}} of revenue and the payment of salaries." He advised to turn it into the ``Department that directs and controls the finances of the country" (ibid). He left Siam going back to Egypt in 1899. In the literature, there is no direct mentioning of how impactful was the financial advising by \citeauthor{innes1913} to the development of Siam. However, thanks to contemprory writers on the history of state building, one can find the following description that indirectly speak of that impact: ``King Chulalongkorn (r. 1868--1906) was responsible for the giant leap forward in fiscal capacity \dots [as b]etween 1868 and 1915 tax revenue increased almost tenfold, from 8 to 74 million baht" \citep[p.~263]{queralt2022}, see also \citep{vella1955}.\footnote{\cite{vella1955} observed similar gains in the efficiancy of revenue collection by the modernized Ministry of Finance of Siam: ``The organization of the Ministry of Finance in 1892 did not in itself solve the considerable problem \dots of centralizing the collection of revenues. On the advice of the British adviser to the ministry, a Comptroller General's Department was created. Heads of other ministries were prevailed upon to deposit their funds there, and other controls were established over the sources of revenue. These measures greatly increased fiscal efficiency; by 1903, receipts were 100 per cent greater than they had been in 1896." \citep[p.~346]{vella1955}} These revenue increases that followed the \citeauthor{innes1986}'s advising in Siam during 1986--1899 might sound as confirmation of the mainstream postulate of the benefits of maintaining the government finances in surplus between revenues and expenditure o at least in balance them. However, this dissertation considers those outcomes from the chartalist perspective.\index{Mitchell Innes, Alfred!Chartalist} Successful state building relies upon such financial system, where own money of account (in Siam it was, and still is, \textit{baht}) is used for denomination of tax libailities and then the government enforces the tax payment discipline. Judging from Siam development literature\footnote{For the economic development of Siam see \citep{innes1896,vella1955,brown1978,brown1979,brown1992,queralt2022}.} \citeauthor{innes1896} provided financial advising which was chartalist in practice. Another detail of that impact was that Siam did retain its causious attitude to external borrowings\footnote{In this regard the following quote is worth mentioning: ``[b]etween 1905 ans 1925, Siam floated five loans overseas for a total of {\pounds}13.6 million, 44 percent of which was used to reinforce credit instead of domestic investment activity. These loans turned out to be extremely political, with British, French, and German representatives competing for access, which reinforced the kings' fears of external finance" \citep[p.~263]{queralt2022}.} and it appears that \citeauthor{innes1913}'s financial advising did not encourage it.\footnote{After depature of \citeauthor{innes1913} from Siam in 1899, two successive Financial Advisors to King of Siam were selected and hired among the representatives of the British Empire. instrumental in lobbying the Siam government, first, to adopt the gold standard and, secondly, to start borrowing funds from the European financiers. However, causiousness towards external debt did remain among top governmental officials \citep[see][]{brown1978,brown1979}.} Lastly, Siam (later Thailand) remained a nation not conquered by a major empire.

Upon return to Egypt in 1899, \citeauthor{innes1913} obtained promotion to the ranks of Under-Secretary of State for Finance in the government of that country. His tenure lasted through 1907, the same year when Cromer retired due to old age and health requirements. During this very period \citeauthor{innes1913} became socially visible. 

On one hand, he engaged himself with local Egyptians to establish first sporting club in the country's capital Cairo. It took place in early 1907. The club later transformed itself into one of the most popular soccer clubs of modern-day Egypt\index{Egypt} \citep[pp.~85-86]{jacob2011}. 

On the other hand, he started writing essays to the general public with an aim of telling of his thinking and ideas that he learned over time, while on the overseas positions with \ac{bfo}. At this time, it appeared he was realising that his service had been transforming from advising to national elite in the top governing positions towards advising the general public on the critical issues of the day.

Most of his writings were critical of the status quo. In them he was speaking of the better ways, while understanding the topics he touched upon. Among the known essays, the first one that got published appeared in 1907 and published in Cairo under the title \textit{To see with theirs' Eyes}, it was already mentioned above, \citep{innes1907}. This essay together with ones published several years latter---\cite{innes1909} and \citep{innes1913_}---cover the topic of justice and social coherence. As it was already said above, another major topic for him was money and finance and talking about it in public he offered a better way. In 1908, he crossed the Atlantic ocean to servce in the British embassy in the Washington D.C. There he published two journal articles on money and finance \citep{innes1913,innes1914}, which are the most referenced of all his writings as of today, and one journal article on justice systems \citep{innes1913_}. Yet, in 1909 and 1910 he submitted for an internal use within the \ac{bfo} two works. One on justice titled \textit{Notes on the East} \citep{innes1909} and another on money titled \textit{Banking and Currency in the United States} \citep{innes1910}. The latter obtained confidential status right upon submission with \ac{bfo}. Both were never published, but they are accessable via archives today. His ideas on both topics as explained in these essays came from his continious inquiring and learning. Also, he realized that his ideas might require long digesting from the general public. On the likely scepticism over his views from the contemprories he suggested the following allegory: ``But the repeatent theif is the best detective" \cite[p.~4]{innes1907}.

The prime focus of this dissertation is at the \citeauthor{innes1913}'s investigative thinking on the topic of money and finance. In full detail this discussion is laid down in the Section (INSERT THE NO \& pp.), below. 

The rest of the current section is devoted to the broad background of the work performed by Mitchell Innes. He belonged to the emerging profession of international financial advisor. In essence people like him performed such an activity that was advising, or providing expertize, to the leadership of the nations on the matters of state building and particularly in such complex matter as finance. Since the nineteenth century, it became a distinctive type of expertize, which at the beginning was undertaken by individuals and later since mid twentieth century by such international organizations as \ac{imf}, World Bank, \ac{bis} and others.

In the mainstream economic literature this type of activity is called money \textit{doctoring}.\index{Money!doctoring} Hence, people on these positions have been called money doctors.\index{Money!doctors} However, this term contains a narrow notion of financial exeprtize and advizing than this dissertation reveals analyzing the work by \citeauthor{innes1913}. For example, \cite{fland2003} considers money doctors as foreign experts whose knowledge was in high demand by nations experiencing financial crises. Those crises, according to the mainstream conceptualization of money doctorying, manifested themselves in the ``run on domestic currency" with ``deposits go[ing] abroad" resulting in exchange-rate depreciation and high inflation (or worth hyperinflation).\footnote{Similar position is taken by \cite{alvarez2024}: ``Money doctors resolve money troubles. Those troubles manifest themselves in inflation or deflation, in exchange-rate depreciation, in financial or economic crises, or in a general distrust in the currency."}  In other words, the money doctors epitomized ``the international flows of expertise" by their arrival into the crisis-hit nations. These were not just symbolic geastures. Indeed, they were ``complements to international flows of money (international capital flows)" intended for those very nations. That is why, by a mainstream narrative about financial advising by foreign experts, ``a defining feature of money doctoring is its tight association with crisis lending. \textit{The money doctor is truly part of the process of brokering money against reform.}" \citep[p.~3-4, emphasis original]{fland2003} Also, this narrative was built upon the employment of the basic devises of metaphorical thinking discussed in Chapter \ref{sec:intro} (Section \ref{sec:metaphor}, pp. \pageref{sec:metaphor}-\pageref{sec:whats_capital}). In particular, the concept of money doctor is an outgrowth of the blood metaphors of money which were initiated by the philosophers of political economy who themselves were trained as physicians:

\begin{quote}
From a metaphorical point of view, expert advice to foreign governments as a
branch of applied economics has to do with the medical practice. Money
doctors inherited an old tradition in economics, dating back to Fran\c{c}ois Quesnay
and the physiocrats in the 18th century, and which in the 19th century was illustrated by Cl{\'e}ment Juglar, among others. This tradition resisted the gradual rise
of an economic science taking physics as its role model, emphasizing concepts of
equilibrium and progressing through deduction. Quesnay, Juglar (who were both
physicians) and followers preferred to look at the economic organization of a
given country as a consistent whole, just as the human organism, and favored
inference. For their late-19th-century counterparts, economic problems such as
exchange crises were the \textit{symptoms} of deeper flaws which economists \textit{as doctors} had to address. This is how the international macroeconomist, when operating in
emergencies, became a "money doctor." \citep[p.~2, emphasis original]{fland2003}
\end{quote}

This abstraction is unhelpful in this dissertation's inquiry into the matters of interantional finance with pervesive capital ``flows/mobility." Hence, the terms of ``money doctor(s,-ing)" are omitted. According to the literature on money doctors,\footnote{See \cite{drake1989}, \cite{rosenberg1999}, \cite{fland2003}, \cite{schiltz2012}, \cite{moneydoctors2021}, \cite{alvarez2024}.} their advising was associated prevailingly with policies that lead to the establishment of the gold standard during the second half of the nineteenth and first half of the twentieth centuries. Later on, these policies resulted in the external debt dependence and subordination. To make difference, this dissertation retains more specific term of international financial advising instead of money doctoring. The former carries with itself a broader and specific meaining rather than narrow and metaphorical one of the latter. The above discussion on the \citeauthor{innes1913} career within the \ac{bfo} provides that broader conceptualization. 

The domain of international finance of the nineteenth century was characterized by two pillars. First, the dominance of the British-led gold-standard system that rested on the Bank of England and the City of London, the major global financial center of that time. Second was the profit-seeking activities of the major financiers from City of London, Paris, Berlin, and other centers in organizing international bond issues with higher-than-average returns, the raised capital from which was supposed to some development (for example, railroad constraction) -- hence, \textit{haute finance}.\index{Haute finance}

One can infer that \citeauthor{innes1913}' advising to foreign governments---to the Kingdom of Siam\index{Siam} and Egypt\index{Egypt}---on financial matters was of two types. In the case of Egypt it was about all things that related to the workings of and with the international finance system of that day. From the bonds related issues to capacity building of the domestic governmental bodies. In the case of Siam it was about just the latter issues thanks to its retained freedom from external debt dependence. Generally speaking, governments of many nations required an advise on the rules of the game in that system of \textit{haute finance}\index{Haute finance} and active mediation with it. Foreign offices of major empires were quite active in conducting this kind of service. However, it must be recognized, such service was provided or withdrawn not purely on the basis of the private finance interest in obtaining larger profits. Instead, they were provided strategically given the tense rivalry among the major empires themselves \citep[see][]{viner1929}. British diplomacy was in the forefront of that activity thanks to the centrality of the City of London for \textit{haute finance}\index{Haute finance} of that time. 

Diplomats of Great Britain cooperated with the financiers of the City of London. They as well cooperated with other foreign diplomats and bankers, when they were dealing with financing of the less developed nation. The following account of railroads construction in China in the early 1900s is illustrative:

\begin{quote}
Bankers and diplomats maintained close ties through \textit{frequent} contact, most times officially and sometimes socially. \dots Were financiers and statesmen aiming at the same ends, or did they have different perspectives? \dots Perhaps the best answer that can be given to the question of shared aims between Westminster and the City, Washington and Wall Street, is to suggest that in neither case was there an absolute congruence of interest and action. There were some shared goals. Both parties in each case sought Chinese dependence upon the West, but profit, not surprisingly, played a stronger part for the bankers, who were certainly not the patriots indifferent to finance that they presented themselves to be. \citep[p.~263, emphasis added]{davis1982}
\end{quote}

The above-mentioned accounts reveal the exposure the diplomats had to the financiers from the City of London and Wall Street of New York and their dealings. It is reasonable to suggest that \citeauthor{innes1913}, while on his international assignments, could not escape the shared feature of the British diplomatic corps. His critical eye was capturing the benefits and consequencies of the private finance. His writings on this topic that he realized the connection between debt and dependency, which was all about control by the lender over the borrower through the means of providing or withdrawing credit. 

And that control by the creditors, and its flip side the dependency of the borrowers, is well documented in the literature. That control/dependency was not a sole prerogative of the nation, Great Britain, that managed to be on the top of the monetary hierarchy under the British gold-standard system. Other nations that adopted gold-standard rules of the game had realized that they can exercise them same power to the nations that happened to be on the lower tires of the international monetary hierarchy. 

Thus, a prime example of that evolution was Japan, which ``chose to emulate the Western imperialist example" during the period referred to as ``a brutal opening up (kaikoku)" \citep[p.~61]{schiltz2012} that lasted fifteen years from 1853 and through 1868. Japan of 1853, being an isolationist state, found itself under pressure of military fleet of four steamboats that entered the harbor of what is today's Tokyo. The foreign military expedition was ordered by US President Millard Fillmore and led by Commodore Perry. Its mission was to put demanding pressure on the Japan authorities to open the country to the international trade. The pressure was a success for the outside world. Inside Japan, it stirred a wave of active re-thought of the behavior of this nation within the present international framework of military and economic power:

\begin{quote}
[T]her [Meiji] Restoration proved to be a revolutionary event designed to accommodate to the norms, rules, and institutions of a new international order. The new leaders [of Japan] themselves had no clearly formed plan for a reformed institutional structure; rather, determined to do whatever was necessary to restore Japanese sovereignty and build national strength, they took their cues from the institutions of the great powers. \citep[p.~68]{pyle2008}
\end{quote}

The political aim of restoring nation's sovereignty in Japan under Meiji, since 1968, was based on the mix of the insights provided both by domestic and foreign thinkers on the particular details of the international system of power, law and finance. Thus, as far as international law system is concerned the following realization made by a top representative of Japanese official class is noteworthy:

\begin{quote}
One cannot depend on international law without having a well-prepared military force. Many countries use the cloak of international law to seek their own interest in dealing with weaker nations. \citep[p.~61]{schiltz2012}
\end{quote}

In 1879, eleven years since Meiji Restoration began, Japan received a foreign advice on the international finance system from the former U.S. President Ulysses S. Grant\index{Grant, Ulysses S.}. The latter had a world-wide tour after his departure from the White House, visiting many capitals of the nations and meeting with leaders of these country. His stop in Japan culminated with a face-to-meeting with Emperor Meiji at Shore Palace on August 10th, 1879. This conversation was well documented as Grant's papers contain its detailed transcript. Grant's remarks on the international monetary system unmistakably provided a straightforward advice on preserving and shoring up Japan sovereignty. They deserve full citation that is following:

\begin{quote}
Another subject that occurs to me is the question of foreign indebtedness. There is nothing a nation should avoid as much as owing money abroad. Any individual, who borrows money from others to such extent that he can not repay at his will, is altogether helpless and becomes enslaved to the principal. Indeed, none can feel more humiliated than he! Being so in the case of any individual it is more so in the case of any nation. Look at Egypt\index{Egypt}, Spain or Turkey; how helpless they are! National resources are all hypothecated to such an extent as they have now nothing that they may call absolutely their own. In Egypt\index{Egypt} the Kedief is forced to abdicate! See in Spain the extravagant rate of internal taxes of all sorts as the necessary result of superfluous foreign loans. There the corruption among revenue officials of all grades are fast ruining the nation, a nation of great capacity and resources. \par
I am glad to hear that the foreign loan of Japan is not so large; at any time it might be paid off if holders of her bonds would receive the money before they are due. The sooner it is paid back the better it will be for Japan. Japan, if possible, ought never to borrow any more from foreign nation. \par
You are doubtless aware that some nations are very desirous to loan money to weaker nations whereby they might establish their supremacy and exercise undue influence over them. They lend money to gain political power. They are ever seeking the opportunity to loan. They would be glad, therefore, to see japan and china which are the only nations in Asia that are partially free from foreign rule and dictation, at war with each other so that they might loan them on their own terms and dictate to them the internal policy which they should pursue. \citep{grant}
\end{quote}

Grant's remarks should be translated to the modern-day reader. Under the gold-standard system an embedded stability was sought after by the participants of this system in terms of convertibility of the domestic money unit of the borrower's nation into the foreign one (of the creditor's nation). It was an explicit commitment of the borrower's national authorities to keep economic environment in a shape that would best fit to the redemption of the foreign loan at stable exchange rate of the national money unit to a foreign one. Exchange rate stability was maintained through convertibility of the credits denominated in the domestic money of account to gold at some predefined level. Hence, when Grant was talking of never borrowing from a foreign nation, it should be translated to what today's \ac{mmt} literature is saying about avoidance of issuing financial liabilities denominated in the foreign money of acoount. Between then and now, the institutions and the language used to describe the matter are different however the meaning is shared.

Back to Japan of the late nineteenth century, internal thinking of the ruling elites turned towards emulation of the Western powers. The Sino-Japanese War (1894--95) is usually mentioned as a tuning point in the Japanese assertiveness towards foreign interventions into the neighboring territories such China, Korea and Taipei. This turn was accompanied by Japan joining the gold-standard system and further imposition of Japanese own monetary control over less developed nations, which were Taipei and Korea. Politicians of the day in Japan sounded a popular excuse for binding up monetary affairs of those nations, i.e. limiting their own sovereignty, by adopting there a gold-standard system controlled from Japan and then providing loans to those lands. The following quote is dated 1894:

\begin{quote}
How was it that British had an excuse for intervention in Egypt\index{Egypt}? Was it not in the fact that England had obtained its position of interest by providing Egypt\index{Egypt} with capital? \citep[p.~75]{schiltz2012}
\end{quote}

Eventually, evolution of thinking inside Japan over the second half of the nineteenth century moved from total isolation of the country towards grasping the basic principles of the international power system, where ``currency imperialism is a powerful means of introducing informal empire" \citep[p.~19]{schiltz2012}. The Japanese then followed the lead of the established major powers of the time. 

\citeauthor{innes1913}, being based in the region in that very period as part of British diplomatic corps, could hardly had left unnoticed those developments during his service as resident advisor to the King of Siam\index{Siam}. It could be the case that it was King of Siam\index{Siam} that felt the pressure of those rapid developments himself and treated them as opportunity and, quite possibly, as a threat if mismanaged from its court side. It is reasonable to infer that \citeauthor{innes1913} was observing the workings of the international system of finance and power firsthand. This affected his understanding of the subject matter he worked with. His advising must had been affected too.  

Modern-day literature on the ``money doctoring" such as \cite{rosenberg1999}, \cite{fland2003}, \cite{schiltz2012}, \cite{moneydoctors2021}, \cite{alvarez2024} omits the episode of the former U.S. President Grant\index{Grant, Ulysses S.} being in the shoes of financial advisor to Japan, not to mention \citeauthor{innes1896}' laboring in this capacity both in Egypt and Siam. The Grant episode was a very short-lived one. However, domestically-written literature on the development of modern economy in Japan does refer to this episode treating it significant as far as ``the Japanese government's later policy" of the twentieth century is concerned \citep[p.~186]{takahashi1969}. Such omission on the side of the above-mentioned ``money doctoring" lierature is due to its narrow definition of a financial advisor (``money doctor"). In this literature, a person qualifies to be a money doctor if she or he conducts financial advising in the foreign country which was in line with monetary orthodoxy. It meant such a person was lobbying the domestic government for greater engagement with \textit{haute finance}\index{Haute finance} and, in parallel, pushing the reforms required to place such a nation on the gold standard. For some nations, especially in Latin America and Asia, unlike in Japan, it resulted in the prolonged external debt dependencies extending towards today. 

In consluion, this section made the point that \citeauthor{innes1913} was practically a financial adviser. His 28-years-long service in the British diplomatic service coincided with a period, when gold standard as international monetary regime and international private finance, or \textit{haute finance}\index{Haute finance}, were in their peak. City of London was the centeral to both and primacy of the British Empire in finance was widely recognized and followed internationally. Hence, for diplomats of the time \textit{haute finance}\index{Haute finance} was a daily job. \citeauthor{innes1913} experience must had been aided by the fact that he worked in Egypt under leadership of Evelyin Baring (Lord Cromer\index{Lord Cromer}). The later, in addition to be a \textit{de-facto} ruler of Egypt\index{Egypt} as part of the British Empire, was part of the Baring family, which was running one of the world's most powerful financial house Baring Brothers \& Co.\index{Baring!Brothers \& Co.} since late eighteenth century. It is documented that Cromer was maintaining close relationship with at least one brother, who was working in that financial house. This factor along with many other, this dissertation infers, did contribute to \citeauthor{innes1913}'s active learning of the subject of finance while on the job. Cromer distinuished his performance by picking him as most experienced candidate to serve as a (fist ever) financial advisor to King of Siam on the request of the latter. The case of the early development and transformation of Siam into a modernized state during the 1890s reveals \citeauthor{innes1913} as an adviser with a thinking of, what later would be defined as, the \textit{chartalist}\index{Mitchell Innes, Alfred!Chartalist} or non-orthodox approach towards monetary matters. Whereas many international financial advisors of the time brought with them  into the nations, which invited them, the thinking of, what later would called as, the \textit{metalist} or orthodox approach to monetary economics. The difference between the two approaches has been not trivial and, in breif, the parting points have been about the historical origins of money and what drives or determines their value.\footnote{See \cite{goodhart} on both approaches and Chapter 2 from \cite{wray1998} for an extensive discussion of the chartalist approach.}

\subsection{The U.S. Monetary Orthodoxy Before First World War}\label{sec:mon_orthodoxy}

In 1908, after his diplomatic-financial missions in the Orient (the Middle East and South-East Asia), \citeauthor{innes1913} headed to the Americas continent to serve as Councilor of the British Embassy in the capital of the United States \citep[p.~4]{wray2004}. He was going to stay in this capacity through 1913 to conclude his \ac{bfo} service on the five-year assignment in Uruguay from 1914 and through 1919.

Yet, writing his 1907 essay criticizing the predominant treatment of the native populations by the foreign rulers from major empires, \citeauthor{innes1907} referenced to the U.S. as the ``great-souled nation" that was giving a proper path to other nations for the future ``fruitful interchange" . Back then, as it implies from his essay, he followed the news coming from the United States attentively, probably anticipating his future asignment there. He was writing enthusiastically about its politics by noting that ``[w]hen [the U.S. Secretary of State] Mr. Root can publicly pronounce \dots words [of promoting mutual interchange and assistance between the American Republics], when President [Theodore] Roosevelt thinks it is not beneath him to sit down to dinner with [a born in slavery] Mr. Booker Washington," then the ruling class of the European empires should follow this example for the sake of greater social coherence between different groups, classes and races of people \citep[p.~47]{innes1907}. These remarks by \citeauthor{innes1907} show he was simpathetic to this country's prospects generally and to the Roosevelt administration in particular. Knowing the country from the outside usually differs from knowing it while being inside. \citeauthor{innes1907}' experience with the U.S. on its own soil must had stimulated his analyical mind to accept the intrecate complexity of the society at that stage of its development.

Upon his arrival to Washington, D.C. he faced the country that just passed the stress of the financial panic of 1907. The American public in large was occupied by the economic question having two parralel pathways: domestic and international. Domestically, reform proposals of the banking and currency systems topped the political agenda. Internationally, the policy of ``dollar diplomacy"\index{Dollar diplomacy} was prevailed \citep{rosenberg1999}.

There are no records showing exactly which sources of information the British Embassy clerks, including \citeauthor{innes1913}, used in their Washington office. As proxy, this dissertation uses the journal articles published in the U.S. at the time to reconstruct the climate of opinions.

\subsubsection*{Domestic Developments}

At that moment of time the country lived under the gold standard, which was in place since 1879.\footnote{See \citep[p.~18]{eichengreen2008} and \citep[p.~6]{woodward2011}.} It was the economy, where Wall Street was the epicenter of the financial industry. Its greatest innovation, not available in other financial centers of the world, was the New York call-loan money market that functioned from late nineteenth century despite the regular ups and downs.\footnote{The call-loan money market managed to survive until Great Depression and seized to exist afterwards.}

The financial and banking panics were not infrequent. During the nineteenth century, contemporaries used to observe that next panic produced a greater economic depression than the previous one.\footnote{See \cite{richardson2015}.} Prior to the panic of 1907, the previous one took place in 1893 and was still fresh in the people's minds. It produced, in addition to a lengthy depression, a new wave in American politics. It would be called much later as ``a refom tradition that led to the Progressive Era and eventually the New Deal" \citep[p.~30]{unterman2017}. But at the very beginning, during the presidental campaign of 1986, it started debating the currency question, being a reform proposal correcting the depressive stringency of the gold standard. It is epitomized by a now-historical speech made by a 36-year-old presidential candidate from the Democratic party William Jennings Bryan, who aimed at the opponents while saying: ``[y]ou shall not crusify mankind upon a cross of gold" \citep[p.~28]{unterman2017}. Despite broad support the reform movement had lost to the conservative side that favored preserving the existing monetary regime.

In the wake of 1907 crisis, the Economic Club of New York held a discussion on February 5th of 1908 betweem four notable figures from the American business and political circles, including Bryan. Their speaches were published in the \textit{The Journal of Accountancy}: \cite{morawetz1908}, \cite{carnegie1908}, \cite{gage1908}, and \cite{bryan1908}. The main topic they touched upon was how to reform the U.S. monetary sphere so that to avoid future financial panics. The mood was decisive and even edged with desperation. Thus, one of the speakers was a leading captain of the industry \citeauthor{carnegie1908} who claimed ``[o]ur banking system is the worst in the civilized world" \cite[p.~357]{carnegie1908}.\footnote{Andrew Carnegie did let know of his opinion expressed during the debate at the Economic Club of New York to President Teodore Roosevelt in a letter dated February 15th, 1908 \citep{carnegie1908_}.} Another participant of the debate, former Secretary of U.S. Treasury \citeauthor{gage1908}, agreed with previous speaker' in the's assessment. He mentioned that ``we have \dots the most imperfect system," which siezed to operate ``at a time of profound peace, at the close of the most prosperous year the country ever saw, in the year 1907," unlike the French system that managed to support the local industries even during ``the rise of the Commune anarchy" \cite[p.~362-363]{gage1908}. \cite{morawetz1909}, speaking from the conservative wing of the political spectrum, hoped that leaders of the Democratic party, referring to Bryan without mentioning his name, would not reopen the currency question of 1896 about the gold standard. Whereas \cite{bryan1908} kept pushing for greater government intervention and his proposal at that very moment of time had two points: (a) government-issued currency to be issued in emergencies, and (b) government backing of bank deposits. It is interesting that the side, representing the interests of major businesses in this debate, did rely on the proper theoretical approach towards the operations of banks, which would later be recognized as \textit{endogenous} money approach.\index{Money!endogenous approach to}\footnote{On endogenous money approach see \cite{wray1990}. In the debate held on February 5th of 1908 by the Economic Club of New York \citeauthor{morawetz1908} elaborated on the bankings transactions invoking endogenious money approach without knowit it: \begin{quote}Nine-tenths of the so-called deposits of the banks and trust companies do not represent the deposit of any
money, and never did represent the deposit of any money. The
use of the expression ``bank deposits" is misleading to many who
are not familiar with the actual course of the banking business.
Nine-tenths of these so-called deposits were created by the banks
and trust companies through mere book entries for the purpose
of making a profit by loaning their credit for a consideration.
When a man applies to a bank to borrow \$100,000 he rarely wants
to carry away the money in a bag. What he wants is credit with
the bank so that he can draw his checks from time to time to pay
his bills. He gives his note to the bank, payable in say three
months, with interest at six per cent., and the bank credits his
deposit account with \$100,000 as if he had actually deposited that
sum, although he never in fact deposited a dollar of money. The
borrower generally receives no money, it being understood that he
will draw upon his deposit credit only gradually, and that he will
at all times leave a substantial balance undrawn. Not a dollar in
currency has been received by the bank. The transaction is
simply a ``swapping" of credits. Again, a man deposits with his,
bank a batch of checks drawn by other people on other banks,
and receives a credit as if he had actually deposited so much
money. These checks are then settled through the clearing houses by turning in to the bank which presents them checks drawn upon it by its own depositors. In this case also no money, or very little money, passes, and the transaction is merely a ``swapping" of credits. \citep[pp.~347-348]{morawetz1908}\end{quote}In other writings of the time by representatives of business community, who offered their opinion on the monetary reform, one can find similar awareness of the proper understanding of the banking operations. The following extract is from essay by president of Commercial National Bank, Chicago: ``[t]he banker is
the bookkeeper and settling agent for the business world" \citep[p.~370]{roberts1909}.} At the same time, they postulated that gold is ``the only ultimate reserve for the payment of liabilities in money," that government-issued money was a heresy and it was an evident that over-expansion of bank credit in double-digit ratio relative to gold reserves caused the panic in the first place. The key proposition from the business side of the debate was to establish a Congressional non-partisan commission on the monetary affairs. The presidential campaign of 1908 ended up with a vote favoring a continuation of the Republican party administration. The Congress did create the monetary commission in 1908 and it would study the other nations' experiences with respect to banking and currency from 1909 and through 1913, when the central bank was created in the US after upon the Federal Reserve Act. By chance, \citeauthor{innes1913}' assignemnt in the United Stated coincided exactly with the period of the most concentrated and publicized debates among the politicians and financial experts this country ever had before on the required function of a central bank.\footnote{The presidential elections campaign of 1896 witnessed the peak of the mass population of voters participating in the monetary debates. It was epitomized by the rivalry between two presidential candidates William Jennings Bryan, the Democrat, and William McKinley, the Republican. The former campaigned against the tyranny of the monetary system based upon the gold standard, hence, Bryan's ``cross of gold" thesis. The latter campaigned on preserving the status quo. The latter won the elections. Hence: ``Bryan's epochal defeat in the ``battle of the standards" marked the beginning of the end of the money question as a popular cause. The organization of currency and credit became much less open to broad-based political challenge thereafter." \citep[p.~205]{sklansky2017} Indeed, during the presidential elections of 1900 and 1904 the proponets of the gold standard within both the Republican and Democratic parties worried that their opponents could manage to reopen the ``currency question" again as in the 1896 campaign \citep{mckinley1900}.}

\subsubsection*{International Developments}  

In the international arena the country's officials and its financial top brass were challenging the dominance of the British-controlled gold-standard system by the means of the ``dollar diplomacy"\index{Dollar diplomacy}: 

\begin{quote}
[Since the turn of the twentieth century the] U.S. officials began to shape, for the first time, a foreign financial policy. They sought to stabilize and open new areas of the world economy to a growing volume of U.S. trade and investment by spreading the gold standard. Gold-standard diplomacy, the work of a cadre of economists who became America's first generation of professional international financial advisers, would quickly broaden into the much larger agenda of ``dollar diplomacy"\index{Dollar diplomacy} and, during the 1920s, into programs to stabilize currencies and rationalize financial practices around the globe. \citep[p.~4]{rosenberg1999}
\end{quote}

It was the period, when professionalism was on the rise generally and in the U.S in particular. Hence, the profession of economists was shaping its form quite distinctively and the dominant vocabulary of the people in this profession was centered around the gold-standard system. The economist profession itself evolved to serve the demands of the day (suited to the ``dollar diplomacy"\index{Dollar diplomacy}) and within it a specialty of international finance economist had emerged. The most publicly visible and ``especially influential" international economists in the U.S. at the beginning of the twentienth century were Charles Conant, Jeremiah Jenks and Edwin Kemmerer \citep[p.~5]{rosenberg1999}.\index{Conant, Charles A.}\index{Kemmerer, Edwin W.}

Conant\index{Conant, Charles A.} was ``one of the crucial figures in formulating policy in the early days of dollar diplomacy" and his book \textit{History of Banks of Issue} first published in 1896 was regarded as ``chief authority [written] in English" \citep[p.~887]{bankersm1908} and ``influential" \citep[p.~8]{schiltz2012}. Jenks was professor at Cornell, who participated in several government assignments and later brought his student Kemmerer into the field of ``dollar diplomacy"\index{Dollar diplomacy} that was gaining strength. 

Since the 1896 presidential elections, Conant maintained his political affiliation with ``Gold Democrats,"\index{Gold Democrats} members of the Democratic party that opposed to anti-gold standard Democrats such as William J. Bryan\index{Bryan, William J.}, mentioned above. Thanks to this affiliation, Conant was publicly active by taking part in the Indeanapolis Monetary Convention of 1898. It aimed at cementing the gold standard monetary regime in the U.S. Eventually, its activities were credited with adoption of the gold standard law in the year of 1900, a victory for Gold Democrats and like-minded Republicans over the Bryan's led movement on monetary reform, \citep[p.~887]{bankersm1908} and \citep[p.~227]{sklansky2017}. It is interesting to note that Indeanapolis Monetary Convention was organized by Hugh H. Hanna, a local businessman, who maintained confident relationships and correspondence with U.S. President Teodore Roosevelt \citep[see][]{hanna1903}. In 1903, President Roosevelt appointed Hanna to lead the three-member commission on international exchange that would be envolved in the international effort with other European major powers of spreading the gold standard regime into the Orient countries, Latin America and Carribeans \citep[p.~243]{sklansky2017}. In addition to Hanna as commission's chair other two members were above-mentioned Conant and Jenks.

In the U.S., Conant\index{Conant, Charles A.} was recognized as a pioneer of the new profession of international financial advisor. He is credited ``[a]s the creator of a new currency for the Philippines under American rule" \citep[p.~208]{sklansky2017}. Conant's work on the theory of money---book \textit{The Principles of Money and Banking} that was published in 1905---cited the 1896 book \textit{Money and the Mechanism of Exchange} authored by W. Stanley Jevones as a basis from which his own view developed. In both works, origins of money presented through barter as a starting point followed by the invention of money as medium of exchange that helped to solve the prblem of double coincidence of wants. In similar wane, these works consider money as, according to Conant definition, ``commodity of intrinsic value acceptable in exchanges" such as gold and silver (the precious metals) coins. Further evolution of the precious metals money system allowed introduction of credit instruments. 

Kemmerer\index{Kemmerer, Edwin W.}, the youngest of the above-mentioned troika of international financial advisers, received his initial assignment to the Philippines, where he stayed for three years through 1906 and was instrumental in realization of the U.S. dollar diplomacy\index{Dollar diplomacy} by carrying out profound reorganization of the local institutions that would fit to the U.S. dollar-led gold-standard system. Upon his return to the US, Kemmerer became a prominent participant of the US involvement in the currency stabilization programs and introduction of the gold-standard systems in the countries that turned to the US for assistance. Given the publication of the book on money in 1944  shortly before his death, Kemmerer throughout his lifetime held the precious metal-based view on the origins of money and its corollary inferences previously mentioned in the relation to Conant.

In 1910, \citeauthor{innes1913} wrote an internal report to the \ac{bfo} on the subject of the banking and currency system in the United States . The content of the report, which at the time obtained a status of confidential document within the halls of the government of the United Kingdom, is investigated in the next section (below).

\subsection{The 1910 Report}

On May 28th 1910 the British embassy at Washington, DC sent a dispatch to the \ac{bfo}. It delivered a copy of the report by \citeauthor{innes1913} titled \textit{Banking and Currency in the United States}. It counted 14 sections with total number of words just below 12,000 and one chart, which is referred to by the author as a table \citep[p.~42]{innes1910}.

It was an internal report intended to describe in concise terms the subject to the officials of two departments of the British government: the Treasury and the \ac{bfo}. After reading it, the government officials concluded that ``it should be treated as a confidential report," according to the accompanying letter signed by the British government official on August 30, 1910 \citep[p.~41]{innes1910}.

The introduction section of the 1910 report answers the question why the author had written it. Innes was writing his report as a reflection to the financial crisis of 1907 that happened the US just few years ago and the US politicians were busy to re-define the monetary system that proved failing by consulting with best practices in Old Continent. It was the time when the US legislators passed emergency currency law and set up Monetary Commission led by Senator Aldrich to address the matter.

By the very first, opening paragraph of the report Mitchel Innes stated:

\begin{quote}
Ever since the causes of the late financial panic have been studied and more or less understood, one thing has become clear; it was largely \textit{artificial}, by which I mean that there was no solid reason for such a panic as occurred. The country was thoroughly prosperous, trade was increasing, there was no shortage of crops. Speculation had, no doubt, overshot the mark, the prices of investments had been pushed on too rapidly; but there was nothing in all this to compel almost every bank in the country to suspend payment. \citep[p.~5, emphasis added]{innes1910}
\end{quote}

This passage on the real economy performance---that ``[t]he country was thoroughly prosperous, trade was increasing, there was no shortage of crops"---sounds like an echo to the contemporary observation by \citeauthor{veblen1904}, who distiquished between brisk and dull times---on the business cycles in the US economy, when economic conditions were shifting from brisk and dull ones while the physical capacity of the society to carry out economic activity was not under any sort of physical destruction \citep[p.~89]{veblen1904}.

This said, Mitchell Innes went on and elaborated on his initial statement:

\begin{quote}
One of the causes of the crash was the currency and banking system, and this was so thoroughly realized when the panic has subsided, that there appeared to be no two opinions on the point that the system required modification. \citep[p.~5, emphasis added]{innes1910}
\end{quote}

That reference that ``there to be no two opinions" on the major cause of the financial crisis (called panic and crash by the author) also revealed the author's belief, he shared with some of the US key politicians and businessmen, that changes in the institutional design of the monetary system were about to unfold in the US. Key expectation was that a central banking institution would be established in the end. The US economy of the time was one that witnessed accelerating increase in the population of its society and, quite importantly, its wealth was ``increasing at a rate of which we of the Old World can hardly form an idea" (ibid). With this economic might, Mitchell Innes envisioned, this country's ``central bank would control finances not so much of a country as of a continent" and ``it will readily be understood what a power in the financial world such a bank would wield." (ibid) 
The above-mentioned passages the author used just to introduce his readers to the core of the report - institutional detail of the banking and currency systems as they were at the time. Sections from II through XIV are describing these systems, while the first section was called ``Introductory" and served appropriately. Out of thirteen sections, the first twelve ones are devoted to the banking system, while the last one (numbered XIV) to the currency system.
What is noteworthy of those descriptions? 
The author begins with a legal side of the matter. In particular, with ``a curious double system of government" (ibid, p.46). Such a system features States and the US Federal Government. The former, i.e. each State, is a self-governing community with its own parliament and executive bodies. Disputes are tried and settled by State courts. The latter, i.e. the Federal Government, has powers too defined by Constitution. Disputes are tried and settled ``by special courts wholly distinct from the State courts" (ibid). Within such a legal structure the banks may operate under any one of the systems of law. Hence, (1) National Banks were operating under ``the virtue of a charter from the Federal Government," (2) State Banks were operating under the virtue of a charter from State Government, and (3) private banks operated outside of the banking laws of the Federal Government and various States. 
During the second half of the 19th century, under the legal framework in place national banks were required to deposit government bonds with Treasury in order to obtain a lawful right to issue of notes in the amount no more than deposited bonds. The former was obligatory, while the latter was optional. Hence, there was a compulsory purchase of government bonds. In such a way a market for government bonds was secured. It enabled the Treasury to issue the government bonds at ``abnormally low rate of 2\%" (ibid). There was too another legal provision that put a constrain on the bank issuing notes and it was associated with paid up capital.
These constraints Mitchell Innes became a burden in the panic time and required intervention by the legislators, who passed in 1908 the Aldrich-Vreeland law or the Emergency Currency Law. It introduced flexibility in note issuance by banks, which became allowed to form a national currency association and the latter would be able to issue notes covered by a wider range of assets that just the above-mentioned government bonds. Mitchell Innes stated that such a law made a stride forward as it recognized an important principle ``which, though by no means unknown and untried, has never been adopted in English or American currency regulation" (ibid, emphasis added). In particular, he elaborated that:
[t]he theoretical importance of the law lies in the fact that it is a tentative move in the direction of what is known as an ``asset currency", or a currency secured by nothing but the ordinary assets of a bank, a principle, which, as I have said, had not before made its appearance in the currency legislation of either the United States or of England. (ibid, emphasis added)
Such a principle being in place and the fact that business community has ``mere knowledge that additional currency can be procured if it is urgently needed will prevent the necessity from arising" and a risk of a panic or crisis would be reduced. To support his argument, Mitchel Innes recalled a recent episode in England where a crisis was averted by ``mere knowledge" that government would ``suspend the Bank Act and enable the Bank of England to issue notes without a covering of gold" (ibid, emphasis added).
The Emergency Currency Law had its own limitation. Mitchel Innes pointed out that such was issuance of asset currency could take place in times of stringency. It was ``obvious illogical" to the author, because ``[i]f such a currency is safe in times of emergency, it is a fortiori safe in normal times."
Another curious element of the US banking system, which Mitchel Innes singled out back then, was legislation clause that National Banks may not have branches. In his opinion, the fact that such a clause was put in effect indicated an over-regulation as well as a tendency of the authorities to become more suspicious of liberty. In Europe, universal practice was of allowing bank branching, while the US opposed it. At the same time, Mitchel Innes' critic of the US practice of limiting bank branching has a particular angle. It points out to that legislators' desire to restrain banking with 
truly official narrow-mindedness by such phrases as ``the place where its operations are carried on," and ``the usual business shall be transacted at an office or banking house"; the words ``place" and ``an office" being held to exclude plural. (ibid)
Business response to this erroneous provision was such that about 25,000 small banks grown up. The author elaborated his criticism by following statement:
The large number of banks independent of each other is generally considered to be a source of weakness, while the superiority of the banking system of Canada, with its great central banks and numerous branches is pointed to as an instance of a sound system. (ibid)
Still, author's critic of restrain on bank branching had another dimension. His explicit reference to the wording of the legislation, where references to place were in singular (not in plural) form. Those singular words-"place" and ``an office"-suggested that legislators had specific idea to contain the banking business particularly in spatial terms. This approach by the legislators run contrary to Mitchel Innes' own credit theory of money, which was laid down later in the 1913 and 1914 papers. 
Nevertheless, the key source of the weakness of the US banking and currency system, according to Mitchell Innes, was the above-mentioned dual system of charters than lack of bank branching. The State banks operated under the lax system of supervision as defined by different states, while the National bank were subject to quite stringent conditions, which the author characterized as being ``constant and increasingly effective supervision."
On bank notes, Mitchell Innes pointed out that supply of notes into circulation was pretty rigid at the time. Thus, National banks could issue notes only against the equal value of government bonds deposited with Treasury (no more than that). So, the size of the notes in the circulation was inflexible, it did not meet the needs of the market. Generally, the issuance of notes was inelastic: ``while there may be a shortage of notes during the bust season, there is a plethora in times of slackness" (ibid).
Another element of weakness of the US banking system was a legal requirement on all national banks to keep a reserve on hand or on deposit with other national bank a minimum of 15-25\%. Mitchel Innes explained his view of this provision in the following way:
The mere fact that a minimum legal amount is fixed for the amount of cash on hand gives the impression that, if the limit is not attained, there is something wrong, that the bank is shaky, although there may be no foundation for the belief. 
(ibid, emphasis added)
This faulty provision, according to the author, might give faulty impression of weakness of the banking institution to the wider business community. It stirred impression that an institution that failed to comply with the regulation would be punished by the authorities. While for the authorities that enforcement of the penalty on such a bank was optional. However, the provision and possibility of its enforcement created ``serous embarrass[ment for] the bank". Those banks aiming to maintain the minimum legal amount as reserve tended to call in their resources when prospects of a commercial crisis grew up. Hence, collectively their actions resulted in the shrinkage of credit right at the moment, when business required credit expansion and credit shrinkage was ``the most disastrous". Such was an outcome for the economy in the absence of the central bank. In situation like that, a central bank would come to the help of bankers via credit creation ``to any extent on good security, irrespective of the state of its gold reserve" (ibid, emphasis added). This way a threat of a crisis must be averted. Mitchell Innes adds that a central bank in the economy ``is practically a State institution, one might say a Government Department, like the Bank of England" (ibid, emphasis added). In the US, until the legislators passed the above-mentioned Emergency Currency Law, there was unfortunately ``no institution with the power to create unlimited credit or unlimited currency (the two are practically identical) inspiring same confidence as gold" (ibid).
Another feature of the US banking system that, Mitchell Innes explained, ``directly tends to the production of the panics which have been so frequent" was the New York call loan market . The author did not refer to it by its name. His description implies that he was talking about precisely of that market. Thus:
In England the bulk of the business of the joint stock banks consist in discounting the bills of their clients - domestic bills. It, therefore, really represents bona fide commercial transactions, purchases and sales of commodities. In the United States a much larger proportion of the business consists of short loans on collateral security, chiefly stocks and shares. (ibid, emphasis added).
It is rather unfortunate that Mitchell Innes did not elaborate his insightful analysis on this element of the banking system in the US. He just briefly mentioned that nature of the transactions at the stock exchange was ``largely speculative" (ibid). Call loans were short term loans extended against the collateral, which had an active secondary market at the Wall Street. 
In the City of London, bankers accommodated via rediscounting of bills of the firms of good standing. And if the latter was unable to get accommodation by banker, it would turn instead to the bill broker for accommodation, which would serve a middleman between the firm and the Bank of England. In the US, where a central back was absent back then, bankers accommodated the firms by extending loans against actively traded stocks. And if the US bankers would stop accommodating clients because of fear of crisis than, ``no one else can" (ibid). The bankers themselves in this situation could not rediscount own bills and reacted by calling in their short loans (that previously were extended via the call loan market). Furthermore, if bankers' borrowers were unable to redeem the call loans, then the bankers were selling the collateral (shares and stocks) into the market at the very moment when ``investments [were] the least sought after".
This more than anything else makes Wall Street most fluctuating market in the world and keeps the fear of a panic always before the eyes of the Americans. The Aldrich-Vreeland Act has somewhat clumsily remedied this situation. The issue of notes by a national currency association would produce the same result as its achieved by re-discounting bills at the Bank of England. (ibid, emphasis added)
Mitchell Innes provides an interesting observation regarding the functioning of the clearing houses, which were the institutions ``for balancing the debts and credits of the banks". In the US, those institutions differed from the London Clearing house to an extent. The latter was a pure balancing-debts-and-credits institution. The former were ``far more than this" as they acted too as ``a bond of union between the banks". This said, however, the major observation by the report's author regarding the clearing houses was on the methods of settlement. They differed between the clearing houses. There were four methods mentioned:
(1)	Cash settlement, where under cash was understood any forms of money, except the national bank notes . Its share was estimated at one quarter (25\%) of total.
(2)	Manager's cheques - these were the cheques drawn by the manager of the clearing house on the debtor bank. Its share was estimated at 25\%, too.
(3)	Clearing house certificates - these were issued against gold coin deposited by participating banks.
(4)	Drafts on another large town such as New York, Philadelphia, Chicago or some others. 
With such a conclusion on the US clearing houses, Mitchel Innes is talking about what is now called inter-bank market (or the fed funds market in the US):
In some clearing houses the balances are borrowed and lent either with or without interest. In Chicago the practice of ``trading" the balances, as it is called, has grown to such an extent that it is estimated that bout 75\% of the balances are dealt with in this manner. A bank with a large balance to its credit may trade portions of it to many different banks. (ibid)
In this report Mitchell Innes indorsed an idea of insurance fund as a method of guaranteeing the deposits of the banks, which was in the public discussions at the time. It, he explained, would be ``far more economical but also more effective means to the end than reserve of gold".
The final section of the report was devoted to the currency system. It consisted of different kinds money in coin and paper. The range of legal tender money seemed a complication to an ordinary observer. It was not to Mitchell Innes. He explained:
But it is only in name that they differ. All are equally good for effecting payments; all are equally convertible at will at the United States Treasury, and the convertibility of all is equally guaranteed either explicitly or practically by the government; so that the difference in the nomenclature has no significance. The various funds in the Treasury, which can be applied for the purpose of redemption, form in reality one whole. 
(ibid, emphasis added).
Mitchel Innes considered gold and gold hoards as a wasteful element of the banking and currency system not only in the US but in other major countries. At the very end of the report, he aimed to dispel the accepted view that US was a leading economy in terms of gold hoards. His analysis of the gold hoards was based on relative terms (gold vs currency and cheques), not on the gross value. He considered the practice in the US, England, France and Germany by concluding the report with a sentence: ``[o]ne cannot help but ask what purpose these hoards serve."

\subsection{The 1913 and 1914 Articles}

In such an environment, \citeauthor{innes1913} arrived at a point of time that urged him to intervene into the public discourse that was overwhelmed by the viewed of the above-mentioned economists. Residing in the Washington DC, \citeauthor{innes1913} realized that to be heard his view on the money matters had to be delivered to the wider audience through the publication media that is based and circulating in the country's financial center. Eventually, he submitted his work to the journal that published monthly and located in the New York City (the publishing of the journal apparently took place at the address: 27 Thames St, N.Y. City ). Hence, in May 1913, the Banking Law Journal published his first article ``What is Money?" and soon after, in just an eight-month interval, a second one ``The Credit Theory of Money" appeared in February 2014. 
The publisher wrote on the pagers of the journal that first article by \citeauthor{innes1913} attracted ``much comment and criticism, from economists, college professors and bankers, as well as from the daily and financial press, because he differed so widely from the doctrine of Adam Smith and the present theories of political economy." 
In his 1913 article aimed to dispel the universally held belief among the economists that gold-based money preceded the credit instruments, which were wrongly believed to be a later evolution that took place out of coins made of precious metals. He aimed to explain that theories of money that were prevailingly held among economists did not correspond to the ``modern research in the domain if commercial history and numismatics, and especially recently discoveries in Babylonia." Hence, he went on, ``in the light of [these findings] it may be positively stated that none of these theories rest on a solid basis of historical proof - that in fact they are false" \citep[p.~378, emphasis added]{innes1913}. 
Innes' explanations are deep in details however his writing style helps to keep the reader engaged in the reading. In some places as the one just mentioned above, where he ends his arguments against the established beliefs among the economist by using a strikingly short ending like ``in fact they are false." By this short phrase used to emphasize his point, \citeauthor{innes1913} resembles Grant's explanation to Emperor Meiji on the dangers of being in debt to a foreign creditor under the gold-standard system of the day. The Japanese transcript of the meeting reveals that Grant concluded his remarks by this short ending ``I beg Your Excellency to understand this." (see footnote 2 on page 8).
One of the major ideas of \citeauthor{innes1913} was to explain the incorrectness of the dominating view that gold-based money in the forms of coins preceded credit instruments that were believed to be a later development. As Keynes commented on the \citeauthor{innes1913}' 1913 article, his whole exposition of the article was to point out the ``extreme antiquity of credit," i.e. credit is the basis of the money systems as we know them in one form or another and credit is money, while coins embodying credit arrived later in the history of humanity and they had limited usage. Another idea of \citeauthor{innes1913} was to explain the error of the dominant belief that scarcity drives the value of money. His exposition of the value theory of credit aims to correct that omission among the economists. 

\subsection{Conclusion}

Personal experience in the British diplomacy did bring \citeauthor{innes1913} into the face-to-face contact with the British led gold-standard system of late 19th century and early 19th century. It mean that he was involved firsthand in the cooperation between diplomats, financial markets and the countries that borrowed from the bankers and required outside advisory. \citeauthor{innes1913} himself mentioned in his article that his exposition on questions of what is money and what is the credit theory that stands behind money is based upon ``the writer's conclusions drawn from a study extending over several years" \citep[p.~379]{innes1913}. He never produced a standalone volume on the credit theory of money after his departure from his Washington assignment. However, this just indicate that during his professional career his curiosity of the debt, financial matter allowed him elaborate on this subject while he was trotting the globe while in Diplomatic Service from England to Egypt\index{Egypt} to East Asia and then back to Egypt\index{Egypt} and then to the US.

\subsection{!!! Additional quotes !!!}

\cite[p.~5]{roberts2012}: ``Before the First World War, no one would have thought to call the United States an economic hegemon." However, Mitchell Innes yet in 1910 suggested that the US central bank once etsablished would be able to become a powerful actor in economic matters of international stage:

\begin{quote}
[The US] central bank would control finances
 not so much of a country as of a continent [and]
 it will readily be understood what a power in
the financial world such a bank would wield. \citep{innes1910}.
\end{quote}

\cite{samuels1976} provide a review of the prevailing practice of teaching monetary theory adopted in the US colleges and universities in the first decades of the 20-th century. In particular, they cover the case of Harvard University professor A. Piatt Andrew (1873-1936), who served as a ``special assistant" to the National Monetary Commission. The latter was created in the wake of 1907 financial crisis and eventually in 1913 led to the institution of the central banking in the US.

\subsection{\capitalizetitle{Mitchell Innes versus other creditists}}

\begin{figure}[!ht]\centering
\includegraphics[width=1.0\textwidth]{\plotsfolder/econs_lives}
\caption[Major economists writing on the credit theory]%
{Major contributors to the credit theory of money\par \small Red lines indicate years, when major works by the named creditists were published: see \cite{thornton1802}, \cite{colwell1859,colwell1860}, \cite{innes1909,innes1910,innes1913,innes1914,innes1914a,innes1932}
}
\end{figure}

%
% -------------- Henry Thornton (1760-1815) -----------------------------------+
%
\subsubsection{Henry Thornton (1760-1815)}\index{Thornton, Henry}

%
% -------------- John Fullarton (1780–1849) -----------------------------------+
%
\subsubsection{Jon Fullarton (1780-1849)}\index{Fullarton, John}

\citeauthor{fullarton1844} talks of ``a simple process of set-off" \citep[p.~31]{fullarton1844}

%
% -------------- Stephen Colwell (1800-1871) -----------------------------------+
%
\subsubsection{Stephen Colwell (1800-1871)}\index{Colwell, Stephen}

In the United State of mid 19th century, a book titled \textit{The Ways and Means of Payment} was published by Philadelphia-based publisher. It was authored by \citeauthor{colwell1859} (1800-1871), an industrialist from Philadelphia and member of the local elite. See \citep{colwell1859}. 

The author ``had considerable abilities as an economic thinker. \dots  In the 1850s Colwell authored numerous books and articles on topics ranging from the state of manufacturing in Pennsylvania, to the national credit system" \citep[chapter~8, pp.~109-121]{davenport}. 

His biography accounts reveal him as a person, who was an impatient learner of social and economic relations. At the age of thirty six, he started running his father-in-law iron manufacturing business. This took place upon his return from Europe, where he traveled by the insistence of father-in-law to study the state-of-the-arts methods and practices of the local industrialists. He visited England, being attracted to observe firsthand its productive capacities in the midst of Industrial Revolution. In addition, he visited France, Sweden and Belgium \citep[footnote 3, p.~235]{davenport}. He picked up from that learning trip a number of observations that impacted him intellectually. Some accounts mention his dissatisfaction with the observed plight of the working people in the most advanced industrialist lands. He read the works of Adam Smith and critically reviewed his ideas in own writings. His curiosity in the changing world made him great collector of books on the subject of banking and general political economy thought. Upon his death, his personal library consisting of about six thousand titles was handed over to the University of Pennsylvania \citeyearpar[p.~110]{davenport}. He strived to influence the way Americans think of economic matters. In addition to own publications, it appears that he put effort to make the works of German economist Friedrich List published in the United States: the same Philadelphia-based publisher that in 1859 published Colwell's payments book in 1856 published \citeauthor{flist1856}'s major work \textit{National System of Political Economy}. \cite[footnote 12, p.~380]{breton1964} mentioned Colwell as translator of the List's book. Indeed, Colwell favorably viewed the writings of the German economist to such an extent that he wrote an extensive introductory essay to that book, see \citep{colwell1856}. \citeauthor{colwell1859}'s own major work on the subject of political economy was the above-mentioned \textit{The Ways and Means of Payment}, see \citeyearpar{colwell1859}. It became ``his finest treatise" \citep[p.~819]{dorfman2}:    

\begin{quote}
Into its 644 pages Colwell poured a large part of the contents of his vast library on the literature of banking. \textit{(ibid)}
\end{quote}

A condensed version of the book was put into an article, which was published a year after  \citep{colwell1860}. Indeed, at the time, this book was considered ``the most impressive single work on banking produced in America before 1860" \citep[p.~494]{selgin1989}. 

In this book, \cite{colwell1859,colwell1860} explains the efficiency of the commercial activities via utilizing the credit system. In a broad extent, the terminology and its meaning is similar to \citeauthor{innes1913} as far as technique of payment is concerned. Thus, Colwell speaks of the mutual claims between transacting counter-parties, which are extinguished by ``set off" \citep[p.~693]{colwell1860}. However, his description lacks detail of differentiating debts and credits in terms of their maturity, which is in the forefront of Mitchell Innes's analysis.

Also, Colwell mentions what later in the Mitchell Innes writings  was described as sanctity of obligation, albeit in some lighter version: ``[s]o long as a man's own debts are to be regarded as a good currency to pay him with, and he is willing to receive such payment." \citeyearpar[p.~691]{colwell1860}. Then that ``[m]en learned to pay their debts with their credits" \citeyearpar[p.~691]{colwell1860}. Lastly, ``[g]old and silver are seldom lent upon interest; they are never sought for as a medium of payment, because a check upon a bank is preferred" \citeyearpar[p.~694]{colwell1860}. 

Both \citeauthor{colwell1859} and \citeauthor{innes1913} underline the primacy of the money of account concept. It is the language of the commerical activities. The usage of money of account is an established habit and ``a \textit{mental} operation" \citep[p.~3, emphasis added]{colwell1859}. 

What makes \citeauthor{colwell1859} different to \citeauthor{innes1913} is his view that such an efficient system of credit innovated from the previous inefficient circulation of coins. This thinking is, in fact, similar to the present orthodox presentation of money that today's credit sprang up from the money based upon precious metal. Hence, Colwell stands with one foot in the standard economic theory while another is in the alternative theory. 

%
% -------------- Henry Dunning Macleod (1821-1902) -----------------------------+
%
\subsubsection{Henry D. Macleod (1821-1902)}\index{Macleod, Henry D.}

Literature: \cite{bezemer2010}, \cite{skaggs1997} from \cite{smithin1997}, \cite{skaggs2003}, \cite{maloney1991} chapter 6 pp.120-141, \cite{hayek1933} from \cite{socenc1933}, \cite{kregel1986}

"Whatever the title of a Macleod book---he wrote more than a dozen books on economics, several of which went through multiple editions---the principal topics addressed were \textit{always money and credit}." \cite[p.~362, footnote 5, emphasis added]{skaggs2003}.

%
% -------------- John R. Commons (1862−1945) ------------------------------+
%
\subsubsection{John R. Commons (1862-1945)}\index{Commons, John R.}

\citep{commons1923}: ``Money \dots is the center of economic theory, instead of an afterthought \dots" p. 646  ``\dots the correct picture" pp. 643-644 

\begin{quote}
It is, indeed, a kind of dictatorship, through
private property, in that it is effective because the laborers and consumers have no voice in raising wages and reducing prices. When the
government starts in to dictate wages and prices, the railroads, for
example, have great difficulty in obtaining enough capital for extensions. Savings are very largely a matter of wage and price fixing and
there is a capitalistic mechanism based on private property and dominated by competition and fear of bankruptcy that practically forces
savings to be made.  \citep[p.~642]{commons1923}
\end{quote}

\begin{quote}
[T]wo debts are created by the transaction at a point of time --- a debt
 of payment and a debt of performance. These debts are equivalent to the value willingly
 agreed upon in the transaction. The debt of payment is released by a payment of legal
 money. The debt of performance is released by physical delivery of the materials, services,
 or labor, as measured by other legal units. \citep[p.~241, footnote 7]{commons1936}
 \end{quote}
 
\citep[p.~528, emphasis added]{tymoigne2003}: ``Commons and Keynes had a \textit{circuitist} approach to money".

\begin{quote}
 \textit{Some} debts can be used as money, but, for Commons, this supposes that two
 conditions are satisfied \dots First, debts used as money must be negotiable;
 their nominative characteristic must be bypassed by the creation of a market that makes
 them impersonal. The second condition necessary to allow the use of debts as money is
 that \textit{no time discount} is applied to these debts. This means that their market value does
 not increase (i.e., time discount decreases) as their maturity comes closer. Thus, debts
 like commercial papers or bills of exchange cannot be considered as money. For
 Commons, only one debt satisfies these two properties: the bankers' debts, namely,
 bank deposits (and more specifically demand deposits). This is so because, unlike
 securities, demand deposits contain \textit{no futurity}. Bank money is a ``debt past due"; its
 term is \textit{immediate}, bankers being obligated to supply their debt on demand once they
 have granted it. Moreover, it is easily transferable from debtors to creditors via the
 intermediate role played by bankers. \citep[p.~529, emphasis added]{tymoigne2003}
\end{quote}

\citep[Commons quoted in][p.~11161, emphasis original]{whalen1993}: ``\dots we have, not a \textit{circulation} system
 of money, but a forecast and repetition system of money."

%
% -------------- Ralf George Hawtrey (1879-1975) -------------------------------+
%
\subsubsection{Ralf G. Hawtrey (1879-1975)}\index{Hawtrey, Ralf G.}

See \cite{mattei2017,mattei2017_,mattei2018}

"He modeled the world economy as a ``great credit machine." Any sort
of economic relation could be understood as an exchange of debt and credit between
buyers and sellers. The concept of currency was secondary to credit (or debt, debt
being just the opposite of credit)." \citep[p.~477]{mattei2018}

\citep[p.~2]{hawtrey1913}: ``The purpose of the principle of payment for services rendered is to put pressure upon all competent members of the community to work (except, of course, those who possess accumulated property)." Here, it appears that \citeauthor{hawtrey1913} left out the government taxation as a factor that fits to the above-mention purpose. Then, \citeauthor{hawtrey1913} articulates ``the term 'money' being taken to cover every species of purchasing power available for \textit{immediate} use, both legal tender money and credit money, whether in the form of coin, notes, or deposits at banks." \citep[p.~3, emphasis added]{hawtrey1913}

\citep[p.~v]{hawtrey1919}: ``This book was written during the War. The first edition went to press in June, 1919, the very month in which Peace was signed."

\citeauthor{kojima1995} states ``that on the view point of monetary analysis, it is \dots Hawtrey who influenced Keynes." Keynes regarded him as ``grandparent", however, ``neither Keynes himself nor [his biographer] Harrod clarified what had been the influence."

\citep{glasner2021} on the intellectual connection between Keynes and Hawtrey. See \citep{laidler1993} on the link between Chicago school and Hawtrey.

\citep{keynes1920}: 

%
% -------------- John M. Keynes (1883-1946) ------------------------------------+
%
\subsubsection{John M. Keynes (1883-1946)}\index{Keynes, John M.}

\cite[vol.XXV, p.~44]{keynes1980}

\cite{wray1999} criticized \citeauthor{keynes1980}'s \ac{icu} proposal as having two problems. First one is that international monetary system cannot be designed as if trade were `goods against goods`. Second is that its banking analogy is confused. On one hand it is correctly asserts that elimination of convertibility of bancor, money of account in the \ac{icu}, into gold nullifies potential runs on the system. On the other hand, its assertion that balances in bancors might be ``employed to finance" further activity is ``flawed," because by endogenous approach it is loans in bancor create the reserve balances in bancor held by surplus nations \citep[p.~194-198, footnote 16]{wray1999}.\footnote{\cite{wray1999} was published in \cite{harvey1999}.}

%
% -------------- Joseph A. Schumpeter (1883-1946) ------------------------------+
%
\subsubsection{Joseph A. Schumpeter (1883-1950)}\index{Schumpeter, Joseph A.}

In \citep{schumpeter2014} there is a dedicated section titled 'How Payment Really Works'.

\citeauthor{wray2009} recognizes that Schumpeter's ``vision of markets was much more orthodox" \citep[p.~810]{wray2009}.

\subsection{On the Controversy ``All Money is Credit, but Not All Credit is Money"}

The literature on the \citeauthor{innes1913}'s contribution to monetary economics contains a common thread. It is a controversy around his statement that `` 