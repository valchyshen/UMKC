\newpage

\subsection{Introduction}

This paper lays down observations that might be helpful to the readers,
who are interested in the explanation of the Russia's war invasion of
Ukraine happened on February 24th, 2022. It must be mentioned early on
that this is a second act of massive destruction of Ukraine. The first
act happened in February 2014, or nearly nine years ago. Hence, Russia's
war on Ukraine has been a multi-year phenomenon. At the same time, as it
has a length in terms of the observable time period, more importantly it
has a length in terms of the chain of the processes underlying that bias
to social and economic destruction.

The narrative of the below writings deciphers, first of all, the
economic reasoning of the war and where the money sphere is the central
point of attention. Evolutionary processes that took place over time and
institutional details are essential, too. In fact, the stages in the
evolution of the monetary system are begging for scrutiny. This said the
paper should not be considered among the readers as an attempt of
historical analysis.

More specifically, the paper's analysis aims to be consistent with the
framework of \acf{mmt}. Because of author's immediate
affiliation, the paper references quite frequently the ``Kansas City"
approach within \ac{mmt}. The latter has been explicitly articulated, for
example, by \cite{wray_2019,wray_2020,bell,tcherneva,fullwiler_2009,fullwiler_2022,tymoigne2014,tymoigne_textbook}.

In addition, certain economic developments are considered from the
theoretical viewpoint of the traditional economic framework, which is
articulated by New Keynesian economists such as \cite{guriev_2022},
\cite{itskhoki2022a,itskhoki2022b}.

In addition, the paper borrows from the political economy literature
such as \citep{dharvey} to relate the above-mentioned evolutionary processes
in the money economics to the transition by the entire region of the FSU
countries to the market-based economic model from the state dominating
economic model. That transition took place in the 1990s and largely is
still ongoing. And lastly, this paper argues that Russia's war on
Ukraine \textit{is} a part of that ongoing process. ``The feature of
capitalism," the term borrowed from \citep{dharvey2022}, is definitely there
and it cannot be avoided, if one applies the analytical framework
mentioned above.

\subsection{The \ac{mmt} Lens}

There are three key theoretical concepts that stem from the \ac{mmt}
literature. These are (1) the rule of ``one nation, one money of
account,"\footnote{By its meaning, this rule is the exact statement of the rule ``one nation, one currency" made by \citep[pp.~41-42]{wray2012} and, similarly, ``one government, one money" rule of \citep{goodhart2017}.} (2) the hierarchy of money, and (3) monetary sovereignty.

\subsubsection{One nation, one money of account}

Historically, countries with political sovereignty established
themselves by choosing national money units of account. By doing so they
have been knitting their respective economies via debts, both public and
private, denominated in those money units of account. The essential part
of the knitting process has been the states' two accompanying powers of
taxation and money creation. These developments have been ``the most
robust regularities," according to \cite{goodhart2017}. Episodes in history,
when politically-sovereign states were abandoning those powers for the
sake of joining a currency union such as euro-zone, have been
``unprecedented events" (ibid). These nations, not rarely, had ended up
in debt crises and economic stagnation. To conclude, the rule of ``one
nation, one money of account" has been \textit{the usual case}, while
abandoning of the own or national money of account has been an act
of divorce with the above-mentioned and accompanying powers of taxation
and money creation:

\begin{quote}
Following Knapp, Innes, Minsky, Mosler, Goodhart, and what we learned from historical experience, we summarized our view as: \ac{mmt} insists that the usual case has been that each nation state chooses its own money of account, issues currency denominated in that money of account, and imposes obligations (such as taxes) payable in the currency.~\citep[pp.~41-42]{wray2012},~\citep[p.~6]{wray_2020}
\end{quote}

Going forward, the term currency is going to be used quite
specifically referencing to paper banknotes, which are normally
liabilities of the central bank and denominated in the national money
unit of account of the country under consideration. This approach
follows \citep[p.~32]{tymoigne_textbook}.

In the today's institutional setup of the monetary system, currency, as
paper banknotes, represents just a fraction of the outstanding stock of
financial liabilities denominated in the same national money of account
as the currency itself. These financial liabilities belong to the
balances sheets, in other words they were issued, by public and private
entities. They might belong, too, to balance sheets of resident and
nonresident entities, which are respectively called too as ``onshore" and
``offshore," with respect to the country's jurisdiction, whose money of
account is under consideration.

As it will be shown in the next section on the money hierarchy, the rule
of ``one nation, one money of account" does not imply that all
entities within an economy maintain balance sheet purity by sticking to
the national or domestic money of account. Even societies governed
by highly totalitarian regimes today, and in the past too, are unable to
make it happen -- it has been just unpractical.\footnote{In the Soviet Union, the balance sheets purity has been heavily guarded by the legal and law-enforcement systems. First, households were not allowed to neither transact with foreign currencies not possess them. If caught violating, a person would end up in jail with a lengthy sentence. Second, the state monopoly on foreign trade was organized via a complex web of institutions, including the state bank monopoly. In the Soviet Union, State Bank was the all-encompassing banking institution with domestic affiliate dedicated to foreign trade, Vneshekonombank, and a small network of foreign-based banks in the ownership of the Soviet government. It was Vneshekonombank, which handled payment clearing in the foreign money units of account and its balance sheet explicitly was populated by the assets and liabilities denominated in the foreign units as money of account. See \citep{valchyshen_ru98}.}

Instead, \ac{mmt} recognizes the institutional set-up of the capitalist,
monetary-production economies \citep[pp.~10-17]{tymoigne2014}. By large, they are
populated with profit-seeking businesses, for which ``the only thing that
matters is that \textit{money} can be generated" (ibid, p.10, emphasis added).
The institutional set-up is such that businesses are facing risks and
uncertain world, where ``future is unknown and cannot be known" (ibid,
p.13). Instead of cold calculations, the business ventures follow
economics conventions, because of doing so ``reduces the darkness of the
future" (ibid). The latter arises from the recognition that ``a
monetary-production economy is inherently unstable" and uncertain (ibid,
p.11). Speculation is normal response to such an environment.
Eventually, ``the aim of any economic activity is to make net *monetary*
gain in order to stay liquid and solvent" (ibid, p.15, emphasis
original). Such an institutional set-up implies that business ventures
are absolutely free to enter into contractual agreements with their
customers and supplies of resident and nonresident types, which
stipulate commitments to pay not only by delivering funds denominated in
the national money of account, but in foreign ones, too (see table
above).\footnote{\cite{intl_acc} defined ``reporting currency" and ``operational currency".
    The former refers to the money of account a business chooses to
    denominate its financial statements. The latter is ``primary currency
    in which an entity does business and generates and spends cash. It
    is usually the currency of the country where the entity is located
    and the currency in which the books of record are maintained" \citep[p.~171]{intl_acc}. This paper uses term currency to mean paper notes, which are
    primarily the central bank liabilities and assets of the entities
    who find them useful. Instead, reporting and functional money units
    of account are the terms being used throughout in this paper for the
    sake of consistency. Then, following the logic of terminology
    defined in \citep{intl_acc}, there are reporting money of account and
    operational money of account that a business uses. These might be
    the same unit. At the same time, they might differ from the national
    money of account. In today's economy, a business entity is free
    to decide what is a reporting and operational money units of
    account. For simplicity, it is assumed that an economic entity uses
    national money of account as its reporting and operational
    money of account.}

At this point, \ac{mmt} succinctly recognizes the great divergence. Broadly,
there are two types of the government balance sheet and its commitments
to pay: (1) monetarily sovereign, and (2) monetarily non-sovereign.
These two cases represent opposite boundaries of the spectrum of
monetary sovereignty \cite{fadhel}. Detailed descriptions are in the
respective section below.

With respect of the rule ``one nation, one money of account," a
monetarily sovereign government does not behave as above-mentioned
business ventures, which must stay liquid and solvent. Instead:

\begin{quote}
[A] sovereign government finances are nothing like those of households
and firms. [...] Of cause, households and firms can and do become
[illiquid and] insolvent when they issue too much debt. But a
sovereign, currency-\textit{issuing} government is nothing like a
currency-using household or firm. The sovereign government cannot
become insolvent in its own currency; it can always make all payments
as they come due in its own currency.\footnote{As it was pointed out above, in \cite{wray2012,wray_2020,wray2022_2} the term ``currency" is used broadly to mean a range of
    financial instruments, which are liabilities of some entities and
    denominated in the national money of account. For the sake of
    consistency, in this paper ``currency" refers to paper banknotes,
    which are normally liabilities of the central bank. This approach is
    consistent with \citep[p.~32]{tymoigne_textbook}. In \citep{wray2012,wray_2020,wray2022_2} there are sections with more detailed descriptions, which put aside general terms and use more specific ones.}~\citep[p.~2, emphasis original]{wray2022_2}
\end{quote}

The Minsky's categories of speculative and Ponzi finance positions does
not apply to a monetarily sovereign government. It must be rather
treated as enjoying Minsky's hedge finance position, meaning that it
``always can meet payments denominated in its currency as they coming
due" \textit[p.~24]{tymoigne2014}.

While a monetarily non-sovereign government must stay liquid and solvent
as owes funds denominated not in own money of account but in
foreign one. Hence, it must borrow to pay upon the previously incurred
debt plus interest. This exercise is called refinancing and that is why
such Hence, such a government acts like a business venture. And it falls
into the Minsky's finance positions ranging from speculative to Ponzi
(ibid).

\ac{mmt} has been explaining the government's own decision to abandon own
money units of account for the sake of joining a monetary union, such as
European Monetary Union (EMU) as a step backward at least and one that
poses risks of sovereign defaults. The following statement of the same
sort, but has succinct ending very much relevant for the main topic of
this paper. It is about the usually overlooked economic concept of
monetary sovereignty, which has specific and practical meaning (see
quote below by Wynn Godley, one of giant thinkers on whose shoulders \ac{mmt}
stands). It is a founding block of the political sovereignty.

\begin{quote}
[T]he power to issue its own money, to make drafts on its own central bank, is the main thing which defines national independence. If a country gives up or loses this power, it acquires the status of a local authority or colony.~\citep[p.~6]{wray_2020}
\end{quote}


\subsubsection{Monetary sovereignty}

As it was pointed out above, there is a spectrum of monetary
sovereignty. The boundaries of this spectrum, on one side, define the
monetary sovereign economy and, on the other side, the non-monetary one.
Inside the spectrum, there is a variety of positions characterizing a
certain economy with its own characteristics.

In the former case---of the \textit{monetarily sovereign} system---the
consolidated government\footnote{By \ac{mmt} convention, consolidated government sector means the central/federal government and the central bank balance sheets being consolidated.} of the nation keeps its liabilities (i)
denominated in the national money of account and (ii)
non-convertible. The first point means the government does not borrow
foreign funds (liabilities of foreign financial institutions denominated
in the foreign money units of account) neither for the sake of
purchasing foreign goods and services nor for the sake of purchasing
domestic good and services. Instead, it spends by expanding its balance
sheet and crediting the bank accounts of the sellers with funds
denominated in the national money of account. The second point
means that central bank liabilities\footnote{Which is part of the consolidated government.} such as reserves and paper
currency are non-convertible at some pre-determined or pre-committed
exchange rate or at the boundaries of the exchange rate band\footnote{This is called sometimes exchange rate corridor.} into liabilities denominated in foreign money units (such as the \acf{usd}) of account or into precious metals (such as gold). Meanwhile, this type
of conversion is possible at the market exchange rate, at which the
private sector entities---those outside of the perimeter of the
consolidated government---are converting private financial liabilities
denominated in the national money of account into the private
financial liabilities denominated in foreign money units of account.

In the latter case---of the *monetarily non-sovereign* system---the
consolidated government keeps its liabilities (i) in full (100\%) share
in the denomination of foreign money units of account, and (ii)
convertible. The first point means that government issues financial
instruments (creates new liabilities on its balance sheet) that are
denominated in the foreign money units of account and stands ready to
face contingency liabilities in foreign money units of account. The
second point means one-to-one convertibility of the domestic resident
\acp{iou} into foreign non-resident financial \acp{iou} and foreign currency of
the country, the money of account is being used for domestic
denomination of financial assets and liabilities.

In between these two cases, there are positions of the national monetary
system, where the government has national money of account, imposes
tax liabilities and enforces their redemption by domestic resident
economic entities and private individuals, but foreign money
denominations do exists too. The latter might stem from the domestic
private sector operations, which are vital for the society as a whole or
due to some vested interests. The reasons that push the government into
this path include the following. First, lack of domestic produce of
vital commodities such as food, energy, medicines, infrastructure
capital goods and so on. Second, high sensitivity of the domestic
consumer goods prices on the exchange rate between domestic money unit
of account and the foreign one (such as the US dollar, the euro, etc).
Hence, the government rather borrows foreign funds denominated in the
foreign money units of account and rationalize it as a ticket to spend
funds denominated in the national money of account. Development of
the domestic government and corporate bond markets is an accompanying
alternative to the foreign borrowing by the government and the private
sector. The domestic bond market is aided by the interest rate policy of
the central bank that rationalizes the rate target and the yield curve
high enough to attract foreign investors. The second point means that
there explicit or implicit policy to support convertibility of the
government liabilities (reserves and paper currency) as well as the
private sector liabilities both denominated in the national money unit
of account into domestic or foreign liabilities\footnote{Respectively, liabilities of the balance sheets of resident and nonresident entities.} denominated in
foreign money units of account. Conversion commitment might take place
via explicit or implicit set-ups. Exchange rate policy of pegging the
domestic money unit value versus the foreign money unit is a prime
candidate of explicit way of the convertibility commitment. Defending of
the exchange rate within a band, or even within a crawling band, is a
type of the explicit commitment to convertibility. Implicitly
convertibility commitment arrives with quasi currency board
arrangements, where government deficits rationalized via either foreign
borrowings or domestic borrowing with returns attractive to foreign
buyers.\footnote{Such as Warren Mosler, who coined ``raw meat" term out of own investing experience. It means several percentage points in yield differential in the carry trade constructed by the fund managers within the available possibilities of investing into domestic government bond markets. See \citep[p.~4]{mosler}.}

\subsection{The Hierarchy of Money}

\subsubsection{The Hierarchy of Money as a Network of \acp{dcr}}

There is key conceptualization within \ac{mmt}, which is of the hierarchy of
money \citep{bell}. Sometime it is referred to as multi-tiered pyramid of
liabilities \citep[p.~148]{wray2020}.

In the most simplest presentation, the hierarchy has four tiers.
Sovereign government liabilities are at the apex (top tier) of the
hierarchy. They are acceptable because they redeem tax liabilities.
Commercial banks' liabilities occupy the next (second) tier of the
hierarchy. They are convertible one-to-one into currency on demand and
government stands ready to backstop banks (a) via the central bank's
operations and (b) via the government's deposit insurance. The next two
tiers, the third and forth, are respectively consisting of non-bank
business liabilities and individuals' liabilities. The former tier has
some acceptability thanks to the long-standing development of the market
for tradable business liabilities. The latter tier enjoy much lower
acceptability.

This exposition of the hierarchy assumes that liabilities are
denominated in the national money of account. Since, sovereign
government accepts its liabilities for redemption of the tax liabilities
it imposes regularly, the businesses and individuals denominated their
own liabilities in the national money of account too \citep{bell}. In
some economies, acceptability of the foreign currencies of a more
economically important nation might be visible (such as the US dollar
and euro). And, hence, bank liabilities might be denominated in the
foreign money of account of that nation alongside with bank
liabilities in national money of account \citep[p.~148]{wray2020}. If
country's laws tolerate domestic usage of the foreign money units of
account and traditionally foreign trade has been taking place by using
foreign funds, then by default resident commercial banks have
liabilities denominated both in national and foreign money units of
account.

The above-mentioned exposition of the money hierarchy grades financial
liabilities from the point of acceptability. In other words, there are
entities whose liabilities are accepted and held as assets by other
entities. ``All of these entities are tied together through a \textit{network}
of \acp{iou}" \citep[p.~10, emphasis added]{wray2022_2}.

Such an exposition of the hierarchy of money might be enhanced by
showing specific balance sheet relationships between key groups of
economic entities. Correspondent banking links are best suited for the
task. They show payment routes for the funds denominated in both
national and foreign money units of account.

As far as national money of account is concerned, the corresponding
banking principle mean that commerical banks maintain correspondent
banking relations with the central bank. These are accounts, where
balances are called reserves and these are central bank \acp{iou} denominated
in the national money of account. Hence, the central bank provides
its balance sheet (liabilities) for the commercial banks' operations. In
their turn, commercial banks provide these balance sheets (liabilities)
to the businesses, non-profit institutions and private individuals. All
of them use their checking or current accounts opened to them by
commercial banks. Hence, reserve accounts opened by the central bank
into the name of the commercial banks are representing correspondent
relationship, similarly checking or current accounts opened by
commercial banks into the name of their clients are representing
correspondent relationship, too. The key point here is what
correspondent relationships the government has. Normally, it has an
account with the central bank and might have accounts with commercial
banks too. But for the simplicity here the case of the governemnt
account with the central bank is assumed. Government expenditures and
revenues are accounted for through this account. The hierarchical
structure of these correspondent relationships reveal that central
balance sheet is \textit{the only} place to accumulate reserves, which are
central bank credits. Commercial banks' balance sheets are a range of
pro-profit businesses that provide own credits or \acp{iou} to the entire
population of the country's non-bank businesses and general public. See
the bottom section of the Figure 1 in the Appendix.

As far as foreign money units of account are concerned, the
correspondent banking principle is different in one crucial aspect. It's
crucial because in the modern day policy-making environment the policy
of financial liberalization mean that national central bank does not
maintain correspondent accounts for national commercial banks that are
denominated in the foreign money units of account. Today, this is
totally prerogative of the pro-profit commercial banks, which themselves
decide with which business entity they are able to maintain
correspondent account denotative in the foreign money units of account.
Normally, a national commercial bank opens a correspondent account with
a foreign bank, which does business on settling payments for foreign
entities. Such a bank might a resident of the country, whose money unit
of account is concerned. For example, a US bank is a correspondent bank
for foreign banks, which provide own customers with payments in the US
dollar. But it is not the universal rule -- cost wise, a national
commercial bank might open a correspondent account for the US dollar
payments not with the US resident bank, but with a commercial bank
residing in a third country. The latter might be not necessarily a
financial center territory.\footnote{For example, a Kyrgyzstan-based commercial bank Demirbank has a correspondent banking relationship with a Kazakhstan-based commercial bank Eurasian Bank for the settlement of payments in the US dollar. In its own turn, Kazakh-based bank has a correspondent banking relationships for the US dollar payments with US-and Austria-based commercial banks. See web pages of Demirbank and Eurasian Bank, respectively, \url{https://www.demirbank.kg/en/about/correspondent-banks/correspondent-banks} and \url{https://eubank.kz/about/correspondent-accounts/?lang=en}.} These relationships are subject to
change as costs considerations force national commercial bank to change
counter-parties in the correspondent banking business. Domestically,
depending on the level of financial liberalization, businesses and
general public might have either access only to resident bank accounts
denominated in the foreign money units of account or they might have
legal rights to have non-resident bank accounts denominated in the
foreign money units of account. For simplicity, the former case is
assumed. See top and bottom parts of Figure 1\textit{For simplicity, currency relationship as a central bank liability and an asset for the holder is assumed away in this exposition}. and Figure 2 in Appendix.

With above-mentioned general structure of the money hierarchy as a
network of correspondent relationships both in the national and foreign
money units of account, let us consider the balance sheets operations in
a more detail. First, we extend the schema of a typical balance sheet
(see Table 1 above) into the one, where correspondent relationship is
explicitly articulated. See Table 2 below. It shows that an economic
entity for each money of account must have correspondent
relationships. For general public this type of relationship is via
commercial banks. For commercial banks --- via central bank or other
commercial banks. For the national central bank the key correspondent
relationship is for settling payments in foreign money units of account,
while for the payments in national money unit it provides own \acp{iou}. For
the national central government this relationship is via the national
central bank, which settles payments for the government in both national
and foreign money units of account.

Such a visualization of the hierarchy of money differs from one by \citep{bell,wray2012,wray2020} in that now a tier represents a collection
of balance sheets that have correspondent relationships with a tier one
notch above the current tier.

Thus, for liabilities denominated in the national money of account,
the apex tier is the central bank balance sheet, while the central
government and commercial banks occupy the next tier as they all have
accounts on the balance sheet of the central bank. Non-bank businesses
and private individuals occupy next tier as they have accounts with
commercial banks.

These tiers (in national money units of account) coincide with tiers for
liabilities denominated in foreign money of account. For the letter
the apex tier is the central bank's liabilities of the foreign country,
in which money of account the domestic economic entities happened
to denominate own liabilities. In the most simplest correspondent
relationship with foreign counter-parties, resident commercial banks of
the country have accounts with foreign commercial banks of the country,
the money of account of which is used domestically. Hence, national
commercial banks share the same tier with foreign firms and general
public as they have accounts with banks of that foreign country. See
Figure 1.

The balance sheet structure as shown in Table 2 shows that assets and
liabilities denominated in the foreign money of account are subject
to *revaluation effect* every time the exchange rate between national
money of account versus the foreign one takes place. Normally, this
is the daily (mark-to-market) procedure as exchange rates fluctuate on a
high frequency fashion.

When an entity's balance sheet has assets and liabilities denominated in
the foreign money of account and they are equal, then exchange rate
movements does not have immediate impact on the entity's net worth. When
such assets are bigger than liabilities than exchange rate
devaluation\footnote{With respect of the national money of account, devaluation
    means an \textit{increase} in the quantity of funds denominated in the
    national money unit that are able to buy one foreign money unit. For
    example, an exchange rate of the \acf{mxn} versus the \acf{usd} moving from 16 to 20 is devaluation and now 20 \acfp{mxn} buy 1 \acf{usd}, while previously it took 16 \acfp{mxn} to
    buy 1 \acf{usd}.} produces immediate and positive revaluation effect on
the balance sheet by increasing net worth. Conversely, exchange rate
appreciation\footnote{With respect of the national money of account, appreciation
    means an \textit{decrease} in the quantity of funds denominated in the
    national money unit that are able to buy one foreign money unit. For
    example, an exchange rate of the \acf{mxn} versus the \acf{usd} moving from 20 to 16 is appreciation. Now, 16 \acfp{mxn} buy 1 \acf{usd}, while previously it took 20 \acfp{mxn} to
    buy 1 \acf{usd}.} yields negative revaluation by decreasing net worth.
In other situation, when entity's assets denominated in foreign money
unit of account than liabilities in this foreign money unit, then
exchange rate devaluation or appreciation respectively decrease or
increase the net worth item of the entity's balance sheet.

Irrespective of the money units of account used for denomination of the
assets and liabilities of the economic entities mentioned in the money
hierarchy as a network of balance sheets, there is the following
observations that are worth to mention. They arise from the balance
sheets' relationships when payments are made.

For example, consider, a standard situation when central government
spends. Let us assume, it has accumulated enough balances on its account
with the central bank. Also, let us assume that government spends the
funds via one transaction\footnote{Sort of the COVID-19 stimulus payments. These means that
    government does buy anything with these funds. They decrease net
    worth of the government and increase net worth of the recipient.} and the recipient of the government
speding is Person 1 (it shown in Figure 1). The payment that government
makes in favor of Person 1 has an instruction to debit the government
account and credit the recipient's account. But the accounts of the
parties to this transactions are seating not only in the different
balance sheets, but in the different tiers of hierarchy. This nuance is
mitigated for the transacting parties by the mere fact that payment
requisites include all needed details about banks serving the parties.
Still, that nuance is quite important for us here. In more detail, since
government has an account with the central bank it gives an instruction
to the central bank to debit its account and credit the account of the
recipient (Person 1), being served by a bank (Bank 1, see Figure 1). The
central bank on its balance sheet has accounts of the government and the
above-mentioned bank, not of the recipient directly. Hence, on the
balance sheet of the central bank an accounting entry is recorded that
debits account of the government and credits the account of the bank
(Bank 1). Notice that balance sheet of the central bank did not contract
nor expanded, it stayed the same in terms of total assets as it was
before the transaction. Also, notice that balance sheet of the
government decreazed by the size of the payment (as it records an
accounting entry of debiting the net work item of the balance sheet and
crediting the central bank's account). Next, the central bank informs
the bank that serves the recipient that there a credit of certain size
in favor of the recipient (Person 1). Hence, the bank's balance sheet
records an accounting entry that debits the bank's reserves account with
central bank and credits the checking account of the recipient (Person
1). In its turn, the bank informs the recipient of the credits in her
favor. And the balance sheet of the reception (Person 1) must record an
accounting entry for the size of the payment: debit of the bank account
and credit of the net worth. At this point the payment transaction is
complete. Notice, that both balance sheets of the bank and of the
recipient were expanded by the size of the payment simultaneously.

To conclude, the payment transaction resulted in the following
modifications of the balance sheets along the money hierarchy in the
national money of account. At the apex tier, the central bank's
balance sheet did not changed in terms of the total assets and
liabilities, only the liabilities side was modified --- the balance at
the account of the government was written down (debited), while the
balance at the account of the recipient's bank was written up
(credited). At the next tier, the balance sheet of the government
contracted while the balance sheet of the recipient's bank increased.
Lastly, at the bottom tier, the balance sheet of the recipient has
increased too.

Another example with operations of the money hierarchy in the national
money of account is a payment transaction that redeems tax
liabilities of Person 2 to the government. Before the payment
transaction takes place, the balance sheets of the taxpayer and the
government record respectively tax liability and tax claim. The
taxpayer's balance sheet records the entry debiting net worth and
crediting tax liability (both are items in liabilities side of the
balance sheet). The government's balance sheet records the entry of
debiting tax claim and crediting net worth. These two mirroring items
(tax liability and tax claim) are going to be cancelled out once the
payment is completed. Now, before the payment the taxpayer (Person 2)
must accumulate enough balance on its bank account (at Bank 2). Let us
assume that taxpayer earns wage income at non-bank business (Firm 2),
which borrows from its serving bank (Bank 2) to pay its own wage
liabilities to own employees. This a bit lengthy explanation of the
origins of the taxpayer's bank balances is provided to show that are
created by the bank by lending. The balance sheet of the bank expands
when it records a loan to the client and simultaneously it records a
deposit to the account of this client. These are pre-conditions of the
tax redemption payment to begin. So, with enough balance on its bank
account to redeem its tax liability Person 2 instructs the bank (Bank 2)
to debit her bank account and credit the government's account. Since,
the latter seats on the balance sheet of the central bank, the payment
must take place through the balance sheets of all counter-parties that
are connected through correspondent banking relationship. The commercial
bank, in its turn, to settle this payment through the central bank must
obtain enough balance of reserves on its account with the central bank.
The most illustrative prior assumption is that commercial bank lacks
enough reserves at the very beginning, hence, it borrows from the
central bank and only after it makes the payment on behalf of its
taxpaying client (Person 2). Hence, bank's balance sheet records two
entries of the money size equal to tax redemption: (1) debit the
reserves account and credit of the loan due to the central bank, and (2)
debit the taxpayer's account and credit the reserves account. The
central bank, too, records two entries of the same money size: (1) debit
the loan due from Bank 2 and credit the reserves account for Bank 2, and
(2) debit reserves account for Bank 2 and credit the government's
account. Finally, government's balance sheet records the entry of debit
the central bank account and credit the tax claim account, while the
taxpayer's balance sheet records the entry of debit the tax liability
account and credit the bank account.

Note how balance sheet expanded first at each tier of the money
hierarchy to initiate the payment transaction. And then, as the
transaction was completed, the balance sheets of the taxpayer and its
employer contracted. Balance sheets of the entities in the two tiers
above the bottom one expanded.

The above two cases of payments in national money---one is government
expenditure and another is tax redemption by a private person---reveal
asymmetric re-configurations of the balance sheets that are
hierarchically linked with each other. When the payer occupies an
*upper* tier within the hierarchy relative to the payee's tier, the
upper tier's entities as a whole *do not* experience a contraction of
the aggregate balance sheet as a result of payment transaction. It is
different, when the payer occupies the *lower* tier with respect to the
payee: in this case, the aggregate balance sheet of entities of this
(lower) tier does experience a contraction of the aggregate balance
sheet as a result of the payment transaction. When the payer and the
payee occupy the *same* tier of the money hierarchy than no change in
the aggregate balance sheet takes place.

The same pattern of balance sheets re-configurations takes place for the
transactions in funds denominated in the foreign money of account.
Consider an example of a US importer, a non-bank business entity, buys
supplies from the Mexican producer with a payment in the US dollar made
via a US resident bank. Then, the Mexican company normally has a US
dollar account with its commercial bank, resident of Mexico. The latter
has a corresponding account opened with a US bank, which is not
necessarily the same bank that serves the US importer. So, when the US
importer instructs its serving bank to debit its account and credit the
account of the Mexican supplier, this payment transaction takes place
via the key nexus of the US bank-correspondent to the Mexican bank.
Since, the US bank serving the US importer occupy the same tier in the
money hierarchy as the US bank-correspondent to the Mexican bank, the
payment transaction within the US jurisdiction evolves the Federal
Reserve balance sheet as well and hence the three bank balance sheets
record each the following entries:

(1) US bank of the payee: debit the checking account of the US importer
    and credit the reserves account at the Federal Reserve;

(2) The Federal Reserve: debit the reserves account of the bank serving
    the US importer and credit the reserves account of the US
    bank-correspondent to the Mexican bank serving the Mexican supplier;

(3) The US bank correspondent to Mexican bank: debit the reserves
    account and credit the account of the Mexican bank.

At the same time, within the Mexican jurisdiction the bank serving the
Mexican supplier records the entry of debiting the US-dollar
correspondent account with US bank and credit the US-dollar account of
the supplier. And finally, the non-bank parties of this payment
transaction---the US importer and the Mexican supplier--record in their
balance sheets the following entries:

(1) the US importer: debit inventories and credit the US bank account
    and credit;

(2) the Mexican supplier: debit the US dollar bank account and credit
    inventories.

The key takeaway of this exposition is that at the tier of the money
hierarchy, where the US commercial banks' balance sheets are positioned,
there is no contraction of the balance sheet in aggregate. Once the US
importer transferred US dollar funds to the Mexican supplier, what this
funds transfer actually mean is just a change of one US bank \acp{iou} for
another US bank \acp{iou}.\footnote{This is an illustrative example that confirms 
the observation in
    \citep{newman} made over the private sector investments into the US
    Treasury secueities: ``\dots the money involved changes hands but cannot
    leave the U.S. financial system." When this statement is understood
    through the analysis of the money hierarchy as a network of balance
    sheets with correspondent relationships, it become clear that this
    statement applies \textit{universally}. In other words, it is not unique
    characterization of the US financial system. If a Mexican importer
    finds a US supplier which is good to be paid in Mexican peso -- the
    same observation will be made: the money as bank \acp{iou} denominated in
    Mexican peso never leave the Mexican financial system, but
    nevertheless the US supplier of the Mexican importer will have its
    Mexican peso account.} In addition, this funds transfer means, too,
that the part of the money hierarchy that trace US dollar denominated
liabilities into non-US jurisdiction, which is Mexico, experiences the
expansion of the balance sheets. The Mexican bank and the Mexican
supplier have their balance sheets expanded for the size of the payment
volume.

However, when the same Mexican supplier decides not to sell its US
dollar balance with own bank but to use it to buy owns supplies from the
US. Then, this particular payment transaction will end up with balance
sheets contraction of both the Mexican bank and the Mexican supplier.
The balance sheets of the US commercial banks collectively do not
contract. Instead, the US bank serving as correspondent to the Mexican
bank has its balance sheet contracted, because it acts on behalf of the
payer, while the balance sheet of another US bank expands, because it
serves the payee.

All in all, the money hierarchy as a network of balance sheets with
correspondent relationships shows:

(1) there is a set of principles in play that do not depend neither on
    the dichotomy of national / foreign money of account nor the
    name of the money of account;\footnote{It suggests that concept of US dollar hegemony is rather
    secondary. A money hierarchy of whatever named money of account
    reveals these principles.}
(2) first principle, mentioned in \citep{wray2020}, is that entities of the
    lower tier are leveraging up (expanding) their balance sheets
    against the \acp{iou} issued by the entities from upper tiers;
(3) entities of different tiers interact via their balance sheets linked
    by correspondent relationships in order to make payments -- hence,
    there are *vertical* and *horizontal* patterns, which are discussed
    below. It is assumed that payer has accumulated enough balance on
    the account with a correspondent entity of the upper tier to
    initiate the payment transaction;
(4) the \textit{horizontal} pattern of payments: when two entities have their
    balance sheets located at the same tier of money hierarchy, then the
    aggregate balance sheet of the entities within this tier do not
    change while individual balance sheets do change -- the payer's one
    contract, while the payee's expands for the size of payment
    transaction;
(5) the \textit{vertical} pattern of payments arises when two transacting
    entities have their balance sheets located at different tiers of
    money hierarchy:

-   when a payer's balance sheet is at upper tier versus the payee,
    which is at lower tier, then in aggregate the balance sheet of the
    upper tier \textit{does not contract}, while the lower tier's
    aggregate balance sheet \textit{expands} thanks to the payment
    transaction;\footnote{This principle is well known, at least by policy-makers, and
    observed in the emerging market economies that care of accumulating
    official foreign-exchange reserves. It take place every time when,
    for example, the central government borrows in the foreign money
    unit of account for the declared purpose of carrying out
    expenditures in the national money of account. Normally, the
    government's foreign money account is opened with the national
    central bank. Hence, this type of borrowing by the government
    increase the official foreign-exchange reserves, which are normally
    accounted for in the balance sheet of the central bank.}

-   when a payer's balance sheet is at lower tier versus the payee,
    which is at higher tier, then in aggregate the balance sheet of the
    lower tier \textit{does contract}, while the upper tier's aggregate balance
    sheet \textit{does not expand} because of the payment transaction.

(6) to facilitate smooth payments the entities of the upper tiers must
    accommodate the entities of lower tiers in their \acp{iou}, if this does
    not taking place the correspondent relations between balance sheets
    is reconfigured.

Within the money hierarchy part with national money of account, the
key node of accommodation is the central bank's balance sheet (monetary
policy) and the central government's balance sheet (fiscal policy). With
the money hierarchy part with national money of account, the
private sector as well as government sector must have a ready access to
the balances of their correspondents, which might be provided by the
third parties (private or official lenders,\footnote{Including government concession loans and national wealth funds.} international financial institutions\footnote{Such as International Monetary Fund (IMF) and World Bank, etc.}, central banks\footnote{Swap lines such as practices by the Federal Reserve, European Central Bank, Peoples Bank of China, etc.}).

This lengthy exposition of the money hierarchy serves as an elaboration
complimentary to the previously stated claim by \ac{mmt} that finance is not
a scarce resource \citep[p.~139]{wray2020}, even in the international context.
It is rather specific social relationship albeit a very dynamic one. In
addition, it provides detailed analysis that might be a ground to
re-consider the concept of international or cross-border capital flows,
a need for which \citep{woodruff2005} suggested:\footnote{Within \ac{mmt}, the mainstream conceptualization of capital flows is questioned by Mosler, who once mused ``money does not move in that sense of motion" \citep{mosler2022}, and implicitly in \citep{newman}. Among the mainstream economic literature, an implicit recognition of slight diversion between real-life economic outcomes and the capital ``flows" was made in \citep[pp.~2-6]{hewson1975}, \citep[p.~216]{stigum2007}, \citep{feygin2020}, and \citep[p.~6]{bis2020}.}

\begin{quote}
 Discussions of international finance regularly rely on metaphors of
 motion: capital \textit{flow}, capital \textit{flight}, capital \textit{mobility}. The
 contribution of an institutional sociological analysis of money is to
 reveal these metaphors as deeply misleading. When an object moves
 through space, it passes from one set of surroundings to another---but
 these are mere surroundings, circumstances around the object, which
 retains its original integrity. Capital---investment in debt or
 equity---is not this sort of self-sufficient entity, a tossed ball
 indifferent whether it is caught or missed. Capital exists only as a
 relationship among parties, as rights and obligations, more or less
 perfectly specified in law or shared expectations. When capital
 ``moves," what happens in practice is reconfiguration of a network of
 such rights and obligations. Those who enjoy these rights or labor
 under these obligations can ascribe significance to them only in the
 context of their broader financial situation, consisting in other
 assets and other liabilities. This is another sense in which the image
 of capital as a self-sufficient object moving through space is
 misleading: the particular balance-sheet contexts in which capital is
 situated have a powerful influence on its effects. (Woodruff, 2005, p.
 36, emphasis original)
\end{quote}

\subsubsection{Value of credits}

With a money of account comes debts and price lists \citep[p.~150]{bell}.
The former, debts, are contracts for deferred payment between debtors
and creditors, while the latter, price lists, are contracts for sale and
purchase or contracts for immediate payment. These two groups of
contracts are identical from the point of view of the social relation
they represent:

\begin{quote}
 By buying we become debtors and by selling we become creditors, and
 being all both buyers and sellers we are \textit{all debtors and creditors}.~\citep{innes1913}
\end{quote}

What each debtor owes to the creditor is referred to as debt, while what
the same creditor owns on its debtor is referred to as credit. This
said, the words debt and credit refer to one, the same, thing -- the
social relationship between debtor and creditor. Usage of one these two
words just indicates from which angle this relationship is looked at
\citep{innes1913,wray2004}. There are government- and
commercial-related relationships between debtors and creditors.

It is also said that debt is an \ac{iou} or promise to pay, which is issued
by an individual or an entity and that is why it is accounted for in the
liabilities side of its balance sheet. The same \ac{iou} is credit for its
holder, who accounts it in the assets side of its balance sheet.

A debtor in order to free herself from the debt that is coming due in
the short-term future seeks credits by selling her own produce or labor
power. Value of credits, which are held by a creditor on her debtor, is
governed by the debtor's ability to set off the equal amount of credits
against the amount of debts that fall due \textit{immediately}. Hence, first
class credits are the most valuable property (ibid). These credits are
denominated in national money of account and might be denominated
in the foreign ones.\footnote{It is assumed that economies under consideration have adopted a legal system liberal approach to the usage of different money units of account alongside with a national money of account.}

Those credits denominated in the national money of account, which
are capable to redeem the debt immediately, are (i) reserves and
currency, which are liabilities of the national consolidated government,
and (ii) balances on the checking or current accounts with resident
commercial banks, which are covered by state insurance. Similar credits
denominated in the foreign money units of account include (i)
currencies, which are liabilities of the foreign consolidated
governments, and (ii) balances on the bank checking accounts of
non-resident and resident commercial banks.

For those credits denominated in the national money of account,
these is one-to-one convertibility between government and commercial
bank liabilities. ``The central bank stands ready to ensure banks make
this conversion on demand" \citep[p.~148]{wray2020}. Hence, both government
and commercials debts owed by individuals and non-bank entities are
redeemed by setting off the banks credits against these debts. Banks
redeem their government and commercial debts by offsetting reserves
against them.

For those credits denominated in the foreign money units of account,
one-to-one convertibility between resident bank \acp{iou} into foreign
currency (as cash notes) is assumed by the private sector participants,
but bears risks of default.

Integrity of money takes place, when fundamental law of debt---the
issuer of an \ac{iou} must accept it back in payment---is adhered
\citep[p.~10]{wray_2020} and \citep[pp.~147-148]{wray2020}. Moreover:

\begin{quote}
The relations of debtor and creditor are the most important artificial
 relations which subsist between human beings ---I say artificial to
 distinguish them from the relations of love and motherhood etc---and
 \textit{law of debt transcends, perhaps, in importance any other human law}.
 \citep[emphasis added]{innes1910}
\end{quote}

Additionally, integrity of money requires creation of credits that come
along not with debts (obligations to pay back in some future time), but
without them. This is done, as \ac{mmt} explains, via fiscal policy with
deficit spending that affects a wider public, whereas monetary policy
operations create credits with payback obligations and to commercial
banks mostly. Still, following Mitchell Innes,\footnote{See \citep{valchyshen_innes} for discussion of unpublished work by Mitchell Innes.} such unconstrained
power on credit or currency creation---in other words, \textit{elasticity} in
terms of creation of the first-class credits on demand---inspires
confidence and integrity into the national money system. Or what is \ac{mmt}
calls sovereign money system.

\begin{quote}
 In countries which have a Central State Bank which is practically a
 State institution, one might almost say a Government Department, like
 the Bank of England, the bank is bound to come to the help of the
 other bankers by creating credit to any extent on good security,
 irrespective of the state of its gold reserve and the threatened
 crisis may thus be averted. But here [in the U.S.] there was, until
 the Aldrich-Vreeland Act came into operation, no institution with the
 power to create unlimited credit or unlimited currency (the two are
 practically identical) inspiring the same confidence as gold.~\citep{innes1910}
\end{quote}

\subsection{Russia's War on Ukraine}

With the above-mentioned economic insights derived with the help of
modern money theory, let us now turn towards institutional details of
the evolution of the monetary system of today's Russia. In parallel, we
are going to do the same for Ukraine, too. However, even a slight
investigation into the matter enforces a researcher to do a bit of extra
work. It means one needs not only to understand the transition of these
economies after the break-up of the Soviet Union from the command or
state-planning system to the market system, it is a must to study the
evolution of the monetary system during the Soviet period.

The following sections described these periods in a small detail. The
last section concludes.

\subsubsection{Monetary system of the Soviet Union from 1920s through 1980s}

We must start with the Soviet Union period. It is not because there was
a Cold War rivalry between the Western capitalist countries and the
Soviet Union led bloc of socialist countries. This period lays
foundations of the economic developments of late 1980s under Gorbachev
and his Perestroika policy, which did not prevent Soviet Union from
breaking up. Then, it had an impact on the transition period of 1990s in
now sovereign countries, including Ukraine and Russia.

The economy of the Soviet Union before the reforms of late 1980s under
Gorbachev had quite specific institutional design.\footnote{See \citep{valchyshen_ru98} for more detailed discussion of the matter.} First, all means
of production were under ownership of the government. Secondly, finance
was reduced to be a utility like service and not the tool for
speculation, i.e. non-productive capital gains.

The design of the Soviet Union economic system that survived with some
minor modifications through mid 1980s with ruble as money unit of
account was elaborated in the 1920s and early 1930s. It was after the
lengthy period during which the Bolsheviks managed to stabilize the
economy after the 1917 revolution and subsequent several-year civil war
on the terrains of the former Russia empire. The Bolsheviks government
crushed the Ukraine's republic militarily in those years.

There was a short period in the 1920s when authorities in the Soviet
Union allow private ownership to carry out business activities. That
short episode was characterized by active commercial debts (private
\acp{iou}) creation. And authorities presiding in the State Bank understood
that process as endogenous money creation. Thus, a State Bank protocol
dated in the 1920s over the economic developments of the time
acknowledge that when bank accepted commercial \acp{iou} (called \textit{veksel}) it
meant ``exchanging commercial money by banking money"
\citep[pp.~40-44]{cbr2008}. However, the central authorities were dissatisfied
with lowered level of control over the economic processes and by
elevated price inflation. Hence, there was a crucial reform done in the
1930s that banned issuance of \acp{iou} outside of the State Bank system. In
effect, by strictly enforcing debtor-creditor relationship via the State
Bank only, the authorities imposed tight ``ruble control" over the real
resource procurement and re-allocation.

The financial system was structured into two segments: domestic and
international. Its domestic segment consisted of two banks: Gosbank (or
interchangeably, State Bank) and Vneshekonombank. The former handled on
its balance sheet all the transactions for government agencies as well
as for the government-controlled industries that were directed via their
respective ministry. The latter handled operations in the domestic units
with outside world. The domestic part of the system was centrally
directed and not strictly a profit-seeking institution. It was concerned
rather with development. The international segment consisted of the
commercial banks owned by Soviet Union government as purely
profit-seeking institutions, which used to, and still do, hire
foreigners for their operations as a practice in those markets \citep{krotov2007,krotov2011}.

Domestically, it was the government via respective ministries overseeing
certain sectors of the economy and the state-planning department (called
Gosplan) that determined assortment, quantities, and prices of the
output. These were yearly assignments (plans). Similarly, on this
temporal basis Gosplan also allocated credit limits per production
union, which was a sector wide agglomeration of enterprises located over
certain area that were connected technologically into a production
chain. State Bank via its branches was accommodating enterprises with
lending within those limits. There was a practice of exchanging credit
limits between production unions, if there was a union in shortage of
the credit limit and there was a union with excess of credit limit.

The State Bank was organize as a single institution with a network of
branches, each of which maintained own balance sheet. A branch
maintained a balance sheet relationship with its customers. The payments
on behalf of the customer of State Bank were organized via the MFO
system, where the MFO abbreviation stands in Russian for *mezhfilialnyi
oborot* or inter-branch turnover in English. It facilitated smooth
functioning of payments between counter-parties in different parts of
the country spanning several time zones. Each branch of State Bank had
an MFO account on its balance sheet. If an enterprise, a client of the
State Bank branch, had not enough balance on its account, the branch
accommodated it with short-term loan to make the payment, charging a low
single-digit interest rate from the customer. If the customer delays the
repayment of the loan than interest rate charge increased. The payment
entry was accounted for on the balance sheet of the branch via the MFO
account (debit the current account of the bank's customer and credit the
MFO account). The MFO account was not an asset only account, it was
rather an asset-and-liability account. Its balance could appear on
either side of the balance sheet. Once a year, State Bank carried out a
procedure of netting out the accumulated balance on the MFO accounts of
its branches.

In addition to the MFO payment system, which was a gross payment system,
there was additional practice of net clearing. It was understood that:
(a) if there was a group of enterprises that operate regularly as
suppliers and customers to each other; and (b) if their payments
"collide with each other [in time]" or, in other words, their payments
to each other were taking place nearly simultaneously, then the net
clearing of these mutual payments would be beneficial to the parties
involved. The benefits of net clearing were: (1) improving the
timeliness of payment settlements; (2) economizing on the costs of
borrowed funds by the enterprises; and (3) lowering the labor input in
the paperwork both by the enterprises and State Bank officers.
Eventually, a number of such entities were established and their only
function was net clearing or matching of the mutual obligations between
enterprises and then bringing up only the net volume for the final
settlement/payment via the MFO system. These entities were called
Bureaus of Mutual Settlement. They enhanced operations of the industrial
enterprises as payments speed up, according to the State Bank documents
of the time. At that time, in the 1930s, the Soviet Union's economy
experienced rapid industrialization, while ``[f]ollowing the Second World
War the Soviet Union made rapid strides in economic development and
attained levels of industrial production second only to the United
States" \citep[p.~612]{dillard2}.

Alongside with industrial development, the Soviet Union authorities
carried out policy of agriculture produce expropriation from the
Ukrainian rural population, which has been traditionally focused on
grain harvesting. Grain consolidation helped to sell it into the foreign
buyers and obtaining foreign bank credits, which were needed to buy
capital goods required for industrialization. In Ukraine, that period of
mass and artificial starvation in the 1930s is called Holodomor, i.e.
enforced deadly hunger.

The Soviet ruble was an inconvertible money of account. As a deputy
head of State Bank put it: there was ``Chinese wall" between ruble and
foreign currencies back then~\citep[p.~19]{krotov2008b}. So, in technical
terms, the ruble was inconvertible, neither into a foreign-currency unit
of account nor into precious metals such as gold (this was despite the
fact that cash notes of ruble of different denomination contained small
print saying ``bank tickets are backed by gold, precious metals and other
assets of State Bank." ). A domestic foreign-exchange market was absent,
and holding of foreign currencies by private individuals was illegal.
Foreign-exchange transactions were under strict control and limited in
terms of: (a) transactions and operations related to commercial
cross-border flow of goods and services; and (b) the transactions being
handled by a designated set of banking units, the above-mentioned
Vneshekonombank and IBEC (a Moscow-based International Bank for Economic
Cooperation). Entrepreneurship as well as financial speculation by
private individuals was also illegal (Bashkirova, Soloviev, \& Dorofeev,
2012, p. 48). In general, as \citeauthor{dillard2} put it, Soviet Union was ``a non-business economy"~\citep[p.~633]{dillard2}.

Mechanics of government finance were the following. A former Deputy Head
of State Bank explained: ``[T]he \ac{mfo} system allowed budget spending on
the local level to be carried out independently of budget revenues.
Balancing of revenues and expenditures was taking place [later on] in
the Center, in State Bank, which credited Ministry of Finance if the
balance of the latter was in shortage"~\citep[p.~16]{krotov2008b}.

After the World War II, the Soviet Union established a bloc of socialist
countries of the Eastern Europe and later included Vietnam, Cuba.
Together these were organized under the umbrella of the Council for
Mutual Economic Assistance (CMEA), which used to be referred to by the
abbreviated name of Comecon. At the beginning the trade within Comecon,
i.e. between different money systems each of which has own money unit of
account, was carried out via Western banks and relying on the funds
denominated in foreign money units of account (not of the countries
members of Comecon). Later on, in 1963 Soviet Union established the
above-mentioned IBEC that took over over the trade flows between Comecon
member states by establishing new money of account *transferable
ruble*. The latter was not convertible into the Soviet ruble, the money
unit of account used in the USSR, nor into the units of accounts of the
Western world. Again, the member states of Comecon retained their
national money units of account for domestic transactions, while dealing
in Western money units for transactions outside the CMEA. See \citep{vincze,kalinski}.

During the 1970s both the Soviet Union and the members of the Comecon
had accumulated foreign-currency debt on the back of: (a) past poor
harvests; (b) support of consumption by domestic consumers via importing
goods; and (c) domestic fixed-investments that required foreign
technology. For more on these developments, for instance, see \citep{coombs1976,gaidar2002,harold}.

Dillard explained that agriculture was one of ``the weakest links of the
Soviet economy"~\citep[p.~628]{dillard2}. As a result, it forced the country
"[i]n 1963 [to enter] the world market to buy grain to feed its growing
population" (ibid). It was quite an unusual episode in the global
economy of the time. Thus, Coombs discusses Russia's significant
offerings of gold during the 1960s to the London gold market,
"reflecting wheat harvest failures in 1963 and 1965."~\citep[p.~154]{coombs1976}.
Later, however, the authorities of the Soviet Union turned to
imports-buying through Western debt instead of selling its stock of
gold. Already in the 1970s, the Soviet Union was already in its fourth
or fifth year of trying to gradually reduce political tensions with the
Western powers. Meanwhile, the Polish government, facing unpopularity,
embarked on ``a build-up of Western debt, a substantial proportion of
which was used for consumption," and eventually, ``Poland become the
world's third-largest wheat importer"~\citep[pp.~89-90]{harold}. The Soviet
Union helped Comecon members with refinancing of debt in foreign money
of account. For example, Poland was assisted in this way for several
billion US dollars \citep{stasi1981}. But already in the early 1980s, top
leadership of the Soviet Union faced situation, which later would be
called as sudden stop: ``Banks have suddenly stopped giving us loans
(USA, FRG [West Germany])." \citep{stasi1981} quoted in \citep[pp.~15-16]{zubok2021}.

The command economy of the Soviet Union of its late period was described
as \textit{hunting} for the ``hard" currency receipts via exports of commodities
such as crude oil and natural gas, see \citep[p.~222]{dibb1988} and \citep{zubok2021}. Back
in the 1980s, the hard/soft currency dichotomy implied: (a) ``hard
currency" meant the Western banks' liabilities and the central banks'
liabilities as physical (paper) currency denominated in the money units
of account of the major economies such the US dollar and Deutsche mark,
(b) ``soft currency" meant the state banks' liabilities denominated in
the money units of account of the socialist bloc of countries, as in~\citep{dibb1988}.


\subsubsection{Soviet Union break-up and transition to market in 1990s through August 1998 default}

Since late 1970s and especially early 1980s, the Soviet Union leadership
nurtured the idea of economic reform \citep{zubok2021}, while specifics of
the reform were vague and debatable. However, key strategy was to
incentivize profit-seeking behavior among the state-owned industrial
enterprises, which were deemed wasting resources while enjoying
accommodation in resources allocation provided by the state. Money as
credit was considered as part of the resources alongside with physical
resources. While foreign banks \acp{iou} (credits) were indeed scarce, as it
was described above, the Soviet Union authorities tended to think of the
domestic bank \acp{iou} (credits) in the same way. In early 1980s, few years
prior to Perestroika launch by Gorbachev, the mid-1980s leader of the
Soviet Union, then another leader of Soviet Union Andropov instructed
State Bank to provide much less accommodation to domestic enterprises
than it used to be. Same kind of incentives were
sought to apply widely throughout Soviet Union:

\begin{quote}
 ``Other ministers should come to you," [Andropov] said to the Minister
 of Finance, ``crawling on their bellies, \textit{begging for money}."~\citep[p.~18, emphasis added]{zubok2021}
\end{quote}

Later on, in early 1986 or just few years after Andropov death, newly
appointed leader of Soviet Union intriduced policy of Perestroika. It
aimed to realize absolute economic calculation among the publicly-owned
enterprises, i.e. push them into profit-seeking mode of operations
instead of obtaining state subsidies. It never intended to introduce
private property on those means of production, see \citep{gorb}. This is what
Andropov, a hard-line leader, was seeking too.

The second half of 1980s experienced an extension of the past shortages
of consumer goods, not elimination of them. As past system---with
Gosplan and State Bank---was still in place, the policy of establishing
business-focused economy had shifted to embrace the previously
unthinkable ideas. First of them was market-determined prices instead of
state-dictated ones on broad range of goods and services. Second of them
was making legal private property on the means of production.

Meanwhile, as these ideas were still digested by policy-makers and
general public, the key institutional change took place. The monetary
system was reformed first out of all other sectors. The rationale was
that business production must be supported by business credit
allocation. While previously there was state monopoly on credit
allocation to enterprises, in the second half of 1990s a law was adopted
that allowed creation of cooperatives, a form of private ownership
organization with for-profit operations. In addition, the changes into
the law allowed cooperatives to create banks. The old banking system was
deem slow and wasteful, hence, there must be modern banks to spring up.
There was no banking oversight from the State Bank as this practice was
unknown to the officials of the old monetary system. It resulted in the
development when new banking industry attracted either highly-smart and
business-capable people with PhD in physics and people with criminal
past and respective incentives \citep{valchyshen_ru98}. These new banks used
to open accounts with State Bank and its new departments -- they
effectively used State Bank balance to net clear the payments their new
customers, to whom they promised more attractive interest rates on
balances than State Bank did. In addition to that the authorities were
gradually relaxing the previously tight restrictions on foreign trade,
which inevitably spawn innovations of allowing access to the \acp{iou} in
foreign money units of account.

However, let us return to the main policy of the 1980s of incentivizing
the state-owned enterprises to become more efficient and operate in a
business, competitive environment. It became popular if not dominant
among the domestic economists of the Soviet Union and the foreign ones
that worked on the economy of the Soviet Union and advised its
policy-makers. That thinking adopted the approach developed by Hungarian
economist Janos Kornai, who argued that socialist economies operated
under *soft budget constraint* (SBC). It is when state-owned enterprises
dominating these economies obtain credit from state bank and subsidies
from the state government on a regular basis. Moreover, the price system
controlled by the state does not incentivize surpluses of goods.
Instead, shortages result.

So, the key of economists within Soviet Union were aligning with a
policy to remedy the Soviet Union economy. It was short with two
prescriptions: (1) moving to the market-based economic system as quickly
as possible, and (2) impose hard budget constraint on state-owned
enterprises, while their privatization was a distant prospect back than.

Economists of Post Keynesian tradition from the quite early criticized
the concept of hard budget constraint as one having no relevance at all
if one bases analysis on the endogenous money view. As the following
quote borrowed from the 1991 article states, both types of the economies
(capitalist or socialist) could be characterized with such monetary
conditions that are, applying SBC terminology, on the softer side of the
budget constraint continuum.

\begin{quote}
 If credit money is truly endogenous in a capitalist system, then a
 hard budget constraint does not exist in capitalist or socialist
 systems; i.e., given the cost of obtaining credit from the banking
 system, firms have a soft budget constraint in \textit{both} economies.
 \citep[p.~330, emphasis added]{szego}
\end{quote}

Indeed, high-profile proposals on the reforming of Soviet Union such as
in \citep{peck} considered the translation towards market economy through the
prism of hardening of budget constraint. For example, @nordhaus
contributing to @peck advised for the fiscal policy ``the target should
be a balanced budget"~\citep[p.~104]{nordhaus}, while for the monetary policy
his advise was:

\begin{quote}
 With respect to credit policies, a substantial tightening of credit
 will be possible and desirable when firms operate with hard-budget
 constraints. Once individual enterprises are subject to hard-budget
 constraints, Gosbank should make credit available only to firms that
 can repay credits; this implies curbing credits to unprofitable
 enterprises.~\citep[p.~108]{nordhaus}
\end{quote}

From the \ac{mmt}'s sectoral balances approach, where domestic non-government
sector is represented by two sectors of households and enterprises in
addition to the government and foreign sectors, then prospect of two
sectors changing their balances from deficit to zero (balanced) or even
surplus suggest that either surplus of the household sector should
contract or foreign sector should increase its deficit against domestic
economy. Given the fact that over 1980s Soviet Union economy was
struggling in obtaining foreign \acp{iou} (credits) from abroad, then we can
assume that foreign sector impact on the Soviet Union economy not that
decisive. Then with major three sectors of the domestic economy, where
two---government and enterprises---are set to reduce their gross
spending to achieve net spending at zero, which balanced budget or
surplus budget outcome as incomes are equal or greater than
expenditures. It means that the remaining sector of households faces
decline in gross incomes. Dynamically, such a prospect spells unstable
rather than stable economy.

Add to this the last minute financial reform of early 1991, when the
Finance Minister of Soviet Union pushed through the policy of calling in
all the 50 and 100 paper currency notes of the State Bank in circulation
and replacing them with currency of newer print. A person was allowed to
exchange these notes in the total amount no more than 10,000 rubles. The
reason of this reform was concern over growing ``overhand of currency"
amid rising prices and persistent shortage of goods. (Already three-year
private banking system operating on the margins of State Bank was left
without due attention.) This policy caused the ``most uproar"~\citep[p.~189]{zubok2021} among the general public.\footnote{Yet, \citeauthor{zubok2021} considers Valentin Pavlov, Minister of Finance and author of this reform, as ``one of the few who knew how the Soviet monetary system worked and what the real causes of its crisis were"~\citep[p.~175]{zubok2021}.}

Politically, Soviet Union was brewing with national revival of
independence and sovereignty that was previously curtailed with military
force. In Ukraine it was in 1920s, while in Baltic countries it was in
1940s. Economic hardship with long-lasting shortages of consumer goods,
which became acute and open during 1980s, fueled wide backlash against
the central authorities in Moscow. In 1991, political disintegration of
the Soviet Union was largely peaceful. Political divorce between Ukraine
and Russia back in 1991 was completely peaceful.

There is wrong money take with respect to Soviet Union collapse such as:

\begin{quote}
 The power of money was central and crucial to the behavior of Soviet
 elites during the last years of the Soviet Union. Had the Kremlin
 ruler made different choices, to tap into this power, turning the
 existing elites into stakeholders of the transition, instead of
 alienating them, even the KGB officers would supported state
 capitalism and privatization, just as they later did under Yeltsin and
 Putin.~\citep[p.~437]{zubok2021}
\end{quote}

It assumes that economic theory was roughly correct in drawing the path
for transition towards the market economy. It does not take into account
the possibility of this economic theory being a bit incorrect.

\cite{dibb2006}: ``Practically *overnight*, the Soviet Union lost its world
power status and, fragmented as Russian society was, plunged into a
nightmare of a collapsing economy, rampant inflation and the
disappearance of savings, jobs and pensions" (emphasis added). It is a
grand mistake to put all these events into one short period of time,
which is 1991, and then allow this one sentence to govern over our
understanding of FSU and Russia developments. Secondly, approximating
``Russian society" to ``the Soviet Union" is another omission usually made
by two groups: (1) Russians themselves, and (2) people who where not not
born and raised inside the Soviet Union.

In 1990-91, today's president of Russia Putin himself had experience of
structuring a cross-border transactions while he served in the city
council of Leningrad, the second largest city in Russia and the FSU.
Now, it is known as city of St Petersburg. Back then the economy of the
collapsing Soviet Union was swiftly disintegrating economically and
politically, see respectively \citep{woodruff2000} and \citep{zubok2021}. The
major urban centers of the former Soviet Union such as city of Moscow
and Leningrad have experienced shortages of consumer goods supplies. It
was due to mass hoarding of vital goods produced domestically as
state-run infrastructures of banking and payments and provisioning of
food and other products were evolving from purely state owned and
centralized to privately owned and decentralized albeit in an extreme
sense of the word. In May 1991, or few months before its ultimate
collapse, the Soviet Union government made a desperate move to sustain
key urban areas with food supplies. It assigned a number of for-profit
businesses, which were just emerging at the time in the Soviet Union, to
be middlemen between foreign food suppliers and itself. The foreign food
suppliers were accepting foreign funds in payment. Both the Soviet Union
government and those for-profit businesses were short of foreign funds.
In order to obtain them, the Soviet Union government guaranteed exports
of valuable raw materials.\footnote{These included ``oil, timber, metals, cotton, and other natural resources"~\citep[p.~121]{gessen2014}.} This way the foreign funds were raised
and available for payment of food imports. In the city of Leningrad in
1991, such a for-profit business was Kontinent, one of the many
companies created by the city council's foreign relations department
that was led by Putin. It entered into an agreement with a German food
supplier. Some biographic accounts of Putin contain mentioning this
transaction. the Russia side committed to deliver natural resources,
while the German side committed to deliver food products. The
transaction involved money of account and mutual money payments. It was
essentially a monetary transaction, even though some authors called them
as barter.\footnote{By definition ``[b]arter exchange of commodities, at a ratio
    agreed by the parties involved, *does not* involve a common money of
    account" \citep[p.~107, emphasis added]{ingham2020}. Hence, barter
    description of the above mentioned transaction is a mistaken
    approach.} It did contain money of account that both sides agreed
to use for denomination of the value of goods purchased and sold. The
Russian side desired to get access to the balances in the German bank
denominated in the money of account of Germany. For the German side it
was rather normal to agree for the use of German money of account. Once
the Russian side delivered the natural resources under its side of the
agreement, the German side credited 90 million Deutsche marks to
Vneshekonobank bank account opened with a German bank. Then the Russian
side paid for the food supplies to the German side. Records showed that
above-mentioned company Kontinent and its supervisors were flexible and
creative in terms of choosing destination for the food products imported
from Germany. There was a mini scandal in the city of Leningrad back in
1991. Amid food rationing in the home city, the Leningrad-based
Kontinent diverted supplied to the city of Moscow instead of Leningrad,
the original destination. The key point of this episode is that people,
who practically handled transactions like these from the Russian side,
understood money as a balance sheet entry as the above mentioned 90
millions Deutsche marks were considered as \textit{credits} or a balance of
funds credited to the account of Russian side at the German bank. See
\citep[pp.~104-105,118-124]{gessen2014} and \citep{gessen2017}.

Meanwhile, the whole region of former Soviet Union was experiencing
hyperinflation during the first half of the 1990s. Dollarization first
of all of personal savings was sweeping Ukraine, Russia and all other
newly independent countries. After the disintegration of the Soviet
Union, over some time the newly independent countries adopted their own
money units of account, while Russia retained ruble. Ukraine changed
money of account from ruble to karbovanets. Incomes were collapsing and
in terms of real (price adjusted) contraction of annual volume of gross
domestic product (GDP) the entire region of the former Soviet Union was
in depression by mid 1990s. It was not even a recession, it was truly
depression as cumulative loss in annual GDP was in double-digit
territory. Human conditions worsened for majority of population in every
newly independent states. Both households and private businesses
effectively adopted US dollar as their operational money unit of
account.

As Ukraine and Russia, as well as other former members of Soviet Union,
now being separate states while still largely interconnected by trade
relationships, the economic developments of contraction persisted and
even accelerated. In Russia, president Yeltsin appointed reform-minded
economist Yegor Gaidar as active head of the government. Gaidar was
widely credited by implementing the price liberalization and launching
the process of privatization of the state-owned enterprises. Russia was
a cheer leader in economic reforms for the rest of former Soviet Union
members. Other nations watched what Russia's government was implementing
and tried to emulate albeit with some lengthy delay. This development is
vividly described by the former chief economic adviser to the then
president of Ukraine, Leonid Kuchma:

\begin{quote}
 We, especially in the first years of independence [of early 1990s],
 copied in good faith all that what was done by the [Russian] 'elder
 brother.' Moreover, our teachers were the same.~\citep[p.~232]{galchynskiy}
\end{quote}

Indeed, Russia was pioneer in price liberalization, mass privatization
program, positive real interest rate policy by the central bank,
\footnote{Implemented in order to incentivize savings in the national money
    unit of account, to discourage dollarization and fight
    hyperinflation.} stock market development and then government bond market. As authorities
turned to the policy of tight credit and high interest rates (trying to
make them positive in real, inflation-adjusted terms), the private
sector was free to pursue a variety of strategies for the sake of
obtaining profits. As commercial bank \acp{iou} (credits) were costly, the
non-bank businesses proliferated in practice of issuing commercial \acp{iou}
and settling mutual obligations in a quite similar way that previously
done by Bureau of Mutual Settlements that operated under the roof of
State Bank and Gosplan from 1930s to 1980s. These new businesses settled
commercial obligation within the borders of newly independent states and
across of their borders too. The most lucrative state-owned enterprises
with their produce in high demand abroad were snapped into private hands
via privatization process. There was pressure to privatize fast. It was
coming from the theory that former state-owned enterprises must be
brought into the market and competitive environment as quickly as
possible to get the economy out of the crisis. As it was discussed
above, the hypothesis of soft budget constrain was governing that
theory. Collapsing demand and hyperinflation were adding urgency to do
the thinking that privatization is quick fix. Lastly, theory reasoned
that once hard budget constrain was imposed then it would quicken the
recovery. Former state-owned enterprises were sold via auctions that
usually attracted few bidders and selling prices of these productive
assets were low but albeit impressive for the ordinary people amid
economic depression. That was general development with the private
sector in the newly independent countries of former Soviet Union.
However, it was Russia public and private sectors---policymakers and
bigwig business people that quickly earned the name of oligarchs---that
were in the lead and impressed their counterparts in the other former
members of the Soviet Union. The center of intellectual activity in
terms of policy-making and in terms of making big money quickly was in
Moscow. For example, Ukraine's policy-makers and business people
normally observed Russia's successes in the strides, respectively, to
the market economy and concentration of wealth with a bit of jealousy.

In the second half of 1990s the \textit{major} economic development was taking
place with respect to government finance. That process was called
shifting from the inflationary financing towards market-based financing.
The former referred to the practice of obtaining direct credit from the
central bank, the operational legacy of the State Bank, which operated
one balance sheet effectively for entire economy. The latter referred to
the practice adopted in the West, where government was selling bonds
denominated in the national money of account to the private
investors, resident and non-resident ones. As additional tool against
high-inflation environment, the laws were adopted throughout former
Soviet Union that made illegal direct lending of the central bank to
government as well as direct purchase of the government bonds via the
primary market.

Russia was first to launch the government bond market. At the begging
the government was able to sell short-term securities or zero-coupon
bonds of maturity of up to one year.\footnote{These were known as GKOs, which in Russian stands for government
    short-term bonds or obligations. The entire episode of the Russia's
    monetary history since transition to the market economy in early
    1990s earned a name as the GKO market. It lasted several years from
    1994 when the Russian government started issuing GKOs denomianted in
    rubles and through August 1998, when the government chose to declare
    default on them.} As interest rates were high at
double digits the government securities were selling at the market
rates. This brought foreign interest in the market that was offering
lucrative yields. Foreign bids into the GKOs also bid up the value of
the exchange rate of ruble versus the US dollar. The central bank as its
own policy mix was supporting a policy of exchange rate corridor, which
aided to the GKO investors' calculation of the expected returns as
future value of ruble to US dollar was expected to survive around the
current level (of level of entry into the GKO positions). At the
beginning the GKO market created an environment of economic progress:
(1) the government was said financing its deficits via non-inflationary
means, (2) exchange rate stabilized at last after a multi-year
devaluation period, (3) private investors, domestic and foreign, were
harvesting high-yield returns and this encouraged more position-making
into GKOs from domestic and foreign bidders (Russia newly created
private banks did that and foreign major investment banking names did
too). Russia's commercial banks made additional business by making
positions on promising delivering US dollar in the future to their
counter-parties.

Ukraine followed the government bond market launch with some delay. But
the major details were similar in terms of short-term maturity of the
bond denominated in the national money of account and high interest
rates at which these bond were sold. It also helped to stabilize the
exchange rate for some period of time as the central bank adhered to the
same policy of high positive real interest rates and pegged exchange
rate against the US dollar.

The exterior of the new means of government financing did hide the
underlying operations of the market. As explained in \citep{valchyshen_ru98},
using Russia's case as an example, the market size at some (and not very
much distant) point reached the level when the government was not able
to redeem the bonds due at a particular date from accumulated balances
at the central bank account, even if proceeds from the primary auction
of GKOs are taken into account. To support smooth functioning of the GKO
market, the central government and the central bank coordinated their
balance sheets in the following way: under an established and
long-lasting agreement the central bank provided the government with
short-term (initally intra-day) loan enough to redeem the bonds. Then,
the government repaid the central bank loans once enough balance was
accumulated. That practices worked for a while until the fall of 1997,
when Asian crisis hit shaking international confidence in riskiness of
emerging markets. At the time, Russia's politics was messy with Yeltsin
barely managing to survive re-elections against ever popular Communist
Party. The latter was collecting votes from the dissatisfed voters,
number of which was fairly big as market economy exposed extreme social
inequality between super rich few oligarchs, who benefited from the
hurried transition from the command economy to the market one, and the
vast majority of rest of the population. Yeltsin kept pressing with
liberal market reforms as economy was struggling and key exports were
not covering imports and Russian government was openly short of foreign
\acp{iou} (credits) it needed to pay past debts and support the exchange rate
convertibility. In late 1997, foreign investors turned quite cautious to
the GKO market. Some were withdrawing from the ruble positions and not
interested in new GKO positions. The exchange rate corridor was under
risk of collapsing a policy commitment. Yeltsin was begging foreign \acp{iou}
(credits) nearly in the same fashion as Gorbachev in his tenure. In
first half of 1998, the central bank reacted to the market uncertainty
with surprise interest rate hike of massive proportion by raising it
from 28\% to 150\% via several spreadout steps. But the hike to the 150\%
level from 50\% in late May of 1998 was considered as a sign of brewing
problem. That rate survived no more than 10 days and was lowered to 60\%.
But the course of the events was indicating that exchange rate policy of
corridor was about to break down. Exchange rate policy was a priority as
it was observed by the general public on a daily basis. Behind the
curtains, the central bank and the central government were at near war
because of the GKO market. It did not function as before anymore in
terms of attracting funds on the account of the government. Instead,
investors were expecting redemption and some were eyeing converting
ruble balances with resident banks into US dollar balance of
non-resident banks, i.e. withdrawing from the Russia's domestic bond
market. The internal loan from the central bank to the government was
quite big and not payable back by the government. What was previously an
inter-day loan turned into a longer term loan and the central bank top
officials worried that a balance sheet disclosure would reveal that the
law, which banned such a credit relationship, was broken. It was the
central bank to officials that pushed the government to declare default
on the ruble-denominated government bonds. On August 17, 1998, the
Russia government declared non-payment decision and invited investors
for restructuring of the debt. As a result, exchange rate policy changed
by re-pricing the ruble at much weaker rate to the US dollar. Inflation
spiked again and economic recession followed. Politically, that Russian
financial crisis of August 1998 paved the way to change in the
government, where liberal reformers were replaced by people of more
experience in security services such as prime-minister Primakov. Lastly,
president Yeltsin handed over the presidency powers to Putin, then
largely unknown security service person of younger generation. The key
economic and monetary take away of this political change is that both
Primakov and Putin preserved economic policy-making machinery brought by
Yeltsin.

In Ukraine, the bond market too experienced a swift departure of
investors and exchange rate peg was broken. Inflation spike and
recession followed. The government restructured its market debt.

\subsubsection{Evolution of the monetary system in 2000s on the eve of \ac{gfc}}

Late 1990s and early 2000s turned out to be a period of turnaround as
commodities markets for crude oil, iron ore, grains, etc. appreciated in
price. The entire region of the former Soviet Union experienced
export-led growth. Russia's economy benefited from foreign trade so much
that its government paid down its foreign debt and enjoyed trade
surpluses of massive proportions. Its leadership must have consciously
adhered to this policy as it did not want to appear again in situation,
in which both Andropov and Gorbachev founds themselves, of begging
constantly for foreign \acp{iou} (credits) to be able just to redeem own debt
was falling due or buy essential, everyday food.

In this period, Russia's policy-making remained broadly aligned with
intellectual underpinnings of liberal market economy. While few former
reformers active in government during 1990s survived into the corridors
of power in Moscow during 2000s. The key person, who spearheaded liberal
reforms along the lines initiated by Gaidar, was Alexey Kudrin. He has
been credited with pushing through reform such as land market (turning
land into tradable commodity), monetizing social benefits (it turned
highly unpopular and harmed popularity of the Kremlin leadership, mainly
Putin), budget rule (deemed the most successful policy of all), pension
age increase (again highly unpopular with general public and cost Putin
popularity).

Meanwhile, its private sector was enjoying lucrative market environment
too. Private owners of the major industrial now privatized enterprises
were welcomed in the major financial centers as respected business
people to do new initial public offerings (IPOs) of shares and US dollar
borrowings via the Eurobond market or via syndicated loan market. As it
was noted above, private sector was adopting the US dollar as its
operational money of account. Even middle- and low-income
households followed this lead as domestic purchases of personal cars and
personal housing was priced and dealt with in the US dollars, including
cash currency.

This market practice was followed nearly universally over the other
newly independent countries, where private sector operations were
thought of in the US dollar terms primarily. Russia's private and
state-run businesses were expanding into the countries of former Soviet
Union, including Ukraine. They were expanding their balance sheets
relying on the US dollar as money of account. For example,
state-run financial institution of Russia called VEB as well as
privately run commercial bank of Russia NRB acquired commercial banks in
Ukraine from the private sellers. Over time, operations of these two
commercial banks revealed that they were expanding their balance sheets
predominantly in the foreign money units of account. Primarily it was US
dollar, not Russian ruble. While they pretty neglected using Ukraine's
national money of account.

In the run-up to September 2008, when \acf{gfc} culminated with Lehman Brothers bankruptcy, both Russian and Ukrainian private sectors were expanding their US dollar balance sheets quite
liberally. The GFC produced deep devaluations and deep recessions.

Thanks to oil and natural gas exports, Russia had sizable official
foreign-exchange reserves. It spent them in 2008-09 to support its
private sector, where leading industrial companies appeared
highly-indebted to private financial institutions abroad and used their
assets as collateral for borrowings. That experience paved the way
towards critical re-consideration of the private monetary relations.
Disciplining of the domestic business to use ruble as their operational
money of account had been initiated in the aftermath of \ac{gfc}.

Meanwhile, Ukraine had never had enough official foreign-exchange
reserves. Since 2005, its economy moved from having foreign trade
surpluses to having chronically foreign trade deficits.

\subsubsection{The first phase of the war of 2014}

Just three years after \ac{gfc} Russia's economy faced two slow-burning
crises:

-   social-political crisis of low approval rating of the Kremlin
    leadership and personally Putin; and

-   economic crisis of slow growth, which was subject of internal
    discussion of Kremlin top economic advisers.\footnote{Namely, by Andrei Belousov, who now holds position of first deputy prime minister of Russian government: \url{http://government.ru/en/gov/persons/123/events/}}

The first crisis was a public reaction over the results of general
elections into the Russian parliament in December 2011. These were
deemed as fraudulent. The ruling party United Russia, the party
representing two personalities of power Putin and Medvedev, won those
elections. But the environment in the public was negative to the ruling
elite. That was the year when Putin served his last year as prime
minister, while president Medvedev was an outgoing official who was not
going to challenge Putin in the presidential elections in March 2012.
So, just three months before presidential elections Russia social life
saw massive protests against authorities. Public dissatisfaction was
fueled too but recent announcement made by Putin in the second half of
2011 that he would take part in March 2012 presidential elections,
seeking re-election. The December 2011 protests calmed down as police
arrested the protest leaders swiftly. Authorities showed showed they
were not going to tolerate public protests. In March 2012, Putin won
presidency. It is noteworthy that the December 2011 parliamentary
elections and the March 2012 presidential elections showed that
Communist Party of Russia has remained second largest political group
after the ruling party United Russia. Protests against the ruling party
took place over 2012 and 2013 but were losing in terms of visibility.
However, surveys such as by Levada, one of the most trustworthy opinion
pollsters in Russia, were recording a wider dissatisfaction over the
ruling party and Putin personally. In 2013, Levada's Putin approval
index hit the lowest point since early 2000s, which was an extraordinary
fact. Levada was forced to revise its survey and drop the index
altogether in 2013, while its components were still produced on monthly
basis and published.\footnote{The Puting approval index was calculated and reported by Levada
    on a monthly basis from two variables: (1) share of the surveyed
    people who approve Putin's policy and (2) share of surveyed people
    who disapprove Putin's policy. The difference between (1) and (2)
    resulted in net approval index.}

Since his re-election as president in March 2012 and following
inauguration in May of same year Putin declared a multi-year policy of
grand ambition that addresses Russian general public nostalgia about the
dissolved Soviet Union. His program aimed to create Customs Union with
the countries of former Soviet Union. Medvedev, a usual substitute for
Putin and who speak what Putin does not not want to say himself as of
yet, was talking about this new policy as a better version of the
European Union. It meant that Russia's government was thinking of
monetary union, too, where ruble was considered as the shared money unit
of account.

At that same year, Ukraine and a number of other countries were in talks
with European Union officials about signing an association agreement
with the EU. In Ukraine, that move was considered as yet one step to
become more closer with European institutions and away from the Russia
dominated ones.

Meanwhile, economic growth in 2012 and early 2013 was slow in Russia
alone. Moreover, during first half of 2013 real GDP was negative in real
seasonally adjusted quarter-on-quarter teams every quarter: in first and
second quarters of 2013. Technically this meant recession, while in
Russia both financial media (domestic and foreign) and professional
economists kept silence on the phenomenon. A bit later, the government's
statistical office revised the data and published new series of
quarterly national accounts, according to which there was no more two
consecutive quarters of real GDP decline \citep[p.~30]{valchyshen_2014}. Also,
by mid 2013 there were two economic factors were observed: (1) Russia
ruble was sizably dislocated up by the metric of real effective exchange
rate, suggesting lost competitiveness according to standard economic
theory \citep[pp.~9-10]{valchyshen_2013}, (2) top economic policy-makers of
Russia, first of all Belousov and then economic minister Ulyukaev,\footnote{He was lately sentenced to several-year term in jail for high-profile corruption scandal.}
were discussing internal, not external, issue that drags Russian economy
and in particular they flagged lost control over costs as the core
issue, which is an indirect reference to \citep[pp.~9-10]{valchyshen_2015}.

In this environment, Russia authorities pressured Ukraine authorities to
give up the EU association agreement and join the Russia-led project of
Customs Union. Ukraine's president Yanukovych refusal to sign EU
association agreement, what was concession to Putin, caused massive and
extremely determined public protests\footnote{Disclosure: author of this paper took part himself in Euromaidan as a supporter of protests during the winter of 2013-14 and article about this experience appeared in \textit{The Westchester Guardian} on February 20, 2014 (link: bit.ly/3q4oE5D).} that ended up with numerous
losses of life on the side of protesters, while president Yanikovych
gave up his powers and fled the country to Russia.

Russia itself not only gave refugee to fugitive Yanukovych, it invaded
into Ukraine with army troops without insignia in February 2014. They
crossed the Ukrainian borders into territories of Crimea and Eastern
Donbas, effectively occupying them since then till now. Ukraine's army
withdrew from Crimea while aimed to fight back in Donbas. The Western
powers, the US and the EU, imposed sanctions that were targeting
financial accounts of Russian officials and restricting access of the
Russian borrowers to international financial system.

\begin{quote}
With its currency near an all-time low, its stock market down twenty percent this year and a marked rise in interest rates, Russia has already started to bear the economic costs of its unlawful effort to undermine Ukraine's security, stability, and sovereignty.~\citep[quoting David S. Cohen, Under Secretary for Terrorism and Financial Intelligence]{ustres2014}
\end{quote}

The above-mentioned quote borrowed from the statement from the US
Treasury dated 2014 reveals common assumption of the Russia economic
conditions on the eve of 2014. This assumption is that Russia's economy
right before the 2014 sanctions was in a kind of normal point, i.e. its
economic parameters were ``right". Indeed, Russian ruble lost in value
versus the US dollar during the later part of 2014 and in 2015, too. Its
stock market dropped and interest rates both on domestic instruments in
rubles and in foreign one in US dollars jumped in response. What if, as
it was explained above, ruble was \textit{not} in ``right" point? What if the
stock market is not a concern for the majority of general public that
votes? What if interest rates on the borrowings in the foreign money of
account is not a point of concern for the Russia authorities? What if by
free-floating the exchange rate of ruble, Russia authorities move away
from the nearly chronic dependence on one commodity exports? It is not
known exactaly whether Russia government were making fun of sanctions or
not, but already in early 2015 Russia finance minister Siluanov was
widely talking the following way:

\begin{quote}
 ``Dutch disease is over," Siluanov said in an interview, referring to
 the fallout from a commodity boom that pushes up exchange rates and
 stalls competitiveness.~\citep{bbg2015,tass2015}
\end{quote}

At this moment, we must recall that 2014-15 was characterized by two
swift developments in the international financial system: (1) steady and
sizable appreciation of the US dollar versus major money units of
account as measured by the DXY index (see Appendix), and (2) similarly
steady and sizable drop in the crude oil price. These developments were
rather problematic for an economy that relies on crude oil exports as
revenues and dollar-based borrowings. This is exactly where Russia used
to be and GFC revealed its vulnerabilities. The following statement by
Russia's deputy finance minister Alexey Moiseev,\footnote{Short bio of Alexey Moiseev is here: \url{https://www.fsb.org/profile/alexey-moiseev/}} made eight years
after the GFC, has been symptomatic and telling about the long-lasting
trends and changes taking place in the Russia's economy:

\begin{quote}
 I do not see a stagnation of the Russia's financial market. I observe
 instead that financial market has been actively developing. If we
 consider the financial market as a whole, we see there are several
 processes that are taking place. These processes realize our
 long-standing dream and they lead to what where we want to arrive
 eventually. In particular, this is sort of import substitution of the
 foreign [financial] markets. Here, I disagree with the statement that
 Eurobond borrowing is a good instrument for raising funds [by a
 Russia's business entity]. You know, the financial crisis of 20014-15
 showed that even to the companies with large share of export revenues
 Eurobonds were very risky instruments for raising funds. There were
 periods when hedging of foreign exchange risks was cheap.
 Nevertheless, we see that even, when they were cheap, they worked out
 at the end not the way the companies expected. I can share information
 that largest companies that hedged FX risks have suffered losses.
 \dots I conclude that our shared task must be putting to rest attempts
 of returning to the international financial markets with new
 borrowings. We have lastly a chance to substitute foreign [financial]
 markets, to which we accustomed ourselves and where it was
 comfortable, with domestic [financial] markets. Today, our financial
 markets is very comfortable. My meeting with foreign investors reveal
 that foreigners do receive the same level of high-quality service in
 terms of clearing, settlement, payments, etc as in major financial
 centers such as London and New York.~\citep{moiseev}
\end{quote}

The key outcomes from the 2014 invasion of Ukraine for Russia are the
following. First, its economy moved to the more flexible exchange rate
policy than it used to be. There is no yet a proper definition of
free-floating in economics. Ruble before 2014 was considered largely
free-floating money unit. However, next few years showed that was
another stage of free floating, which is pretty different from the past
one in terms of exchange rate upper and lower boundaries. Second,
popularity of Kremlin and Putin personally skyrocketed in 2014 (see
Figure 5 in Appendix). Effectively, public discontent with authorities
largely evaporated.

It, however, went south by 2018 again. Authorities pushed through the
pension reform, where retirement age was raised. It was adopted during
the meeting of the government headed by Medvedev, who served as prime
minister after serving as president for a while. The meeting was held
right on the eve of the opening of the World Cup held in Russia.
Nevertheless, the government decision caused uproar among the general
public. Kremlin and Putin lost popularity at an extent, see Figure 5.

\subsubsection{The second phase of the war of 2022}

Russia invaded Ukraine on February 24th, 2022 with a bigger military
force than in February 2014. The Western powers---the US and the
EU---responded with wave of sanctions with a freeze of the Russia's
official foreign-exchange reserves on the correspondent accounts with
banks in their jurisdictions. In the wake of the invasion, there was a
usual spike recorded in public approval of Kremlin and Putin, in
particular (Figure 5). However, this paper recognizes that correlation
does not mean causation. Why than such striking dependency exist?

\cite{dibb2022} put it this way: ``[T]here is obsession of Ukraine [in the
Putin's Russia]. ... Putin like some other Russians probably many has a
peculiar attitude toward Ukraine."\footnote{On the video, watch since 19:36 timing.}

Russia propaganda in the Russian language and later in foreign language
media has been traditionally portraying in derogatory terms the
sovereignty of Ukraine. There has been a whole tradition in the Soviet
Union to antagonize, denegrade, mock Ukrainian leaders that fought for
country's sovereignty in 1920s. It gained new talking points, when
Ukraine factually gained sovereignty in 1991.

Under Putin's political regime it has been progressively doubling down
on these terms. Internationally, there are two big political events that
made modern Ukraine known: Orange Revolution of 2004 and Euromaidan
Revolution of 2014. In Russia these events are called colored
revolutions and talked about in derogatory terms by the Putin's regime.
In fact, the Ukrainian people of different nationality and speaking
normally Ukrainian and Russian, who gathered for those two events, knew
from the very beginning that their protest was against the Putin's
regime spread into Ukraine. Since 2012, when Putin got re-elected as
president of Russia the tide of anti-Ukrainian stance and the vocabulary
of derogatory terms were on the rise and reaching new highs. It has been
in a grand display on the eve of the war invasions by Russia into
Ukraine since 2014 and 2022. In Russia, domestically the atmosphere seen
in the public discourse is properly characterized by the following
terms:

\begin{quote}
[T]he paranoid style of \dots politics, in which the nation is
depicted as besieged and threatened by enemies from within and
without."~\citep[pp.~82-83]{dharvey}
\end{quote}

With all domestic policies that has been implemented in Russia in the
1990s and since 2000s, which have been unpopular but still implemented
for the sake of macro-financial stability, modern day Russia's
government do qualify for David Harvey's ``a growing column of neoliberal
state aparathuses worldwide"~\citep[p.~29]{dharvey}. Ukraine's governments
under different presidents, too, qualify. Even though, Russia explains
that it wages war against the West at large, not Ukraine in particular,
its intellectual underpinning for the organization of economic
relationships is derived from the very modern-day West. Let us recall
that second most popular political force in modern-day Russia is
Communist Party of Russia. Putin's regime speaks of some possibility of
restoration of the Soviet Union, which is popular among the voters who
follow the Communist Part, but he never speak of abolition of the
private property institution on the means of production of large-scale
type. Instead, domestically Russia's government retains the language
that carefully preserves the business friendly environment. Business
institutions are developing along the best practices and guidelines
borrowed from the West. Financialization and organization of the economy
upon market-based finance has been there for a lengthy while already and
never curtails conceptually. Putin's regime has been explicitly
underlining that it is open to business as usual.

Monetarily, Russia's domestic economy experience the following
development on the back of the war on Ukraine lasting since 2014. Yet 15
years ago, about the time of GFC, general public as a whole held 90-120
billion in US dollar equivalent as commercial bank \acp{iou} denominated in
the foreign money units of account. Normally, \acf{usd}, euro and pound
sterling were the top ones. As of November 1st, 2022 the size of the
household deposits with local banks in foreign money units dropped by
about 40\% to 57 billion in US dollar equivalent, while yet at the
beginning of 2022 it amounted to 94 billion \citep{zadornov}. This speaks to
the above-mentioned development of disciplining domestic businesses and
private individuals to adopt national money of account as
operational instead of the foreign one.

\subsection{Conclusion}

Russia has been carrying its war on Ukraine as matter of fact for
several years now. It has been sanctioned with numerous rounds for its
open destruction of the Ukraine's society and underlying economy. For
the sanctions of 2022 as well as for the sanctions of 2014 type the
following consideration apply:

\begin{quote}
Sender governments [that impose sanctions] face numerous barriers to
imposing economic sanctions that are sufficiently costly to force a
target government to make concessions. Senders' first major challenge
is \textit{knowing} how costly the sanctions against a target state need to
be in order to make the target concede.~\citep[p.~169, emphasis added]{sanctions2021}
\end{quote}

Hence, the sanctions' most simple formula $\text{Costs}>\text{Benefits}$ worked for
Russia so far in the following way $\text{Costs}<\text{Benefits}$.

There is underlying theory behind sanctions, where conceptually scarcity
of resources, including money and finance, is the first stepping stone
of entire theoretical corpus. Economic logic behind sanctions forges
expectations that Russia's government will experience government budget
deficit deterioration, leading to ``money printing" and hence high
inflation. There is kind of thinking that things that seemingly ruined
the Soviet Union economy must be in play with Russia's this time.

\ac{mmt} rebukes this stance by arguing ``finance is not scarce resource"
\citep[p.~139]{wray2020}. Its conceptualization of money hierarchy as well as
the sectoral balances and endogenous money concepts (not discussed in
due detail in this paper) gives a rational for proper understanding of
the Soviet Union economy demise in the 1980s and tendency to accelerated
inflation of the early 1990s.

Proper knowing of the motives of Russia's war on Ukraine includes
realization of Russia own economic and social crises,\footnote{From the speech by Andrey Belousov on April 22, 2013: ``Today's situation of economic slowdown, unfortunately, at a great extent is not because of the global economy, it is due to \textit{internal} factors"
(emphasis added). Link: \url{}.} from which Russian population attention diverted by the Putin's regime onto manufactured crisis/war in Ukraine.
