\newpage

\section{\uppercase{Introduction}}\label{sec:intro}

%\subsection*{Abstract}This introductory chapter contains five sections. The first one delivers the problem statement. It starts off acknowledging it has been an established norm in economics to regard the concept of capital flows as central to entire field. Nevertheless, controversy about it does persist. The second section reviews the literature with standard critique of capital flows as an economic concept. The third one considers a more radical critique that rejects capital flows conceptualization as the metaphor of motion. Section four details the capital flows as a metaphor, and major liquistic studies are relied upon here. The last fifth section reviews the concept of capital, separately from flows, relying on heterodox economics. The latter is adopted throughout this work.      

\subsection{Capital Flows: The Central, and Yet the Least Understood Topic}

In academic and policy-making circles worldwide the concept of \textit{capital flows}, which is also referred to as capital mobility or movement with its international dymansion, is one of the widely accepted, used and discussed. The prevailing thinking is that capital flows play a \textit{central} role in the global economy and its analytical framework developed so far is solid \citep{koepke2019}. Yet some economists acknowlege that the concept itself remains obscure. It is ``the most controversial and \textit{least understood}" \citep{eichengreen2001}.\footnote{On October 5th, 2024, \citeauthor{fukuyama2024} acknowledged that promotion of capital mobility since the 1980s was a ``big mistake."~\citep{fukuyama2024}} At another angle, there are many economists holding a view that capital flows are beneficial and, hence, must be encouraged. At the same time, a nearly equal number of economists is out there, who think capital flows are dengerous and, hence, must be handled with care or controlled.  

In economics, it has been this way for more than a century now. Every time a financial crisis of global proportions took place, economists and policy-makers turned their attention to the capital flows and underlying forces behind them. The period between the First and the Second World Wars was filled with dramatic economic developments and extreme efforts by economists on rethinking the doctrines they held. Those did not to fit into the realities of the everyday life. For example, in 1933 \citeauthor{keynes1933}, being already an authoritative public voice on the economic matters, observed: ``orientation of my mind is changed; and I share this change of mind with many others" \citep[p.~177]{keynes1933}. He was talking about the long-held doctrine of ``Free Trade."\footnote{For contemprory elaboration on the free trade doctrine with respect to capital see \cite{viner1926}, who was not on the side of Keynes and did not change his mid over the doctrine's key postulates niether during the 1930s nor in the post-war period \citep{viner1947}. \citeauthor{viner1926}'s paper written in 1926 is interesting in that it observed, firstly, that trade in capital was more free than trade in commodities: ``it is a valid generalization that the international movement of capital is as free from protectionist restrictions as the international movement of commodities is subject to them" \citep[p.~40]{viner1926}. Secondly, by the mid of 1920s ``there has made its appearance in many of the lending countries a literature, substantial in volume if not in quality, which brings to the support of financial protectionism economic reasoning similar in its general character to that which, reinforced by non-economic considerations, has for so long persuaded most of the world that commercial protectionism is sound national policy" (ibid).} It was developed by England's phylosophers during the nineteenth century and considered by Keynes' contemproraies and and their predecessors as an absolute wisdom. The doctrine essencially held that free and unfettered international trade in goods and services as well as in financial instruments (or what is known today as capital flows) yields prosperity to all the parties involved. At the time, Keynes' rethinking of the doctrine in the face of economic troubles of the day and fragile peace among the major countries forced him to  conclude:

\begin{quote}
Ideas, knowledge, science, hospitality, travel these are the things which should of their nature be international. But let goods be homespun whenever it is reasonably and conveniently possible, and, above all, let finance be primarily \textit{national}. \citep[p.~181, emphasis added]{keynes1933}
\end{quote}

Today, this period in history is known as one that deliminated two long-lasting episodes of economic and financial globalization. 

The first episode of blossoming of the international investors' activity was the one when above mentioned Free Trade doctrine ruled supreme and lasted through 1914. The First World War outcomes and soon the Great Depression in the US shook the doctrine's foundations by the late  1920s and the early 1930s. Already at that time, \citeauthor{keynes1933} was not alone, while considering the economic phenomenon known as ``the flight of capital" as one to be rolued out \citep[p.~180]{keynes1933}. In the U.S., the New Deal policies eventually arrived at the point of letting finance to be national, i.e. free from foreign constraints. As \cite[p.~185]{schlesinger1958} put it the Roosevelt administration adopted the liberal version of economic order being ``internationalist in trade, nationalist in finance," breaking decisively away from the previous administration's conservative version of economic order that used to be ``nationalist in trade, internationalist in finance". This profound rethinking of the Free Trade doctrine was lasting through the Second World War. Heavy advocacy for a grand re-design of the post-war operations of the international payment and finance systems was taking place since the early 1940s. It featured two broad groups of economists discussing the matter and effectively shaping the operations of the international money system: one advocating capital controls, another for retaining the features of the financial liberalism of the Free Trade doctrine. The historical momentum and moral authority in the eyes of the public was on the side of the former group, while the latter got itself marginalized. The outcome was the post-Second Wold War monetary system of Bretton Woods agreement, which was coupled with domestic policies such as New Deal in the United States. In short, it was restrictive towards international finance practices in the version of the Free Trade doctrine and considered it as less important compenent of the new international economic order \citep[p.~50]{helleiner1994}. As the U.S. Treasury Secretary Morgenthau said about the purpose of the post-war arrangement that it was ``to drive the private money lenders from the temple of international finance" \citep{kregel2009}. For a while, it seemed that Keynes' suggestion was holding true that by the mid-point of the twentieth century ``it may be that our habits of mind and what we care about will be as different from nineteenth century methods and values as each other century's has been from its predecessor's" \citep[p.~178]{keynes1933}. However, things started to shift so that very nineteenth century methods and values got re-established themselves.

The second episode of expanded role of international investors worldwirde, or revival of liberal finance and capital flows, has been observed since early 1970s and through present \citep{kregel2009}. Some would argue that this second episode started a bit earlier in the 1960s \citep{coombs1976,helleiner1994}. Hence, there was a U-shape trajectory of the prominence of international capital flows in economics, indicating a revival of Free Trade doctrine in theory and practice so that private-driven international capital flows of the second episode are at least on par  with the level seen during the first one \citep[see][p.~62]{milne2014}. Instability of international finance has been an essential feature of the second episode, too. Volatility in exchange and interest rates resulted in a series of outsized financial crises in Latin America in 1980s and South-East Asia and Russia in the 1990s \citep{kregel1996,kregel2009}. Then, the \ac{gfc}, which arrived into the developed market economies in 2007-08, reminded that top capitalist economies had not immunized themselves completely from a financial crises of grand proportions. And lastly, since 2022 the world woke up to realize that there has been a hot war in Europe waged by Russia against Ukraine. As in the 1930s, question of true peace (not the one that validates aggressor's claims) has been raised by the mid-point of the 2020s.

Nowadays, after all the past set-backs observed with the experience of international capital flows during both of above-mtioned episodes, the prevailing thinking among the mainstream economists is that they are beneficial. The logic is the following: liberalized international trade in financial claims is equated with free international trade in goods and services. This is because both types of international trade share key characteristic that is overcoming the domestic \textit{scarcity} of the resources. Both of them on net produce positive welfare effect to the domestic community \citep[p.~5]{milne2014}. It is recognized that advanced economics retain the largest share of international capital flows. At the same time, the emerging market economies require them for the sake of own financial stability. This is because they are ``particularly exposed to swings in the availability of foreign capital" \citep[p.~516]{koepke2019}. Such a view held by the modern-day's mainstream economists hardly deviates from the Free Trade doctrine of the ninteenth century. Thus, compare it with advocacy of the capital flows from the 1920s:

\begin{quote}
Countries where capital is scarce relative to the opportunities for its profitable investment usually invite the entrance of foreign capital. Occasionally, however, they are afraid of it. \citep[p.~39]{viner1926}
\end{quote}

It has been case for more than a century now that topic of capital flows was central, while problematic given the crises associated with them. Yet, with every next financial crisis still happening the line of prominent economists, saying our understanding of the matter has shortages, has been getting longer \citep[see][]{roosa1983,eichengreen2001,kforbes2005,abdelal2007}.  

\subsection{Literature Review of the Standard Critique}

Consider the 2023 Michel Camdessus lecture on the central banking at the \ac{imf}, an annual flagship event of this international financial institution. The invited keynote speaker, governor of the central bank of \ac{sar}, had a topic on international capital flows. He was talking about experiences of his own country. As governor of the central bank he regularly meets with foreign institutional investors, who practice face-to-face meetings with key decision-makers and economists of a domestic economy they invested. Their job is to make sense of the future prospects of such an economy and possible policy responses to the evolving environment. The \ac{sar} central bank governor explained that from his side such meetings serve the following purpose. ``The fundamental goal of these engagements is to encourage investment." In other words, to encourage capital \textit{inflows}. But there is important another observation by him. Every time such meetings end, he would have an internal meeting with central bank employees. ``Then I return from these meetings, and we have policy sessions where staff want to talk about the \textit{dangers} of capital flows. But the investors I just met are the people who are responsible for the capital flowing. So, I wonder -- which part of my time am I wasting? Do we want these capital flows or not?" \citep[emphasis added]{sarb2023}.

This observation is characteristic of the capital flows topic. There are benefits and costs. As far as benefits are concerns the mainstream economic theory asserts:

\begin{quote}
These [capital] flows, and capital mobility more generally, allow countries with limited savings to attract financing for productive investment projects, foster the diversification of investment risk, promote inter-temporal trade, and contribute to the development of financial markets. In this sense, the benefits from a free flow of capital across borders are similar to the benefits from free trade [in goods and services]. \citep[p.~4]{ostry2010}
\end{quote}

The major point is that free trading in widgets and dollars are identical.\footnote{The IMF staff admits that advocacy for capital flows liberalization via identity between capital flows and free trade is done ``at the \textit{simplest} analytical level" \citep[p.~1137, emphasis added]{citrin_fischer2000}. \citep{kregel2008} provides due critique and overview of this advocacy ``from the straightforward application of the neoclassical approach" at the sophisticated level of analysis.} It might be called trade homogeneity idea.
The metaphor of trade in widgets and dollars is from \cite{bhagwati1998a}. Respectively, it speaks about international trading in goods (widgets) and in capital (dollars).
Because capital is another word for money, hence, ``dollars" must be understood as money of different origins. It might be---in the parlance of capital flows terminology---US dollars, or Australian dollars, or Japanese yens, and so on. \cite{bhagwati1998a,bhagwati1998b} criticized this approach right in the aftermath of the financial crises of late 1990s.

There has been another point explaining benefits of international capital flows. It is scarcity of resources. And since finance has been considered as a scarce resource as, for example, some a rare earth mineral, hence, the scarcity idea and and trade homogeneity idea appear fitting each other:

\begin{quote}
Inequalities of endowment the world over require, for welfare maximization, the flow of capital across national boundaries. \citep[p.~212]{shaw1973}
\end{quote}

Back then, on the eve of the Asian and then the Russian financial crises of 1997-98 the ruling idea with respect to development was about financial deepening and liberalization. Opening of domestic economy to international capital flows was the path to future prosperity. While retaining barriers to them via capital controls was a sure policy to stay underdeveloped and poor. This idea was spelled out in early 1970s and became part of a set of policies for market-based development, which later became known as Washington Consensus \citep[see][]{shaw1973,mckinnon1973,williamson2004,kforbes2005}.

It took financial crises of 1990s and then Global Financial Crisis of 2007-08 to shake the financial liberalization consensus.
The costs of international capital flows became broadly evident and hard to deny.
Hence, policymakers and mainstream economists embraced some capital controls.
It was ``a shift \dots that broke with the Washington consensus view" \citep[p.~ 276]{acharya2018}.

In 2012, International Monetary Fund (IMF) issued an institutional view (IV) on the issue of capital flows management (CFM), where word ``management" is just a useful and convenient substitute for capital controls.
It recognized substantial risks associated with financial liberalization and capital flows.
Now, IMF has ``no presumption that full liberalization is an appropriate goal for all countries at all times"  \citep{imf2012}.
Days had past, when IMF was considered ``the bastion of [rapid] capital market liberalization"  \citep[p.~154]{kforbes2005}.
In 2022, the IMF under new leadership and staff have updated the IV on CFM  \citep{imf2022}.
Now, instead of rapid liberalization, there is a notion of well-designed liberalization.
It means in essence that same theoretical framework is in place, while the pace of steps that domestic policymakers must undertake for liberalization is more measured.

To conclude, the mainstream economics have modified just a bit its view of international capital flows by learning from experiences that revealed costs.
There was no effort put into inquiry to abandon the Eichengreen's description of capital flows as ``least understood" topic  \citep{eichengreen2001}.

The underlying theoretical premise of the capital flows/mobility concept has been retained all along.
It is not only about the idea that international capital flows not only identical with international trade in goods and services in terms of the perceived welfare benefits \citep{ostry2010,milne2014,koepke2019}. More importantly, as this dissertation aims to explain, mainstream theory considers capital (money) as much as physical goods being movable across borders. That is the term capital flows itself implies there are both the point of departure and the point of destination. In other words, a jurisdiction from which an investment funds were originated had given up a scarce resource, while a jurisdiction where a recipient of the investment funds is located had gained that scarce resource.

International economics and its popular sub-field of exchange rate determination has vast literature. Both orthodox and heterodox economists have theorized about it in a extent. However, it appears that they all come to a consensus that there is lack of theory.

There have been mainstream and heterodox economists, who criticized the above-mentioned theory of international economics. Their key arguments were: (1) benefits from the free movement of capital are overestimated, while risks are underestimated, and (2) there is no ground to equalize international trade in goods and services with international trade in financial instruments.

As some dust had settled since the Asian and Russian financial crises of 1997 and 1998 respectively, \cite{bhagwati1998a} described it is a ``capital myth" that free international trade in widgets and dollars is identical.
His explanation of the myth was formulated through the prism of political economy's concept of power.
The latter arises from the private financial sector's interest in gaining additional market share or, in other words, traffic of business activity.
The author never went beyond the established framework, where notion of capital movement remained central.

Heterodox economists, too, pointed out at the flaw of equalizing trade in goods and services with trade in financial instruments (capital flows). Both \cite{carvalho2009} and \cite{bonizzi2017} articulate that in order to correctly conceptualize capital flows one should not confuse foreign financial investment with goods. And, instead, a ``monetary analysis needs to consider capital flows as ``flow of funds", rather than transfers of ``real" resources" \citep{bonizzi2017}.

\cite{mason2015} takes a similar view arguing that foreign portfolio investments, which constitute a sizable portion of capital flows, are not providing the recipient economies with production capital in the form of ``new machines or software or engineers or land".
Instead, these are investments into financial instruments, which are sold in some significant extent via the secondary markets not the primary ones.
Hence, the author admits his preference to avoiding the terms ``capital flows" and ``capital mobility" altogether.
However, the ``capital flight" term remains in his vocabulary, see \citep{mason2016}.

\cite{kregel2008,kregel2015} explains that mainstream theory behind capital flows ``is based on faulty theoretical justification" \citep[p.~219]{kregel2008}.
This kind of critique is one of the most elaborated.
Early proponents argued that capital was flowing (being exported) abroad in pursuit of higher rates of return \citep{viner1928_1}.
Later on this proposition was extended assumption of higher productivity of physical capital.
Developed, savings-rich, and aging economies have lower marginal productivity of capital, while still-developing, savings-poor and population-growing economies have higher marginal productivity of capital.
The counter argument in \citep{kregel2008,kregel2015} is that theoretically and empirically those assumptions proved overstated.
And, by Keynes, finance must be homespun.

\cite{davidson2000} criticized the capital flow liberalization by distinguishing two theoretical hypotheses.
First one is the efficient market theory (EMT), a mainstream approach that has a direct link to the above mentioned idea that free trade in widgets and dollars is identical.
Second one is Keynes' liquidity preference theory (LPT).
Following LPT, the EMT's promises of efficiency gains are impossible to achieve.
In other words, ``EMT is not applicable to real world financial markets" \citep[p.~1117]{davidson2000}.

\cite{tobin2000}, too, criticized the capital flow liberalization by starting with impossibility to equate free international trade in goods and services with international trade in financial instruments. The latter is totally different activity versus the former. The gross volume of the international financial transactions dwarfs trade in goods and services (or what is call current account of the balance of payments). Also, ``nine tenths of these transactions [of gross capital flows] are reversed within a week, 40\% within a day" \citep[p.~1101]{tobin2000}.\footnote{Similar data was cited in \citep[pp.~3-4]{haq1996} and \citep[p.~330, end-note 3]{kirshner1999}.} Hence, capital controls are required both on capital inflows and outflows.

Considering the above-mentioned critique of the capital flows conceptualization, it must be noted that there have voices that demanded more.

Back in early 1980s Roosa asserted ``the world is \textit{still awaiting} the appearance of that all-embracing theoretical description of the economics of international trade and capital movements" \citep[p.~7, emphasis added]{roosa1983}.
Two decades later, in early 2000s, when mainstream economists faced a series of emerging market financial crises, Eichengreen stated ``[c]apital account liberalization, it is fair to say, remains one of the most controversial and \textit{least understood} policies of our day" \citep[p.~341, emphasis added]{eichengreen2001}.
Another book written as a response to financial crises of that period concluded: ``both scholarly and popular understandings of the origins and politics of financial globalization [i.e. international capital flows/mobility] should be \textit{significantly revised}" \citep[p.~3, emphasis added]{abdelal2007}. International economists, either from 1930s or from 1990s, occasionally pointed out at the remaining confusion while dealing with subject matter. In the early 1930s, \cite{iversen1936} observed (p. 35, emphasis original): ``The attempts sometimes made to express all international transactions in terms of imports and exports, visible or invisible, necessitate a certain distortion of popular language which may cause confusion. It seems better, therefore, to speak of debits and credits. Even the phrase ``export of capital" might be misleading to the layman, seeing that an export of commodities occasions a payment \textit{to} the exporting country, whereas an export of capital involves a payment by the exporting country to the Pee vie ATI RS country. It has been suggested that it might be better to speak of an ``import of securities" in order to stress the fact that the import of securities (export of capital) like the import of commodities is a debit item in the balance of payments (Barrett Whale, International trade, London 1932, p. 31)."

While in the late 1990s, \cite{kant2002} referred to the group work of \cite{abalkin1999}, which was an attempt of international group of economists to clarify the difference between capital flight in emerging-market countries and developed-market countries. It suggests that in the wake of the late 1990s financial crises there were questions still to be answered. The Asian financial crises of 1997 forced the following acknowledgement: ``A sad commentary on our understanding of what drives capital flows is that every crisis spawns a new generation of economic models." \citep[p.~58]{rodrik1998} as published in the collection of essays by orthodox economists \citep{princeton1998}.

Yet, in some short period after the \ac{gfc} and subsequent Euro-zone crisis, there has been a prevailing view that financial liberalization must revitalized.
Advocates of the new arrangement named it ``financial globalization 2.0" \citep{bis2016}.
In a broad extent, its basic idea remained in line with view expressed by \cite{eichengreen2000,eichengreen2001}: a new set of policies must allow to reap the benefits while containing the risks.

Nevertheless, in early 2020s the old issues remain. This, in \citep{blouin2022} the authors ``demonstrate that globalization affects governance, by increasing a country's vulnerability to sudden capital flight". It is the ``increased threat of capital flight can discipline governments and improve governance and welfare by placing countries in a 'golden straitjacket'". \cite{slobodian2018}, too, points out to this threat posed by capital flows as a tool of discipline:

\begin{quote}
[T]he world economy exercises discipline on individual nations through the perpetual threat of crisis, the flight of investment that punishes expansion in social policy, and speculative attacks on currencies in reaction to increases in government spending.
\citep[pp.~270-271]{slobodian2018}
\end{quote}

\cite{woodruff2016,woodruff2019} traces back this tendency of ``governing by panic," i.e. via austerity-minded policy-making that relies on the threat of capital flight, to the crisis in Eurozone of 2010-12 and then further back in history to the periods right after the First World War, when major economies were abandoning the monetary regime of gold standard one by one.

Since 1970s, there has been a subfield in sociology that critically reviewed the process of globalization. It captured and conceptualized the increased significance of risk as a potential threat to society. It has been having a disciplining forced upon the society if its government attempted to carry out some social programs requiring higher public expenditures and government deficits. It corresponds to the above-mentioned conceptualizations of the ``golden straitjacket" and ``governing by panic" just mentioned above.

Thus, \citeauthor{beck2000} considers seven theorems of the modern-day economic globalization, a phenomenon he termed as ``capitalist world society". One of the theorems, a second one by counting order, states that ``transnational corporations have interest in `weak states`" \citep[p.~96]{beck2000}. The author refers to Zygmunt Bauman writing as originator of this formulation. Meanwhile, the author applies traditional way of thinking borrowed from the standard economic approach to explain how exectly the states become weak. Here is the explaination: ``National states suffer from a sickness unto death -- falling tax revenue. Transnational states must therefore plug the tax loopholes if they are to develop power and competence in politicis and social policy." \citep[p.~112]{beck2000} Generally, \citeauthor{beck2000} formulates the \textit{risk} society, the concept that aims to describe the complex relationships accompanying the process of globalization. He mentions late 19th century critique of free trade by Max Weber, who uttered that ``it is to a nation's advantage to eat cheap bread, but not if this happens at the expense of future generations" However, \citeauthor{beck2000}'s risk society rests on the semantic hegemony of globalism:

\begin{quote}
Globalism, then, draws only small part of its strength from what is at present the case. Its potential force comes more from \textit{staging of threats}. This is the realm of the 'might', the 'should' and the 'if \dots then'.\par It is thus from a variant of \textit{risk} society that the transnational corporations derive their power. It is not 'actual damange' from \textit{economic} globalization \dots, but the threats of the same in public discourse, which stirs up fears, intimidate people, and perhaps ultimately compels political and trade-inion players to stave off the worst by themselves undertaking what investors want to see done before they are willing to invest. The \textit{cemantic hegemony} of globalism, its publicly fomented ideology, is a source of power from which the corporate sector draws its strategic potential.
\citep[p.~122, emphasis original]{beck2000}
\end{quote}

\citeauthor{bauman1998} also speaks within this sociological framework of imposed risk and undertainty\footnote{This piece was published in \citep{beilharz2001}.}: 

\begin{quote}
Global finance, trade and information industry depend for their liberty of movement and their unconstrained freedom to persue their ends on the political fragmentation, the \textit{morcellment} of the world scene. They have all, one may say, developed vested interests in `weak states` --- that is, in such states as are \textit{weak} but nevetheless remain \textit{states}. Deliberately or subconsiously, such inter-state institutions as there are exert coordinated pressures on all member or dependent states to destroy systematically everything that could stem from or slow down the free movement of capital and  limit market liberty. Throwing wide open the gates and abandoning any thought of autonomous economic policy is the preliminary, and meekly complied with, condition of eligibility for financial assistance from world banks and monetary funds. \dots As Michel Crozier pointed out many years ago, domination always consists of leaving as much leeway and freedom of manoeuvre to oneself as possible, while imposing as close as possible constraint of the decision-making of the dominated side; to rule, said Crozier, is to be close to the 'source of uncertainty'. This strategy was successfully applied once by state powers, which now find themselves on its receiving end -- it is now world capital and money that are the focus and source of uncertainty. \citep[p.~303, emphasis original]{bauman1998}
\end{quote}

According to \cite[p.~75, emphasis added]{kuttner2018}: ``The so-called Washington Consensus [since 1980s] became a one-size-fits-all recipe for developing nations. Its norms were exactly the inverse of the original Bretton Wood system. Poor countries needed private capital. To attract private capital, according to the Washington Consesus, they needed balanced budgets, modest taxes, privatization, limited welfare states, and \textit{above all}, free capital movements." Hence, \citeauthor{kuttner2018} observes there has been a ``massive power shift to capital" that all recent and massive crises ``did \textit{not} shake the political hegemony of finance."

In the Quinn Slobodian writings we find regular references to the topic of ``cross-border flows of money" \citep{slobodian2018,slobodian2023}, however, a due and more deeper analysis of these processes called ``flows" of money or financial capital as they are was not undertaken.
General assumption has been so far that our understanding of them had been perfected already.
At the same time, a slight sense of incompleteness remained. For example, Slobodian acknowledges respective limitation and accompanying convention in the introduction section of his book \textit{Globalists}:

\begin{quote}
Covering the better part of a century as it does, my account is necessarily \textit{incomplete}. \dots It does not explore the worthy topics of the conversion of the IMF and World Bank to the policies that became known as the ``Washington Consensus." Similarly absent are the transformations in the international monetary governance, including the rise of monetarism, the end of Bretton Woods system, the introduction of the euro, and changes in the central bank policy. This means \textit{leaving out all-important question of finance}, which was perhaps the single most important transformation in global capitalism since the 1970s. One reason for the omission is that these topics have been covered comprehensively by other authors. \citep[pp.~23-24, emphasis added]{slobodian2018}
\end{quote} 

The proper inquiry into the centrality and controversy of capital ``flows/mobility" must be based on the investigation of finance at its core. The next section provides a literature review of a more radical critique of the concept of capital flows. 

\subsection{Literature Review of the Radical Critique}\label{sec:rad_critique}

There is a view that rejects conceptualization of international financial transactions by utilizing metaphors of motion such capital ``flows/mobility" (``inflows/surges" and ``outflows/flights").

During second half of 1990s and early 2000s, \citeauthor{woodruff2005} investigated the underlying currents of the financial processes of two crisis-hit economies of Argentina and Russia \citep[see][]{woodruff2000,woodruff2005}. The author was puzzled about proliferation of monetary surrogates in the named economies as they were trying to rain in high inflation and achieve macroeconomic stabilization at that period of time. His book \textit{Money Unmade} \citeyear{woodruff2000} provides an extended inquiry into that matter with respect to Russia's economy transition towards a market-based economy during 1990s. However, it was his \citeyear{woodruff2005} paper, where he highlighted the need for the explicit balance-sheet analysis in order to sharpen our understanding of the economic phenomena. That paper concluded:

\begin{quote}
Discussions of international finance regularly rely on metaphors of motion: capital \textit{flow}, capital \textit{flight}, capital \textit{mobility}. The contribution of an institutional sociological analysis of money is to reveal these metaphors as deeply misleading. When an object moves through space, it passes from one set of surroundings to another---but these are mere surroundings, circumstances \textit{around} the object, which retains its original integrity. Capital---investment in debt or equity---is not this sort of self-sufficient entity, a tossed ball indifferent whether it is caught or missed. Capital exists \textit{only} as a relationship among parties, as rights and obligations, more or less perfectly specified in law or shared expectations. When capital ``moves," what happens in practice is reconfiguration of a network of such rights and obligations. Those who enjoy these rights or labor under these obligations can ascribe significance to them only in the context of their broader financial situation, consisting in other assets and other liabilities. This is another sense in which the image of capital as a self-sufficient object moving through space is misleading: the particular balance-sheet contexts in which capital is situated have a powerful influence on its effects. \citep[p.~36, emphasis original]{woodruff2005}
\end{quote}

Along the similar line, \cite{mosler2022,mosler2023} asserts: ``money does not move in that sense of motion". His preferred analogy is watching football on the TV screen. It might be live or prerecorded broadcast. By observing the game on the screen, the viewer sees the players running and ball moving, too. However, at a closer inspection the viewer with knowledge of how technology works will recognize that it is numerous pixels on the TV screen that go on and off that create this impression of moving objects. And it is similar with money system:

\begin{quote}
When you talked about people's idea of money [that] it moves from one account to another, and it moves from the banks and it goes offshore and it does all these things. But I'd like to tell the analogy I like to get people is if you look at your TV screen you're watching a football game you'll see people moving across the screen, but if you get right up close to the screen there's just dots going on and off. There's no nothing moving on that screen. And apart from actual cash which is a very small part of the economy the money is just dots going on and off of people's bank accounts. One number is going higher, another one's going lower. There's nothing moving from one account to another. As a matter of accounting, when you increase one account for that by decreasing another, but that doesn't mean anything actually move. If I buy something from a non-resident my bank account goes down, his bank account at (let's say I have a bank account in Citibank [and non-resident's bank, as well]\footnote{The institutional arrangement behind this statement is the following. The major assumption is the parties to the transaction agreed to settle the payment in the US dollars. Then, the reference to non-residents means that they have accounts with their domestic banks, which are denominated in the domestic money unit of account and in the foreign money unit of account, US dollar. The latter must have corresponding accounts with banks in that jurisdiction, where the buyer's bank does business. In the Mosler's hypothetical example he is the buyer of a German car produced by a Germany-based company Mercedes. For simplicity, the seller is the German car producer that is Mercedes. It has a checking/current account with a Germany-based bank, Deutsche Bank. Now, by assumption Deutsche Bank has a correspondent account with Citibank to settle payments in the US dollar.}) goes up. They'll say: the money left the country, because it went to a German resident. [Let's assume] it's a Mercedes or something. They have an account in Citibank. But what did it actually do: the dollars were debited from my account and credited to Mercedes' account. They didn't \textit{go anywhere}. \citep[emphasis added]{mosler2023}
\end{quote}

Essentially, both Woodruff and Mosler have raised identical proposition: one must apply a balance-sheet analysis to actual operations of commercial banks in settling cross-border (and domestic) payments. This is a type of the analysis that delves towards ``the operational, nuts and bolts (debits and credits) level" of the monetary system \citep[p.~36]{mosler2010}. Once done this way, it will be realized that metaphors of motion, indeed, are misleading and cannot be applicable.

In his 2010 book \textit{The 7 Deadly Innocent Frauds of Economic Policy}, \citeauthor{mosler2010} provides description of the fifth deadly innocent fraud, which is stated as ``the trade deficit is an unsustainable imbalance that takes away jobs and output" \citep[pp.~59-62]{mosler2010}.
The conventional thinking was that US was suffering from sustained trade deficits, such state of affairs was deemed unsustainable. Some expressed thier concern by saying that the US was behaving like a drunken sailor by borrowing from abroad to finance its purchases of foreign produced goods.
To reveal this thinking as faulty, Mosler goes into some technical details about the payments with respect to a hypothetical example involving a Chinese car producer:

\begin{quote}
Assume you live in the U.S. and decide to buy a car made in China. You go to a U.S. bank, get accepted for a loan and spend the funds on the car. You exchanged the borrowed funds for the car, the Chinese car company has a deposit in the bank and the bank has a loan to you and a deposit belonging to the Chinese car company on their books. First, all parties are ``happy." [...] Where's the ``foreign capital?" There isn't any! The entire notion that the U.S. is somehow dependent on foreign capital is inapplicable. Instead, it's the foreigners who are dependent on our domestic credit creation process to fund their desire to save \$U.S. financial assets. It's all a case of domestic credit funding foreign savings. We are not dependent on foreign savings for funding anything. \citep[pp.~61-62]{mosler2010}
\end{quote}

The above case with Chinese car producer is identical to the case with a German producer mentioned above.
The former was formulated in 2023, the latter in 2010.
The latter was created to reject the metaphor of a drunken sailor who supposedly acted irresponsibly by borrowing from abroad so that purchase of the foreign car took place.
The former appears to reject still-prevailing adoption of the metaphors of motion in the conventional discussions about the money system.

The logic behind these two identical cases is the same.
It is to show that the monetary systems---domestically and internationally---are debt-credit systems, not transportation systems.

Such way of thinking can be found in the writings of other economists.

In the middle of 1910s, the European countries being at war turned to the U.S. as a supplier of vast variaty of goods and capital (financing). \citeauthor{taussig1929} observed the rising trade surplus of the U.S. with the European countries such as France and the U.K. The U.S. producers extended their exports towards Europe at sizable extent and overall commodities exports were ``reaching sums never before dreamed of" \citep[p.~308]{taussig1929}. At large extend that great rise in export was paid by borrowing by above-mentioned European countries in the domestic capital market of the U.S. via different means: (a) early on via standard bonds issued by European states and bought by the U.S. investors, (b) later on via secured borrowed of the same countries but collaterlized by securities issued by U.S. entities, (c) and lastly, when the private U.S. investors considered the European borrowers as being streatched and of higher risk, via Liberty bonds issued by the U.S. Treasury.

\begin{quote}
There was much gloating over the vast ``favorable" balance of trade that was being built up; what could do more to enrich the country? \dots It is obvious that in reality, thru it all, \dots the thing that happened was that Americans bought from Americans. One group of Americans produced goods which were exported; another group supplied funds for paying them. \citep[p.~314]{taussig1929}
\end{quote}

All in all, the account provided by \citeauthor{taussig1929} noted that ``never before was the nature of capital export more completely misunderstood" \citep[p.~316]{taussig1929}. This critical attitude towards the established theoretical apparathus with respect to the matters of international economics lived on via the students of \citeauthor{taussig1929}. Two of them, \citeauthor{angell1925} and \citeauthor{white1933}, produced extensive works critical of the standard approach \citep[see][]{angell1925,white1933}. 

The above-mentioned observation by \citeauthor{taussig1929} is similar to the one by \citeauthor{mosler2010}. The similarity comes from the recognition that it is the U.S. domestic financial sector that accomodates the sale or purchase of goods. The only difference is about the origins of the produced commodities: in the former case it is produced domestically in the U.S., while in the latter one it is produced abroad.

In the U.K., economic discussions of the inter-war period of the first half of the 20th century and especially during the 1930s were dominated by Keynes and Hayek. The point of the discussion was the monetary regime held domestically and internationally. An insightful review of the core ideas of that discussion is by \cite{kregel1986}, being published as a chapter contribution to \cite{drukker}. By the time, Keynes had its \textit{Treaties on Money}, being first published in 1930, and was writing in 1940 that ``[o]ne by one, the currencies of the world, like their national economies were becoming independent of one another" \citep[p.~3]{keynes1980_25}. Also, in 1940, Keynes started to elaborate on the post-war monetary system and his origional proposal was about the \ac{icu}, which is based on ``the essential principle of banking":

\begin{quote}
The idea underlying such a Currency Union is simple, namely to generalise the essential principle of banking, as it is exhibited within any closed system. This principle is the necessary equality of credits and debits, of assets and liabilities. If no credits can be removed outside the banking system but only transferred within it, the Bank itself can never be in difficulties. It can with safety make what advances it wishes to any of its members with the assurance that the proceeds can only be transferred to the bank account of another member.
\citep[p.~3]{keynes1980_25}
\end{quote}

At the same time, Hayek had a series of lectures that turned into a published volume in 1937 and was titled \textit{Monetary Nationalism and International Stability}. According to \citeauthor{kregel1986}, Keynes observed: ``In truth, the gold standard is already a barbarous relic ... A regulated non-metallic standard has slipped in unnoticed. It \textit{exists}" \citeauthor[p.~34, emphasis original]{kregel1986}.
Hayek considered negatively the Keynes' position of advocating a replacement of the gold standard for a more flexible monetary regime.
In his book \citeauthor{hayek1937} distinguished three different types of national monetary systems.
These are (1) homogeneous international currency system, (2) ``mixed" system, and (3) the system of independent currencies.  The third type was what Keynes was advocating, while being objected by Hayek. It is interesting that \citeauthor{hayek1937}, while describing the method of capital flows of the third system, observed: 

\begin{quote}
\textit{No money actually passes from country to country}, and whatever redistribution of money between persons may be involved by the redistribution between countries has to be brought about by corresponding changes inside each country. \citep[pp.~17-18, emphasis added]{hayek1937}.
\end{quote}

As far as the author of dissertation can say, both of the above-mentioned authors did not go further into analysis of the cross-border relationships via details of debt-credit relationships between transacting counterparties. 

During the early post-war period, when the Bretton Words arrangement was already in place, \citeauthor{rueff1972}, a leading French economist, was writing about the economic phenomenon reminiscent to those described above. He referred to it as the Anglo-Saxon mechanism or procedure called the gold-exchange standard.  It replaced the classical gold standard that functioned from the mid nineteenth century and through 1914, when the Great War erruprted that later become known as the First World War. According to \citeauthor{rueff1972}, this mechanism worked during 1926-29 and during the 1958-61. Its major deviation from the pre-1914 gold standard was in the following. It allowed ``the banks of issue [of the countries other than the U.S. and the U.K.] \dots to create money which is backed not only by claims denominated in the national currency and by their gold stock, but also by foreign exchange payable in gold - that is to say, after the First World War, payable in pounds sterling and dollar." Hence, the nature of capital movements had become ``of an entirely new character and apparantly very unusual. In fact, the liquid funds, although entering into the economy of the recipient countries \dots did not leave the countries of origin overseas" \cite[p.~21-22]{rueff1972}.   

\begin{quote}
The use of such a mechanism has the considerable drawback of damping the effects of international capital movements in the financial markets that they affect. For example, funds flowing out of the United States into a country that applies the gold-exchange standard increase by a corresponding amount the money supply in the receiving market, without reducing in any way the money supply in their market of origin. The bank of issue to which they accrue, and which enters them in its reserves, leaves them on deposit in the New York market. \citep[p.~17]{rueff1972}
\end{quote}

This development \citeauthor{rueff1972} considered led to the Whall Stree crash in 1929 and subsequently to the Great Depression. He concluded a crisis of the similar magnitued was breweing in the early 1960s. Overall, he called this regime an ``absurdity" and ``the monetary sin of the West." His ideal regime is the classical gold standard, which does not create such ``fundametal disequilibrium" as described above.  He retained this view, while participating in the internal debates within \ac{mps} over the future monetary regime, whether it had to be based on the (old or classical) gold standard or on the flexible exchange rates and extensive capital mobility \citep{schmeltzer2012}. Intellectual solidarity between \citeauthor{rueff1972} and \citeauthor{hayek1937} under the umbrella of \ac{mps} appears to have an additional line of explanation, which is about their objection towards a monetary system that does not feature the assumed ``flows/mobility" mechanism of capital, where capital assumed to be scarce as gold and transported from one location to the other.   

Also, \cite{ingham2004} reviews the heterodox economic literature with respect to the conceptualization of money (Chapter 2, pp. 38-58). It is a concise while a quite broad review of heterodox strands of thinking: from early Banking school to German Historical School and from Keynes to Post-Keynesian Theory and revival of Chartalism. It appears the author's preferred way of judgement of each strand is how decisive was the intellectual break with (neo)classical economics. In one particular section of this review Ingham discusses 'circuitists.' These are the French and Italian economists who followed Keynes and developed the monetary circuit theory to the Post-Keynesian Theory. It is interesting that within these economists Ingham had identified those who were more radical in their writings:

\begin{quote}
Other formulations have begun to move further away from the traditional metaphors of 'things' that 'flow', or even move in a 'circuit'. As Cencini argues, following Tooke, credit-money \textit{does not flow}; rather, it is an emission that 'appears' and 'disappears' in the credits and debits of double-entry bookkeeping. \citep[p.~54, emphasis added]{ingham2004}
\end{quote}

Ingham's \textit{Nature of Money} appears a volume that continuously engages with the basic idea urged upon us by \cite{woodruff2005} and \citep{mosler2022,mosler2023}. That is our understanding of economy requires emancipation from the centuries old metaphors of motion that are embedded into thinking about money both in domestic and international dimensions. 

\begin{quote}
In advancing their argument, some credit theorists have referred to money as 'circulating debt'. However, this traditional metaphor, taken from the commonly sued seventeenth-century analogy with blood, is \textit{inappropriate} (Cencini 1988: 74). Rather, money consists in vast dense networks of overlapping and interconnected multilateral credit-debit relationships which are mediated by the issuers in a process referred to by Tooke in the early nineteenth century as 'efflux and reflux'. This is more obvious in the case of the 'clearing' of debits and credits in a bank giro, where money-stuff does not actually 'flow' around or through the accounts. The conception is even clearer in the case of debit cards. On one level, they are media of exchange, but on another they are the means of contracting a three-cornered relation of credit and debt between the buyer, seller, and card-issuing bank. Coins and notes should also be seen in this light, and might be referred to as 'portable debt' (Gardiner 1993: 224). \dots The coin is simply reusable credit in myriad credit and debit relations. This re-usability has been conceptualized as 'velocity' in orthodox economic theory , but this is \textit{misleading}. It is not so much a matter of the same 'money' circulating serially, as the creation of credit-debit relations denominated in a money of account for which there is an ultimate means of final settlement. Schumpeter's quip bears repetition: unlike commodity, money can have 'a velocity so great that it enables a thing to be in different places at the same time' (Schumpeter 1994 [1954]: 320). \citep[p.~73-74, emphasis added]{ingham2004}
\end{quote}

This doubt over the conceptualization of money as circuit, if it ever belonged to the nature of money, can be observed in the late writings of \cite{earley1994}, which appeared in \cite{dymski1994}. \citeauthor{earley1994} clearly followed the \cite{copeland1952} methodology on financial accounts of major sectors within an economy, while not referencing it directly in his papers cited here.
His one of the most relevant observations with respect to this dissertation proposal is: ``The notion of money is a \textit{thing} that \textit{circulates} is extremely difficult to slough off" \citep[pp.~346-347, emphasis original]{earley1994}.\footnote{Now, it is recognized that traditional institutionalists shared the view that conceptualization of money via circulation/velocity is obscure and analytically weak. See, for example, discussion on the analytical framework of John R. Commons by \cite{fiedler2022}, pp.284-285 emphasis added, from the \cite{biggiero2022} edited volume: ``[Commons] is namely arguing \textit{against} the concept of monetary circulation that underlies the conventional quantity theory of money and is connected with the generally recognized money formula, together with the speed of circulation (velocity)."}

\cite{kumhof2020} paper rejects the global savings glut hypothesis as explanation of the US sustained trade deficits and argues that it is the domestic US banking system that extends credit to purchases of imported goods by US consumers and firms.
At this point this paper is on the same side with \cite{mosler2010,mosler2023}.
However, the paper authors retained the motion-based terminology while inquiring into the phenomena of cross-border financial transactions.
Thus, explaining the shock to the domestic economy that comes from capital inflows, the authors wrote: ``[t]he shock triggers a gross financial flow into domestic banks \textit{away} from foreign banks" \citep[p.~4, emphasis added]{kumhof2020}.
They endorsed the general shift among economists in analyzing gross capital flows instead of net ones.
One of the major takeaways of the paper that must be broadly in line this dissertation proposal, when all motion-based metaphors are being discounted, is the following statement:

\begin{quote}
[A]ll financial flows necessarily consist of a pair of gross inflow and outflow components that are inseparable as a matter of \textit{accounting}, and that are therefore necessarily perfectly correlated. \citep[p.~11, emphasis added]{kumhof2020}
\end{quote}

In 2021, two former SWIFT employees published a book titled \textit{The Pay Off}.
One of the authors, Gottfried Leibbrandt, held the position of chief executive officer at this international payments company in 2005-2019. His 2004 PhD dissertation thesis was on the adaptation of the payment technologies \citep{leibb2004}. The book is full of recognition that metaphors of motion used to depict payments are ``actually" not what is taking place.

\begin{quote}
When we talk about payments, we talk about moving money and sending money; about channels and conduits; about flows and movements; about rails and routes; about traffic, transit, travel, transfers and transmissions. All these words imply movement but, truth to be told, the vast majority of payments are simply a sleight of hand: entries changed in book ledgers. \citep[p.~13]{payoff2021}
\end{quote}

Payments are entries in the balance sheets of the transacting counter-parties. There are payment systems. These are domestic and foreign ones. Their backbone is correspondent banking business:

\begin{quote} 
Despite widespread talk of 'money sending abroad,' no 'money' is actually 'sent'. What use would it be, after all, if pounds were sent to Frankfurt or euros to London? When you stop to consider it, it's obvious that money doesn't 'move abroad'. The vast majority of foreign payments were -- and are -- handled through a system called corresponding banking. The system was designed precisely to avoid the need to ship gold or other coinage around. \citep[p.~166]{payoff2021}
\end{quote}

Indeed, these observations by \citeauthor{payoff2021} are in line with ones made by \cite{woodruff2005} and \cite{mosler2010,mosler2022,mosler2023}. There is an extension, however. It suggests that balance-sheet analysis must pay attention to the payment system and correspondent banking relationships. Hence, the payment system and the balance sheets of banks are intertwined, enabling each other.

Quite similar observation was made by \citeauthor{tobin2000}, who did not mention balance-sheet analysis explicitly and never rejected openly the metaphors of motion behind the capital flows.
Nevertheless, the following statement is worthy to mention as its implication is clear:

\begin{quote}
\textit{Nothing} is involved in financial transactions beyond exchanging pieces of paper or making entries in electronic ledgers. The communications revolution makes these transactions easy, fast and cheap. No physical frontiers have to be crossed by financial assets. The only barriers to financial transactions are national regulations. \citep[p.~1101, emphasis added]{tobin2000}
\end{quote}

While discussing modern-day technology in payments and digital assets, \cite{payoff2021} quotes senior official of Bank of England, who made the following remark about underlying substance of the money transactions:

\begin{quote}
While the technology has changed beyond all recognition, all of these systems are economically the equivalent of 18th century bank clerks with quill pens altering their banks' ledgers *to debit one account and credit another* whilst addressing shortcomings. \citep[p.~9, emphasis added]{boe2020}
\end{quote}

Early literature analyzing the Eurodollar market from, at least, the 1970s did utilize the balance sheet analysis to comprehend the underlying processes of monetary operations by private businesses. There is, too, explicit recognition of the correspondent banking relationships, an essential element of the domestic and international finance. And this literature's observations turned out to be in line with above mentioned sources. For example, in \cite[pp.~1-16]{hewson1975}, there is a discussion of the typical case of Eurodollar operation, involving a non-resident company with respect to the U.S. jurisdiction and its monetary system:

\begin{quote}
Suppose that Company A, a European resident, receives a demand deposit for \$1,000,000 in City Bank, a New York commercial bank, as payment for an export of goods to the United States. Wishing to keep its deposits in dollars for future trading purposes, but encouraged by the prospect of an interest return, Company A transfers this deposit to a London bank, Eurobank I. That is, Company A instructs City Bank to transfer its deposit to Eurobank I where it is credited with (say) a three-month time deposit. These changes are recorded in the balance sheets of City Bank and Eurobank I as shown below. For City Bank, total deposit liabilities have \textit{remained unchanged}---the demand deposit of Eurobank I has simply replaced the demand deposit previously held by Company A. However, both the liabilities and assets of Eurobank I have been \textit{increased} by the amount of the deposit. \citep[p.~2, emphasis added]{hewson1975}
\end{quote}

This example of Eurobanking is depicted in \mbox{Figure \ref{fig:eurodollar1}}, \mbox{p. \pageref{fig:eurodollar1}}.
The author makes a couple clarifications in the footnotes: (1) this example is given to the Eurodollar market, which is the market of the US dollar denominated deposits, however, the principles are general and applicable to all ``Eurocurrency markets," which to deposits denominated in any foreign money unit of account; (2) for simplicity, it is assumed that City Bank and Eurobank I are banking correspondents, which means that City Bank has an account in the US dollar opened for Eurobank I. This account is a liability for City Bank, while it is an asset for Eurobank I.

\vspace{.1in}\begin{figure}[!ht]
\captionsetup{width=.8\linewidth,labelfont=bf}
  \centering
  \begin{tikzpicture}
  \draw[help lines,white] (0,0) grid (12,8);
        %
        % City Bank (New York) -------------------------------------------------
        \matrix (m) [matrix anchor=north, matrix of nodes, nodes in empty cells,
             nodes = {text width=2.3in, minimum height=.2in, align=left},
             column sep=1em,
             row 2/.style={align=center, font=\bfseries},
            ] at (6,8)
        {
                      &             \\
        Assets        & Liabilities \\
        \hspace{7mm}  & Demand deposit \\
        \hspace{7mm}  & \hspace{.1in} to Company A \hspace{.1in} --\$1,000,000 \\
        \hspace{7mm}  & Demand deposit \\
        \hspace{7mm}  & \hspace{.1in} to Eurobank I \hspace{.1in} +\$1,000,000 \\  
        };
        \node[fit=(m-1-1)(m-1-2)]{\textit{\MakeUppercase{City Bank (New York)}}};
        \draw[thin]  (m-2-2.south -| m.west) -- (m-2-2.south -| m.east);
        \draw[thin]  (m-1-1.south east) -- (m-6-1.south east);       
        %
        % Eurobank I -------------------------------------------------------
        \matrix [matrix anchor=north, matrix of nodes, nodes in empty cells,
             nodes = {text width=2.3in, minimum height=.2in, align=left},
             column sep=1em,
             row 2/.style={align=center, font=\bfseries},
            ] (n) at (6,3.5) 
        {
                      &             \\
        Assets        & Liabilities \\
        Demand deposit & 3-month deposit \\
        \hspace{.1in} at City Bank \hspace{.1in} +\$1,000,000  & 
        \hspace{.1in} to Company A \hspace{.1in} +\$1,000,000 \\
        };
        \node[fit=(n-1-1)(n-1-2)]{\textit{\MakeUppercase{Eurobank I}}};
        \draw[thin]  (n-2-2.south -| n.west) -- (n-2-2.south -| n.east);
        \draw[thin]  (n-1-1.south east) -- (n-4-1.south east);                  
  \end{tikzpicture}
  \caption[Eurobanking example \#1]%
  {Eurobanking example \#1: Company A, a non-resident to U.S., obtained \$1 million proceeds from selling goods in the U.S. on its account in City Bank (New York) and then transfers these funds on 3-month deposit in Eurobank I (London). \\ \\Source: \citep[p.~2]{hewson1975}}
\label{fig:eurodollar1}
\end{figure}

The analysis by \citeauthor{hewson1975} follows two prescriptions outlined above by \citep{woodruff2005}, \citep{mosler2010,mosler2022,mosler2023} and \citep{payoff2021}. There must be (1) institutional details of how cross-border financial transactions actually carried out, in other word, there is recognition of correspondent banking relationships, and (2) balance-sheet analysis. The first and second points are deeply intertwined.

And \cite{hewson1975} raises similar conclusions. As cross-border transaction being finalized, the funds never left the United States jurisdiction, while the United Kingdom jurisdiction has a commercial bank whose balance sheet expanded in the US dollar terms. Hence, when Company A instructed a New-York based City Bank to transfer the funds from the checking account to the US dollar term account opened with London-based Eurobank I, the 'money' did not move. What actually happened is both banks changed their balance sheets by recording respective accounting entries. This is perfectly in line with \citep{woodruff2005}, \citep{mosler2010,mosler2022,mosler2023} and \citep{payoff2021}.

Similar analytical observation can be found in the popular money-market textbook by \cite{stigum2007}. Here, as well, a reader has an introduction into the Eurodollar market operations. There is a case, where a London office of the U.S. bank which deposited U.S. dollars with the London branch of French bank Credit Lyonnais. It is depicted in \mbox{Figure \ref{fig:eurodollar2}}, \mbox{p. \pageref{fig:eurodollar1}}. Key assumptions in this case are the London-based branches of the banks have correspondent accounts denominated in the US dollar with banks based in the United States. And the latter are members of the Federal Reserve, implying they have reserve account with the Federal Reserve Bank of New York. The latter settles US dollar payments between all member banks. The key takeaway was summed up by the authors in the following observation:

\begin{quote}
Note again that the dollars \textit{remain} in New York, even though they are now held by the London branch of a French bank. \citep[p.~216, emphasis added]{stigum2007}
\end{quote}

Again, this observation is in line with all above-mentioned sources.
This example from \citep{stigum2007} is more extended one than the example from \citep{hewson1975}.

\vspace{.1in}\begin{figure}[!ht]
\captionsetup{width=.9\linewidth,labelfont=bf}
  \centering
  \begin{tikzpicture}
  \draw[help lines,white] (0,0) grid (16,16);
        %
        % Chase (London) -------------------------------------------------------
        \matrix (m) [matrix anchor=north, matrix of nodes, nodes in empty cells,
             nodes = {text width=1.2in, minimum height=.12in, align=left},
             column sep=1em,
             row 2/.style={align=center, font=\bfseries},
            ] at (4,16)
        {
                      &             \\
        Assets        & Liabilities \\
        Time deposit at Credit Lyonnais (London) & \hspace{.1in} \\
        \hspace{12mm} +\$20MM & \\
        \hspace{.1in} & \hspace{.1in}\\
        Demand deposit at Chase (New York) & \hspace{.1in} \\
        \hspace{12mm} --\$20MM & \hspace{.1in} \\ 
        };
        \node[fit=(m-1-1)(m-1-2)]{\textit{\MakeUppercase{Chase (London)}}};
        \draw[thin]  (m-2-2.south -| m.west) -- (m-2-2.south -| m.east);
        \draw[thin]  (m-1-1.south east) -- (m-7-1.south east);  
        %
        % Credit Lyonnais (London) -----------------------------------------
        \matrix [matrix anchor=north, matrix of nodes, nodes in empty cells,
             nodes = {text width=1.2in, minimum height=.12in, align=left},
             column sep=1em,
             row 2/.style={align=center, font=\bfseries},
            ] (n) at (12,16) 
        {
                      &             \\
        Assets        & Liabilities \\
        Demand deposit at Morgan (New York) & Time deposit of Chase (London) \\
        \hspace{12mm} +\$20MM & \hspace{12mm} +\$20MM \\ 
        };
        \node[fit=(n-1-1)(n-1-2)]{\textit{\MakeUppercase{Credit Lyonnais (London)}}};
        \draw[thin]  (n-2-2.south -| n.west) -- (n-2-2.south -| n.east);
        \draw[thin]  (n-1-1.south east) -- (n-4-1.south east);  
        %
        % Chase (New York) -----------------------------------------------
        \matrix [matrix anchor=north, matrix of nodes, nodes in empty cells,
             nodes = {text width=1.2in, minimum height=.12in, align=left},
             column sep=1em,
             row 2/.style={align=center, font=\bfseries},
            ] (o) at (4,9.5) 
        {
                      &             \\ % Title row: bank's name
        Assets        & Liabilities \\
        Reserve account at the Fed (New York) & Demand deposit of Chase (London) \\
        \hspace{12mm} --\$20MM & \hspace{12mm} --\$20MM \\ 
        };
        \node[fit=(o-1-1)(o-1-2)]{\textit{\MakeUppercase{Chase (New York)}}};
        \draw[thin]  (o-2-2.south -| o.west) -- (o-2-2.south -| o.east);
        \draw[thin]  (o-1-1.south east) -- (o-4-1.south east);
        %
        % Morgan (New York) ------------------------------------------------
        \matrix [matrix anchor=north, matrix of nodes, nodes in empty cells,
             nodes = {text width=1.2in, minimum height=.12in, align=left},
             column sep=1em,
             row 2/.style={align=center, font=\bfseries},
            ] (p) at (12,9.5) 
        {
                      &             \\ % Title row: bank's name
        Assets        & Liabilities \\
        Reserve account at the Fed (New York) & Demand deposit of Morgan (London)\\
        \hspace{12mm} +\$20MM & \hspace{12mm} +\$20MM \\ 
        };
        \node[fit=(p-1-1)(p-1-2)]{\textit{\MakeUppercase{Morgan (New York)}}};
        \draw[thin]  (p-2-2.south -| p.west) -- (p-2-2.south -| p.east);
        \draw[thin]  (p-1-1.south east) -- (p-4-1.south east);
        %
        % Federal Reserve (New York) ---------------------------------------
        \matrix [matrix anchor=north, matrix of nodes, nodes in empty cells,
             nodes = {text width=1.2in, minimum height=.12in, align=left},
             column sep=1em,
             row 2/.style={align=center, font=\bfseries},
            ] (p) at (8,5) 
        {
                      &             \\ % Title row: bank's name
        Assets        & Liabilities \\
        \hspace{7mm}  & Reserve account of Chase (New York)\\
        \hspace{7mm}  & \hspace{12mm} --\$20MM \\ 
        \hspace{7mm}  & \hspace{7mm} \\
        \hspace{7mm}  & Reserve account of Morgan (New York)\\
        \hspace{7mm}  & \hspace{12mm} +\$20MM \\
        };
        \node[fit=(p-1-1)(p-1-2)]{\textit{\MakeUppercase{Federal Reserve (New York)}}};
        \draw[thin]  (p-2-2.south -| p.west) -- (p-2-2.south -| p.east);
        \draw[thin]  (p-1-1.south east) -- (p-7-1.south east);
  \end{tikzpicture}
  \caption[Eurobanking example \#2]%
  {Eurobanking example \#2: A Eurodollar placement example: the \$20 millions are placed by Chase (London) with the London branch of Credit Lyonnais. \\ \\Note: MM = millions. Source: \citep[p.~216]{stigum2007}}
\label{fig:eurodollar2}
\end{figure}

The former involves the Federal Reserve Bank of New York as transacting banks both based in London have U.S. dollar correspondent account in separate US-based banks.
But in both cases, it is obvious from the balance sheet analysis that U.S. banks in aggregate do not experience shrinkage of the balance sheet size.
This confirms with statement made by former US Treasury official in his book \textit{Freedom from National Debt}: ``the [U.S.] dollars \textit{cannot go} to another country" \citep[emphasis added]{newman2013}. This line of thought was formulated by \citeauthor{earley1981}, who analyzing the Eurodollars system observed that ``the resident US bank dollars ``never leave home," being simply transferred from one party to another" \citep[p.~225]{earley1981}.

Borrowing from Stigum's example, \cite{feygin2020} was describing operations of the US banks with the Eurodollar markets. He concluded that ``actual dollars" never ``left the United States":

\begin{quote}
As we can see, Eurodollars are initially funded through an interbank transaction with a U.S. domiciled bank and its foreign branch. From the point of view of the onshore money system, \textit{no actual dollars have left the United States}. However, credit has now been issued that can ultimately be multiplied many times over to create a system of dollar funding. \citep[emphasis added]{feygin2020}
\end{quote}

But these observations by \citep{hewson1975}, \citep{earley1981}, \citep{stigum2007}, \citep{newman2013} and \citep{feygin2020} describe a phenomenon that is \textit{general} and not exclusive to the U.S. monetary system. In other words and to be a bit more precise, it is not only the funds denominated in the US dollar as money of account that behave this way.

The key is that transactions are taking place in the national money unit of account.
For example, if counter-parties of the financial transaction being situated in different jurisdictions and at the same time voluntarily agree that funds are going to be denominated in the Mexican peso, then Newman's statement would apply to the pesos.
It means that in aggregate the Mexico's banking system would not contract in terms of the balance sheet size.
A Mexican commercial bank, resident of Mexico, must have a balance on the peso correspondent account with its transacting foreign bank.
This peso account is a liability on the balance sheet of the Mexican bank.
For the latter bank, non-resident to Mexico, the peso account is an asset.

Financial market professionals tend to think on the same terms. Such as in this market commentary made by an asset management adviser:

\begin{quote}
Bank deposits can't really leave the system except through loan repayment and Quantitative Tightening.\footnote{``Deposits can technically leave through other sources such as RRP, TGA, etc, but loan repayment and QT are the primary ways that this is impacted in a structural manner across the last decade." \citep{roche2023}} So all deposits are held by someone until they're retired. \citep{roche2023}
\end{quote}

Another commentary was made by Ousm\'ene Mandeng.
He is a visiting fellow of London School of Economics ``who has spent decades working in the field of central bank reserve management" \citep{ft2022}. Previously, he worked at the IMF and authored several working papers on sovereign debt restructurings, foreign-exchange reserves management and Bretton Wood system,\footnote{ See \cite{brettonwood2016}.} see \cite{mandeng2003,mandeng2004,mandeng2016}. He was quoted by the \textit{Financial Times} as saying \citep[emphasis added, square brackets contain author's comments]{ft2022}:

\begin{quote}
Foreign exchange reserves are not held by central banks [in their heavily gaurded vaults]. Securities and money \textit{never move}, everything is external [meaning these are double-sided debt-credit relationships]... \par In the case of securities, central banks would ask their brokers to sell the asset in question. In the case of, say, a German government bond [that the CBR owns], the broker in Frankfurt will call other brokers to announce the sale and, once a price is agreed, will instruct the custodian of the security to transfer it to the buyer. Upon receipt of payment into a bank account, typically in Frankfurt, the custodian will instruct the central security depository to assign the buyer as the new owner.\par The central bank's then credited the proceeds at their account with the broker. The proceeds could then be used to instruct the broker, or foreign exchange dealers, most of whom are in London, to buy the ruble at a specific rate. The seller will usually be a Russian commercial bank. Seller and buyer may well share the same correspondent bank. Once the purchase is made, the Bank of Russia would instruct its correspondent bank to credit the seller's account with euros.
\end{quote}

The above mentioned three paragraphs describe two key features of the international monetary system. First, the money system is based upon debt-credit relationships, where correspondent banking relationships are principal. Second, the securities system is similar as it is based on the accounts. In other words, securities ownership operates via the debt-credit relationships.
Operations of the custody services and money system are deeply intertwined as discussed in \citep{occ2002} and \citep{gcastodian2023}.

Why is that dollars, bank deposits cannot leave the system? This is because money originate from the balance sheets. Not the other way around. These balance sheets reflect activities of the variety of economic entities: from government to commerce banks and firms to households. Balance sheets record creation of money, they are record their destruction. This is a continuous, regular process as government provisioning and business activity progress along.

A Post Keynesian economist writing in honor of Basel Moore's contribution to economics\footnote{The entire volume is \citep{moore2006}.} observed: ``[B]ookkeeping is central to the endogeneity of money. It is because bank money is both a liability and an asset with *no existence outside* banks' balance sheets" \citep[p.~21, emphasis added]{gnos2006}. This is a narrow view. For the purpose of this dissertation proposal, it must be re-stated in a better, general way: book-keeping is central to the concept of money, it is because money is both liability and asset with no existence outside economic entities' balance sheets.

Among international financial institutions and central banks, there are occasional instances of recognition of the above-mentioned characteristics of the monetary system. IMF's high-ranking official made the following remark about cross-border payments, which is, if obscurity is removed, in line with \citep{payoff2021}:

\begin{quote}
When a Moroccan ceramics business exports dinnerware to nearby Spain, it receives money in its account through a complex web of inter-linkages between banks, possibly going through Paris and New York. The payment is routed through banks that know and trust each other. \textit{Money does not really change hands; instead, each bank offers credit to the next one in line}. As a result, the small Moroccan business may face delays in receiving money and will pay high fees, hurting its bottom line. \citep[emphasis added]{adrian2023}
\end{quote}

This recognition has been explicitly stated in the 2003 joing working paper of \ac{imf} and \ac{wb} covering the project on the informal funds transfer systems such as \textit{hawala} system\index{Hawala}\footnote{Hawala is an Arabic term that denotes ``transfer," where the root
\textit{H-w-l} means ``transform" or ``change"  \citep[p.~328]{redin2014}.} and others. Due to the 9/11 terrosist attack of 2001, the anti-terrorist investigation launched by the U.S. and other countries requested a thorough investigation of the operations of such informal systems as terrorist group (as Al Qaeda) used them. In 2002, the joint group of the \ac{imf} and \ac{wb} researchers visited a number of countries, where these systems persist from ``very remote times"\footnote{Quote from \cite[p.~11]{imf2003}} till today. Their report features a paragraph, which is in line with above-mentioned statements by the \ac{imf} top-ranking official \cite{adrian2023}, but with non-\ac{imf} affiliated observers such as \cite{woodruff2005} and \cite{mosler2010,mosler2022,mosler2023}:

\begin{quote}
Assertions that hawala ``sends money without sending money" are misleading. Many discussions of remittances through the informal funds transfer systems give the impression that this kind of transaction is very different from making international payments through established institutions, such as banks or money exchanges. \dots In fact, \dots the modalities of hawala transmission are \textit{similar} to other kinds of international payments, including those that go through formal banking systems. The accounting details are also \textit{similar} \dots [R]emittance and payment systems generally rely on transmission of a payment order that is based on receipt of some funds at the remitting end of the transaction. Actual payment made to the beneficiary out of balances at the receivening end; settlement follows, or in cases where there are no exchange control issues, institutional accounts can be debited/credited congruently. The point of this example is to demonstrate that payment modalities around the world are \textit{similar} in terms of mechanics; the main difference among them is selection of formal or informal channels. \citep[p.~14]{imf2003} 
\end{quote}

In other words, the above statement says that functionally, i.e. while providing their clients with payment services, the work done by the \ac{nyc}-based J.P. Morgan from Wall Street is similar to the Kabul based  hawaladar\index{Hawaladar}\footnote{Hawaladar is a person that acts as operator called as dealer in the the hawala system \citep[p.~328]{redin2014}.} from the local money bazaar. Aside of this, in the vast majoriy of other aspects these operators are very much different.

Researchers of \ac{bis} have been producing top-quality papers on the usage of swap line by major central banks. This has been especially true during the Covid-19 pandemic, which required re-activation of the swap lines after previous occasion that was during the \ac{gfc}. In one of such works, \ac{bis} economists undertook the technique of balance-sheet analysis, which by default requires recognition of the correspondent banking relationships.
Their conclusion, provided as a quote below, is in line with the one mentioned above by \citep{hewson1975}, \citep{earley1981}, \citep{stigum2007}, \citep{newman2013} and \citep{feygin2020}:

\begin{quote}
The use of the swap lines expands the balance sheets of the central banks \textit{as well as} those of the commercial banks in both countries. \citep[p.~6, emphasis original]{bis2020}
\end{quote}

This phenomenon is also well known and being discussed by the central banks of emerging market economies. For example, on June 29th 2023, \ac{nbu} issued an official statement regarding the \ac{imf}'s loan facility worth of 886 million US dollars:

\begin{quote}
The funds will be channeled for [the government] budget support. Furthermore, this money will support Ukraine's [official] international reserves, which will strengthen the NBU's capacity to ensure the stability of the [domestic] FX market. \cite{nbu2023}
\end{quote}

Implicitly this statement is speaking about two things. First, the Ukraine's government has a US dollar account with the central bank, while the latter has a correspondent account with a US foreign bank. Hence, the payment of the \$886m is taking place on the balance sheets of the Ukraine central bank's US dollar correspondent bank, the central bank itself and lastly the government. Secondly, the government exchanges the US dollar balance at its US dollar account with the central bank for the balance denominated in the domestic money unit of account (Ukrainian hryvnia or UAH). Then, the government carries out its UAH payments, while the central bank has higher foreign-exchange reserves. The latter must serve as a buffer to support the exchange rate of domestic money versus the US dollar. While the above-mentioned funds of \$886m once credited to the US bank account of Ukraine's central bank are accounted in two balance sheets: (1) the US-based bank accounts them as own US dollar liability, (2) the central bank of Ukraine accounts them as US dollar asset.

Among academics, writing these days on the matters of international finance and payments, there has been a growing tendency to subject the standard presentation of the matter to a new scrutiny. They were accompanied by visibility of former industry professionals such as \cite{payoff2021}, which was already mentioned above, and \cite{scott2013,scott2022,scott2024_}. This developemnt has been taking place against the background of overlapping episodes that featured finance in good and bad light. Thus, the Covid-19 lockdowns worldwide showed that some (not all) countries were able to sustain their economies with government expenditures even though the economic activities of private businesses and individuales dropped dramatically. Next, Russia re-launhced its war against Ukraine in February 2022 after its initial aggression of 2014. It caused new wave of sanctions, including economic and financial ones, from the U.S. and other countries. Hence, additional interest turned towards the international payments systems and their key operational units such as \ac{swift}. Lastly, financial scandals continued to happen: private company Wirecard of Germany, once a celebrated champion in the fast-growing digital payments industry, ended up being bankrupt and its managers accused of a multi-million fraud.

As researchers started to look a bit more closer at the operations of the business units like \ac{swift} they started to pick up the ideas that question the conventional way of thinking about the international transactions. The following quote is from the paper that looked at what exactly \ac{swift} is doing: 

\begin{quote}
Figure 1 shows how correspondent banking relationships between the two 
correspondent banks transfer funds in a particular currency corridor; it also illustrates 
that \textit{messages but not money} cross borders. There is a series of domestic 
transactions with the two banks ending up with more and less money in their 
respective correspondent accounts (McCune 2014). Accounts (a/c) are debited and 
credited (settled) in response to messages transmitted between banks, all this via the 
SWIFT system. \citep[p.~7, emphasis added]{robinson2018}
\end{quote}

The authors of \cite{robinson2018} took their analysis further into the correspondent banking relationships. They referred to a blog article \citeauthor{mccune2014} written by a payment specialist and borrowed from it the chart (named as ``Figure 1") depicting the scheme of cross-border payment involving the correspondent banks. Originally, \cite{mccune2014} explained the chart by saying: ``[a]t the end of the day, \textit{money doesn't cross borders}. There is no international wire, just a series of domestic transactions. One bank ends up with more money in its correspondent account: the other bank ends up with less money in its correspondent account" (emphasis added). Their description relates to the above-mentioned reasioning about irrelevance of the usage of metaphors of motions while discussing international finance. There is a missing element in the disussions by \cite{mccune2014} and \cite{robinson2018}. They did not take into account the debt-credit nature of the correspondent banking relationships -- in other words, it is not enough to recognize (1) the existance of the correspondent banking relationships, and (2) that these accounts are being debited and credited depending which of the corresponding banks is the originator of the payment and which one is the receipient of the payment. If only (1) and (2) are considered, then the analysis is not very much different from the description of the system under the classical gold standard, where, to use \citeauthor{rueff1972}'s language, a money supply \textit{increase} in one location (jurisdiction) is compensated or offset with a money supply \textit{decrease} in another. A more thorough analysis of the international payment first of all recognizes (1) the money of account in which the payment is undertaken, and (2) then respective correspondent banking relationships that facilitate the transaction. This is discussed in detail in Chapter \ref{sec:payments} on page \pageref{sec:payments}.

\citeauthor{scott2022}, a former financial industry professional in the London financial center, has been writing prolifically about the payment industry and its spread on the back of new technologies. The following passage from his blog post is worth mentioning here. Hence, the advice on the proper analysis of the monetary system is to avoid the delusional imagination of money or capital as free-floating commodities, instead those are \acp{iou}:

\begin{quote}
The actual monetary system is a politically anchored multi-layered web of \acp{iou}, in different forms, that holds the so-called economic realm together. Analysing the separate elements of that foundation as if they were free-floating commodities subject to monetary cost considerations (that the foundation itself underpins) is delusional on multiple fronts. \citep{scott2024}
\end{quote}

More than that, his advice extends further. When conceptualizing the financial transactions our imagination must take into account this: ``The only thing that moves in a digital money system is messages \dots" \citep[p.~75]{scott2022}. ``Re-assing[ing]" is what a (commercial/central) bank does while obtaining a standardized payment order/instruction/message from its customer---the word used in \citep[p.~75]{scott2022}---as they both maintain a debt-credit relagtionship with each other, where the bank in on the debt side and customer is on the credit side of it. ``Digital money 'movement' is just promise-editing" \citep[p.~72]{scott2022}.

With continued financial frauds of grand scale such as with a recent case Wirecard and history of illicit flows, there has been tendency in the academia to treat the promise of \ac{cbdc} baseed on \ac{dlt} and empowered by \ac{ai} as sort of the cure from the traditional finance with its ever-present shortages of dealig with capital and financial flows. The latter proved to be difficult to trace, especially given the ever-present concept of illicit flows. With new technology of decentralized finance , there is a hope among people that aim to solve the present issue of poor tracability of money ``flows". Already now, they recognize once the shift towards the \ac{dlt}-based system is made then, it is beleived, the concept of money would change. So that, all the money in use would be stored as records in the ledgers -- hence, when we will transact the ledgers would be only modified or change, nothing else. ``Hence, what we understand as money does not 'flow', but is rather updated" \citep[p.~615]{westermeier2023}. 

This view expressed by \citeauthor{westermeier2023} is noteworthy given the technological development. It appears still containing the mainstream view on money with its conceptualisation of it as a standalone and scarse matter. This dissertation aims to show that money by its very nature, no matter of its forms and technological solutions, has a debt-credit relationship. Hence, as these relationships have been expressed as balance sheets of the transacting economic units, money transactions have been all the time about modifications in the balance sheets or ledgers. The issue of poor tracibility, which has been observed historically, is rather political than a natural to the pre-\ac{dlt} forms of money. 

Note how close in their thinking \cite{westermeier2023} and \cite{scott2022} are! They are not alone: yet in 1998 an official representing Federal Reserve Bank of New York wrote in his article for a legal journal that ``[m]oney is (carefully regulated) talk." \citep[p.~7]{sommer1998}.

The above-mentioned review of the literature points out at the requirement to reconsider the conceptualization of capital flows/mobility. The capital or money is not moving in spatial terms as a standalone physical (full-body) object would do.

Instead, capital or money is a balance sheet concept, see \citep{bell2001}. In other words it is ``representing debt-credit relationships" \citep[p.~17]{kregel1996}. And money of account does matter for considering the phenomena of cross-border financial transactions.

Summing it all up. There is a much more ambitious and radical critique of capital ``flows/mobility" as expressed by \cite{woodruff2005} and \cite{mosler2005,mosler2010,mosler2022,mosler2023}. What unites these two authors is that they strike at the core by refuting the movement metaphor altogether. Their argument is simple: using this metaphor is just erroneous and deeply misleading.\footnote{Writings by \citeauthor{scott2022} are of this category, too. Consider this description of finance: \begin{quote}Economists frequently use blood metaphors for money, seeing it as a substance of value that 'flows' around the economy. Financiers love this vision of money as a circulatory system, because it creates an image of their sector as the 'beating heart' of the global economy. \par Under this metaphor, money carries a payload, as if it were plasma transporting nutrients to cells. It's seen as a delivery system for value. This belief is reinforced by those who characterize it as a 'store of value', as if value resides 'in' the money. Many people in the finance and fintech sector talk about payments platforms as 'value transfer systems', as if PayPal, Stripe and the banking sector provided pathways for the 'store of value' to carry its payload from point A to point B. \par These liquid value metaphors are \textit{deeply misleading}, and obscure the true nature of finance. Money isn't like blood pumping through veins.~\citep[emphasis added]{scott2024_}\end{quote}} In other words, there is no such thing as money movement.
These authors never developed further their views into a coherent theoretical framework.
Their ideas advocated for an analysis based upon these two methods: (i) analysis of the institutional detail of the actual operations between different economic entities that engage in international trade and investing, and (ii) analysis of the balance sheets of the transacting counter-parties.\footnote{A recent work \citep{braun_kodd2022} in their edited book \citep{braun2022} made an attempt to follow this line. It, too, rejected ``capital flows" metaphor and conceptualized cross-border financial transactions as financial claims that constitute relations between their issuers and holders.   Also, this work recognized the importance to utilize ``the language of balance sheets" while inquiring into the phenomena of international financial transactions.  However, this work missed institutional detail of the very relations that exists between the balance sheets of the economic entities involved in a typical set of cross-border transactions.}

By questioning the very foundation, this is the most radical critic of the established concept of capital flows/mobility.\footnote{Being radical has been a characteristic feature of an analysis done in line with \ac{mmt} as remarked by \citep[p.~251]{pistor2022}. This dissertation proposal aims to qualify as an \ac{mmt} consistent.   Also, ``[i]f emerging markets are to achieve their objective of joining the ranks of industrialized, developed countries, they must use their economic and political influence to support \textit{radical} change in the international financial system" \citep[p.~235, emphasis added]{kregel2015}.}

A re-think of the phenomena of international financial transactions must be helpful as a policy advice for the very countries that are experiencing the external constraint, or that exposed and hence dependent on the availability of foreign capital. It must be a tool of engagement with mainstream economists on their next revision of the concept.\footnote{\cite{gopinath2017} provides an example of rethinking undertaken within the IMF.}

\subsection{Capital ``Flows" as a Metaphor}\label{sec:metaphor}

Indeed, the way we speak, the language we use and narratives we follow do matter. This is especially true, when we inquire into the vital economic matters. In the field of economics there are ongoing debates on the analytical aparathus, terminology and true meaning of the processes that actually take place. For example, \citeauthor{tymoigne2023} in their paper say it explicitly and firmly that they ``[reject] the terminology of ``deficit financing", ``deficit spending", or ``monetizing deficits" when applied to a consolidated government that is monetarily sovereign" \citep[p.~5]{tymoigne2023}. \citeauthor{fullwiler2022} do the same thing with resepect to the terminology of ``printing money" or ``monetization" by expalining there is no such thing these terms invite us to imagine, ``because it is not operationally possible" \citep[p.~401]{fullwiler2022}. 

This dissertation is building a similar argument with respect to capital ``flows/mobility". This quest appears inevitable. Since ``[t]he question of the cultural origins and authority of natural-scientific concepts leads us to the question of language and discourse, and one of its most intriguing elements: metaphor" \citep[p.~45]{walker2020}. 

\citep[p.~ 376, emphasis original]{gilbert1989}: ``\textit{the world is authored by humans through langauge}, and rationality is constracted from intersubjective ineractions. Claims to scientific validity depend on their social context and genesis. Though truth is linguistically dependent, members of scientific or other communities consider their own truths as absolute. \dots Denial of the linguistic, rhetorical nature of epistomology and ontology results in use of ``naive and immoral rhetoric" \cite[p.~88]{brown1987}."

There is a number of sources that emphasize on importance of language and metaphors such as \citep{brown1987,klamer1994,alborn1994,mirowski1994,lakoff1980}. 

Indeed, ``metaphors are pervasive in everyday life" \citep[p.~3]{lakoff1980}. It is thanks to the work by \citeauthor{lakoff1980}, people realize that ``[e]verything you understand is through framing \dots you cannot understand without metaphors -- you cannot think without stories" \citep[p.~290]{wray2015}. It is not only about our understanding of economics, it is general impact of metaphors or of metaphorical language on our knowledge. While streassing that our thinking process takes place through metaphors, the linguistic scholars point out that metaphors tend to highlight specific aspects of the concept, while hiding the other aspects about it \citep[p.~10]{lakoff1980}. 

The field of economics features the power of metaphors very much clearly. This is because, ``economics is metaphorical" \citep[p.~44]{klamer1994}. It is even more so with finance, where ``discourses about money and the world of finance seem to be teeming with metaphors" \citep[p.~49]{nunning2015}.\footnote{This paper was published as part of \cite{gil2015}. It extends its statement by saying: ``money and the world of finance belong to those domains that people commonly write and talk about in figurative language" \citep[p.~49]{nunning2015}.}

Major metaphors pervassive in the economic thought are centuries old and originate from attempt to explain economic phenomena in terms of the other observable developments. The very first visual of the economic processes and their logic was provided by Quesney's \textit{Tableau}. ``The idea of cicular movements in national economy is an old one which first came from the field of biological science. The idea was given great prominence in physiocratic thought, namely, in the circulation of the \textit{prouit net} presented in the \textit{tableau economique}." \citep[p.~160, emphasis original]{zweig1950}. Hence, one of the major metaphors in economics is \textit{circulation} or \textit{cicular movements}. Another detail of the orgions of this metaphor comes from \citeauthor{walker2020}:

\begin{quote}
Prior to the eighteenth century, the dominant choice of metaphor for the discussion of the national wealth was physiological. \dots Quesnay was a physician who, influenced by William Harvey's discovery of the circulation of the blood, believed that the 'health' of the national economy resolved to the unobstaructed circulation of a living value substance (denoted as \textit{bl{\'e}}, or wheat) derived ultimately from the fertility of the soil. Like blood, the life giving fundamental of agriculture was essential to the health of all other organs of the economy. \citep[p.~90, emphasis original]{walker2020}
\end{quote}

It turns out the very personal experiences of the early economic philosophers allowed them to bring up the metaphor that became an essential and authoritative way of thinking for the future generations. Being medical practitioner, Quesnay picked up discoveries in this field to refer them to them economic sphere. Other prominent writers from the eighteenth century had similar backgrounds as Quesnay's. In particular, one the most significant discoveries of the time was William Harvey's circulation of blood made in the seventeenth century \citep{weston2013}.

\begin{quote}
Economy is now seen to act in the world; it causes events, creates effects. Because the economy is not found as an empirical object among other worldly things, in order for it to be ``seen" by the human perceptual apparatus it has to undergo a process, crucial for science, of representational mapping. \dots  The French Physiocrat Francois Quesnay provided the first such map in 1758. \dots  It is significant that like many early political economists, Quesnay was trained as a physician. The circulation of wealth was to him the lifeblood of society. There was a medieval precedent for this metaphor. Even before William Harvey's seventeenth-century physiological theories, the description was common of money ``circulating" through the ``body politic." Thomas Hobbes spoke of money as blood. \dots Sir William Petty, John Locke, and Nicholas Barbon (author of \textit{A Discourse of Trade} [1690]) were all trained in medicine. Petty studied anatomy in Holland and later wrote The Political Anatomy of Ireland. Locke joined the household of the Earl of Shaftesbury as a physician. \citep[p.~440]{buck-morss1995}
\end{quote}

History scholar \citeauthor{alborn1994} points out that during the Victorian period the England society witnessed public discourse on the economic matters and on the currency matter in particular, which extensively utilized the  metaphor of circulation. In effect, this is the metaphor of motion. \citeauthor{alborn1994} calls, too, the ``circulatory language." Goods and currency (in the form of paper notes issued by private banks) were visible as they changed places and hands. During the 18th century the public health issues were at the center of debates. Hence, the circulatory language about economy was using the analogy of blood circulation within the body of a human being. 

Later on, during the nineteenth century in England active debates on key economic questions took place against the backgorund of new discouveries in the natual siencies such as biology and physics. The economic thought continued searching for morem sophisticated narratives. \cite[p.~10, emphasis added]{mirowski1994} observed: ``In appropriating the formalisms of mid-ninteenth-century energy physics and adapting them to the language of utility and prices, the progenitors and their epigones adopted a ceratin worldview, one that had to stress the \textit{extreme near identity} of physics and economics."
 
During the 19th century episode, when industrial revolution took place, the circulatiry language used the analogy of ``steam circulated through an engine" \citep[p.~176]{alborn1994}. \citeauthor{alborn1994} points out that during the 19th century it became obvious that commerical transactions took place thanks to ``\textit{non}circulating checks and other bookkeeping expedients" \citep[p.~177, emphasis original]{alborn1994}. In particular,  Walter Bagehot introduced new body analody, while writing on the money market of the City of London: 

\begin{quote}
\dots the nervous system had replaced the circulation of blood as the central metaphor to describe the money market. \dots Bagehot found a more useful physical analogy in the nervous system, with its ``connective tissue" and \textit{instant} electronic messages from the brain to all parts of the body. \citep[p.~190, emphasis added]{alborn1994}
\end{quote}

It is interesting to note that \citeauthor{copeland1952}, while writing in the middle of 20th century, himself objected the circualtory language adopted by comteprorary economists. Back then, the metaphor of motion relied on the analogy from hydro mechanics, where money funds or purchasing power were flowing via the imagined system of pipes. His description of the matter coinsides with Bagehot's in one detail that financial transactions are carried out in a \textit{instant} manner: 

\begin{quote}
It has been customary, at least since Herbert Spencer, to portray the money circuit in terms of an hydraulic analogy. \dots [T]he hydraulic analogy has fostered a number of \textit{serious misconceptions} concerning the nature of the money circuit, and will suggest comparing that circuit with an electrical circuit instead. The electrical analogy is by no means complete, but it is essentially as complete as the hydraulic analogy and it avoids the hydraulic misconceptions. In this electrical analogy the reservoirs become batteries and the conduits become wires. A major advantage of an electrical over an hydraulic analogy is that the velocity of an electric current is so great that for most practical purposes one can assume transmission to be \textit{instantaneous}. There are some technical difficulties in regarding transmission in the money circuit as instantaneous\dots But on, the whole, instantaneous transmission is a far better assumption than a significant transmission lag. \citep[pp.~29-30]{copeland1952}
\end{quote}

In this passage \citeauthor{copeland1952} speaks of the misconception that arise from the usage of the hydrolic metaphor of circulation. Here, he mentions the notion of ``transmission lag" that comes alongside the latter. Instead, the proper notion is about instantenuous transactions. It assertion is valid especially today. Just consider what advancement in connectivity, communications and computetional power took place if compared to the 1950s, when \citeauthor{copeland1952} observed there were ``some technical difficulties" in carrying out transactions through the monetary system.

Despite the above-mentioned attempt by \citeauthor{copeland1952} to change the metaphorical thought about the functioning of the money system during the 1950s, since then academic economists and practicioners from the financial industry just kept relying on their preferred way of imagining it through hydrolic circulation metaphor.  The following passage by former employee of the private financial institution from the City of London suggests an answer to the reasonable question why does it happen this way:

\begin{quote}
It is common for economists to use blood metaphors when speaking about money, characterising it as a substance of value that 'flows' around the economy. Financiers love this metaphor because it presents their sector as the 'beating heart' of the global economy. Bit the use of this circulartory system metaphor \textit{prevents} us from seeing the true nature of finance. \citep[p.~20]{scott2022}
\end{quote}

Indeed, the hydrolic and blood metaphors remain the most popular among the key speakers on the subject of money system, how it opeartes domestically and cross-border. For example, Mark Carney, former governor of Bank of England, used the metaphor of hydraulic pipes during the 2019 \ac{imf} panel on capital ``flows": ``We need to think of the [international financial] system as a whole, where does that capital flow through. Think of the system in terms of pipes" \citep{imf2019}.\footnote{This way of thinking is well established among the mainstream economists. Thus, \citeauthor{duflo2017} in her 2017 paper published in \textit{American Economic Review} called economsists to apply ``plumber's mindset" to their engagement with real-world policy-making projects. Such a mindset is said has a trait of paying a painstaking attention to details. Yet, in the paper's section titled ``Designing the Taps, Laying Down the Pipes: Two Examples" the author wrote: ``Although I will often talk about details as a general way to describe the handiwork of plumbers, there are two different kinds of plumbing in policy design. First is the "design of the tap" work: taking care of apparently irrelevant details, such as the way the policy is communicated or the default options offered to customers. Second is "layout of the pipes" work: important logistical decisions that are fundamental to the policy's functioning but often treated as purely mechanical, such as \textit{the way money flows from point A to point B}, or which government worker has sign-off on what decisions."~\citep[p.~5, emphasis added]{duflo2017}}

Some academics turned to more innovative metaphors in attempt to outperform the established way of thinking. Thus, \cite{clark2005} proposed a metaphor of ``money flows like mercury," which is ``designed to explicate the spatial and temporal logic of global capital flows." A more extended rationale behind this proposal is in the following: 

\begin{quote}
Characteristically, mercury tends to (1) run together at speed, (2) form in pools, (3) re-form in pools if disturbed, (4) follow the rivulets and channels of any surface however smooth it may appear to be, and (5) is poisonous in small and large doses if poorly managed. I would contend that global finance has similar characteristics, even if we should take care not to exaggerate commonalities between mercury and money. After all, the value of a metaphor is its suggestiveness rather than the precise relationship between the allusion and its object. Indeed, a metaphor works because of the apparent tensions between the allusion and its reference point. Put slightly differently, whereas a metaphor can suggest ways of conceptualizing social issues and institutions, they are neither sufficient in specifying the characteristics of the metaphorical object nor do they stand as adequate representations of underlying economic and social processes. Metaphors are instruments of inspiration, and they are instruments of communication, as any participant in related industry conferences will immediately recognize. \citep[p.~105]{clark2005}
\end{quote}

The post-\ac{gfc} vocabulary in use by the officials from the Federal Reserve System featured the blood metaphor by custom. Its chairman wasinvoking by saying "[c]redit is lifeblood of market economy" \citep{bernanke2009}.\footnote{In 2009, U.S. President Barack Obama used similar expression in his public speech: ``the flow of credit is the lifeblood of our economy" \citep{weston2013}.} Yet in 2023, Agustin Carstens, head of \ac{bis}, utilized the blood metaphor: ``We need to provide the heart of the system. We can think [of it] as a physical heart that pumps the blood to the system. The blood being the money" \citep{bis2023}.  In the Covid-19 pandemic experimenting with metapohors of motion on money continued. For example, \cite{mcdowell} adopted ``a heating, ventilation, and air conditioning (HVAC) system as an analogy," while describing the cross-border payment system. With a promise of new techonolgy such as \ac{dlt} and future launch of \ac{cbdc}, \ac{bis} turned the botanic metaphor of the money system as a \textit{tree} with a trunk and canopy, where trunk represents the central bank money and the canopy symbolizing the private sector money. Such an approach has been picked up and popularized by academic writers from different fields of science such as \citep{swartz2023}. 

\begin{quote}
When it comes to \ac{cbdc}, the \ac{bis} has been curiously botanical. In the following sections, we discuss three key metaphors it has used in its \ac{cbdc} imaginary that describe 
the economy at different levels: money as flower petals, the central bank as a tree trunk and the global economy as a forest canopy.\dots We argue that the \ac{bis}'s use of botanical metaphors helps to naturalize \ac{cbdc}. The metaphors do not address the ambivalences and sociopolitical implications of tech nological change. We want to be clear that we do not take these metaphors as emblematic of a comprehensive or coherent theory of money or even \ac{cbdc} as advocated by the \ac{bis}. However, metaphors do more than visualize a particular concept. They contain more than they show, or readers can make more of them. They contain more meaning than the creators perhaps intended. \dots The \ac{bis} money flower metaphor is like a dried and pressed flower preserved in a herbarium. It allows us to study money scientifically and create taxonomies and new cultivars. The stamen of innovative cryptocurrency pollinates the pistil of trusted central bank money. The flower bears fruit, and its seed sprouts new hybrid plants with more resilient and competitive blossoms in the changing economic climate. Money no longer seems natural but instead resembles agriculture. \citep[pp.~13-14]{swartz2023}
\end{quote}

Lastly, Among the insiders of the payments industry there is a tendency to depict it as a system of \textit{rails} \citep[see][]{stanescu2023}. While financial maket professionals and academics writing on its matters consider it as a \textit{plumbing} system and themselves as \textit{plumbers} \citep[see][]{davidson2000,norman2007}. 

Indeed, over a lengthy while there has been no shortage of innovative approaches by academics and professionals to shaping the general public imagination and habitual way of thinking about the financial system and especially its international part. From blood circulation as in human body to hydrolic-based images that rely on pipes and plumbers and from nervous system to a botanic image of trees with outgrowth of conapies. What unites these imagined constracts or metaphorical way of thinking is that come from older and more fundamental metaphor of circulation. 

\citeauthor{klamer1994} observe that ``economics is metaphorical" and when scolars ``begin talking about metaphor, science moves" \citep[p.~44]{klamer1994}. 

Since economics is based upon metaphorical discourse, \citeauthor{klamer1994} distonuish three basic types of metaphors used by economists: (1) pedagogical, (2) heuristic, and (3) constitutive. 

1. Pedagogical metapohors ``simply serve to illumintae and clarify an exposition." Their effectiveness is by providing the ``mental images with which the audience [of students] can visualize an otherwise complicated concept" and ``good teachers are equipped with numerious such metaphors" \citep[p.~31]{klamer1994}.

2. Heuristic metaphors are more influencial than pedgogical as they catalyze the thinking of the audience by ``helping to approach the penomena in a novel way." In other words, these are ``thought-propelling metaphors" \citep[p.~32]{klamer1994}.

3. Constitutive metaphors are more special than previous two. This is because they ``work on an even more fundamental level" as they serve as ``conceptual schemes through which we interprete a world that is \dots unknown" \citep[p.~39]{klamer1994}.

This said, circulation falls into the group of constitutive metaphors. Indeed, it works on that fundametal level of the established way of thinking. Where as the metafor of pipes through which the capital is flowing might be considered as qualifying to be as pedagogical or heuristic metaphor.  In the former (pedagogical) case one might say that thinking or imagining through such a metaphor illuminates us and reduces our ignorance due inexperience with this matter. In the latter (heuristic) case one might argue that such a metaphor propells our thought into some stage of comrehension of the matter. However, there is a weakness in the typology developed by \citeauthor{klamer1994}. It is about not allowing for a metaphor to conceal. This characteristoc of a metaphor explicitly is found in the work \cite{lakoff1980}. They call it highlighting and hiding and generally applied to any metaphor. Another apsect of metaphors, which \cite{lakoff1980} are explicitly talking about, is their experiential and physical bases. For example, in the metaphor ``understanding is grasping"  \citeauthor {lakoff1980} explain there is a refrence to our regular exeprience of taking a physical object by hand (grasping it) and taking a careful look at it. With economic activities there is a substantial part that is not observable. The early writers on the economic matter where obviously puzzled by the very unobservable part of the economic matters. Their struggle over grasping it was resolved through references to the observables that were substantial breakthrough in knowledge of that time. Hence, the eighteenth century writers with their own professional experience in medical practice or who were influenced by new discoveries in medicine (see above) came up with a metaphor of circulation for general economic matters and for money it is being modified into circulation like blood. Since then, these metaphors have survived till now. However, the above section \ref{sec:rad_critique}, pp. \pageref{sec:rad_critique}-\pageref{sec:metaphor}, shows that there has been long tradition to question this metaphor's ability to educate and illuminate our understanding. Using \cite{lakoff1980}, this metaphor did poor job in terms of highlighting the major aspects of the matter, instead it has been rather good job in terms of hiding those very basic aspects.

Among the contemprory scholars of lingustrics and anthropoligy, and particularly those working on the intersection of culture and money relationship, there has been a recognition of explicit connection between metaphorical language and political purposeness:

\begin{quote}
\dots metaphors of money and financial crises serve to narrativize and naturalize complex cultural, economic and political transformations, projecting \textit{ideologically charged} plots onto the developments they purport merely to represent or to illustrate. In doing so, they arguably do creative work in serving to define how the cultural transformations associated with the current economic and financial problems are understood by contemporaries, familiarizing people with complex processes that are largely beyond their ken or understanding. Generating a whole network  of ideological implications and normative entailments, the metaphoric mappings involved in the metaphor of 'financial crisis' also play ``acentral role in the construction of social and political reality" (Lakoff/Johnson 1980:159). Since metaphors have ``the power to define reality" (Ibid., 157), they even constitute a license for policy change and political and economic action (cf. Ibid.,156). \citep[p.~63-64, emphasis added]{nunning2015}
\end{quote}

As the section on the international financial advising shows the use of language and the metaphoric mapping was indeed in the toolkit of the ``money doctors," some of them, while being in the country they advised, did not shy from shaping domestic climate of opinions (see Section \ref{sec:money_docs}, pp.\pageref{sec:money_docs}-\pageref{sec:mon_orthodoxy}).

So far, the apparent longivity of the circulation metaphor and its sister metaphors such as capital ``flows" made the following assirtation seemed right: ``[t]he metaphors we live by may turn out to be the metaphors we die by" \citep[p.~47]{walker2020}. This said, it must be recalled that ``\textit{language is class-based}, and it contains the value-orientations of the social position of the leaders" \citep[see][p.~ 375, review of \citep{brown1987}, emphasis original]{gilbert1989}. In the end, the guiding principle of this dissertation work has been found in the assirtation of another (higher) quality, which is ``[when scolars] begin talking about metaphor, science moves" \citep[p.~44]{klamer1994}. 

\cite{klamer1992} argues that accounting used to be a muster metaphor in economics, which was abandoned by economists as of late. According to the metaphor typology of \citeauthor{klamer1994}, accounting as a metaphor qualifies to the level of the constitutive metaphors. Such a metaphor has encourage the way of thinking that would be aware of "an irrelevant accountancy" \citep[p.~186]{keynes1933} of neoclassical type and leave that aside. This key underlying argument of this dissertation work is that accounting catalyze the thinking of capital as an economic phenomena beyond the current convention of ``flows/mobility".

\subsection{What is Capital}\label{sec:whats_capital}

Let us start with the term ``capital", while leaving the word ``flows" aside for a while. Under capital it is understood as financial capital and not the capital goods, directly. However, it is recognized there is a relationship between them.

This approach is in line with ``Keynes' views on the dominance of money over the real sector in a capitalist economy" \citep{kregel1986a}. And Keynes was not alone by adopting this way of thinking. \cite{wray2009} discusses evolution of capitalism from the finance capitalism stage to the money manager capitalism stage. In all of these stages the above-mentioned dominance of money over the real sector is in display.   

A number of contemporary thinkers shared the views held by Keynes. Take for example Alfred Mitchell Innes and Micha{\l} Kalecki. Note that this dissertation work has dedicated chapter on the intelectual legacy of Alfred Mitchell Innes (see Chapter \ref{sec:innes} on page \pageref{sec:innes}). Thus, he held the view that ``[e]very banker and every commercial man knows that there is \textit{only} one kind of capital, and that is \textit{money}. Every commercial and financial transaction is based on the truth of this proposition, every balance sheet is made out in accordance with this well-established fact" \citep[p.~151, emphasis added]{innes1914}. According to \citeauthor{toporowski2020}, Micha{\l} Kalecki held a similar view: ``By 'capital' Kalecki clearly meant \textit{money} capital, rather than titles of ownership of valuable productive equipment." \citep[p.~139, emphasis original]{toporowski2020}.

Then, with dominance of finance capital over the real sector's capital goods comes realization that there is a relationship between the two. Veblen, Keynes and Minsky traced it. In the Veblen's description capital goods are ``the aggregate of the material items engaged in industrial output" \citep[p.~811]{wray2009}. In the descriptions by Keynes and Minsky these are capital goods (assets) which produce output. By the selling the output for money and adjusting them for costs of the produced output, the owners are to obtain profits that are called prospective yields \citep{keynes1936}. In the Veblen's terms capital assets have ``presumptive earning capacity" \citep[p.~811]{wray2009}. In other words, capital assets are expected to bring a stream of future earnings, which are also called quasi rents \citep{keynes1936}. 

\begin{quote}
If it is true that capital assets are valuable because they yield profits, it follows that the market price of a capital asset depends upon current expectations of future profits and the way expected profits are transformed are transformed into future value. \dots From a Wall Street perspective, if a collaction of capital assets is not capable of generating revenues for current or prospective owners, it has no value. \citep[pp.~227-228]{minsky1986}
\end{quote}

It is worth noting that for Veblen the very present value of the capitalized presumptive earning capacity is capital, which is, however, subject to flactuation and manipulation \citep[p.~811]{wray2009}.

Minsky expaunds on Keynes and explains the relationship between finance capital and capital goods (or real assets):

\begin{quote}
There is a multitude of real assets in the world which constitute our capital wealth --- buildings, stocks of commodities; goods in the course of manufacture and of transport and so forth. The nominal owners of these assets, however, have not infrequently borrowed \textit{money} [Keynes's emphasis] in order to become possessed of them. To a corresponding extent the actual owners of wealth have claims, not on real assets, but on money. A considerable part of this ``financing" takes place through the banking system, which imposes its guarantee between its depositors who lend it money and its borrowing customers to who it loans money with which to finance the purchase of real assets. The interposition of this veil of money between the real assets and the wealth owner is a specially marked characteristic of the modern world. \citep[p.~115, quoting Keynes ]{minsky1975} 
\end{quote}

In his 1986 \textit{Stabilizing and Unstable Economy} book Minsky further expanded on this relationship, which is organized and maintained primarily by bankers. And under bankers Minsky understood broad range of private and for-profit firms, which include commercial and investment banks as well as miscellaneous money managers \citep[p.~249]{minsky1986}. If bankers emphasize on the collateral or expected present value of assets, on the back of which the financing is provided, then such a practice is conducive to the emergence to a fragile financial structure (relationship). A robust financial structure (relationship) arises, when bankers emphasize on what Minsky termed as ``cash flows" of thier counterparties.       

Now, to conclude this section on the broad meaning of the term international capital it is time to mention ``international." As it was just discussed above, capital is understood financial capital that maintains via the bankers a relationship with capital and other assets within a domestic economy. Meanwhile, international financial capital does the same thing with only difference that it acts globally, i.e. within a number of economies, jurisdictions.

Essentially, this term is describing an institutional design for a broad range of jurisdictions, within which the banking industry broadly defined---banks and money managers of resident and non-resident types---are transacting across borders. These entities earned the name of \textit{international investors}. By design, they are free to buy and sell real or financial assets for the sake of profitable opportunities.

