\newpage

\section{\MakeUppercase{Cross-Border Payments in the Real World}}\label{sec:payments}

\subsection*{Abstract}

\subsection{Introduction}

\subsection{aaa}

On \ac{gfn} see \citep{wojcik2018}.

\cite{usdot2009} on the \ac{usd} status as reserve currency with key factors explaining it. 

\begin{quote}
In 2010, the fastest way to move money on the same day from New York to 
London was to catch a flight from JFK to Heathrow and deliver it yourself. Now, 
you can initiate a secure, real-time payment that's sent and received into your 
account in seconds at virtually no cost and in any currency. Not just from New 
York to London, but from New York to anywhere - including space. The last 
decade for payments has been extraordinary. However, payments are on the 
cusp of an even greater revolution. \citep{jpmorgan2021}
\end{quote}

On the issue of 'dollarization'\index{Dollarization}\index{Dollarization!of 1990s} see \citep{alesina2001}\index{Dollarization!in Latin America},\citep{jameson2003},\citep{havrylyshyn2003},\citep{johnson2008}\index{Dollarization!in Russia}.

\begin{quote}
 Residents in developing economies save and borrow in foreign currencies. At the end
 of 2000 the share of domestic bank deposits denominated in foreign currencies was
 35\% on average in all developing economies, and 44\% in those among them where
 dollar deposits are not illigal. \citep[p.~63]{ely2006}
\end{quote}

\citeauthor{clark2024} considers modern-day geopolitics between US and China: "I think China is accepting short term econmic pain to help conjure a non-US dollar financial system into existanc." He also aludes to the case of sanctions imposed on Russia for its military envasion of Ukraine:

\begin{quote}
The issue is that like Japan in 1980s, even if China suddenly became democratic, it is hard to see how Americans could ever accept China having a larger economy. If you look at US-Japanese relations in the 1980s it was all about how to restrain Japanese economic might, even with Japan being democratic and staunch US ally. Given the history of China relations with the West, relative economic decline is not political acceptable in China. This leaves China with only one option, to replace the US financial system with it own system. Strangely, China seems to me to be making enormous progress in this aim. \par
First of all, with Russia's invasion of Ukraine, the West looked to isolate and destroy the Russian economy. This policy has signally failed, with Russian exports of energy to the rest of the world remaining stable. This trade is largely conducted outside of the US dollar financial system, not only for energy products, but for the shipping of these assets too. Russia exports 10\% of total oil exports, all of which now needs to be outside of the US financial system. The recipients of this oil, must also have the financial system in place to handle this trade, meaning that the non-US dollar financial infrastructure is now in place in major nations such as India, China and Brazil to name a few. \citep{clark2024}
\end{quote}

In the words of the manager of the Canada Payments, the operator of the national pyament system in Canada: "In the payment industry, we know that when payments work how they're supposed to, they go \textit{unnoticed}. This means we're doing our job right. It also means that many people \textit{don't know} about the critical payment systems that underpin our economy." \citep[emphasis added]{larocque2024}

\begin{quote}
There is no such a thing as "capital flight". Transactions are voluntary: if one is carried out
 it is because everyone is happy with the terms of trade (prices reflect indifference levels).   In other words, for someone to buy (say, the one "flying capital") there must be someone
 selling. Of course, once the government has an exchange rate target it must stand ready to
 buy/sell the quantity of foreign exchange the market desires (and if dollar reserves are going
 out, we may call that "capital flight"). \citep[p.~374, footnote 8]{mario2021}
\end{quote}

On the postwar \ac{epu}\index{European Payments Union} see \citep{kaplan1989} as it details the EPU operations by utilizing the termilogy and logic of the \textit{offsetting} balance sheet technique. 

\uppercase{Important!} On \ac{fps} see \citep{bis2024}.

See \citep{hemmer2008} on \ac{hc} and \ac{mm} accounting issues that arose on the back of \ac{gfc}.

\citep[p.~820]{wray2009}: "Rather than 'mark to market', prices
 were actually set by 'mark to model'; later, this was called 'marking to myth' as the models
 broke down."
 
\citep{fed2024} on the book-entry securities operations.

\begin{quote}
At the core of current international monetary arrangements lies \textit{an
 interlocking network of national and international payments systems} which
 facilitate the exchange of funds associated with almost all international
 trade and financial transactions. \dots At first glance, the payments side of a transaction appears to be straightforward--the payor instructs his bank to transfer an agreed upon
 amount of funds to the bank of the payee. \dots there is the problem of how best to
 achieve a simultaneous exchange of goods or securities for money so as to
 minimize the risk that one party to the transaction could renege on his
 payments or his delivery obligation, while the other party is fulfilling
 his side of the transaction. This risk can be minimized by closely tying
 the delivery of securities or goods to the receipt of good funds by the
 seller. \textit{The payments system is, therefore, closely linked to the system
 of transferring securities} and legal arrangements for securing title to
 goods. \citep[emphasis added]{folkerts1990}
\end{quote}

\begin{quote}
Money is (carefully regulated) talk. \citep[p.~7]{sommer1998} \dots Although this Article's definition of bank liabilities is standard, its definition of payment is not. "Payment"
 is clearly-and wrongly-treated in law as something requiring the
 tender of currency, or at least the satisfaction of a right to receive
 currency.2" This law is clearly inconsistent with market practice. The
 honor of a check seldom involves dollar bills; commercial wire trans
fers almost never do. This Article takes a different approach, one more consistent with
 practice. For our purpose, "payment" is the result of extinguishing a
 bank liability.\citep[p.~10]{sommer1998}
 A bank payment, therefore, is a shifting of bank liabilities
 through a highly stereotyped communications process. Indeed, we
 can view most of the commercial law of checks or wire transfers as
 regulation of the speed, reliability, and integrity of this process.
 The purpose of the payment system is admittedly not the shifting
 and extinction of bank liabilities, but rather discharging money-de
nominated obligations among end-parties. But the payment system
 functions by the extinction and novation of bank liabilities. Most pay
ment law takes the purpose of payment systems for granted, concen
trating instead on the communications that extinguish and novate
 bank liabilities. \dots
 This Article has defined a payment as the result of extinguishing a bank liability. This
 definition has led to a complementary relationship between payments
 and money. Payments are the mechanics of money; money is the pur
pose of payments. Money is what payments do. Understanding money
 requires understanding the corresponding payment system. But the
 converse is also true: Payment systems cannot be understood without
 understanding money. \citep[p.~11-12]{sommer1998} \dots
  money and payment systems are inextricably interrelated:
 social teleology and instrumental functionality. Neither makes sense
 without the other. Because money cannot be viewed independently of
 the payment system that implements it, each payment system is its own
 money. This implies that checks are different money than coins,
 although one employing the same unit of account. If this statement
 strains intuition, try buying a house with coins, or finding a soda
 machine that accepts checks. In other words, money \textit{is} what a payment system \textit{does}. \citep[p.~18]{sommer1998}
 As long as the underlying system operates reliably, and
 the rules for external use of the system are separable from the internal
 operating rules, a user of the payment system does not have to know
 the operating details of the system at all.  \citep[p.~21]{sommer1998}
\end{quote}

\citep[p.~46]{shen1997}: "Fedwire is used mainly for interbank payments,
 many of which are related to federal funds transactions or payments for the purchases of government securities."
 
See as well \citep{cpss1995,cpmi2022,fed2023} on the complex of payment system and other elements of the \ac{fmi}. On Citi as global castodian see \citep{gcastodian2022,gcastodian2023}.

See \citep{boz2022} for empirical work on the composition of international trade by money of account (called "currency" in this paper).

"SWIFT is a network that enables financial institutions to send and receive messages on financial transactions to and from one another in a secure and harmonised manner. Yet SWIFT neither clears or settles payments nor is it directly involved in the transfer of funds. Instead, SWIFT messages amount to \textit{payment orders} settled via correspondent accounts that banks hold with each other." \citep[p.~4, emphasis added]{boz2022}. See also \citep{robinson2018,robinson2023,robinson2024}. 

\cite{robinson2018}: "Often characterized as the 'plumbing' of the global financial system, and perceived as "boring, low margin and not strategic" (Lord et al. 2015, 6), it is poorly understood beyond its well-known intermediary 'facilitating' role. A fundamental function of FMI is the transfer of (exchange) value, more simply known as payment.".

\cite{maurer2012}: "\dots recent attention to the ghosts in the financial machine neglects the infrastructures of payment that make finance possible."

\begin{quote}
While the established system enables the flow 
of money via intermediaries such as SWIFT and corre
spondent banks, these intermediaries are not given in the 
new monetary system, which has money as its basis and 
enables interaction via APIs. \citep[p.~15]{swartz2023}
\end{quote}

On \ac{ifrs}: "IFRS was created in the Anglo-Saxon accounting
style with a \textit{capital market orientation}, instead of a tax orientation as is common in Continental European accounting (Zeff, 2012)." \citep[p.~6, emphasis added]{trimble2024}, which was published as a chapter in \cite{amat2024}.

It is worth to note that \ac{bea}, the state-run agency that has been responsible for the collection and publication of the balanace of payments records of the United States since 1920s, has dropped the minus sign from presentation of the debit items, see \cite{bogen2014}.   

Also, \ac{bea} adheres to the consistency between \textit{positions} and flows \citep[p.~41]{bea2022}, which contrasts a bit with stock-flow consistency analysis of Godley. This work proposes to abandon the terms stemming from the metalist view of money, which invoke the thinking of money as things, which include both stocks and flows. Instead, positions and changes in them are to be disegaged from utilizing metaphors of motion. Let us use financial positions instead of "stocks" and change in the financial positions instead of "flows".

The technique of the New York Clearing House by settling mutual indebtedness in 19th century is descirbed in \citep[pp.~292-342]{gibbons1858}. \citep{kregel2023,kregel2024_}

\citeauthor{roover1948} describes inter-bank transfer between two clients of two separate banks \citep[see][pp.~272-275]{roover1948}. This description at the very basic principles of the transfer procedure is similar to the description of the workings of the \textit{Hawala}\index{Hawala} system, see \citep{ballard2005} and \citep{qorchi2002}.

See \citep{schenk2023} on payments today. On the mainstream literature on correspondent banking \citep{schenk2024}, \citep{panza2019}, \citep{cipriani2023}.

\citeauthor{schenk2023} considers correspondent banking as practice that emerged in full extent during the first wave of globalization, which took place in the second half of the 19th century. Technological innovations in communications sustained its usage since then. Overall, according to the author, the history of correspondent bankings accounts for about 150 years now. Prior history of this practice is downlayed since communication technologies might be considered now as rudimentary if compared to subsequent technological breakthroughs. First, it was "the spread of undersea and coastal telegraph in the second half of the nineteenth century" \citep[p.~281]{schenk2023}. Then, invention of telex machines in the 1920s, the spread of the computers in the 1970s and the internet since the 1990s all magnified the trend towards greater globalization of trade and finance. As of today, rise of new digital technologies---such as \ac{dlt} and \ac{ai} respectively since late 2010s and early 2020s---are yet to be tested if they are capable to replace the correspondent banking mechanism \citep[p.~300]{schenk2023}.

The following description of the cross-border payment is made by \citeauthor{cipriani2023}, a team of economists from Federal Reserve Bank of New York. It is quite simplified, indeed, despite the usage of financial jargon. Nevetheless, its major shortcomings are of totally other nature. These must be highlighted and corrected. Let us first consider the description first, which reads:

\begin{quote}
 [C]ross-border payments occur through 
``correspondent banking": banks use the services of ``correspondent banks" in order 
to execute cross-border payments \dots A "correspondent account" is an account that a ``respondent bank" has at a foreign correspondent bank usually in the foreign bank's currency. Both banks will keep a record of this account, and common terminology here is to refer to \textit{Nostro} and \textit{Vostro} accounts, where the terms are the Italian words for ours and yours. The record kept by the respondent bank of the money that it keeps with its correspondent bank is the \textit{Nostro} account, whereas the record kept by the correspondent bank of its respondent bank's money is the \textit{Vostro} account. \dots
[I]f X wants to send money to Y, who resides in another country, X's bank will instruct its correspondent bank in the country where Y resides to send money to Y's bank through the domestic payment system of Y's country. The \textit{Nostro} account of X's bank (an asset on the bank's balance sheet) will be debited for the amount paid; similarly, the \textit{Vostro} account (a liability on the correspondent bank's balance sheet) will also be debited. \citep[pp.~39-40, emphasis original]{cipriani2023}
\end{quote}

Let us now pay attention to the above-mentioned shortcomings. There are two. First, it invokes a metaphor of motion that creates an impression of the payment system being alike to the transportation system. This is due to the usage of such terms as \textit{send} and \textit{flow} with respect to money. Secondly, it uses the term \textit{debit} in quite imprecise way so that an inexperienced reader would not get it what is exactly is going on within the payment system. 

Describing the banking industry in New Orleans during an antebellum era, \citeauthor{green1972} depicts local banks having a large network of correspondent banking relationships. He notes that a typical private bank once launched was prioritizing of establishing a correspondent relationship with another bank, see below.    

\begin{quote}
Because of New Orleans' involvement in national and world trade, nearly every Louisiana bank maintained a correspondent relationship with some bank in New York, Philadelphia, or Boston, and with some European city (usually London, but often Liverpool, Paris, or Amsterdam). For example, the Commercial Bank, which concentrated on the Western trade and earned its profits largely from its wide correspondent ties, maintained accounts with 71 banks and had "active correspondence" with 40 of these. One of the \textit{first concerns} of a newly opened bank was to establish correspondent ties. \citep[pp.~72-73, emphasis added]{green1972}
\end{quote}

Another illustrartive example of primacy of correspondent banking is by \cite{adams1978}. The author provides an account of the operations of the private, unchartered bank based in Philadelphia during the first half of the 19th century. The bank was established in 1812 by Stephen Gerard, a Frenchman born in Bordeaux who active in mechant trade between commercial centers of Europe and North America over second half of the 18th century. He settled in Philadephia during the American Revolutionary War, being already a successful merchant with personal wealth embodied in "ships, cargoes, and funds spread all over the European continent \dots in Amsterdam, Hamburg, Saint Petersburg, Copenhagen, Riga, London, and other 
trading centers" \citep[p.~8]{adams1978}. He kept active correspondence with banks, with which his funds were held, and one of the major bankers he was communicating with were "Messrs. Baring Brothers \& Co. of London" (ibid, p. 9). In early 1812, Gerard commited himself to launch and run a private bank in Philadelphia. Already in May of same year he wrote another letter to the Baring Brothers,\index{Baring!Brothers \& Co.} in which "he inquired about a possible correspondent relationship" with the London-based bank (ibid, p. 23). He got a positive responce from them. Overall, by June 1982 the Girard's bank established correspondent ties with banks in six major cities in the US and with two banks in London (ibid, p. 25). This example shows that establishing correspondent relationships was a matter of top priority for a banker.

The overseas trade by the eighteenth and early nineteenth centuries major merchant houses is also discussed in \cite{marichal2007}. It provides an example of a European merchant with experience of being a correspondent to the London's merchant house of Baring Brothers (ibid, p. 177). 

Another illustrative example can be found in \cite{joslin1963}, which provides a historical account of establishment and operations of the British banks in the nineteenth century's Latin America. The following quote demonstrate the built up of the operational capacities of the London \& River Plate Bank, which was incorporated in London in 1862 with prime purpose of expanding the British commercial activities in that continent:

\begin{quote}
Before the branch [in Bueno Aires] was opened and during the succeeding years Head Office [in London] was active in building a network of correspondents, so that the bank could offer a wide range of services to those who wished to make payments or remit funds to the River Plate. In London the City Bank became its banker and a discount and drawing account was opened at the Bank of England. In Paris its agents were Messrs. Bischoffshcim and Goldschmidt, and other correspondents were soon appointed at Hamburg, Amsterdam, Berlin, and Genoa. Across the Atlantic a correspondent in New York was necessary to handle business between the United States and the River Plate. For Brazil the bank chose the London and Brazilian Bank as its original correspondent, though die agency was transferred to the Brazilian and Portuguese Bank when die London and Brazilian Bank proposed to operate in the River Plate. In the River Plate a furdier net� work of correspondents and agents was needed for the bank to carry out the collection of bills and extend the circuladon of its notes, which the agents were to honour. There were further additions as the years passed. By 1871 diere were more correspondents in France, the Low Countries, Spain, Portugal, and Italy, and an agent at St. Petersburg [capital of Russia Empire]. Careful arrangements were made with the correspondents, specifying limits within which they might draw on the bank or be drawn upon. \citep[pp.~29-30]{joslin1963}
\end{quote}

Similar arrangements preceeded for the launch of the London \& Brazilian Bank, the Brazilian \& Portuguise Bank, the London Bank of Mexico \& South America and many other financial establishemnts aimed at that continent.  Further, a wide and equally efficient network of correspondents is vital for the banking business acting as a clearing hours for debts and credits. Thus, from the experience of the ninteenth century banking in Brazil: ``To gain the maximum advantage in the exchange market, a bank needed branches or correspondents at many points along the coast. Only in this way could drawings easily be covered by remittances, since at Rio Grande do Sul and at Sao Paulo exports greatly exceeded imports, while the converse held true at Pernambuco and Salvador do Bahia" \citep[p.~64]{joslin1963}.



\citep{bernstein1984}: "At a dinner he gave to Walter Gardner
 and me in 1943, Keynes said that the United States would be glad that the
 new institution would allow control of capital flows, as there could be a flight
 from the dollar after the war. I asked whether that was because there might
 be a depression in the United States and a fear of a possible devaluation of
 the dollar. Keynes said that was what he meant. After Bretton Woods,
 however, Keynes came to believe that the dollar payments problem would be
 solved by expansion in the United States."
 
\citep{coombs1976} on the swap lines in mid-1960s. 

\begin{quote}
If money moves from Italy to Switzerland, the
Bank of England and Bank of France couldn't care less, but we become automatically involved in a problem of what to do about the Swiss dollar surplus
and the Italian dollar deficit. In the past, these shifts of funds between
foreign countries were largely settled by gold transfers, but now we have
reached the stage where the adjustment is often effected through debits and
credits under our swap network, or similar credit operations. \dots In any
event, the New York Bank has now become virtually the balance wheel of
the international financial machinery and I suspect that the complexity and
magnitude of our operations will continue to grow. \citep{coombs1964}
\end{quote}

According to \citep[p.~98]{viner1943}: "Even under pre-1914 gold standard, moreover, actual gold transfers played only a residual role in settling international liabilities; aside from clearing or offsetting transactions, there was an elaborate pattern of use of international credits to meet---or to postpone---immediate laibilities to payment."

\citeauthor{viner1943} says: ``clearing means matching or offsetting debt bt reciprocal debt, and credit by reciprocal credit" \citep[p.~86]{viner1943}. While showing awareness of the offegting principles of payment, this passage omits the time-matching priciple, which is a debt due today (emmidiately) must be offset only with credit that is due today (emmidiately). Otherwise, offesting does not work.

\citep{cattani2002}: "In banking correspondent banking is the nearest analog to exporting. Here one bank, the correspondent, makes payments to (generally local) parties, for a fee, at the request of its client bank located in another county, state or country.". This is an example of misrepresentation as balance sheet analysis was not applied here.

\citep[p.~38]{spero1980}: ``Foreign branches gave Franklin[ National Bank]'s rivals a competitive advantage beyond service. With foreign branches they had greater access to funds for use both at home and abroad. For example, when credit became less available to banks in the late 1960s due to the Federal Reserve's anti-inflation policy, banks with foreign branches were able to use those branches to borrow funds that were transferred to the head office for use within the United States." From the balance sheet analysis the last line in this quote misrepresents the banking business. Depending on the transacting counterparties and their arrangements via correspondent accounts there are might be different combinations of the trasnactions -- but any case the end result is hierarchical (a) re-assignemnt of debt-credit relationships, or/and (b) creation of the new ones.

\citep[p.~382]{ruggie1982} on Schacht device: ``had the Germans succeeded in their quest to
 establish a "New International Order" after World War II, the designs
 Hjalmar Schacht would have instituted were the very mirror image of Bret-
 ton Woods"-obviously, differences in social purpose again provide the
 key." In the footnote 11 (ibid) he adds: ``11 A brief description may be found in Armand Van Dormael, Brettoh Woods: Birth of a Monetary System (London: Macmillan, 1978), chap. 1. The classic statement of how it actually worked remains Albert O. Hirschman, National Power and the Structure of Foreign Trade, expanded ed. (Berkeley: University of California Press, 1980)."
 
\begin{quote}
 one of the first assignments
 that Keynes undertook after he joined the British Treasury in 1940 was to
 draft the text of a radio broadcast designed to discredit recent propaganda
 proclamations by Walther Funk, minister for economic affairs and president
 of the Reichsbank in Berlin, on the economic and social benefits that the
 ``New Order" would bring to Europe and the world. Keynes was instructed
 to stress the traditional virtues of free trade and the gold standard. But this,
 he felt, "will not have much propaganda value." Britain would have to offer
 ``the same as what Dr. Funk offers, except that we shall do it better and
 more honestly." He had reached the conclusion that only a refinement and
 improvement of the Schachtian device would restore equilibrium after the
 war. \citep[p.~387-388]{ruggie1982}
\end{quote}

On the Schachtian device see \cite{beyen1949}, on Germany's trading practices of the 1930s see \citep{hirschman1945}.

 
 See \citep{hirschman1951} on convertibility. 
 
\subsection{Englisn Monetary Orthordoxy on Cross-Border Transactions}

See \citep[chapter 1]{goschen1864} who developed from \citep[chapter XX]{mill1852}